%!TEX root=documentation-bachlorthesis-speicherarchitektur-lstucker.tex



\documentclass[
oneside, % einseitiges Dokument
a4paper, % Papierformat
pdftex,
fontsize=11pt,
%headsepline, % use headinclude also! (see M. Kohm)
% footsepline, % use footinclude also! (see M. Kohm)
% headinclude, % count head to text body (not to margin)
% footinclude, % count foot to text body (not to margin)
% BCOR8mm, % set extra margin for book fixation
openany,
titlepage, % es wird eine Titelseite verwendet
draft=false, % Status des Dokuments (final/draft)
ngerman % für Umlaute, Silbentrennung etc.
]{scrbook}

%\usepackage{helvet}
% Umlaute ----------------------------------------------------------------------
%   Umlaute/Sonderzeichen wie ‰¸ˆfl direkt im Quelltext verwenden (CodePage).
%   Erlaubt automatische Trennung von Worten mit Umlauten.
% ------------------------------------------------------------------------------
\usepackage[utf8]{inputenc}
\usepackage[T1]{fontenc}
\usepackage{textcomp} % Euro-Zeichen etc.

% Anpassung an Landessprache
\usepackage[ngerman]{babel} %Thomas hatte noch English drin

%!TEX root=documentation-bachlorthesis-speicherarchitektur-lstucker.tex

% meta-information -----------------------------------------------------------
%   Definition von globalen Parametern, die im gesamten Dokument verwendet
%   werden können (z.B auf dem Deckblatt etc.).
%
%   ACHTUNG: Wenn die Texte Umlaute oder ein Esszet enthalten, muss der folgende
%            Befehl bereits an dieser Stelle aktiviert werden:
%            \usepackage[latin1]{inputenc}
% ------------------------------------------------------------------------------
\newcommand{\titel}{Speicherarchitektur für Massendaten einer Webapplikation}
\newcommand{\untertitel}{Webapplikation}
\newcommand{\art}{Bachlorthesis}
\newcommand{\fachgebiet}{Betriebsysteme}
\newcommand{\autor}{Lucien Stucker}
\newcommand{\autoremail}{stuckluc@students.zhaw.ch}
\newcommand{\studienbereich}{Informatik}
\newcommand{\matrikelnr}{06-557-540}
\newcommand{\dozent}{Beat Seeliger}
\newcommand{\dozentemail}{xsel@zhaw.ch}
\newcommand{\jahr}{2012}
\newcommand{\ort}{Zürich}
\newcommand{\logo}{logo_zhaw.png}

\usepackage[fleqn]{amsmath}

\usepackage{geometry}
\usepackage{fancyhdr}
\fancyhead[L]{\leftmark}
\fancyhead[R]{}

\usepackage{graphicx} % Grafiken
\graphicspath{{media/}} % hier liegen die Bilder des Dokuments

% sorgt daf¸r, dass Leerzeichen hinter parameterlosen Makros nicht als Makroendezeichen interpretiert werden
\usepackage{xspace}

\usepackage{enumerate}

% Schrift
\usepackage{helvet}
% \usepackage{lmodern} % bessere Fonts
\usepackage{relsize} % Schriftgröfle relativ festlegen
\usepackage{ascii}

\usepackage{setspace} % Einfache Definition der Zeilenabstände 
\onehalfspacing % Zeilenabstand 1,5 Zeilen



% zum Einbinden von Programmcode
\usepackage{listings}
\usepackage[parfill]{parskip}
\usepackage{color}
\usepackage[table]{xcolor}
\usepackage[T1]{fontenc}
\usepackage[utf8]{inputenc}
\usepackage[toc,page]{appendix}
\usepackage{multirow} 
\usepackage{tabularx}
\usepackage{longtable}

\usepackage{bibgerm}

% Farben
\definecolor{hsztblue}{cmyk}{0.946,0.452,0,0.349}
\definecolor{hellgelb}{rgb}{1,1,0.9}
\definecolor{light-gray}{gray}{0.95}

% URL verlinken, lange URLs umbrechen etc. -------------------------------------
\usepackage{url}
%% Define a new 'leo' style for the package that will use a smaller font.
\makeatletter
\def\url@leostyle{%
  \@ifundefined{selectfont}{\def\UrlFont{\sf}}{\def\UrlFont{\small\ttfamily}}}
\makeatother
\urlstyle{leo} % Now actually use the newly defined style.




%\fancyfoot[RO, LE] {\thepage}

% http://stackoverflow.com/questions/586572/make-code-in-latex-look-nice
%\lstset{breaklines=true,frame=single,basicstyle=\ttfamily}

\lstset{
basicstyle=\small\ttfamily,
numbers=left,
numberstyle=\tiny,
frame=b,
columns=fullflexible,
showstringspaces=true,
breaklines=true
}

\lstdefinelanguage{JavaScript}{
     keywords={attributes, class, classend, do, empty, endif, endwhile, fail, function, functionend, if, implements, in, inherit, inout, not, of, operations, out, return, set, then, types, while, use},
     keywordstyle=\color{blue}\bfseries,
     ndkeywords={},
     ndkeywordstyle=\color{black}\bfseries,
     identifierstyle=\color{black},
     sensitive=false,
     comment=[l]{//},
     commentstyle=\color{black}\ttfamily,
     stringstyle=\color{red}\ttfamily
  }


% \renewcommand{\familydefault}{\sfdefault} % Standardschriftart Helvet

\newcommand*\oldurlbreaks{} % Handle long urls
\let\oldurlbreaks=\UrlBreaks
\renewcommand{\UrlBreaks}{\oldurlbreaks\do\a\do\b\do\c\do\d\do\e%
  \do\f\do\g\do\h\do\i\do\j\do\k\do\l\do\m\do\n\do\o\do\p\do\q%
  \do\r\do\s\do\t\do\u\do\v\do\w\do\x\do\y\do\z\do\?\do\&}



\usepackage{titlepic} % http://typethinker.blogspot.com/2008/08/picture-on-title-page-in-latex.html

\usepackage{hyperref}
\hypersetup{
    unicode=false, % non-Latin characters in Acrobat’s bookmarks
    pdftoolbar=true, % show Acrobat’s toolbar?
    pdfmenubar=true, % show Acrobat’s menu?
    pdffitwindow=false, % window fit to page when opened
    pdfstartview={FitH}, % fits the width of the page to the window
    pdftitle={\art - \titel}, % title
    pdfauthor={\autor}, % author
    pdfsubject={\titel \untertitel}, % subject of the document
    pdfcreator={\autor}, % creator of the document
    pdfproducer={\autor}, % producer of the document
    pdfkeywords={solr} {nutch} {information} {retrieval}, % list of keywords
    pdfnewwindow=true, % links in new window
    colorlinks=true, % false: boxed links; true: colored links
    linkcolor=black, % color of internal links
    citecolor=black, % color of links to bibliography
    filecolor=black, % color of file links
    urlcolor=black % color of external links
}


% http://en.wikibooks.org/wiki/LaTeX/Glossary
%\usepackage[toc,xindy,acronym]{glossaries}

% \usepackage{acronym} % [printonlyused]


%\usepackage[toc,xindy]{glossaries}
%\makeglossaries
%\input{Include/Glosar_Terme}

% für Index-Ausgabe mi
\usepackage{makeidx}
\setcounter{secnumdepth}{4} %nunmbers
\setcounter{tocdepth}{3} %inhaltsverzeichnis
\makeindex

% Glossar
\usepackage[
acronym,      %ein Abkürzungsverzeichnis erstellen
toc]          %Einträge im Inhaltsverzeichnis]{glossaries}
{glossaries}
\makeglossaries

% Seitenraender -----------------------------------------------------------------
\setlength{\headheight}{26pt}
\geometry{a4paper, top=25mm, left=40mm, right=25mm, bottom=30mm,
headsep=10mm, footskip=12mm}
% Uni Freiburg entfield links 2,5 cm; rechts 2,5 cm; oben 2,5 cm; unten 2 cm

\usepackage[font=small,labelfont=bf]{caption} % Caption auch in non-float tabellen/bildern setzen. Benutzung: \mycaption{table|figure}{Titel}

\DeclareCaptionFont{white}{\color{white}}
\DeclareCaptionFormat{listing}{\colorbox{gray}{\parbox{\textwidth}{#1#2#3}}}
\captionsetup[lstlisting]{format=listing,labelfont=white,textfont=white}

\newcommand{\mycaption}[2]{
\begin{minipage}[c]{0.8\linewidth}
\renewcommand{\figurename}{Abbildung}
\captionsetup{type=figure}
\captionof{#1}{#2}
\end{minipage}
}

\newcounter{qcounter}
\newcounter{para} \setcounter{para}{0}
\newcommand{\newpara}{%
  \refstepcounter{para}%
  \noindent\llap{\thepar. }\quad}
\newcommand{\oldpara}[1]{%
  \noindent\llap{\ref{#1}. }\quad}

\newcommand{\refbf}[1]{\textbf{\ref{#1}}}
\newcommand{\refsoll}[1]{\textbf{\nameref{#1}}}
\newcommand{\refko}[1]{\textbf{\nameref{#1}}}
\newcommand{\reftab}[1]{\textbf{Tabelle (\ref{#1})}}
\newcommand{\reflisting}[1]{\textbf{Listing \ref{#1}}}
\newcommand{\refabb}[1]{\textbf{Abbildung \ref{#1}}}
\newcommand{\refchap}[1]{\textbf{Kapitel \ref{#1}}}
\newcommand{\refsec}[1]{\textbf{Absatz \ref{#1}}}
\newcommand{\refeql}[1]{\textbf{Gleichung \ref{#1}}}
\newcommand{\refeqlb}[1]{\textbf{Berechnung \ref{#1}}}



\usepackage{tabularx}
\newcolumntype{L}[1]{>{\raggedright\arraybackslash}p{#1}} % linksbündig mit Breitenangabe
\newcolumntype{C}[1]{>{\centering\arraybackslash}p{#1}} % zentriert mit Breitenangabe
\newcolumntype{R}[1]{>{\raggedleft\arraybackslash}p{#1}} % rechtsbündig mit Breitenangabe
\newcommand{\ltab}{\raggedright\arraybackslash} % Tabellenabschnitt linksbündig
\newcommand{\ctab}{\centering\arraybackslash} % Tabellenabschnitt zentriert
\newcommand{\rtab}{\raggedleft\arraybackslash} % Tabellenabschnitt rechtsbündig

\setlength{\baselineskip}{16.5pt} % 16 pt usual spacing between lines

