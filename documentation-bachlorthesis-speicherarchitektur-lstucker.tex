%!TEX root=documentation-bachlorthesis-speicherarchitektur-lstucker.tex



\documentclass[
oneside, % einseitiges Dokument
a4paper, % Papierformat
pdftex,
fontsize=11pt,
%headsepline, % use headinclude also! (see M. Kohm)
% footsepline, % use footinclude also! (see M. Kohm)
% headinclude, % count head to text body (not to margin)
% footinclude, % count foot to text body (not to margin)
% BCOR8mm, % set extra margin for book fixation
openany,
titlepage, % es wird eine Titelseite verwendet
draft=false, % Status des Dokuments (final/draft)
ngerman % für Umlaute, Silbentrennung etc.
]{scrbook}

%\usepackage{helvet}
% Umlaute ----------------------------------------------------------------------
%   Umlaute/Sonderzeichen wie ‰¸ˆfl direkt im Quelltext verwenden (CodePage).
%   Erlaubt automatische Trennung von Worten mit Umlauten.
% ------------------------------------------------------------------------------
\usepackage[utf8]{inputenc}
\usepackage[T1]{fontenc}
\usepackage{textcomp} % Euro-Zeichen etc.

% Anpassung an Landessprache
\usepackage[ngerman]{babel} %Thomas hatte noch English drin

%!TEX root=documentation-bachlorthesis-speicherarchitektur-lstucker.tex

% meta-information -----------------------------------------------------------
%   Definition von globalen Parametern, die im gesamten Dokument verwendet
%   werden können (z.B auf dem Deckblatt etc.).
%
%   ACHTUNG: Wenn die Texte Umlaute oder ein Esszet enthalten, muss der folgende
%            Befehl bereits an dieser Stelle aktiviert werden:
%            \usepackage[latin1]{inputenc}
% ------------------------------------------------------------------------------
\newcommand{\titel}{Speicherarchitektur für Massendaten einer Webapplikation}
\newcommand{\untertitel}{Webapplikation}
\newcommand{\art}{Bachlorthesis}
\newcommand{\fachgebiet}{Betriebsysteme}
\newcommand{\autor}{Lucien Stucker}
\newcommand{\autoremail}{lstucker@hsz-t.ch}
\newcommand{\studienbereich}{Informatik}
\newcommand{\matrikelnr}{06-557-540}
\newcommand{\dozent}{Beat Seeliger}
\newcommand{\dozentemail}{bseliger@hsz-t.ch}
\newcommand{\jahr}{2011}
\newcommand{\ort}{Zürich}
\newcommand{\logo}{logo_hszt.jpg}


\usepackage{geometry}
\usepackage{fancyhdr}
\fancyhead[L]{\leftmark}
\fancyhead[R]{}

\usepackage{graphicx} % Grafiken
\graphicspath{{media/}} % hier liegen die Bilder des Dokuments

% sorgt daf¸r, dass Leerzeichen hinter parameterlosen Makros nicht als Makroendezeichen interpretiert werden
\usepackage{xspace}

\usepackage{enumerate}

% Schrift
\usepackage{helvet}
% \usepackage{lmodern} % bessere Fonts
\usepackage{relsize} % Schriftgröfle relativ festlegen
\usepackage{ascii}

\usepackage{setspace} % Einfache Definition der Zeilenabstände 
\onehalfspacing % Zeilenabstand 1,5 Zeilen



% zum Einbinden von Programmcode
\usepackage{listings}
\usepackage[parfill]{parskip}
\usepackage{color}
\usepackage[table]{xcolor}
\usepackage[T1]{fontenc}
\usepackage[utf8]{inputenc}
\usepackage[toc,page]{appendix}
\usepackage{multirow} 
\usepackage{tabularx}
\usepackage{longtable}

% Farben
\definecolor{hsztblue}{cmyk}{0.946,0.452,0,0.349}
\definecolor{hellgelb}{rgb}{1,1,0.9}
\definecolor{light-gray}{gray}{0.95}

% URL verlinken, lange URLs umbrechen etc. -------------------------------------
\usepackage{url}
%% Define a new 'leo' style for the package that will use a smaller font.
\makeatletter
\def\url@leostyle{%
  \@ifundefined{selectfont}{\def\UrlFont{\sf}}{\def\UrlFont{\small\ttfamily}}}
\makeatother
\urlstyle{leo} % Now actually use the newly defined style.


\usepackage{minitoc} % ToC for chapters
\minitoc[n] % No caption for mini ToCs



%\fancyfoot[RO, LE] {\thepage}

% http://stackoverflow.com/questions/586572/make-code-in-latex-look-nice
%\lstset{breaklines=true,frame=single,basicstyle=\ttfamily}

\lstset{
basicstyle=\small\ttfamily,
numbers=left,
numberstyle=\tiny,
frame=b,
columns=fullflexible,
showstringspaces=true,
breaklines=true
}

\lstdefinelanguage{JavaScript}{
     keywords={attributes, class, classend, do, empty, endif, endwhile, fail, function, functionend, if, implements, in, inherit, inout, not, of, operations, out, return, set, then, types, while, use},
     keywordstyle=\color{blue}\bfseries,
     ndkeywords={},
     ndkeywordstyle=\color{black}\bfseries,
     identifierstyle=\color{black},
     sensitive=false,
     comment=[l]{//},
     commentstyle=\color{black}\ttfamily,
     stringstyle=\color{red}\ttfamily
  }


% \renewcommand{\familydefault}{\sfdefault} % Standardschriftart Helvet

\newcommand*\oldurlbreaks{} % Handle long urls
\let\oldurlbreaks=\UrlBreaks
\renewcommand{\UrlBreaks}{\oldurlbreaks\do\a\do\b\do\c\do\d\do\e%
  \do\f\do\g\do\h\do\i\do\j\do\k\do\l\do\m\do\n\do\o\do\p\do\q%
  \do\r\do\s\do\t\do\u\do\v\do\w\do\x\do\y\do\z\do\?\do\&}



\usepackage{titlepic} % http://typethinker.blogspot.com/2008/08/picture-on-title-page-in-latex.html

\usepackage{hyperref}
\hypersetup{
    unicode=false, % non-Latin characters in Acrobat’s bookmarks
    pdftoolbar=true, % show Acrobat’s toolbar?
    pdfmenubar=true, % show Acrobat’s menu?
    pdffitwindow=false, % window fit to page when opened
    pdfstartview={FitH}, % fits the width of the page to the window
    pdftitle={\art - \titel}, % title
    pdfauthor={\autor}, % author
    pdfsubject={\titel \untertitel}, % subject of the document
    pdfcreator={\autor}, % creator of the document
    pdfproducer={\autor}, % producer of the document
    pdfkeywords={solr} {nutch} {information} {retrieval}, % list of keywords
    pdfnewwindow=true, % links in new window
    colorlinks=true, % false: boxed links; true: colored links
    linkcolor=black, % color of internal links
    citecolor=black, % color of links to bibliography
    filecolor=black, % color of file links
    urlcolor=black % color of external links
}

% http://en.wikibooks.org/wiki/LaTeX/Glossary
%\usepackage[toc,xindy,acronym]{glossaries}

\usepackage{acronym} % [printonlyused]


%\usepackage[toc,xindy]{glossaries}
%\makeglossaries
%\input{Include/Glosar_Terme}

% für Index-Ausgabe mi
\usepackage{makeidx}
\setcounter{secnumdepth}{4} %nunmbers
\setcounter{tocdepth}{3} %inhaltsverzeichnis
\makeindex

% Glossar
\usepackage[
acronym,      %ein Abkürzungsverzeichnis erstellen
toc]          %Einträge im Inhaltsverzeichnis]{glossaries}
{glossaries}
\makeglossaries

% Seitenraender -----------------------------------------------------------------
\setlength{\headheight}{26pt}
\geometry{a4paper, top=25mm, left=40mm, right=25mm, bottom=30mm,
headsep=10mm, footskip=12mm}
% Uni Freiburg entfield links 2,5 cm; rechts 2,5 cm; oben 2,5 cm; unten 2 cm

\usepackage[font=small,labelfont=bf]{caption} % Caption auch in non-float tabellen/bildern setzen. Benutzung: \mycaption{table|figure}{Titel}

\DeclareCaptionFont{white}{\color{white}}
\DeclareCaptionFormat{listing}{\colorbox{gray}{\parbox{\textwidth}{#1#2#3}}}
\captionsetup[lstlisting]{format=listing,labelfont=white,textfont=white}

\newcommand{\mycaption}[2]{
\begin{minipage}[c]{0.8\linewidth}
\renewcommand{\figurename}{Abbildung}
\captionsetup{type=figure}
\captionof{#1}{#2}
\end{minipage}
}

\newcounter{qcounter}
\newcounter{para} \setcounter{para}{0}
\newcommand{\newpara}{%
  \refstepcounter{para}%
  \noindent\llap{\thepar. }\quad}
\newcommand{\oldpara}[1]{%
  \noindent\llap{\ref{#1}. }\quad}

\newcommand{\refbf}[1]{\textbf{\ref{#1}}}
\newcommand{\refsoll}[1]{\textbf{\nameref{#1}}}
\newcommand{\refko}[1]{\textbf{\nameref{#1}}}
\newcommand{\reftab}[1]{\textbf{Tabelle (\ref{#1})}}
\newcommand{\reflisting}[1]{\textbf{Listing \ref{#1}}}
\newcommand{\refabb}[1]{\textbf{Abbildung \ref{#1}}}
\newcommand{\refchap}[1]{\textbf{Kapitel \ref{#1}}}
\newcommand{\refsec}[1]{\textbf{Absatz \ref{#1}}}
\newcommand{\refeql}[1]{\textbf{Gleichung \ref{#1}}}
\newcommand{\refeqlb}[1]{\textbf{Berechnung \ref{#1}}}

\usepackage[fleqn]{amsmath}

\usepackage{tabularx}
\newcolumntype{L}[1]{>{\raggedright\arraybackslash}p{#1}} % linksbündig mit Breitenangabe
\newcolumntype{C}[1]{>{\centering\arraybackslash}p{#1}} % zentriert mit Breitenangabe
\newcolumntype{R}[1]{>{\raggedleft\arraybackslash}p{#1}} % rechtsbündig mit Breitenangabe
\newcommand{\ltab}{\raggedright\arraybackslash} % Tabellenabschnitt linksbündig
\newcommand{\ctab}{\centering\arraybackslash} % Tabellenabschnitt zentriert
\newcommand{\rtab}{\raggedleft\arraybackslash} % Tabellenabschnitt rechtsbündig

\setlength{\baselineskip}{16.5pt} % 16 pt usual spacing between lines


%Footnote
\newcommand{\footnoteremember}[2{\footnote{#2}\newcounter{#1}\setcounter{#1}%{\value{footnote}}}
\newcommand{\footnoterecall}[1]{\footnotemark[\value{#1}]}

% Meta-Informationen -----------------------------------------------------------
%   Informationen über das Dokument, wie z.B. Titel, Autor, Matrikelnr. etc
%   werden in der Datei meta.tex definiert und können danach global
%   verwendet werden.
% ------------------------------------------------------------------------------
%!TEX root=documentation-bachlorthesis-speicherarchitektur-lstucker.tex

% meta-information -----------------------------------------------------------
%   Definition von globalen Parametern, die im gesamten Dokument verwendet
%   werden können (z.B auf dem Deckblatt etc.).
%
%   ACHTUNG: Wenn die Texte Umlaute oder ein Esszet enthalten, muss der folgende
%            Befehl bereits an dieser Stelle aktiviert werden:
%            \usepackage[latin1]{inputenc}
% ------------------------------------------------------------------------------
\newcommand{\titel}{Speicherarchitektur für Massendaten einer Webapplikation}
\newcommand{\untertitel}{Webapplikation}
\newcommand{\art}{Bachlorthesis}
\newcommand{\fachgebiet}{Betriebsysteme}
\newcommand{\autor}{Lucien Stucker}
\newcommand{\autoremail}{lstucker@hsz-t.ch}
\newcommand{\studienbereich}{Informatik}
\newcommand{\matrikelnr}{06-557-540}
\newcommand{\dozent}{Beat Seeliger}
\newcommand{\dozentemail}{bseliger@hsz-t.ch}
\newcommand{\jahr}{2011}
\newcommand{\ort}{Zürich}
\newcommand{\logo}{logo_hszt.jpg}


\begin{document}

\frontmatter
\pagestyle{empty}
\pagenumbering{alph}
%!TEX root=documentation-bachlorthesis-speicherarchitektur-lstucker.tex

\thispagestyle{plain}
\begin{titlepage}
\sffamily
	\renewcommand{\headheight}{4.5cm}
	% \renewcommand{\familydefault}{\sfdefault} % Standardschriftart Helvet 
    \begin{center}

	\huge{\textbf{\titel}}\\
%	 \vspace{1.5cm}
%	\LARGE{\untertitel}\\
	 \vspace{1.5 cm}
	\LARGE{\textbf{\art}}\\
	 \vspace{2cm}
	
    	\includegraphics[width=80mm, keepaspectratio = true]{\logo}

    \normalsize
    \vspace{2cm}
    \vspace{2.5cm}

 \normalsize{
    \begin{tabular}{ll}
     Abgabe: & 22. Mai 2012\\
     Präsentation: & 6. Juli 2011\\\\
     Student: &\autor, \autoremail\\
     Dozent: & \dozent, \dozentemail \\
     Studienbereich: & \studienbereich\\
    \end{tabular}\\
    }
\end{center}
\rmfamily
\end{titlepage} %title
\cleardoublepage

\pagestyle{fancy}

\pagenumbering{roman}
\setcounter{page}{1}

\phantomsection
 \renewcommand{\contentsname}{Inhaltsverzeichnis}
\renewcommand{\chaptername}{Kapitel}
% \renewcommand{\listfigurename}{Abbildungsverzeichnis}


\tableofcontents
\cleardoublepage

\listoffigures % Abbildungsverzeichnis
\mtcaddchapter\addcontentsline{toc}{chapter}{\listfigurename}
\cleardoublepage

% Tabellenverzeichnis
\listoftables
\mtcaddchapter\addcontentsline{toc}{chapter}{\listtablename}
\cleardoublepage

\lstlistoflistings  % Listings-Verzeichnis
\mtcaddchapter\addcontentsline{toc}{chapter}{Source Code Verzeichnis}
\cleardoublepage %tableofcontents
\cleardoublepage

\mainmatter
\pagenumbering{arabic}
\setcounter{page}{1}
% Load at the beginngin that we can use it
%!TEX root=../documentation-bachlorthesis-speicherarchitektur-lstucker.tex

\newglossaryentry{RFC}{ name={RFC}, description={Request for Comments (RFC) sind Dokumente, über Internet, inklusive der technischen Spezifikation und Richtlinien, welche von der Organisation Internet Engineering Task Force entwickelt wurde. "'Das RFC wird erst nach erfolgter Diskussion unter der Aussicht des Internet Architecture Board (IAB) herausgegeben und fungiert als Quasistandard. Jedes RFC enthält eine eindeutige, vorlaufende Nummer, die kein zweites Mal zu gewiesen wird."' \cite{MicrosoftComputerLex}  \url{http://www.rfc-editor.org/}}}

\newglossaryentry{UDP}{ name={UDP}, description={  adfajsdfjadslkfjaödjfölaksdjfajsklfj }}

\newglossaryentry{IBM}{ name={IBM}, description={International Business Machines Corporation (kurz IBM) ist ein führendes unternehmen in Software, Hardware und IT-Dienstleistung Bereich. }}

\newglossaryentry{IBM}{ name={CIFS}, description={Common Internet File System (kurz CIFS) wurde 1996 von Microsoft eingeführt und beschreibt eine erweiterte Version von SMB. CIFS und SMB sind eine  Netzwerkdateisystem vergleichbar mit NFS und wird vorwiegend im MS Windows Bereich eingesetzt. }}


\newglossaryentry{XDR}{ name={XDR}, description={Die eXternal Data Representation (kurz XDR) Spezifikation stellt ein Standardisierte Verfahren zur Präsentation von gebräuchlichsten Daten Typen über das Netzwerk zur Verfügung. Dies löst das Problem der verschiedenen Byte-Reihenfolge (Big Endian), Speicherausrichtung auf unterschiedlichen Kommunikations Partner}}

\newglossaryentry{POSIX}{ name={POSIX}, description={Portable Operating System Interface (kurz POSIX) ist eine von IEEE entwickelter Standard, welche die Schnittstelle zwischen Applikation und Betriebsystem darstellt. Die aktuelle Version des Standards ist IEEE Std 1003.1-2008 \url{http://www.opengroup.org/austin/papers/posix_faq.html} }}

\newglossaryentry{FUSE}{ name={FUSE}, description={Filesystem in Userspace (kurz FUSE), ermöglicht die Implementierung eines voll Funktionsfähigen Dateisystem in Userspace. Normaler weise laufen 
FUSE wurde urspünglich Entwickelt um AVFS zu unterstützen, ist jedoch heute ein seperates Projekt. \url{http://fuse.sourceforge.net/}}} 

\newglossaryentry{RPC}{File locking erlaubt es einen Prozess den exklusiven Zugriff auf eine Datei oder teile einer Datei und zwingt ander Prozesse die auf die selbe Ressource zugreifen wollen zu warten bis das Locking aufgehoben wurde.} 

\newglossaryentry{FileLocking}{File locking erlaubt es einen Prozess den exklusiven Zugriff auf eine Datei oder teile einer Datei und zwingt ander Prozesse die auf die selbe Ressource zugreifen wollen zu warten bis das Locking aufgehoben wurde.} 

\newglossaryentry{API}{ name={API}, description={Application Programming Interface (kurz API) auch Anwendungsprogrammierschnittstelle genannt. "'Ein Satz an Routinen, die vom Betriebsystem des Computers für die Verwendung aus Anwendungsprogrammen heraus angeboten werden und diverse Dienste zur Verfügung stellen."' \cite{MicrosoftComputerLex} }}

\newglossaryentry{MIT{ name={MIT}, description={Die MIT Lizenz stammt von Massachusetts Institute of Technology und erlaub die die Verwendung von Software welche Quelloffen ist als auch software welche nicht Quell geschlossene ist. Die genauen Lizenz Bestimmungen sind unter folgenden URL zu finden \url{http://www.opensource.org/licenses/mit-license.php}}}

\newglossaryentry{GNU GPL{ name={GNU GPL}, description={Die GNU General Public License Lizenz auch GPL genannt stammt von der Free Software Foundation und regelt die Lizenzierung von Freie Software. Es gibt drei Versionen der GPL welche unter folgenden URL beschrieben sind \url{http://www.gnu.org/licenses/}}}

\newglossaryentry{Ruby{ name={Ruby}, description={Ruby ist eine interpretierte und objektorientierte Programmiersprache und beinhaltet einige bewährte Prinzipien wie z.B. "'DuckTyping"' und "'Principle of Least Suprice"'. Die Entwickler von Ruby stellen sich selber den Anspruch eine Programmiersprache zu schaffen, die durch Ihre Natürlichkeit einfach erlernbar ist und es den Programmierern ermöglicht, einfachen und übersichtlichen Code zu schreiben, welcher aber nicht seine Mächtigkeit und innere Komplexität verliert.
Ruby hat sich in den letzten Jahren von einer kaum beachteten Programmiersprache zu einem Publikums-Magneten entwickelt. Es gibt eine stetig wachsende offene Community "'Gemeinschaft"', welche sich und die Sprache durch Austausch von Erfahrungen und Ideen weiterbringen möchte.
Ein Grund für die hohe Bereitschaft der Community die Sprache Ruby weiter zu bringen ist der Umstand, dass die Programmiersprache vollständig OpenSource ist und unter der Lizenz der Ruby-License und GPL steht. Zudem ist die Sprache fast beliebig erweiterbar und bestehende Funktionen können einfach durch eigene Funktionen ausgetauscht werden.}}

\input{chapter/vorwort}
%!TEX root=../documentation-bachlorthesis-speicherarchitektur-lstucker.tex
\cleardoublepage
\chapter{Zusammenfassung}

Der Auftraggeber, ein Zürcher Startup Unternehmen, betreibt und entwickelt eine Bild-Archiv Webapplikation für Kunden mit hohen Ansprüchen an die Qualität ihrer Bilder. Die Anwendung erlaubt das Speichern, Archivieren, Verwalten und die Druckaufbereitung von qualitativ hochauflösenden digitalen Bildern. Ein einzelnes Bild kann eine Speichergrösse von bis zu 2 Gigabyte beanspruchen. Das Archiv belegt schon heute mehr als 2.5 Terrabyte Diskplatz.

Die Aufgabe dieser Arbeit besteht darin, ein geeignetes Speichersystem zu evaluieren, welches die Anforderungen für eine Speichergrösse von 11.5 Tebibyte in Szenario 1 und für 218 Tebibyte in Szenario 2 am besten erfüllt. In der Arbeit sollen die Anforderungen bezüglich Skalierbarkeit der Speicherkapazität und Anzahl Datenabfragen, der Datendurchsatz, die Datenverfügbarkeit, die Datenintegrität und Wirtschaftlichkeit des Gesamtsystems untersucht werden.

Bis anhin werden die Daten auf einem einzelnen gehosteten Server gespeichert, auf welchem auch die Webapplikation betrieben wird. Um die Daten vor Verlust zu schützen, sind die Festplatten des Servers mit einem RAID-5 System konfiguriert. Die verfügbare Speicherkapazität ist bereits zu über 50\% ausgelastet und kann nicht ohne Upgrade auf ein neues System erweitert werden. Bei gleichbleibendem Datenwachstum ist die freie Kapazität innert sieben Monaten ausgeschöpft. Das bestehende System kann die Anforderungen bezüglich Verfügbarkeit und Skalierbarkeit nicht erfüllen. Das Überdenken der künftigen Speicherarchitektur drängt sich deshalb auf.
 
Die heutigen am Markt erhältlichen Speicherarchitekturen lassen sich in der obersten Kategorie in Block-, Datei- und Objekt-basierte Systeme unterteilen. Die Block-basierende Speicherarchitektur ist wohl die traditionellste und die am weitverbreitetste Form von allen. Block-basierende Speicher können mittels FibreChannel oder iSCSI über ein Speichernetzwerk (SAN) an mehrere Server-Systeme zur Verfügung gestellt werden. Mit dem Aufkommen von Desktop-Computern, wuchs der Bedarf die Daten zentral für alle Benutzer ablegen und dezentral auf diese zugreifen zu können. Aus dieser Anforderung heraus wurden die Datei-basierenden Speicherarchitekturen entwickelt. Dazu gehört Network File System (NFS). Objekt-basierende Speichersysteme werden zunehmend bei sehr hohem Speicherkapazitätsbedarf oder wo die Daten für Datenanalyse aus Performanceoptimierung auf verschiedene Systeme gespeichert werden sollen.

Für die Evaluation wurde das Analytic Hierarchy Process (AHP) Verfahren gegenüber der Nutzwertanalyse bevorzugt. Bei AHP wird durch den hierarchischen Analyseprozess die Evaluation stark strukturiert. Der Analyseprozess unterstützt mit einer Softwareanwendung erlaubt die neutrale Berechnung der zu untersuchenden Lösungsvarianten. Bei vielen zu vergleichenden Kriterien erwächst zwar schnell ein etwas höherer Arbeitsaufwand, das Ergebnis der Analyse ist aber jederzeit gut dokumentiert und nachvollziehbar.

Für die Evaluation wurde jeweils ein Produktevertreter für jede der zu untersuchenden Speicherarchitekturen gewählt. Es wurden die heute am Markt angebotenen Architekturlösungen berücksichtigt, wie die Block-basierenden, Datei-basierenden und Objekt-basierenden Produkte. Ferner wurden ein Vertreter von Online Speicher (Amazon S3) und der bisherige Anbieter der bestehenden Lösung einbezogen. Anbieter von Cloud-Speicherplatz wie Google Drive, iCloud, myDrive oder Dropbox bieten zwar Speicherlösungen für Endanwender (consumer frontend services) an, konnten aber wegen der fehlenden Möglichkeit die bestehende Anwendung des Auftraggebers einzubinden, nicht berücksichtigt werden. 

Untersucht wurde zudem die Eignung von einer Private Cloud. Private Cloud beinhaltet den Aufbau eines eigenen Datencenters basierend auf der OpenStack Object Storage Lösung, währendem bei Public Cloud Hosting-Anbieter gemeint sind, welche die gesamte IT-Infrastruktur mit einer bestimmten Speicherarchitektur samt Unterhalt und Verwaltung als Service anbieten. Es hat sich gezeigt, dass der Betrieb eines eigenen Datencenters mit eigenen Mitarbeitern zwar interessant ist, sich finanziell im Vergleich zu Hosting-Lösungen jedoch nicht rechnet.

Als Vertreter für Block-basierende Speicher wurde das Speichersystem des Herstellers NetApp gewählt, welches den Speicher über iSCSI den Applikations-Servern zur Verfügung stellt. Als Vertreter für Datei-basierende Speicher wurde dieselbe NetApp gewählt, welche den Speicher auch über NFS den Applikations-Servern zur Verfügung stellt. Als Vertreter von Objekt-basierenden Speichern wurde OpenStack Object Storage gewählt, die ebenfalls für Online Speicher eingesetzt wird. OpenStack Object Storage verwendet für die Speicherung gewöhnliche Computer-Systeme und erreicht durch redundante Verteilung der Daten eine hohe Verfügbarkeit. Dabei sind alle Daten auf mindestens drei Computer-Systemen gespeichert. Als Vertreter von Online Speicher wurde Amazon S3 berücksichtigt.
Das Evaluationsresultat empfiehlt als Gewinner für beide der oben beschriebenen Speicherbedarfsszenerien Amazon S3. Amazon S3 erhielt wegen den niedrigen Gesamtkosten, der beliebig skalierbaren Speicherkapazität, der hohen Redundanz und für die Sicherstellung der Datenintegrität die meisten Punkte. An zweiter Stelle bei Szenario zwei platzierte sich NetApp iSCSI dicht gefolgt von OpenStack Object Storage, welche beide aufgrund der hohen Personalkosten sich hinter Amazon S3 klassierten.

Für die OpenStack Object Storage wurde ein geeignetes System für die Machbarkeitsstudie aufgebaut und erfolgreich getestet. Aufgrund dass sich Amazon S3 und OpenStack Object Storage ändeln, kann abgeleitet werden, dass auch die empfohlene Lösung mit Amazon S3 umsetzbar ist. 
!TEX root=../documentation-bachlorthesis-speicherarchitektur-lstucker.tex

\cleardoublepage
\chapter{Definitionen}

\section{Speichereinheit}
In der Arbeit werden die Angaben zu den Speichereinheiten anhand des SI-Standards \cite{Technology1998} verwendet, welcher zwischen Dezimal Präfixe, wie sie bei Festplatten angewendet werden, und Binär Präfixen unterscheidet.

\section{RAID-5}\label{RAID-5}

Ein RAID-5 Festplatten-Array besteht aus N identischen Festplatten auf welchen die Daten verteilt gespeichert sind. Die Einheit an Daten, welche auf derselben physischen Festplatte platziert ist, bevor weitere Daten auf eine andere Festplatte geschrieben werden, wird als eine Stripe-Einheit bezeichnet.
In der \refabb{abb:RAID-5 Architektur} wird eine Stripe-Einheit durch Blöcke dargestellt. Z.B. stellt A1 eine Stripe-Einheit dar. \cite{Kuratti1995}

Stripe-Einheiten, welche auf allen Festplatten verteilt denselben physikalischen Platz bzw. Adressbereich belegen, bezeichnet man als Stripe. In der \refabb{abb:RAID-5 Architektur} sind diese durch gleichfarbige Blöcke dargestellt. Hier repräsentieren z.B. die Stripe-Einheiten A1, A2, A3 und $A_{p}$ einen Stripe.
Jeder Stripe enthält mehrere Daten Stripe-Einheiten und eine Parität Stripe-Einheit. Eine Parität Stripe-Einheit ist eine XOR-Verknüpfung aller Daten Stripe-Einheiten eines Stripes. Durch die Speicherung und Berechnung einer Parität Stripe-Einheit pro Stripe, ermöglicht es in einem RAID-5 den Ausfall einer einzelnen Festplatte zu kompensieren. Die verlorenen Stripe-Einheiten der ausgefallenen Festplatte können durch das Lesen aller vorhandenen Daten Stripe-Einheiten und der Parität Stripe-Einheit der noch intakten Festplatten wiederberechnet bzw. rekonstruiert werden. Die Anzahl der Stripe-Einheiten eines Stripes ist durch die Stripe Width  ($W_{s}$) definiert. Die Anzahl der Daten Stripe-Einheiten in jedem Stripe ist durch $W_{s} -1$ definiert. \cite{Kuratti1995}

\begin{center}
\includegraphics[width=\linewidth, keepaspectratio = true]{media/Raid-5.png}
\mycaption{figure}{\label{abb:RAID-5 Architektur} RAID-5 Architektur \textit{RAID-5} \cite{Wikipedia2006}}
\end{center}

Die einzelnen Parität Stripe-Einheiten werden in rotierender Folge über die Festplatten verteilt. Dieses Verfahren trägt dazu bei, die I/O Last, welche durch Anfragen eine Aktualisierung der Parität veranlassen, auf die einzelnen Festplatten besser zu verteilen. Weil die Daten in einem RAID-5 auf mehrere Festplatten verteilt sind, erhöht sich die Wahrscheinlichkeit, dass eine Festplatte bei einer I/O Operation beteiligt ist, was wiederum den Datendurchsatz und die I/O Rate eines Array erhöht. \cite{Kuratti1995}

Für die Speicherung der Parität, muss ein Teil der gesamten Festplattenkapazität reserviert werden. Die für die Daten verfügbare Nettokapazität eines RAID-5 Arrays mit n Festplatten und s Speicherkapazität pro Festplatte lässt sich mit folgender Formel \cite{Kuratti1995} berechnen:

\begin{equation}
\mbox{Speicherkapaziät RAID-5}=(N-1)*S
\label{eqn:MaxSpeicherkapazitätRAID5}
\end{equation}
 
\section{Spare-Disk}
Spare-Disk sind leere Festplatten, welche bei einen Ausfall einer Festplatte im RAID vom RAID System automatisch als Ersatz Festplatte verwendet werden kann. Durch den Einsatz von Spare-Disk verkleinert sich die MMTR, dadurch das die Wiederherstellung des RAID sofort nach dem erkennen des Festplattenausfalls starten kann. 

\section{MTTR}
MTTR  ist die Abkürzung von "Mean Time To Recovery"' und bedeutet so viel wie die durchschnittliche Zeit, welche eine Komponente benötigt, um sich nach einem Fehler wieder herzustellen. Die MTTR in einen RAID-5 berechnet sich aus der Zeit bis eine Ersatzfestplatte installiert ist und der Disk Wiederherstellungszeit. 

\begin{equation}
MTTR_{RAID}=Replacement+Rebuild_{RAID}
\label{eqn:MTTR-RAID-5}
\end{equation}

Die Disk Wiederherstellungszeit in einem RAID-5 kann von den folgenden Faktoren abhängig sein:
\begin{itemize}
\item Anzahl der Festplatten
\item Speicherkapazität
\item Periodisierung der Rekonstruktion gegenüber dem normalen I/O
\item  normale I/O-Last während der Rekonstruktion
\item  CPU Last
\end{itemize}

Für die Ist-Analyse stehen keine aussagekräftigen Statistiken für die Disk Wiederherstellungszeit zur Verfügung. Es wird deshalb angenommen, dass heute eine allfällige MTTR 24 Stunden betragen würde.


\section{MTBF}
MTBF ist die Abkürzung von "Mean Time Between Failure"' und bedeutet so viel wie die durchschnittliche Betriebszeit einer Komponenten bis ein Fehler eintritt. Hersteller von Festplatten geben diesen Wert an, um die durchschnittliche Lebensdauer einer Festplatte zu beschreiben. 

Die MTBF mit n Festplatten berechnet man in einem RAID-5 \cite{Chen1994} wie folgt:
\begin{equation}
MTBF_{RAID-5}=\frac{MTBF_{Disk}^2}{N*(N-1)*MTTR_{Disk}}
\label{eqn:MTBF-RAID-5}
\end{equation}

\section{Verfügbarkeit}
Ein Service bzw. ein System gilt als verfügbar, wenn es seine dafür bestimmte Tätigkeit vollständig erfüllt. Die Wahrscheinlichkeit, in welcher ein Service in einer definierten Periode verfügbar ist, bezeichnet man als Verfügbarkeit \cite{Held2004}. Im Idealfall darf eine Service nie ausfallen und soll immer verfügbar sein. In der Realität ist eine perfekte Verfügbarkeit kaum gewährleistet. Es wird angestrebt, die erforderliche oder gewünschte Verfügbarkeit möglichst genau auszudrücken, damit mit Kennzahlen und Metriken die Verfügbarkeit definiert und gemessen werden kann. 

Die Verfügbarkeit wird aus dem Verhältnis der verfügbaren Zeit (Updatime) und der nicht verfügbaren Zeit (Downtime) eines Services \cite{Held2004} bemessen:

\begin{equation}
\mbox{Verfügbarkeit} = \frac{\mbox{Uptime}}{ \mbox{Downtime} + \mbox{Updatime} }
\label{eqn:Verfügbarkeit}
\end{equation}

\section{Datenverfügbarkeit}
Die Datenverfügbarkeit beschreibt die vom Hardware Speicherhersteller und dem Serviceanbieter als Speicherdienstleister "'Storage Service Provider (SSP)"'  gewährleistete Verfügbarkeit der Daten und die zugesicherte Antwortzeit beim Datenzugriff während dem Normalbetrieb \cite{TechTarget2001}.

In dieser Arbeit wird der Begriff dazu verwendet, die gewährleistete Verfügbarkeit für den Datenzugriff des Speichersystems zu beschreiben. Die Datenverfügbarkeit aus der Sicht des Endbenutzers der Webapplikation wird mit diesen Begriff nicht definiert. 

\section{Hochverfügbarkeit}
Die Autorin Andrea Held beschreibt in Ihren Buch den Begriff "Hochverfügbar" wie folgt:
\begin{quotation}\em
Ein System gilt als hochverfügbar, wenn eine Anwendung auch im Fehlerfall weiterhin verfügbar ist und ohne unmittelbaren menschlichen Eingriff weiter genutzt werden kann. In der Konsequenz heisst dies, dass der Anwender kein oder nur eine kurze Unterbrechung wahrnimmt.\end{quotation}\cite{Held2004}

Hardvard Research Group hat die Verfügbarkeit in Verfügbarkeitsklassen eingeteilt:

\begin{itemize}
\item Conventional (AEC-0): Funktion kann unterbrochen werden, Datenintegrität ist nicht essentiell
\item Highly Reliable (AEC-1): Funktion kann unterbrochen werden, Datenintegrität muss jedoch gewährleistet sein.
\item High Availability (AEC-2): Funktion darf nur innerhalb festgelegter Zeit beziehungsweise zur Hauptbetriebszeit minimal unterbrochen werden.
\item Fault Resilient (AEC-3): Funktion muss innerhalb festgelegter Zeiten beziehungsweise während der Hauptbetriebszeit ununterbrochen aufrechterhalten werden.
\item Fault Tolerant (AEC-4): Funktion muss ununterbrochen aufrechterhalten werden. Der 24*7 Betrieb (24 Stunden, 7 Tage die Woche) muss gewährleistet sein.
\item Disaster tolerant (AEC-5): Funktion muss unter allen Umständen verfügbar sein.
\end{itemize}
\cite{Held2004}

In der Arbeit werden zur Verfügbarkeitsbestimmung die Verfügbarkeitsklassen von Harvard Research Group verwendet.

\section{Daten Integrität}
Daten können durch Manipulationen oder durch Bit Fehler verfälscht werden. Solche Verfälschungen können zu Informationsverlust in den Daten bis hin zu vollständigen Korruption der Daten führen. 
Bernd Panzer-Steindel hat zusammen mit Tim Bell, Olof Barring und Peter Kelment am CERN die Integrität von Daten untersucht. Ihre Untersuchung hat gezeigt, dass die Fehlerrate auf dem Niveau von $10^{-7}$ liegt, und hat verschiedene Ursprünge. Eine Ursprung stellt die Festplatte selbst dar, für den Untersuch haben Sie auf 3000 Server alle zwei Stunden ein 2 Gigabyte Datei mit einen speziellen Bit Muster geschrieben, neu eingelesen und mit dem Muster verglichen. Nach einer Laufzeit von 5 Wochen hatten es 500 Fehler auf 100 Server gegeben.\cite{Panzer-steindel2007}

\section{Standortübergreifend}
Für die Gefahren von Datenverlust oder den Verlust der Verfügbarkeit bei einen Ausfall eines Standort zu vermeiden, werden die Daten Standort übergreifend verfügbar gemacht. Häufig werden dazu die Daten an zwei Standort gleichzeitig geschrieben und optional an einen dritten weiter entfernten Standort zeitversetzt geschrieben. 
%!TEX root=../documentation-bachlorthesis-speicherarchitektur-lstucker.tex

\cleardoublepage
\chapter{Ist-Analyse}
Ziel der Ist-Analyse ist es, mit der Situationsanalyse den aktuellen Stand des bestehenden Systems bzw. der Speicherinfrastruktur zu beschreiben und zu bewerten.
Die Angaben für die Ist-Analyse wurden in diversen Interviews mit dem Auftraggeber und aus den von ihm gelieferten schriftlichen Dokumenten erhoben und zusammengestellt.

\section{Bestandesaufnahme}
\subsection{Anwendungs-Architektur}
Die genaue Anwendungs-Architektur wird hier nicht beschrieben, da dies die Geschäftsgeheimnisse des Auftraggebers verletzen würde. Es werden deshalb nur die Teile davon beschrieben, welche die Geschäftsgeheimnisse des Auftragsgebers nicht verletzen.

Die Applikation ist eine Web-Applikation, welche auf dem Web-Framework Ruby On Rails aufsetzt.
Wie dem Namen "Ruby On Rails" zu entnehmen ist, basiert das Framework auf der Programmiersprache Ruby. Ruby ist eine dynamische, Objekt-orientierte, interpretierbare Programmiersprache, von Yukihiro Matsumoto 1995 hervorgebracht. Ein Ruby Programm benötigt für deren Ausführung keine Kompilation.

Ruby On Rails auch Rails oder RoR genannt, implementiert eine Model-View-Controller Architektur. Die drei Sub-Frameworks, spielen dabei eine signifikante Rolle in der Separierung des Codes: Active Record, Action View, und Action Controller. Die drei Sub-Frameworks sind auch im refabb{abb:Rails-Architektur} dargestellt (nach \cite{Bachle2007}).

\begin{center}
\includegraphics[width=\linewidth, keepaspectratio = true]{media/rails-architecure.png}
\mycaption{figure}{\label{abb:Rails-architektur}Rails Architektur \textit{RAID-5} \cite{Wikipedia2006}}
\end{center}

Beides, Ruby On Rails (auch nur "Rails" genannt) und Ruby sind Open-Source Programme. Die Quellprogramme sind für jeden Programmierer offen. Ruby wird unter der Lizenz "'Ruby License"' und GPL veröffentlicht, während Rails unter der "MIT License" veröffentlicht ist.

\subsection{System-Architektur}\label{System-Architektur}
Die bestehende System-Architektur besteht aus einem einzelnen Server, welcher von einem bekannten deutschen Webhosting-Provider gemietet wird. Beim gemieteten System handelt es sich um einen x64 Mikroarchitektur.

\subsection{Speicher-Architektur}
Die Speicher-Architektur besteht aus einem internen Block-basierendem RAID-5 Speicher. Die im Server verbauten Festplatten (RAID-5) werden mittels dem Linux Software RAID Kernel\footnote{\url{https://raid.wiki.kernel.org/}} und dem Verwaltungstool mdadm\footnote{\url{http://neil.brown.name/blog/mdadm}} zu einer logischen Einheit zusammengefasst. Für das RAID-5 werden drei  "'SATA 2"' Festplatten verwendet, welche jeweils eine Speicherkapazität von 2 Terabyte bzw. 1.818 Tebibyte aufweisen.

Im RAID Speicher ist ein root Dateisystem für das Betriebsystem, die Webapplikation, Datenbank, Logdateien und Bilddaten installiert.

\subsection{Speicherkapazität}
Die von RAID-5 zur Verfügung gestellte Speicherkapazität beträgt gemäss \refeqlb{eqn:RAID-5-3disk} 3.636 Tebibyte, davon sind nach Installation des Dateisystems $\approx 3.5$ Tebibyte bzw. $\ca. 3.8 $ Terabyte effektiv verfügbar. Davon sind anhand des \textit{df} Befehls,  wie im\reflisting{df} ersichtlich, $\ca. 2.5$ Terabyte für das Betriebsystem, die Webapplikation, Datenbank, Logdateien und Bilddaten reserviert. Als freier Speicherplatz verbleiben $\ca. 1.3$ Terabyte.

\begin{equation}
\mbox{Speicherkapazität RAID-5}= (3 - 1) * 1.818  \, \mathrm{TiB} =  3.636 \, \mathrm{TiB}
\label{eqn:RAID-5-3disk}
\end{equation}

\begin{lstlisting}[label=df, language=Bash, caption=Report Dateisystem Speicherplatz Belegung in Dezimal Prefix ]
root@www1:~# df -h
Filesystem            Size  Used Avail Use% Mounted on
simfs                      3.8T  2.5T  1.3T   67% /
\end{lstlisting}
\footnote{\url{http://www.debianadmin.com/manpages/dfmanpage.htm}}

\subsection{Datenwachstum}
Das \refabb{abb:disk-usage-by-year} zeigt den Speicherzuwachs von Mitte Juni bis Ende November.

\begin{center}
\includegraphics[width=\linewidth, keepaspectratio = true]{media/disk-usage-by-year.png}
\mycaption{figure}{\label{abb:disk-usage-by-year}Verhältnis von Speicheplatzverbrauch zur Speicherkapazität in Prozent in einer Zeitspanne von einen Jahr}
\end{center}

\subsection{Zugriffswachstum}
Für die Ist-Aufnahme konnten keine historischen Messdaten zur Verfügung gestellt werden.

\subsection{Skalierbarkeit Datenvolumen}
Während dem Betrieb ist ein Ausbau der Speicherkapazität (Skalierbarkeit) nicht möglich. Eine Vergrösserung der Speicherkapazität ist nur durch einen Wechsel auf ein anderes Hostingprodukt möglich. Dies erfordert eine Migration auf eine neue Server-Plattform. 

\subsection{Skalierbarkeit Datenzugriffe}
Keine Skalierbarung möglich. 

\subsection{Daten Durchsatz I/O}
Für die Ist-Aufnahme konnten keine Messdaten zur Verfügung gestellt werden.

\subsection{Daten Redundanz}

Die Daten-Redundanz wird durch das RAID-5 gewährleistet.

\subsection{Datenverfügbarkeit}
Stromausfälle und Netzwerkstörungen im Rechenzentrum des Hosting Providers führten zu Störungen und Ausfällen in der Webapplikation und in der Datenauslieferung. Die Störungen wurden jedoch nicht dokumentiert, weshalb keine statistische messbare Aussage über die erreichte Datenverfügbarkeit gemacht werden kann.

\subsection{Daten Sicherheit}
Die Daten sind durch physischen Fremdzugriff mittels Verschlüsselung geschützt. Für die Verschlüsselung wird das Device Mapper Module dm-crypt eingesetzt. dm-crypt verschlüsselt die Daten bereits auf Blockebene und ist somit für das Dateisystem tranRAID-5nt.

\subsection{Daten Integrität}
Zu jedem Bild wird eine Hash Prüfsumme gespeichert, die bei jedem Sicherung geprüft wird. Die Daten-Integrität ist somit sichergestellt.

\subsection{Sicherung}
Es wird täglich mittels dem Tool ccollect\footnote{\url{http://www.nico.schottelius.org/software/ccollect/}} ein R-Sync Sicherung der Daten erstellt. Das Sicherung wird ausser Hause gelagert.

\subsection{Wirtschaftlichkeit}
Die Kosten für die Speicherung der Daten inklusive Webinfrastruktur sind vergleichsweise zu alternativen Lösungen tief.

\subsection{Lokalität}

Die Web-Applikation inklusive dessen \gls{Primären-Daten} werden am selben Standort betrieben. Die Sicherungsdaten werden an einen seperaten Standort gelagert. Der Zugriff auf Sicherungs-Daten ist innerhalb 30 Minuten möglich.

\section{Analyse-Ergebnisse}

\subsection{System-Architektur}

\subsection{Datenwachstum und Speicherkapazität}
Die Messdaten aus dem  \refabb{abb:disk-usage-by-year} zeigen, dass das Datenvolumen seit  Juni 2011 bis Ende November 2011 mit Ausnahme einer unregelmässigen Spitze und einem grösseren Wachstumsschub Ende Oktober kontinuierlich zugenommen hat. Die etwas erhöhte Spitze lässt sich durch eine vorübergehende technische Änderung am System erklären und hat somit für die Auswertung keine Relevanz. Während der erwähnten Zeitspanne hat das Datenvolumen von 1,634 Terabyte auf 2,546 Terabyte zugenommen. Dies entspricht einer Datenzuwachsrate von 0,912 Terabyte bzw. 55,8\% in fünf Monate. Das Durchschnittliche Datenwachstum beträgt 0,182 Terabyte bzw. 11,1\% pro Monat.

Setzt sich der bisherige Trend fort, so ist die verfügbare Speicherkapazität von 3,8 Terabyte in weniger als 7 Monate erschöpft. Berücksichtigt man bei diesem durchschnittlichen Datenwachstum mögliche Wachstumsschübe für Neukunden, welche durch das Einlesen von deren bestehende Bilddatenbanken entstehen könnten, so reduziert sich die Kapazitätsreserve nach unten und würde eine vorzeitige Umstellung auf ein neues System bedingen.

Gemäss des Auftraggebers ist ferner ein neues Feature für die Web-Applikation geplant, welches den Speicherplatzbedarf verdoppeln würde. Die aktuelle Speichersituation lässt momentan den produktiven Einsatz des neuen Feature nicht zu. Das durchschnittliche Datenwachstum würde voraussichtlich auf 0,364 Terabyte pro Monat steigen.

Die jetzige Speicherkapazität erfüllt schon heute die Anforderungen nicht mehr.

\begin{equation}
\mbox{Datenwachstum}_{Monate5} = 2,546  \mathrm{\ TB} - 1,634 \mathrm{\ TB} =  0,912 \mathrm{\ TB}
\label{eqn:Verfügbarkeit_5Monate}
\end{equation}

\begin{equation}
\mbox{Datenwachstum}_{Monate5} = \frac{0,912 \mathrm{\ TB}}{\frac{1,634 \mathrm{\ TB}}{100 \ \%}} =  55,813 \ \%
\label{eqn:Verfügbarkeit_5Monate_in_Prozent}
\end{equation}

\begin{equation}
\mbox{Durchschnittliches Datenwachstum}_{Monate} = \frac{0,912 \mathrm{\ TB}}{5\mathrm{m}} =  0,1824 \mathrm{\ TB}
\label{eqn:Verfügbarkeit_1Monate}
\end{equation}

\begin{equation}
\mbox{Durchschnittliches Datenwachstum}_{Monate} = \frac{55,813\ \%}{5 \mathrm{m}} = 11,162 \ \%
\label{eqn:Verfügbarkeit_1Monate_in_Prozent}
\end{equation}

\subsection{Skalierbarkeit Datenvolumen}\label{AnalyseSkalierbarkeitDatenvolumen}
Die Eingesetzte Speicherarchitektur lässt eine Skalierung des Datenvolumen mittels hinzufügen von weiteren Festplatten oder durch Migration auf grösseren Festplatten zu. Die maximale Anzahl von Festplatten wird dabei durch die Anzahl vorhandenen SATA Anschlüsse im Server bestimmt. Beim eingesetzten Server handelt sich um eine Hosting Produkt welche meist begrenzt ausbaubar ist. Bei Hosting Produkte erfolgt eine Skalierung meist durch eine Wechsel auf eine besser ausgebautes Hosting Produkt. 

\subsection{Skalierbarkeit Datenzugriffe}
Eine mögliche Skalierung bezüglich des Datendurchsatzes könnte durch schnellere Festplatten, durch eine optimierte Verteilung der I/O-Operationen auf verschiedene Festplatten oder durch die Verteilung der Daten auf weitere Systeme erreicht werden. Zu beachten sind auch hier dieselben Prämissen, wonach wie beim Ausbau des Datenvolumens das bestehende System nur sehr begrenzt ausgebaut werden kann bzw. auf ein anderes Hostingprodukt migriert werden müsste.

\textbf{Skalierung durch schnellere Festplatten:}
Moderne Festplatten, mit Ausnahme von den teuren Solid State Disk (SSD), bieten zwar eine grössere Speicherdichte aber aufgrund der erreichten physikalischen Grenzen keine bedeutend effizientere IOPS versprechen. Eine Festplatte mit 7200 RPM erreicht durchschnittlich ein IOPS zwischen 75 und 100 IOPS. Bei SSD Disks wurden schon 1'190'000 IOPS in einem einzelnen PCI Geräte gemessen.\cite{Symantec2011} \cite{Fusionio} 
Ein Wechsel auf die genannten SSD Festplatten, ist aktuell noch mit sehr hohen Kosten pro Gigabyte verbunden. Gartner geht davon aus, dass 2012 die Preise pro Gigabyte bei SSD auf durchschnittliche 1\$ sinken werden. Gegenüber den heutigen Preisen bei konventionellen Festplatten von 30 Cents pro Gigabyte ist dies immerhin 3x teurer. Aus diesem Grund werden SSD Disks bei grossen Datenvolumen meist noch nicht eingesetzt, ausser spezielle Anforderungen an die Performance rechtfertigen die höhere Investition. \cite{AgamShah2011}

\textbf{Skalierung duch Verteilung der I/O Operationen:}
Ein RAID oder Volume Manager bietet die Möglichkeit, die I/O Operationen auf mehrere Festplatten zu verteilen. In einem RAID-5 verkleinert sich wie in Kapitel \refchap{AnalyseSkalierbarkeitDatenvolumen} beschrieben, die MTTF mit jeder weiteren Festplatte. Zudem sind die verfügbaren Anschlüsse in einem Server ebenfall ein zu berücksichtigen limitierender Faktor.

\textbf{Skalierung durch Verteilung der Daten:}
Nicht alle Kategorien von Daten haben die gleichen Anforderungen an den Durchsatz. Datenbanken z.B. benötigen in der Regeln einen höheren IOPS als statische Daten wie z.B. Bild-Daten, welche weniger oft abgefragt werden. Für die Verteilung der Daten ist eine Änderung in der System-Architektur und allenfalls in der Anwendungs-Architektur notwendig. Ein Beispiel für eine solche Anpassung der System-Architektur wäre die Auslagerung der Datenbank auf ein separates System, welches mit schnellen Solid State Disk ausgerüstet ist. 

\subsection{Redundanz}
Eine RAID-5 System bietet wie im \refsec{RAID-5} beschrieben, keine 1:1 Redundanz. Die Daten sind im RAID-5 nicht doppelt gespeichert, sondern werden bei einen Datenverlust mittels XOR Operation aus den Parität Stripes und den vorhandenen Daten Stripe-Einheiten berechnet. 

Bei einen Zugriff auf einen verlorenen Daten Stripe werden die Daten online berechnet. 

Treten bei einer Festplatte Fehler auf, werden die verlorenen Daten bei einem Zugriff aus dem Parität Stripe Einheit.??? Der Zugriff auf die Daten ist durch die Berechnung der Daten aus dem Parität Stripe Einheit Online möglich. Wird die ausgefallen Festplatte durch eine neu intakte Festplatte ersetzt, können die Daten durch einen Rebuild während des Betriebs wieder hergestellt werden. Durch die Berechnung und Schreib/Lese Operationen im RAID während des Wiederherstellungprozesses verschlechtert sich der Datendurchsatz und I/O Rate für den Datenzugriff wesentlich.

Einen Ausfall einer weiteren Festplatten kann das RAID-5 System nicht mehr kompensieren und führt zu einem physischen Datenverlust aller Online-Daten. Die Daten müssen in der Folge mittels des Sicherung eingelesen und wiederhergestellt werden, was jedoch nur im Offline-Betrieb möglich ist. 

Bei Festplatten welchen aus dem gleichen Produktionszyklus stammen, kann die Wahrscheinlichkeit eines weiteren Ausfall höher sein, im vergleich zu Festplatten die aus unterschiedlichen Produktionszyklen stammen.

\subsection{Service- und Daten-Verfügbarkeit}
Die bisherige System-Architektur gemäss dem Ist-Bestand ermöglichen keine AEC-2 Hochverfügbarkeit. Grund dafür ist die Fokussierung auf einen einzelnen Server in der System-Architektur. Der Server stellt eine "'Single Point of Failure"' (SPOF) dar. Tritt beim Server eine Störung auf, wie Sie zum Beispiel Aufgrund eines Gerätedefekts oder Softwarefehler auftreten können, kann der Service nicht von einem redundantem System übernommen oder weitergeführt werden. Eine Störung kann also die Verfügbarkeit während den Betriebszeiten starl gefährden und entspricht nicht den heutigen Anforderungen.

Die Netzwerkverfügbarkeit wird vom Hosting Anbieter gemäss den Allgemeinen Geschäftsbedingungen \cite{Ag2009} mit einer Verfügbarkeit von 99\% im Jahresmittel gewährleistet. Die Verfügbarkeit könnte somit bei einem 24x365 Stunden Betrieb während 87,6 Stunden nicht gewährleistet sein.

Das Speichersystem für sich selbst betrachtet kann die Verfügbarkeit und die Daten-Integrität gewährleisten, sofern keine Störungen durch einen Festplattendefekt eintreten. Das Speichersystem, welches ein interner Block-basierenden RAID-5 Speicher darstellt, ist ein Bestandteil des Serversystems und weist deshalb die selbe Verfügbarkeitsstufe wie das Serversystem auf. Die Service und Datenverfügbarkeit weist aus den genannten Gründen eine Verfügbarkeit Stufe von '"Highly Reliable"' AEC-2 auf.

Die im \refsec{System-Architektur}  beschriebene System-Architektur gewährleistet keine Hochverfügbarkeit der Daten. Einer der Gründe dafür ist, dass die Applikation nur auf einem einzelnen Server betrieben wird, dessen Hardware mit Ausnahme der Festplatten keine Redundanz aufweist und nicht für Hochverfügbar ausgelegt ist. 

Die Applikation wird auf einem einzelnen Server betrieben. Der Server und dessen Hardware mit Ausnahme der Festplatten stellen einen "'Single Point of Failure"' (SPOF) dar. Fällt der Server wegen eines Hardwaredefektes oder einem Softwarefehler aus, sind Daten und Anwendung für den Endbenutzer nicht mehr verfügbar.

Zudem werden die Daten und Applikation nur an einem Standort betrieben. Treten unvorhersehbare Ereignisse auf, wie z.B. ein lokaler Stromausfall, ein Brand oder eine Naturkatastrophe, kann der Betrieb nicht ohne die Wiederherstellung des ganzen Systems inklusive Daten von Sicherung wieder aufgenommen werden. Sofern nicht ein gemieteter Ersatzserver bereitsteht, muss ebenfalls die Beschaffung und Installation eines neuen Systems bei einem anderen Hosting Provider zur Ausfallzeit (Downtime) mit einberechnet werden. Der Service wäre während einer längeren Zeit nicht mehr verfügbar. Ein Kundenverlust und weitere Konsequenzen könnten die Folge sein.

\subsection{Daten Sicherheit}
Durch die Festplattenverschlüsselung sind die Daten bei abgeschaltetem Server vor unberechtigten Zugriffen geschützt. Befindet sich der Server jedoch im laufenden Betrieb, bietet die Verschlüsselung keinen Schutz vor Datendiebstahl. 

In einem Interview mit Ben Schwan im Technology Review hat Edward W. Felden, Professor für Informatik an der Princeton University den Grund dafür erklärt. 

\begin{quotation}\em
... ,die Dechiffrierungsschlüssel bei einer Festplattenverschlüsselung sitzen immer irgendwo im DRAM-Speicher. Um an sie heranzukommen, schaltet der Angreifer zunächst den Strom des Rechners aus und stellt die Energieversorgung dann gleich wieder her. Dann bootet er die Maschine in ein spezielles, böswilliges Betriebssystem hinein. Zu diesem Zeitpunkt enthält der Speicher noch immer die Originalinformationen, die verfügbar waren, als der Rechner noch nicht abgeschaltet wurde – die gewünschten Schlüssel natürlich auch. Das kurze Abschalten des Stroms hat daran rein gar nichts geändert. Der Angreifer kann dann die gewünschten Dechiffrierungsschlüssel aus dem Speicher auslesen und damit die geschützten Informationen auf der Festplatte jederzeit entschlüsseln\end{quotation}\cite{Schwan2008}

Des weiteren bietet eine Festplatten keinen Schutz bei Schwachstellen in der Software oder  der Konfiguration im laufenden Betrieb. Grund dafür ist, dass das Betriebsystem und Anwendungen über eine Verschlüsselungsschicht unverschlüsselt zugreifbar sind.


%!TEX root=../documentation-bachlorthesis-speicherarchitektur-lstucker.tex

\cleardoublepage
\chapter{Szenarien Beschreibung}
Der Auftraggeber hat gewünscht bei der Evaluation der Speicherlösungen mehre Szenerien in der Entwicklung des Speicherbedarfs zu Berücksichtigen. 

Die beschriebenen Szenarien, sind mit den Auftraggeber besprochen worden. Die folgenden Annahmen wurden nicht aufgrund eines vorhandenen Geschäftsplan, und können von Auftraggeber definierte Geschäftsziele abweichen. 


\section{Szenario-1 Schwaches Datenvolumen und Datenzugriff Wachstum}\label{Szenario1}
Bei Szenario-1 wird angenommen, dass sich der Kundenzuwachs, welche die Dienstleistung zur Speicherung und Aufbereitung Ihrer Bilddaten verwenden, marginal steigert. In diesen Fall beträgt das durchschnittliche Datenzuwachstum 0.25 Tebibyte pro Monat und bewegt sich in einem Vergleichbaren Umfang wie im Ist-Zustand.


Datenwachstum in Tebibyte pro Monat: 0.25 Tebibyte

Bilder\footnote{Durchschnittsgrösse von 100 Mebibyte pro Bild} Zuwachs pro Monat: 4'690 Bilder

Speichervolumen in 36 Monaten: 11.5 Tebibyte

Anzahl Bilder\footnotemark[\value{footnote}] in 36 Monaten: 168'821 Bilder

\section{Szenario-2  Starkes Wachstum Daten / starkes Wachstum der Abfragen}
Das Szenario-2 wird davon ausgegangen, das sich die Kundenbasis im Vergleich zum Ist-Zustand stark anwächst. Durch die Starke Zunahme an neuen Kunden 
beträgt das durchschnittliche Datenwachstumg 6 Tebibyte pro Monat.


Datenwachstum in Tebibyte pro Monat: 6 Tebibyte

Bilder\footnotemark[\value{footnote}] Zuwachs pro Monat: 12'583 Bilder

Speichervolumen in 36 Monaten: 218,5 Tebibyte

Anzahl Bilder\footnotemark[\value{footnote}] in 36 Monaten: 458'228 Bilder
 

%!TEX root=../documentation-bachlorthesis-speicherarchitektur-lstucker.tex
\cleardoublepage
\chapter{Soll-Analyse}
Bei der Soll-Analyse sollen die erarbeiteten Szenarien aus dem Szenarien Beschreib berücksichtigt werden. 

\section{Szenario-1}\label{Soll-1}

\subsection{Verfügbarkeit}
Die Verfügbarkeit soll durchgehend von Speichersystem bis zum Web-Service mindestens dem AEC-2 Standard von Harvard Research entsprechen. Dass bedeutet die Verfügbarkeit der Applikation beziehungsweise der Daten darf nur innerhalb festgelegter Zeit beziehungsweise zur Hauptbetriebszeit minimal unterbrochen werden. Die bestehende Infrastruktur besteht jedoch nur aus einen einzigen Web-Server. Der Web-Server ist zugleich mit seinen internen Festplatten, welche mit einen RAID-5 Zusammengefasst sind, das Speichersystem der Infrastruktur. Wie in der Ist-Analyse beschrieben, erfüllt das Speichersystem des Server zwar eine Verfügbarkeit des Harvard Research von AEC-2. Die Restlichen Komponenten bzw. Software des Web-Services, erfüllen diesen Standard jedoch nicht. Die bestehende Speicherarchitekur kann jedoch den Speicher, nicht ohne Anpassung der Architektur, an weitere Web-Server zur Verfügung stellen.

Die Daten müssen mindesten in einfacher Redundanz vorhanden sein. 

Um den AEC-2 Standard zu erfüllen ist es nicht notwendig, dass die \gls{Primären-Daten} Standortübergreifend verfügbar sind.

\subsection{Datenzugriff}
Die bestehende Speicherarchitektur konnte die bisherigen Datenzugriffe mit einem Web-Server bisher zufriedenstellend erfüllen. Es wird davon ausgegangen, dass sich die Anzahl Datenzugriff für die Bildaufbereitung und Speicherung von neuen Bilddaten, nicht weiter steigert. Durch die Anforderungen in der Verfügbarkeit ist es notwendig, dass der Datenzugriff auf das Speichersystem von mindestens einen weiteren Web-Server Zugegriffen werden kann. 

Der \gls{POSIX-IO} Zugriff soll nach Möglichkeit für die einfache Integration in die Web-Applikation unterstützt werden.

Der Simultane Lesezugriff auf die gleiche Objekte muss unterstützt werden.

Der Simultane Schreibzugiff auf ein Objekt muss nicht unterstützt werden.

\subsection{Speicherkapazität}
Für die Erfüllung der Speicheranforderungen des Szenario-1 muss das Speichersystem den Ausbau auf mindestens 16.1 Tebibyte unterstützen. In den 16.1 Tebibyte ist eine Reserve von 40\% für allfälliges zusätzliches Wachstum oder Migration Reserve einkalkuliert exklusive notwendiger Speicherkapazität für die Redundanz.

Das Speichersystem soll 400'000 speicherbare Objekte unterstützen.

Das Speichersystem muss die Speicherung von Objekten von mindestens 2 Gibibyte grösse unterstützen.

\subsection{Datenqualität}

Die Selbstheilung von Objekten muss nicht unterstützen werden und gilt als optional.

Die Sicherung und Wiederherstellung der Daten aus einer Sicherung muss möglich sein. Die Sicherung der Daten muss bei Fehlenden standortübergreifende Verfügbarkeit der Daten an einen anderen Standort möglich sein.

\subsection{Vergleich mit Ist-Zustand}
Vergleicht man die Anforderungen der Soll-Analyse mit dem Ist-Zustand ergeben sich folgende Vorteile:

\begin{itemize}
\item Durchgehende Verfügbarkeit nach AEC-2 Standard
\item Redundante Web-Server
\item Kein Singel Point of Failure
\item Erfüllt Speicherkapazität Anforderungen für die nächsten 36 Monate
\item Unterstützung von simulatne Lesezugriffe auf die gleiche Objekte
\end{itemize}

\section{Szenario-2}

\subsection{Verfügbarkeit}
Die Verfügbarkeit soll dem AEC-4 Standard von Harvard Research entsprechen. Dass bedeutet die Verfügbarkeit der Applikation beziehungsweise der Daten muss ununterbrochen aufrechterhalten werden.  Der 24*7 Betrieb (24 Stunden, 7 Tage die Woche) muss gewährleistet sein. Bei hoher Kundenzahl und Speichervolumen, würde einen Unterbruch bei der Verfügbarkeit des Dienstes zu einen Repräsentation Schaden verursachen und Unsicherheit der Zuverlässigkeit bei den bestehenden Kunden in Bezug Ihrer Daten Verursachen.

Die Daten müssen mindesten in einfacher idealer weise in doppelter Redundanz gespeichert sein. 

Die \gls{Primären-Daten} müssen mindestens an zwei Standorte verfügbar sein.

\subsection{Datenzugriff}
Durch die Zunahme der Datenzugriff, muss es möglich sein die Daten an mehre Web-Server zur Verfügung stellen.

Der \gls{POSIX-IO} Zugriff soll nach Möglichkeit für die einfache Integration in die Web-Applikation unterstützt werden.

Der Simultane Lesezugriff auf die gleiche Objekte muss unterstützt werden.

Der Simultane Schreibzugiff auf ein Objekt muss nicht unterstützt werden.

\subsection{Speicherkapazität}
Für die Erfüllung der Speicheranforderungen des Szenario-2, muss das Speichersystem 306 Tebibyte unterstützen. In den 306 Tebibyte ist eine Reserve von 40\% für allfälliges zusätzliches Wachstum oder Migration Reserve einkalkuliert exklusive notwendiger Speicherkapazität für die Redundanz.

Geht man von einem Datenwachstum von 6 Tebibyte pro Monat aus, wird das Speichervolumen für das Speichern aller Bilddaten nach 36 Monaten 218,5 Tebibyte betragen. Rechnet man mit einer zusätzlichen Reserve von 40\% für Eventuelles zusätzliches Wachstum oder Migration Reserve, muss das Speichersystem mindestens 306 Tebibyte exklusive der Datenredundanz an Speicherkapazität zur Verfügung stellen können.

Das Speichersystem soll 9'500'000 speicherbare Objekte unterstützen.

Das Speichersystem muss die Speicherung von Objekten von mindestens 2 Gibibyte grösse unterstützen

\subsection{Datenqualität}
Wegen der grossen Datenmenge soll die Selbstheilung von Objekten unterstützt werden, diese verringert den Bedarf an manuelle Wiederherstellung.

Die Sicherung und Wiederherstellung der Daten aus einer Sicherung soll möglich sein. Die Sicherung der Daten muss bei Fehlenden standortübergreifende Verfügbarkeit der Daten an einen anderen Standort möglich sein.

\subsection{Vergleich mit Ist-Zustand}
Vergleicht man die Anforderungen der Soll-Analyse mit dem Ist-Zustand ergeben sich folgende Vorteile:

\begin{itemize}
\item Durchgehende Verfügbarkeit nach AEC-7 Standard
\item Redundante Web-Server
\item Kein Singel Point of Failure
\item Erfüllt Speicherkapazität Anforderungen für die nächsten 36 Monate
\item Unterstützung von Selbstheilung von korrupten Objekten
\item Unterstützung von simulatne Lesezugriffe auf die gleiche Objekte
\end{itemize}
%!TEX root=../documentation-bachlorthesis-speicherarchitektur-lstucker.tex

\cleardoublepage
\chapter{Entscheidungsfindung bei Evaluationen}\label{kab:Entscheidungsfindung}
\section{Grundlagen}
Evaluation ist der systematische Prozess der Datensammlung und Analyse, um eine Entscheidung treffen zu können.
\section{AHP}
Für das Evaluations- bzw. Bewertungsverfahren wurde das Analytische Hierachie-Prozess-Verfahren (AHP) angewendet. AHP wurde in den siebziger Jahren von Thomas L. Saaty zur Lösung mehrkriterieller Entscheidungsprobleme entwickelt und basiert unter anderem auf einem mathematischen Modell.
\cite{Reichardt2003}

Der Entscheidungsprozesse ist beim AHP wie der Name ausdrückt eben analytisch und hierarchisch. Die Analyse beruht auf mathematischen und logischen Entschlüssen. Auf das genaue mathematische Verfahren wird in dieser Arbeit nicht eingegangen. Sie kann in diverser Literatur nachgelesen werden. \cite{Reichardt2003}

Im AHP Verfahren werden alle Kriterien derselben Ebene in Paarvergleiche bewertet und anhand der 9-Punkte-Bewertungsskale aus der \reftab{tab:9PBewertungsskala} bzw. der umgekehrten Relation aus der \reftab{tab:UmgekehrteBewertungsskala} gewichtet.

Nach der Berechnung der Kriterien-Prioritätenbestimmung, werden die Alternativen in Paarvergleiche zu jedem Kriterium anhand derselben 9-Punkte-Bewertungsskale verglichen und bewertet. Anschliessend wird wiederum durch Berechnung der Gewinner ermittelt.

\begin{table}[htbp]
\caption{9-Punkte-Bewertungsskala \cite{Reichardt2003}}
\begin{tabular}{|c|L{3.5cm}|L{8.5cm}|}
\hline
\multicolumn{1}{|l|}{} & Definition & Interpretation \\ \hline
 1 &  gleiche Bedeutung &  Beide verglichenen Elemente haben die gleiche ""Bedeutung für das nächsthöhere Element. \\ \hline
3 &  etwas grössere "" ""Bedeutung & 
Erfahrung und Einschätzung sprechen für eine ""
etwas größere Bedeutung eines Elements im 
Vergleich zu einem anderen \\ \hline
5 &  erheblich grössere "" Bedeutung & 
Erfahrung und Einschätzung sprechen für eine "" 
erheblich größere Bedeutung eines Elements im ""
Vergleich zu einem anderen \\ \hline
7 &  sehr viel grössere ""Bedeutung & 
Die sehr viel größere Bedeutung eines Elements 
hat sich in der Vergangenheit klar gezeigt. \\ \hline
9 &  absolut dominierend &  Es handelt sich um den größtmöglichen ""
Bedeutungsunterschied zwischen zwei 
Elementen \\ \hline
\multicolumn{1}{|l|}{2,4,6,8} & Zwischenwerte &  \\ \hline
\end{tabular}
\label{tab:9PBewertungsskala}
\end{table}

\begin{table}[htbp]
\caption{Umgekehrte Relationen der Bewertungsskala \cite{Reichardt2003}}
\begin{tabular}{|c|L{3.5cm}|L{7.3cm}|}
\hline
\multicolumn{1}{|l|}{} & Definition & Interpretation \\ \hline
1 & gleiche Bedeutung & Beide verglichenen Elemente haben die gleiche 
Bedeutung für das nächsthöhere Element. \\ \hline
 1/3 & etwas geringere Bedeutung & Erfahrung und Einschätzung sprechen für eine 
etwas geringere Bedeutung eines Elements im 
Vergleich zu einem anderen.  \\ \hline
 1/5 & erheblich geringere Bedeutung & 
Erfahrung und Einschätzung sprechen für eine 
erheblich geringere Bedeutung eines Elements im 
Vergleich zu einem anderen \\ \hline
 1/7 & sehr viel geringere Bedeutung & 
Die sehr viel geringere Bedeutung eines Elements 
hat sich in der Vergangenheit klar gezeigt \\ \hline
 1/9 & absolut unterlegen & Es handelt sich um den größtmöglichen 
Bedeutungsunterschied zwischen zwei 
Elementen \\ \hline
\multicolumn{1}{|l|}{1/2, 1/4, 1/6, 1/8} & Zwischenwerte &  \\ \hline
\end{tabular}
\label{tab:UmgekehrteBewertungsskala}
\end{table}

\subsection{Software Unterstützung}
Der Berechnungsaufwand nimmt mit zunehmender Anzahl Alternativen und Kriterien zu. Aus diesem Grund ist es empfehlenswert, die Evaluation Software-unterstützt durchzuführen.

Alle Berechnungen in dieser Arbeit wurden mit der Software "'AHP Decision"' von "'True North Software"' durchgeführt.

%!TEX root=../documentation-bachlorthesis-speicherarchitektur-lstucker.tex
\cleardoublepage
\chapter{Speicherarchitekturen}

Sowohl Private Personen als auch Unternehmen, haben individuellen Anforderungen an die Speicherung Ihrer Daten. Während früher der Speicher der Daten in der Regeln aus internen Speicher eines Computersystems gehandelt haben, haben sich heute eine grosse Bandbreite an Speicherlösungen am Markt entwickelt. Ein Grund dafür ist die kontinuierliche steigende Anforderung an die Speicherkapazität aber auch Anforderungen an das Datenverwaltung, Datensicherung und Daten zur Verfügung zu stellen, machte es bis anhin laufend notwendig neue Lösungsansätze zu entwickeln oder bestehende Lösungsansätze zu erweitern. 

Die heutigen am Markt erhältlichen Speicherarchitekturen lassen sich in der Obersten Kategorie in  in Block- (Block-Based), Datei- (File-Based) und Objekt-Basierende adressierende Systeme unterteilt werden. Wobei man die Kategorisierung für die Einteilung nicht ganz exakt verallgemeinern kann, da einige Speicherlösungen Mischungen von mehren Kategorien sein können.

\section{Block-Basierend}
Die Block-Basierende Speicherarchitektur ist wohl die traditionellste und weit verbreitetste Form zum Speichern und Zuzugreifen von Daten. Die meisten Computersysteme, sei es Server, Desktop-PCs, Tablet-PC, Smartphones, Spielkonsole, speichern Ihre Daten in einen Blockbasierenden Speicher ab. Als Speicher werden in diesen Geräte meist magnetischen Festplatten, Solid State Disk oder Flash-Speicher eingesetzt.

Bei Block-Speicher werden Daten in Blöcke gelesen und gespeichert (adressiert), ein Block bildet sich aus einer Sequenz von Bits bzw. Byte. Die Grösse eines Blocks wird als Blocklänge bezeichnet, und ist bei allen Blöcken einer Einheit gleich gross. 

Experten wie Mike Mesnier, Greg Ganger und Erik Riedel, sehen jedoch bei zunehmender Speichergrösse und Komplexität von Systemen fundamentale Limitierungen von Block Schnittstellen.

\begin{quotation}
\em Since the first disk drive in 1956, disks have grown by over six orders of magnitude in density and over four orders in performance, yet the storage interface (i.e., blocks) has remained largely unchanged. Although the stability of the block-based interfaces of SCSI and ATA/IDE has benefited systems, it is now becoming a lim- iting factor for many storage architectures. As storage infrastructures increase in both size and complexity, the functions system designers want to perform are fundamentally limited by the block interface. \end{quotation}\cite{Mesnier2003}

Vergleicht man die erste Festplatte, welche von IBM produziert wurde, mit einer Seagate von 2011, hat sich die Speicherdichte von 2000 bit per Quadratzoll auf 625 Gigabyte und in der Geschwindigkeit von 8 Kilobyte auf 600 Megabyte verbessert\cite{Seagate2011}\cite{Seagate2011a}

Für den Zugriff auf Blockbasierende Speichersysteme werden meist Schnittstellen Protokolle wie Small Computer System Interface (SCSI) oder Advanced Technology Attachment (kurz \gls{ATA}) verwendet. Diese Protokolle wurden jedoch in einer Zeit entwickelt, wo man davon ausging, dass ein Block Speicher jeweils nur von einem Computersystem verwendet wurde und nicht mit mehreren Computersystemen geteilt wird. Dies Annahme stimmt in Consumer Elektronik Bereich meist noch heute, in Bereichen wo jedoch grosse Speicherkapazitäten oder eine grössere Verfügbarkeit gefordert ist, wie Sie im Geschäftsbereich vorkommen, stimmen diese Annahmen nicht mehr.

Blockbasierende Speicher, welche nicht aus Internen Speicher eines Server gebildet werden, unterscheidet man in Direct Attached Storage (kurz DAS) und Storage Area Network (kurz SAN). 

\subsection{Direct Attached Storage}
Bei DAS handelt es sich, wie es aus der englischen Bezeichnung zu entnehmen ist, um Speicher, welche direkt an ein Computersystem angeschlossen wird. Bei DAS Enclosure handelt sich um ein Gehäuse mit mehren verbauten Festplatten, welche üblich über einen Host-Bus-Adapter an ein Computersytem angeschlossen wird. Als Schnittstellen Protokoll werden \gls{ATA}, \gls{STA}, \gls{eSATA}, \gls{SCSI}, \gls{SAS} und Fibre Channel eingesetzt. DAS können mit mehreren Computersystemen geteilt werden, sofern genügen Schnittstellen zur Verfügung stehen.

\subsection{Storage Area Network}
Die Storage Networking Industry Association (kurz \gls{SNIA}) definiert ein Storage Area Network (kurz SAN) als ein Netzwerk, welch primärer Bestimmungszweck ist Daten zwischen Computersysteme und Speicherelemente und unter Storage Elemente zu transferieren. Ein SAN besteht aus einer Kommunikations-Infrastrukture, welches eine physische Verbindung und eine Management-Schicht beinhaltet, welches die Verbindungen, die Speichereinheiten und das Computersystem organisiert, sodass der Datentransfer sicher und robust erfolgen kann. Der Begriff SAN wird normalerweise (aber nicht notwendigerweise) mit dem Block I/O Service in Verbindung gebracht und weniger mit dem Datei-Zugriff-Service.\cite{SNIA2011}

Je nach SAN Implementierung kommen folgende Geräte bzw. Komponenten vor:
\begin{itemize}
\item Server
\item Host Bus Adapter
\item Gigabit Interface Converter
\item SAN-Switch
\item Speichersystem
\item Tape Library
\item Logical Unit
\end{itemize}

\paragraph*{Server} 
Der Server greift über das SAN auf Ressourcen von Speichersystem oder Tape Library. In einzelnen Fällen kann der Server selbst über SAN anderen Server Speicher zur Verfügung stellen.

\paragraph*{Host Bus Adapter}
Host Bus Adapter (kurz HBA) für das SAN sind intelligente Hardwareschnittstellen, welche für die Verbindung von Server in einem SAN verwendet werden. Sofern die Server nicht bereits mit einen Host Bus Adapter ausgerüstet sind, können die meisten durch Host Bus Adapter in form von Steckkarten erweitert werden. Der Host Bus Adapter selber hat pro Port ein Einschub in welche ein Gigabit Interface Converter eingebaut wird. \cite{Christopher2009}

\paragraph*{Gigabit Interface Converter}
Der Gigabit Interface Converter sind modulare Schnittstellen, welche Elektrische Signale in optische Signale umwandeln.\cite{SNIA2011}

\paragraph*{SAN-Switch}
Der SAN Switch ist ein Switch, welche dezidiert für das SAN Umgebung verwendet wird.

\paragraph*{Speichersystem}
Das Speichersystem stellt im SAN den geteilten Speicher zur Verfügung. Gemäss den \gls{IT} Marktforschung und Analyse Unternehmen Gartner gehören \gls{EMC}, \gls{IBM}, \gls{NetApp}, \gls{Dell}, \gls{HP}, \gls{Hitachi} zu den Marktführer\cite{RogerW.CoxPushanRinnenStanleyZaffos2011}.

\paragraph*{Tape Library}
Tape Library sind Bandbibliothek in welche ein oder mehre Bandlaufwerke und mehrere Magnetbänder befinden und der automatische Bandwechsel mittels eines Roboter realisiert wird. Die Tape Library werden für die Sicherung von Daten auf Band eingesetzt.

\paragraph*{Logical Unit}
Ein Logical Unit ist ein Gerät, welche über SCSI Protokoll andressiert mittel Logical Unit Number (kurz LUN) wird, weshalb oft auch wenn technisch nicht korrekt das Gerät als LUN bezeichnet wird. Im Speichersystem werden mehre Festplatten mittels RAID zu einer Einheit zusammengefasst, sofern keine weitere Virtualisierung von dem Speicherhersteller zum Einsatz kommt, wird die zusammengefasste Einheit wiederum in Speichereinheiten aufgeteilt und diese als LUN dem Server zugeteilt.\cite{SNIA2011}

\subsubsection{Fibre Channel}
SCSI ist zwar sehr populär, ist jedoch mit 80 Mbps Geschwindigkeit, maximal 25 Meter Bus länge, und mit maximal 32 Geräte pro Bus, ein limitierender Faktor für viele Anwendungen. Unteranderem wegen erwähnten Limitierungen von SCSI hat, das American national Standards Institute (ANSI) die Fibre Channel Technik entwickelt. Fibre Channel ist ein mehrschichtiges Netzwerk, welche die Charakteristische und Funktionen für die Übertragung von Daten über ein Netzwerk definiert. Der Standard beinhaltet von der physikalischen Schnittstelle, Daten Codierung, Übertragungssteuerung (Link Control), Fluss Kontrolle, bis hinzu den Protokoll Schnittstellen. In Vergleich zu anderen Netzwerken, beinhaltet die Fibre Channel Architektur einen signifikanten Anteil von Hardware Prozesse um eine hohe Performance zu erreichen.\cite{Gupta2002}\cite{Christopher2009}

Beim Design von Fibre Channel hat man darauf geachtet die Besten charakteristischen Eigenschaften von I/O Bus Kommunikation (Channel) zwischen zwei Geräte und der Netzwerk Kommunikation zwischen Mehren Geräte zu kombinieren. Die Channel Kommunikation ist im Vergleich zur Netzwerk-Kommunikation, Hardware-Intensive, schnell und produziert wenig Overhead. Netzwerk Kommunikation ist hingegen, abhängig von der Software Implementierung genannt Protokoll, unterstützt aber die Kommunikation von einer grossen Anzahl Geräten.

Anders wie es der Namen von Fibre Channel vermuten lässt, ist Fibre Channel nicht auf Fiberoptik-Kabel als Kommunikations-Medium beschränkt, sondern lässt sich auch auf Kupferkabel betreiben. Aufgrund von physikalischen Eigenschaften ist hier Fiberoptik-Kabel in Geschwindigkeit kombiniert mit Distanz dem Kupferkabel überlegen. So liegt die maximale Distanz bei Kupferkabel bei 30 Metern bei einer Geschwindigkeit von 1 Gbps, bei höheren Geschwindigkeiten wird die maximale Distanz noch weiter reduziert. Bei Fiberoptic-Kabel wird die maximale Distanz von der Qualität der Installation, des Fiberoptic-Kabel-Typ, der Kern-Durchmesser, der Lichtwellenlänge Rundreise Latenz und eingesetzter Hardware bestimmt. Je weiter das Licht innerhalb des Kabels übertragen werden muss, desto grösser ist der Verlust des Ursprünglichen Signal stärke. Mit spezieller Hardware können auch Distanzen von bis zu 600 Kilometer \footnote{\url{http://www.enterprisestorageforum.com/industrynews/article.php/2171801/Synchronous-SAN-Sets-Fibre-Channel-Distance-Record.htm}} erreicht werden.

Es gibt drei Fibre Channel Topolgien:
\begin{itemize}
\item Point-to-Point
\item Arbitrated-Loop
\item Switched-Fabric
\end{itemize}

\paragraph*{Point-to-Point-Topologie}
Die Point-to-Point-Topologie ist die direkte Verbindung von zwei Fibre Channel Geräte, meistens handelt es sich bei der Verbindung von einem Server und einen Speichersystem, wie Sie im Direct Attached Storage (kurz DAS) Umfeld vorkommt. \cite{Christopher2009}

\paragraph*{Arbitrated-Loop-Topologie}
Bei der Arbitrated-Loop-Topologie können bis zu 126 Knoten (NL\_Ports) an einen geteilten Bus Ring zusammengeschlossen werden. In diesen Ring kann eine Verbindung zwischen zwei Ports aktive sein, alle anderen Ports fungieren währende diese Verbindung aktive ist als Repeater und leiten das Signal weiter. Die Arbitrated-Loop-Topologie ist deshalb von der Architektur ähnlich wie dem Token Ring.\cite{Gupta2002}\cite{Christopher2009}

\paragraph*{Switched-Fabric-Topologie}
Die Klassischen SAN Topologie ist die Switched-Fabric-Topologie. Eine Switched-Fabric-Topologie besteht aus einer oder mehreren Switches die zu einer Fibre Channel Fabric zusammengeschlossen werden. Die einzelnen FC-Geräte, wie Server bzw. Storagesystem, werden über eine oder mehre Ports an eine Switched Fabric angeschlossen. In eine Fabric können bis zu $2^{24}$ Ports angeschlossen werden.\cite{Gupta2002}\cite{Christopher2009}

Mit der Switched-Fabic-Topologie lassen sich verschiedene Fabric Topologien bilden.
Die einfachste Topologie, welche das Design Ziel die Eliminierung von Single "'Point of Failure"' erfüllt, ist die Dual Switch Topologie, wie in der \refabb{abb:DualSwitchTopologie} dargestellt dient jeder Switch als eigenständige Fabric. Die FC-Geräte wie Server und Storagesystem werden jeweils pro Fabric bzw. Switch mit mindestens einen FC-Port angeschlossen. Durch den Einsatz von Path Management Software auf dem Server, kann eine vom Speichersystem zugeteiltes Logical Unit über mehre Path angesprochen werden. Diese Implementierung bietet gleich mehre Vorteile. Wenn ein Path oder eine ganze Fabric ausfällt übernimmt der andere Path automatisch für den ausgefallen Path. Bei Wartungsarbeiten an Komponenten einer Fabric kann der Service ohne Downtime weiter betrieben werden. Moderne Path Management Software und Speichersystem unterstützen zudem die Lastverteilung (Loadbalance) des I/O Last über alle Pathe.\cite{Christopher2009}

\begin{center}
\includegraphics[width=\linewidth, keepaspectratio = true]{media/}
\mycaption{figure}{\label{abb:DualSwitchTopologie}Fibre Channel SAN mit Dual Switch Topologie}
\end{center}

Die Meshed Fabric Topologie erhöht die Ausfallsicherheit zusätzlich innerhalb den einzelne Fabric. Für die Meshed Fabric sind pro Fabric mindestens vier Fibre Channel SAN Switches erforderlich. Jeder Switch wird wie in \refabb{abb} ersichtlich ist mit mindestens einen Path, den sogenannten Inter Switch Link (kurz ISL), zu allen anderen Switches in der Fabric Verbunden. Die Meshed Fabric kann den Ausfall von Mehren Kabel und Switch verkraften ohne das die ganze Fabric ausfällt.\cite{Christopher2009}

\begin{center}
\includegraphics[width=\linewidth, keepaspectratio = true]{media/}
\mycaption{figure}{\label{abb:MashedFabricTopologie}Fibre Channel SAN mit Mashed Fabric Topologie}
\end{center}

\subsubsection{iSCSI}
Das SCSI-Protokoll ist ein populäres Protokoll für die Kommunikation mit I/O Geräten, spezielle für Speicher Geräte. SCSI weist die Client-Server Architektur auf, wobei der Clients bei SCSI Interface als "initiators" bezeichnet wird und die logische Einheit vom Server als "target".

SCSI Protokoll wurde schon über Protokolle transportiert, jedoch waren all die Transport Protokolle limitiert in der Distanz. \gls{IBM} startete 1996 mit der Forschung für die Übertragung von SCSI über das Ethernet, dabei untersuchte \gls{IBM} ob sich der Transport mittels IP oder \gls{TCP/IP} besser eignen würde. Messungen zeigte da zumal, dass in einem lokalen Netzwerk, der Transport mittels IP besser eignet, anstelle von TCP/IP, mit der Extrapolation in die Zukunft, und den Transport über die lokale Netzwerkgrenze hinweg war aber der Weg mittels TCP/IP die bessere Wahl. 1999 hatten sich \gls{IBM} und \gls{Cisco} geeinigt "SCSI over TCP/IP" gemeinsam in eine Internet Engineering Task Force (kurz \gls{IETF}) Standart weiter zu entwickeln. \cite{JohnL.202} Die fertige Spezifikation von SCSI over \gls{TCP/IP} ist im \gls{RFC} 3720 mit dem Namen iSCSI in April 2004 fertiggestellt worden.\cite{J.Satran2004}

% \paragraph*{Kosten}
Für den Betrieb eines Fibre Channel SAN sind spezielle Hardware und Fibre Channel Kenntnis notwenig. Aufgrund, dass bei iSCSI die selbe Technik wie im Computernetzwerk verwendet wird, benötigt es für den Betrieb keine zusätzliche Ausbildung, Netzwerk Infrastruktur und Management Software Lösungen, was die Gesamtbetriebskosten (kurz TCO) senkt.

Grundsätzlich kann jeder Computer, welcher mit einem Netzwerkanschluss ausgerüstet ist und einen iSCSI Software Treiber hat, iSCSI nutzen. Computer welche Genügen Prozessor Leistung haben können die zusätzliche Last für die Verarbeitung von iSCSI mit konventionellen Netzwerkkarten lösen. Bei Computersystem, welche die Verarbeitungsgeschwindigkeit kritisch ist, wie bei Server kann diese zusätzliche Last negativ sein. Vergleichbar wie für Fibre Channel gibt es für iSCSI spezielle Netzwerkkarten bzw. Host Bus Adapter, welche mittels TCP/IP Offload Engine (kurz TOE) und volle iSCSI Offload Engine im eignen Chip die \gls{TCP/IP} bzw. iSCSI Pakete verarbeiten. Solche Netzwerkkarten entlasten durch die Verarbeitung der \gls{TCP/IP} und iSCSI Pakete im eignen Chip die Central Processing Unit (kurz \gls{CPU}) des Servers und weisen bessere Werte in der Latenz auf.

% \paragraph*{Netzwerk}
In einem Ethernet Netzwerk verwaltet sich jeder Switch mehr oder weniger autonome und führt eine eigene Weiterleitungstabelle, mit welcher der Switch entscheidet, über welchen Port eine Ethernet Packe ausgeliefert werden muss. Dazu enthält die Weiterleitungstabelle pro MAC Adresse den dazugehörigen Port. Trifft eine Ethernet Paket mit noch unbekannter MAC Adresse ein, leitet der Switch das Paket über alle Ports weiter. Durch die Rückantwort des Zielsystems lernt der Switch, über welchen Port, das System erreichbar ist. Werden in einen Switch Netzwerk benachbarte Switches untereinander über mehre Pfade Verbunden, kann es bei in einen solchem Szenario eintreffen, dass das Paket wieder am ursprünglichen Switch ankommt, wenn der benachbarte Switch die MAC-Adresse ebenfalls nicht kennt. Es entsteht dadurch eine Verdoppelung, der Ethernet Paket im Netzwerk bzw. es kommt zu einer Schleifenbildung, was wiederum zu Netzwerkstörungen führt. Mittels Spanning Tree Protokoll (STP) sollen solche Schleifen vermieden werden. Das Spannung Tree Protokoll erstellt eine Baum Topologie mit jeweils einer aktiven Path zwischen zwei Switches. Diese Topologie hat mehre Nachteile: 

\begin{itemize}
\item Beim Topologie-Wechsel wird im Netzwerk der Spanning Tree neu ausgehandelt, während dieser Neuaushandlung kommt es zu einem mindestens 15-Sekunden-Unterbruch. In dieser Zeit werden keine Ethernet Paket weiter geleitet. Eine Topologie Wechsel kann, zum Beispiel durch einen Pfad Ausfall zwischen zwei Switches hervorgerufen werden.

\item In einer Baum-Hierarchie, müssen die Pakete innerhalb des Baumes weitergeleitet werden und können nicht über einen theoretisch direkteren Pfad weitergeleitet werden. Befindet sich das Ziel auf der anderen Baum Seite, muss das Paket die ganze Hierarchie hinauf weitergeleitet werden und auf der anderen Seite hinunter bis zum Ziel System. Könnten die Switches über mehre Pfade miteinander Kommunizieren, könnte ein direkter Weg gewählt werden und in der Kommunikation währen weniger Switches belastet.
\end{itemize}

Hersteller wie Brocade haben diese Problematik für den Betrieb von iSCSI SAN erkannt und haben Lösungen entwickelt welche das Prinzip von Fibre Channel Fabrics für Ethernet Netzwerke umsetzen. Bislang ist darauf jedoch noch keinen allgemeinen Standard entstanden, weshalb es bei solchen Lösungen um porträtiere Lösungen handelt.

%\paragraph*{Sicherheit}
Wie beim Fibre Channel SAN sollte im Geschäftsumfeld iSCSI über ein dediziertes Netzwerk laufen. Die Abgrenzung erhöht die Sicherheit, das Storage Netzwerk ist somit klar abgeschottet von restlichen Netzwerken. Fehlerhafte Firewall Regeln im Computer Netzwerk haben keinen direkten Einfluss auf die Sicherheit des Datennetzwerkes. Störungen oder Überlast im Computer Netzwerk haben beeinflussen nicht die iSCSI Verbindungen. Mittels IPsec kann die Sicherheit durch eine sichere Authentifizierung und optionaler Verschlüsselung der Verbindung weiter erhöht werden. 

%\paragraph*{Integität}
iSCSI stellt die Integrität des übermittelten Paketes mit dem CRC-32c Digests sicher. Die Integrität auf anderen Ebenen, wie Speichersystem, Computer Bit Fehler auf Speichersystem-Ebene oder Memory des Computersystems können. 

\paragraph*{Datenverfügbarkeit / Redundanz}


\paragraph*{Skalierbarkeit Datenvolumen}
Einen Server können mehre Logical Unit zugeteilt werden. Durch den Einsatz eines Volume Manager können mehre Logical Unit zu einer grossen Logischen Volume zusammengefasst werden. Sollte die Kapazität einer Speichersystems nicht ausreichen kann ein weiteres Speichersystem ans SAN angeschlossen werden.

\paragraph*{Durchsatz I/O}
Die Firma Netapp zählt zu den Marktführen in Bereich Unternehmens NAS Speicherlösungen. Neben dem "Network File System" (NFS), beherrschen die Speicherlösungen von Netapp auch das zur Verfügungsstellen von Logical Units über iSCSI als auch über Fibre Channel. Saad Jafri und Chris Lemmons von Netapp haben die Bereitstellung von Speicher über die verschiedenen Verfahren, bezüglich Performance für eine VMWare vSphere Umgebung untersucht. Netapp weisst in Ihren Report nicht die konkreten Messzahlen aus, sondern die Werte in Vergleich zu einen 4Gb Fibre Channel.

Wie im \refabb{abb:NetappIOPS} von Netapp zu entnehmen ist sind die I/O pro Sekunden Werte von iSCSI in einen 1Gb Ethernet Netzwerk in Vergleich zu 4Gb Fibre Channel rund 8\%tiefer. Wobei höhere Werte bei I/0 pro Sekunden besser sind. Im 10 Gb Ethernet Netzwerk erreicht iSCSI dieselben Werte wie Fibre Channel in einen 8Gb Netzwerk.\cite{Jafri2011}

\begin{center}
\includegraphics[width=\linewidth, keepaspectratio = true]{media/netapp_iops.png}
\mycaption{figure}{\label{abb:NetappIOPS}Netapp IOPS durchsatz für alle unterstützte Protokolle in Vergleich zu 4Gb FC mit 8K Block grösse\cite{Jafri2011}}
\end{center}

Der Report von Netapp hat ebenfalls die Latenz verglichen. Bei Latenz möchte man möglichst einen tiefen Wert erreichen. Gemäss \refabb{abb:NetappLatenz} ist die Latenz von iSCSI in einen 1 Gb Ethernet Netzwerk rund 9\% höher als bei Fibre Channel in einen 4Gb Netzwerk. Bei iSCSI in 10Gb Ethernet waren die Werte gleich wie Fibre Channel in 4Gb Netzwerk. Fibre Channel in einen 8Gb Netzwerk hatte jedoch rund 1\% tiefere Werte.\cite{Jafri2011}

\begin{center}
\includegraphics[width=\linewidth, keepaspectratio = true]{media/netapp_latence.png}
\mycaption{figure}{\label{abb:NetappLatenz}Netapp Latenz für alle unterstützte Protokolle in Vergleich zu 4Gb FC mit 8K Block grösse \cite{Jafri2011}}
\end{center}
 
\subsection{Logical Volume Manager}
Logical Volume Manager oder Dateisysteme welche die Logik eines Logical Volume Manager Kombinieren, ermöglichen es mehre Block-Geräte bzw. Logical Units zu einer grossen Logischen Volume zusammen zufassen. Diese hat den Vorteil, das die maximale Grösse eines Block-Gerätes nicht der limitierende Faktor des darauf installierten Dateisystems ist. Neben dem erstellen von Logischen Volume, können einige Logical Volume Manager den Last-Zugriff auf die verschiedenen Block-Geräte durch Striping optimaler verteilen und sorgen damit für eine besser Performance beim Datenzugriff. Eine weitere Aufgabe, die Logical Volume Manager übernehmen, ist die redundante Haltung der Daten durch Spiegelung. Mittel serverseitige Spiegeln (Host-Based Mirroring) können die Daten mittels der Spiegelungsfunktion des Logical Volume Manager auf zwei Standorte zur Verfügung gemacht werden. Dazu werden von zwei Speichersystemen, welche an unterschiedlichen Standorten installiert sind, gleich viele und gleich grosse Logical Units über das SAN dem Computersystem zugeteilt werden.

Klassische Server Linux Distributoren, wie Red Hat, Suse und Debian inkl. Ubuntu liefern den Quelloffenen Logical Volume Manager 2 (LVM2) in Ihrer Distribution mit. LVM2 kann theoretisch auf einen 64Bit System eine Logical Volume von 8 Exabyte bilden und adressieren.\cite{Levine2009}

Das ursprünglich von Oracle als ZFS Ersatz entwickelte Btrfs Dateisystem, könnte sich zukünftig als Standard Dateisystem vieler Server Distributionen entwickeln. Das für Solaris entwickelte ZFS als auch Btrfs kombinieren Dateisystem und Logical Volume Manager. Btrfs selbst wurde jedoch noch nicht als stabile Version veröffentlicht und ist deshalb für den produktiven Betrieb noch nicht zu empfehlen.\cite{Redler2011}


\subsection{Datei System}
Betriebssysteme adressieren die Daten auf einem Block-Geräte nicht direkt an, sondern greift über ein Dateisystem zu. Das Dateisystem organisiert wie und wo Dateien auf dem Block-Geräte abgelegt werden und Verwalten den freien Speicher. Einige Datei Systeme regeln auch die Zugriffsberechtigungen auf Dateien. Ein Block-Geräte (Logical Unit) können im SAN oder DAS an Mehren Computersysteme gleichzeitig zugeteilt werden, jedoch ist es die Aufgabe des Dateisystems sicherzustellen, dass mehre Computersysteme gleichzeitig auf das gleiche Dateisystem lesen bzw. schreiben können. Konventionelle Dateisysteme wie Ext3 unter Linux gehen von einer exklusiven Nutzung des Speichers aus, weshalb diese Dateisysteme keine Funktionen implementiert haben, die den gleichzeitigen Zugriff regeln. Problematik beim gleichzeitigen Zugriff ist die Sicherstellung der Konsistenz. Schreiben zum Beispiel zwei Computersysteme gleichzeitig in dieselbe Datei, kann nicht sichergestellt werden, welche Änderung gültig ist, und führt zu Inkonsistenz. Dateisysteme, welche den gleichzeitigen Zugriff von Mehren Computersysteme unterstützen, regeln den Zugriff auf Dateien mit Sperrmechanismen (locking). Schreibt ein Computersystem in eine Datei, wird die Datei vom Dateisystem vor der Änderung anderen Computersystem gesperrt. Cluster Dateisysteme wie Red Hat Global Filesystem (GFS) und Red Hat Global Filesystem 2 (GFS2) unterstützen dieses Sperren. Das Dateisystem Red Hat GFS Version 2 unterstützt bei einem 64-bit-System theoretisch Dateisysteme bis 8 Exabyte, Red Hat gewährleistet jedoch nur einen Support von maximal 100 Terabyte.\cite{Levine2011}

Dateisysteme wie ZFS und Btrfs stellen die Integrität der Daten vor Veränderung, wie Sie zum Beispiel durch ein Bit Fehler auf dem Block Geräte oder in Memory entstehen können, mittels Prüfsumme sicher. Dabei wir zur jeder Datei eine Hash Prüfsumme abgespeichert, wird die Datei gelesen wird die Prüfsumme aus der Datei neu berechnet und mit der abgespeicherten Prüfsumme verglichen, sofern das Dateisystem ebenfalls gespiegelt ist, korrigiert das Dateisystem die fehlerhafte Datei aus der intakten Kopie. Dieses Verfahren wird auch als Selbst heilend (Self-healing) bezeichnet. \cite{Bonwick2005}\cite{Oracle}

Dateisysteme können mittels Sicherungssoftware auf Bandlaufwerke oder andere Speichermedien gesichert werden.


\subsection{Zusammenfassung}

\begin{table}[htbp]
\caption{Umgekehrte Relationen der Bewertungsskala}
\begin{tabular}{|L{3.5cm}|L{10cm}|}
\hline
Redundanz & Logical Volume Manager oder Raid (Hard bzw. Software) \\ \hline
Standortübergreifend & Einsatz von SAN und Logical Volume Manager notwendig \\ \hline
Skalierbarkeit Datenzugriffe &  \\ \hline
Performance &  \\ \hline
POSIX IO & Bei Unix Dateisystem \\ \hline
Simultaner Lesezugriff & Nur mit einen Cluster fähigen Dateisystem möglich \\ \hline
Simultaner Schreibzugriff & Nur mit einen Cluster fähigen Dateisystem möglich \\ \hline
Skalierbarkeit Speicherkapazität & Mit einen Logical Volume Manager, können mehre Blockgeräte zu einen grossen Zusammen gefasst werden \\ \hline
Max Anzahl speicherbare Objekte &  \\ \hline
Max Objekte Speichergrosse &  \\ \hline
Datenintegrität & Dateisysteme wie Btrfs und ZFS speichern für jede Datei einen Hashwert anhand dieser kann die Integrität sichergestellt werden, jedoch ist Btrfs noch nicht Produktive einsetzbar und ZFS läuft unter Linux nur im Userspace (FUSE) und nicht im Kernel. \\ \hline
Selbstheilung von Objekten & Dateisysteme wie Btrfs und ZFS haben diese Funktion integriert, jedoch ist Btrfs noch nicht Produktive einsetzbar und ZFS läuft unter Linux nur im Userspace (FUSE) und nicht im Kernel. \\ \hline
Sicherung & Normales Sicherungsverfahren \\ \hline
Sicherheit & Wird von Dateisystem bestimmt \\ \hline
\end{tabular}
\label{UmgekehrteBewertungsskala}
\end{table}

\begin{table}[htbp]
\caption{Umgekehrte Relationen der Bewertungsskala}
\begin{tabular}{|L{3.5cm}|L{10cm}|}
\hline
Redundanz & Mit Switched-Fabric-Topologie können redundante SAN Netzwerke erstellt werden. Zudem können mehre Speichersysteme im SAN Verfügbar gemacht werden. \\ \hline
Standortübergreifend & Eine Fibre-Channel SAN kann Standortübergreifend über eine ISL Verbindung Verfügbar gemacht werden. Beim Betrieb von zwei Standorte kann pro Standort je eine Speichersystem und eine Cluster Node platziert werden. \\ \hline
Skalierbarkeit Datenzugriffe & Durch hinzufügen von mehren Verbindungen durch gehend von Server, über die Switches bis hin zu den Speichersystem, kann der Datenzugriff skaliert werden. \\ \hline
Performance &  \\ \hline
POSIX IO & wird durch Dateisystem sichergestellt \\ \hline
Simultaner Lesezugriff & hängt von Dateisystem ab \\ \hline
Simultaner Schreibzugriff & hängt von Dateisystem ab \\ \hline
Skalierbarkeit Speicherkapazität & Einen Server kann druch zur Verfügungstellen weiterer Logical Units (Blockgeräte), mehr Speicher zur Verfügung gestellt werden. Die Speicherkapazität im ganzen SAN kann durch hinzufügen von weiteren Speichersystem oder grosseren Speichersystem skaliert werden. \\ \hline
Max Anzahl speicherbare Objekte & hängt von Dateisystem ab \\ \hline
Max Objekte Speichergrosse & hängt von Dateisystem ab \\ \hline
Datenintegrität &  \\ \hline
Selbstheilung von Objekten & hängt von Dateisystem ab \\ \hline
Sicherung & - \\ \hline
Sicherheit &  \\ \hline
\end{tabular}
\label{UmgekehrteBewertungsskala}
\end{table}

\begin{table}[htbp]
\caption{Umgekehrte Relationen der Bewertungsskala}
\begin{tabular}{|L{3.5cm}|L{10cm}|}
\hline
Redundanz & Durch hinzufügen von mehren Verbindungen durch gehend von Server, über die Switches bis hin zu den Speichersystem, und den einsatzt von Link Aggregation. Bei einen Ausfall eines Switch bzw. Topologie wechesel kann es beim einsatz von Spanning-Tree zu einen mindestens 15-Sekundigen Unterbruch führen. Alternative Ethernet-Fabric, ist jedoch nicht Standardisiert sondern Hersteller Spezifisch. \\ \hline
Standortübergreifend & Eine ISCSI-SAN kann Standortübergreifend Verfügbar gemacht werden. Beim Betrieb von zwei Standorte kann pro Standort je eine Speichersystem und eine Cluster Node platziert werden. \\ \hline
Skalierbarkeit Datenzugriffe & . \\ \hline
Performance & Bei 10 Gigabit Vergleichbar mit 8 Gigabite Fibre-Channel \\ \hline
POSIX IO & wird durch Dateisystem sichergestellt \\ \hline
Simultaner Lesezugriff & hängt von Dateisystem ab \\ \hline
Simultaner Schreibzugriff & hängt von Dateisystem ab \\ \hline
Skalierbarkeit Speicherkapazität & Einen Server kann druch zur Verfügungstellen weiterer Logical Units (Blockgeräte), mehr Speicher zur Verfügung gestellt werden. Die Speicherkapazität im ganzen SAN kann durch hinzufügen von weiteren Speichersystem oder grosseren Speichersystem skaliert werden. \\ \hline
Max Anzahl speicherbare Objekte & hängt von Dateisystem ab \\ \hline
Max Objekte Speichergrosse & hängt von Dateisystem ab \\ \hline
Datenintegrität & ISCSI Pakete sind zusätzlich zur 16bit TCP Prüfsumme durch eine CRC-32c Prüfsumme vor Bit-Fehler oder manipulationen gesichert. \\ \hline
Selbstheilung von Objekten & hängt von Dateisystem ab \\ \hline
Sicherung & - \\ \hline
Sicherheit & Kann durch IP-Sec gesichert werden \\ \hline
\end{tabular}
\label{UmgekehrteBewertungsskala}
\end{table}


\section{Datei-Basierend}
Bei Datei basierte Speicherarchitekturen werden Daten nicht wie bei Block basierten Speicherarchitekturen über Blocke adressiert, sondern über Dateien.

Mit dem Aufkommen von Desktop-Computer, wurden die Daten nicht mehr Zentral auf einen Mainframe bearbeitet, sondern vermehrt verteilt auf den einzelnen Desktop-Computer. Ohne Vernetzung der Computer mussten die Daten mittels portablen Speichermedien ausgetauscht werden. Diese mag noch in kleinen Umgebungen praktikabel gewesen, sobald jedoch die Anzahl Teilnehmer steigt, wird es schwierig, den Überblick über seine Daten zu haben. Die reine Vernetzung der Computer und den direkten Austausch zwischen den einzelnen Desktop-Computer über eine Netzwerk bietet hier ebenfalls keine Lösung. Die bessere Lösung dieses Problems ist eine Zentraler Speicher in welcher die Dokumente (Dateien) gespeichert sind und mittels Netzwerk Protokoll den Desktop-Computer zur Verfügung gestellt werden. Somit ist es für den Anwender klar welche Dateien existieren und hat zugriff auf auf die aktuelle Version. Unternehmen wie Sun Microsystem, IBM, Microsoft und Apple erkannten ebenfalls dieses Bedarf und entwickelten für Ihre Betriebsysteme Software und dazugehörige Protokolle für den geteilten Datenzugriff.

Zu den bekanntesten und weitverbreitesten Lösungen zählen NFS und \gls{CIFS} (SMB).


\subsection{Network File System}
Das ursprünglich rein von der Firma SUN Microsystems (heute Oracle) 1984 entwickeltes Network-File-System, ermöglicht den gemeinsamen Zugriff von mehreren Computersystemen auf das Dateisystem eines anderen Host (Server), als ob er zugriff auf ein lokales Dateisystem stattfindet. Die zweite Version von NFS erschien 1989 und war die erste Version welche von Internet Standard Request for Comments (kurz \gls{RFC}) Standardisiert wurde und unter der \gls{RFC} Nummer 1094 \footnote{\url{http://tools.ietf.org/html/rfc1094}} veröffentlicht wurde. Für den Transport des NFS Protokoll wurde bis Version zwei ausschliesslich das \gls{UDP} Transportprotokoll unterstützt.
Die Version 3 von NFS (\gls{RFC} 1813)\footnote{\url{http://tools.ietf.org/html/rfc1813}}, die 1995 veröffentlicht wurde, war die erste Version, welche Maschinen, Betriebsystem und Netzwerk Architektur, und Transport-Protokoll unabhängig ist. Die Unabhängigkeit wird mit der Verwendung von Remote Procedure Call (\gls{RPC}), welches wiederum ein eXternal Data Representation (kurz \gls{XDR}) verwendet, erreicht. \cite{Stern2001}

Wie im \refabb{} zu entnehmen ist, ist NFS eine weitere Schicht, welche auf dem Dateisystem und dessen Block-Geräte des Computersystems bzw. Speichersystem aufbaut. So ist es zum Beispiel die Aufgabe des Dateisystems bzw. des Block-Gerätes sich um die Redundanz und Integrität der gespeicherten Daten zu kümmern. 

Die Konsistenz der Daten, bei gleichzeitigem Zugriff von Mehren Computersysteme, stellt NFS mit einem separaten Protokoll, genannt Network Lock Manager (kurz NLM) sicher. Der Network-Lock-Manager sorgt dafür, dass eine Datei, welche von einem Computersystem geändert wird, vor der Änderung anderen Computer Systeme gesichert ist. Wenn ein Client eine Sperrung angefordert hat, muss der Client, nach nicht mehr gebrauch der Sperrung, dem Server die Entsperrung mitteilen. Dieser Implementierung der Sperrung führt jedoch zu Problemen, wenn der Client ein System Absturz erleidet, in diesen Fall kann der Client die Entsperrung dem Server nicht mitteilen und die Datei bleibt für gesperrt.
NFS setzt mit NLM das Advisory Locking Sperr-Verfahren ein, dass bedeutet, dass ein weiter Client beim Zugriff auf eine gesperrte Datei nur hingewiesen wird, dass die Datei gesperrt ist, aber nicht den Client zwingt, keine Änderung vorzunehmen.\cite{Stern2001}


Seit Version 4 von NFS Protokoll (\gls{RFC} 3530) ist das Sperre-Verfahren (Locking) im Protokoll selber implementiert, dadurch entfällt der zusätzliche Einsatz von Network Lock Manager. NFS Version 4 beherrscht das Sperren von Bytes-Bereich in einer Datei. Zudem erhält der Client von Server nur einen Leasing-Zeitraum für eine Sperrung (Lock), welcher er vor Ablauf wieder erneuern muss, um die Sperrung aufrecht zu halten. NFS in der Version 4 unterstützt das Sperr-Verfahren Mandatory Locking, dass bedeutet ein weiter Client kann, sich über die Sperrung nicht hinwegsetzen.\cite{Callaghan2003}

Dadurch, dass NFS auf TCP/IP als Kommunikation Protokoll aufbaut, kann eine NFS Freigabe, Standort übergreifend verfügbar gemacht werden, jedoch gilt auch hier das die Latenz und die Bandbreite der Verbindung zwischen den Standorten der limitierende Faktor ist.

NFS selbst hat keine eigene Implementierung für die Sicherstellung der Integrität der übermittelten Daten, stattdessen verlässt sich NFS seit Version 2 auf TCP und Ethernet Fehler Erkennung. TCP prüft im Standard verfahren die Integrität mit einer 16-Bit-Integer Prüfsumme. Die 16-Bit-Integer Prüfsumme erkennt Fehler im pseudo IP-Header, TCP-Header und Daten. Das Verfahren hat jedoch schwächen bei Einzel Bit-Fehlern Erkennung. Ethernet verwendet für die Fehlererkennung einen CRC32 Prüfsumme, diese gilt als effektive in der Erkennung Behebung von Bit Fehler, bietet aber keine durchgehend (End-to-End) Schutz. Grund dafür ist, dass beim Wechseln des Pakets der Kollisionsdomäne, wie es bei einen Swtich oder Router der Fall ist, jedes Mal eine neu CRC32 Prüfsumme erstellt wird.\cite{JohnL.202}. Bei NFS ab Version 4 kann der Datentransfer zusätzlich mit Kerberos abgesichert werden. Kerberos hat einen strengen Schutz gegen Manipulationen am Datenpaket und stellt somit die Integrität der Daten sicher. Nachteil ist aber das Kerberos zusätzlich eingerichtet werden muss. 

Bis und mit Version 4, wahren die Verarbeitung der Metadaten und die Verarbeitung der Daten in einem Protokoll und Server implementiert. NFS skalierte deshalb bis anhin bei der Verarbeitung von Dateien mit grosser Speicherplatz bedarf nicht ausreichend. Fragt einen NFS Client einen NFS Server für eine Datei an prüft der Server die Metadaten, die Metadaten enthalten den Speicherort, die Grösse, das Erstellungsdatum und das Änderungsdatum einer Datei, und wandelt die Anfrage in einen Disk I/O um, die Daten der Datei werden gesammelt und über das Netzwerk übertragen. Bei kleinen Dateien verwendet der Server die meiste Zeit für das sammeln der Daten, bei Grössen Dateien ist der limitierende Faktor der Transfer der Daten über das Netzwerk selbst.
Mit der Entwicklung von pNFS wurde der Transfer der Daten parallelisiert. Die Architektur von NFS wurde dazu in mehre Komponenten aufgeteilt. Der NFS Server besteht neu aus einen getrennten Metadaten Server und eine oder mehre Daten-Servern. Die Aufgabe des Metaservers ist die Verwaltung der Daten wo und wie die Daten gespeichert sind. Die Daten Servern, wo die Dateien gespeichert sind, kümmern sich um lese und schreib Anfragen von den Clients.
Bei einer anfrage an einer grosse Datei können Mehrere Daten-Server parallel Teile von der Datei dem Client ausliefern, der Client kann dann die Verschiedenen teile der Datei wieder zusammen setzen zu einer ganzen Datei. \cite{Shepler2010}\cite{Group2010}

Mit der NFS Version 4.1 wurde pNFS Bestandteil von NFS und ist seit 2010 im \gls{RFC} 5661 standardisiert. Server Linux Distributoren, wie Red Hat haben allerdings NFS 4.1, erst als Vorschau in Ihrer Distribution integriert.\cite{EastJacquelynnMichaelHidep-Smith2011}




\subsection{NAS Appliance}

Network Attached Storage (kurz NAS) sind Speichersystem mit angepassten Datei System für den gemeinsamer Dateizugriff in einen Hetrogenen Computer Netzwerk welche über ein LAN angeschlossen sind. Als Speicher verwenden NAS je nach Typ interne Festplatten, Direct Attached Storage oder über eine SAN angefügten Speicher.
An Clients stellen NAS Ihren Speicher über NFS, CIFS, ISCSI zur Verfügung. High-End NAS können Ihren Speicher wiederum über Fibre-Channel zur Verfügung stellen.

Gemäss Gartner gehören die Anbieter \gls{IBM,} \gls{EMC} und \gls{NetAPP} zu den führenden NAS Anbieter in Midrange und High-End bereich. Wobei gemäss Garnter Magic Quadrant Netapp zusammen mit EMC zu den innovativsten Anbieter.

\begin{quotation}
\em 
\textbf{Strengths}
\begin{itemize}
\item NetApp remains one of the few truly unified storage providers among all top-tier vendors, with its software features continuing to be industry benchmarks. The company was able to regain some of the NAS revenue market share that it had lost in 2009. Its fast revenue growth in 2010 was driven by its successful campaign targeted at midsize enterprises with the value propositions of NFS supporting VMware and unified storage in consolidating Windows application storage.

\item In 2010, NetApp increased its aggregate up to 100TB with Data ONTAP 8.0.1 and introduced compression to complement its popular deduplication capability. It added a RESTful object storage interface (based on its acquisition of Bycast) to its unified storage, targeting global content repositories. On the hardware side, it launched new systems with better performance and denser disk shelves.

\item NetApp's new software bundles have simplified the procurement process and made software pricing more affordable. For customers seeking converged infrastructure, NetApp launched FlexPod for VMware with its partners Cisco and VMware, offering packages including servers, storage and switches.
\end{itemize}
\textbf{Cautions} 
\begin{itemize}
\item The vast majority of the Data ONTAP 8.0 adoption was on the 7 mode (instead of the cluster mode) for larger aggregates, while the early adoption of the cluster mode focuses on high- performance NFS file services. The cluster mode is not ready for mainstream enterprise customers who require those 7-mode features that are still missing in the cluster mode. The ONTAP 8.1 scheduled for release later this year will likely continue to support the two modes: clustered and nonclustered modes of operation.
\item While NetApp continues to enjoy its leading edge in unified storage, it's facing fiercer competition in the high-end NAS market, where file systems larger than 100TB are required and where high performance without the expensive Flash Cache is desired.
NetApp is also challenged in the low-end NAS and unified storage market with new products from both major and emerging competitors.
\end{itemize}
\end{quotation}\cite{IEEE2003}



\subsection{Zusammenfassung}


\section{Objekt-Basierend}

\subsection{Verteilte Dateisysteme}
Unternehmen wie Google, welche Web-Applikationen mit Millionen von Anwendern Betreiben und der Speicherbedarf von hunderten Terrabyte bis Petabyte an Daten,  haben hohe Anforderungen an Ihr Speichersystem. Google hat für Ihren Bedarf eine eigenes Verteiltes Dateisystem genannt Google Filesystem entwickelt. Google hat beim Designe und des Dateisystem angenommen, das es auf gewöhnlichen und günsige Hardware laufen kann, welche aber öfter Komponenten Fehler habe. Daher ist der Ausfall von Komponenten nicht eine Sonderfalls sondern gehört zur Normalität. Zudem handelt sich bei den gespeicherten Dateien eher um grössere Dateien mit 100 Megabyte bis Multi Gigabyte an grösse. Die Auslastung ist primär durch zwei Arten von lese Vorgänge verursacht. Das lesen eines ganzen Datenstroms und das regellos Lesen. Die Schreibbelastung wird durch grosse Sequentiele schreib Vorgänge verursacht und Dateien werden erweitert als modifiziert. Als Architektur hat Google einen Cluster gewählt bestehend aus einen einzigen Metadaten Server und mehren Chunksserver. Die Daten werden bei Google in Einheiten genannt Chunks unterteilt und jeder Chunk bekommt eine eindeutige Identifizierung. Diese Chunks werden über mehre Chunkserver repliziert um die Ausfallsicherheit zu gewährleisten. Der Metadaten Server speichert in seinen Arbeitsspeicher die ganze metadaten bestehend aus Namensraum, Berechtigung Informationen, die Zuordnung der Datei zu den Chunks, und die Speicherort der Chunks. Google hat Ihre Dateisystem bis anhin nicht veröffentlich, hat jedoch eine  Studie über die Desinge und Architektur Prinzipen veröffentlicht. Einige erhältliche Verteileten Dateisysteme, wie Hadoop Distributed Filesystem (kurz HDFS), CloudStore und GLORY-FS beruhen auf den selben Architektur Prinzipen aus der Studie.


 Diese Studie galt für eine Verteilte Dateisysteme, wie Hadoop Distributed Filesystem (kurz HDFS), CloudStore, GLORY-FS,
 Das Google Filesystem 

welche die


 Google stellte an sein Dateisystem folgende Anforderungen: 
\begin{itemize}
\item Komponenten Fehler sind die Norm
\item Multi-Gigabyte Dateien sind häufig
\item Dateien werden werden hauptsächlich erweitert als überschrieben
\end{itemize}

Für diese Hauptanforderungen hat Google eine eigenes Dateisystem Entworfen und Entwickelt.




%!TEX root=../documentation-bachlorthesis-speicherarchitektur-lstucker.tex
\cleardoublepage
\chapter{Marktübersicht}

Dieser Abschnitt behandelt den Speichermarkt für primäre Speicher. Mit der Marktübersicht wird einen Überblick verschaffen, welche Speicherlösung Kategorien, Marktsegmente existieren und Hersteller sich im Markt behauptet haben. Zudem wird einen kurzen überblick gegeben, welche Trends im Speichermarkt aktuell sind.

\section{Speicherlösungen}
Die erhältlichen Speicherlösungen lassen sich in die folgenden Kategorien aufteilen: 
\begin{itemize}
\item Konsumer Speicher
\item NAS Speicher
\item Modulare Disk Array Speicher
\item Verteilte Dateisystem Cluster Speicher 
\item Online Speicher auch Cloud Storage genannt.
\end{itemize}

\paragraph*{Konsumer Speicher}
Unter Konsumer Speicher, werden die Speicher für Konsumer Elektronik, wie Notebook, PC, Smartphones, Mulitmedia Center, Audio Player, Camcorder usw. verstanden. Der Speicher basiert vorwiegend auf Blockgeräten wie Festplatten, Flash Disks, Solid State Disk etc.

\paragraph*{NAS Speicher}
NAS Produkte sind gemäss Gartner Speichersysteme, welche mit optimierten Dateisystemen gemeinsamen Dateizugriff für die im LAN angeschlossenen heterogenen Computer Systeme ermöglichen. Die NAS Produkte können ihren Speicher von internen Disks oder Direct-Attached Storage sowie von SAN Array Speicher zur Verfügung stellen. Die NAS Produkte verwenden für den gemeinsamen Dateizugriff Industrie Standardprotokolle, wie zB. Network File System (NFS) in Unix Umgebungen, oder Common Internet File System (CIFS) für Windows-Umgebungen. 
Viele NAS Produkte unterstützen heute natives ISCSI und in einigen Fällen Fibre Channel, um den Speicher auch über Logical Units zur Verfügung zu stellen. Die NAS Produkte werden mit einem für ihre Aufgaben optimierten Betriebsystem betrieben. \cite{RogerW.CoxPushanRinnenStanleyZaffos2011}

\paragraph*{Modulare Disk Array Speicher}
Modulare Disk Array Speicher sind Speichersysteme mit doppelten Controllern oder Node Cluster Architektur, welche den Speicher über Block Zugriffsprotokolle wie Fibre Channel oder ISCSI zur Verfügung stellen. Sie werden mit vom Hersteller vorkonfigurierten Festplatten ausgeliefert. Die Festplatten werden mit eigener Konfiguration oder mittels Konfiguration des Herstellers für die gewünschte Redundanz im Speichersystem in RAID-Einheiten zusammengefasst. Die Modularen Disk Arrays werden vorwiegend im SAN, manchmal auch im DAS Bereich eingesetzt.

\paragraph*{Verteilte Dateisystem Cluster Speicher}
Verteilte Dateisystemspeicher sind Speicher Cluster, welche den Speicher verteilt über mehrere handelsübliche Computerhardware zu einem grossen Speicher zusammenfassen und diese über eine API Anwendung zur Verfügung stellen. Die gespeicherten Daten werden meist in mehrfacher Redundanz über mehre Cluster Nodes im Speicher Cluster verteilt. Neben wenigen spezialisierten Anbieter werden die meisten verteilten Dateisystem Cluster-Speicher als individuelle Lösungen auf eigener Computer-Hardware betrieben.

\paragraph*{Online Speicher (Cloud Storage)}
Online Speicher, oder Cloud Storage genannt, wird von Gartner als Speichersystem, welches seine verfügbare Kapazität über eine Wide-Area-Network inklusive dem Internet als Dienstleistung zur Verfügung stellt. Als Dienst ist die Speicherkapazität nach oben und nach unten skalierbar und wird nach dem jeweiligen Bedarf des Benutzers verrechnet. Dies ist mit der Versorgung von Strom durch einen Elektrizitätsversorger vergleichbar. \cite{AdamW.Couture2010}

\section{Marktsegment}
Der Speichermarkt kann in die Marktsegmente Heimanwender, Kommerz und Grossdatenanbieter unterteilt werden.

\paragraph*{Heimanwender/ Homeoffice} 
Der Heimanwender hat im Vergleich zu den anderen Marktsegmenten einen geringen Speicherbedarf. Seine Speicherlösungen beschränken sich in der Regel auf den internen Speicher seines Computersystems und seiner Elektronikgeräten. Für Heimanwender, welche einen etwas grösseren Speicherbedarf benötigen, (zum Beispiel Multimedia Inhalte oder Home Office) hat sich ein Markt für einfache NAS Systeme etabliert, welche in der Regel Speicherplatzgrössen bis zu 9 Tebibyte erlauben.


\paragraph*{Kommerz}
Zum Marktsegment 'Kommerz' gehören KMUs und Grossunternehmen in der Sparte Handel, Industrie und Dienstleitung. Diese Anbieter haben einen mittleren bis hohen Bedarf an Speicherkapazität, im Tebibyte Bereich. 

Unternehmen, welche tiefe Anforderungen an die eigene IT-Infrastruktur haben, verwenden ihre Speicherlösungen primär für den gemeinsamen Datenzugriff. 

%Dazu kommen oft NAS Speicherlösungen oder im Server integrierte Speicher zum Einsatz.

Unternehmen mit hohen Anforderungen an die eigene IT-Infrastruktur (zB. Finanzdienstleister), verwenden ihre Speicherlösung für den gemeinsamen Datenzugriff auf Datenbanken und unstrukturierte Daten als hochverfügbare Systemumgebung.

%Die hochverfügbaren Systeme für den Betrieb von grossen relationalen Datenbanken, verwenden Modulare Disk Array Speicher, welche mittels Storage Area Network (kurz SAN) ihren Computersystemen redundant gesicherten Datenspeicher zur Verfügung stellt. Für den gemeinsamen Datenzugriff setzen diese Unternehmen ebenfalls auf NAS Speicherlösungen.

\paragraph*{Grosse Datenanbieter}
Zu den grossen Datenanbieter zählen Webdienstleister wie Google, Facebook, Yahoo, Amazone und Unternehmen aus der Multimedia Industrie wie Pixar Studios, RedBull, aber auch Forschungseinrichtungen wie Cern oder Bibliotheken wie die Amerikanische Library of Congress.
Diese verwalten in ihren Speichern grosse Datenvolumen im Bereich von mehreren hundert Tebibyte bis Exbibyte. 

Für die Speicherung solcher Datenmengen stützen sie sich auf verteilte Dateisystem Cluster Speicher oder hochleistungs NAS Speicherlösungen. 

\section{Gross Datenanbieter}
Gemäss den Anforderungen des Auftraggebers und den beshriebenen Szenarien zählt das zu evaluierende Speichersystem zum Marktsegment der Grossen Datenanbieter. Diese Arbeit und Marktanalyse beschränkt sich deshalb auf dieses Marktsegment.

Demzufolge beschäftigen wir uns mit Speicherlösungen in der Kategorie NAS Speicher, Modulare Disk Array Speicher, Verteilte Dateisystem Cluster Speicher und Online Speicher (Cloud Storage).

\section{Welche Speicherlösungen haben sich im heutigen Markt etabliert, }\label{MarktEtabliert}
%welche Systeme habe sich in jüngster Vergangenheit durchgesetzt und welcher Trend ist zu erwarten?

\paragraph*{NAS Speicherlösungen}
Gartner hat die Anbieter von NAS im mittleren und oberen Marktsegment auf Ihre Marktchancen hin untersucht. Gartner hat diese in Marktführer, Herausforderer, Visionäre und Nichen-Anbieter gegliedert. Es wurden nur Anbieter berücksichtigt, die NAS Lösungen ab einem Preis von 25'000\$ anbieten, die mindestens eines der Protokolle NFS oder CIFS unterstützen und ein Betriebseinkommen von mindestens 5 Millionen Doller ausweisen. 

Als Marktführer wurden Anbieter, welche einen bedeutenden Marktanteil haben, ausreichend Marketing- und Verkaufs-Kapazitäten haben und technologisch führend und innovativ sind.

Als Herausforderer gelten Anbieter mit einem starken Produkt, welche einen namhaften Marktanteil besitzen und über Ressourcen verfügen, diesen ausbauen zu können. Die Anbieter sind jedoch zu wenig Visionär, um sich als Marktführer zu qualifizieren.

Als Visionäre gelten Anbieter, die ein einzigartiges, innovatives Produkte anbieten, welches operationale oder finanziell wichtige End-Benuzter Probleme anspricht, jedoch noch nicht bewiesen haben, einen substantiellen Marktanteil gewinnen zu können.

Als Nischen-Anbieter gelten Hersteller, welche clevere Produkte vermarkten, die auf kundenspezifische Bedürfnisse oder Marktsegmente ausgerichtet sind.

Wie aus der \refabb{abb:MagicQuaderNAS} zu entnehmen ist, stuft Gartner die NetApp dicht gefolgt von EMC als Marktleader ein. Als Herausforderer gilt Oracle und als Visionär gilt BlueArc.

Gartner sieht die Stärken von NetApp darin, dass dieser als einer der wenigen wirklichen Storage Anbieter alle Marktstufen abdeckt und mit Software-Features weiterhin die Messlatte der Industrie vorgibt. NetApp hat seinen Unternehmensgewinn in 2010 stark gesteigert. Zu deren Schwächen gehören, dass in der neuen Version ihrer Betriebsystem-Software noch nicht alle Funktionen von der alten Version integriert werden konnten, welche aber weitgehend zu den allgemein geforderten Features zählen.

Zu den Stärken von EMC zählt Gartner die Übernahme von Isilon's, einem visionären Anbieter, welcher starke Wachstumschancen im traditionellen Datencenter und im Cloud Service Markt hat. Gleichzeitig schwächt die Übernahme von Isilon den Hersteller, weil sich die Produkte nun den Funktionen und Einsatzgebieten überlappen und sich die Kunden fragen, wie der Fahrplan der künftigen Entwicklung aussieht, bevor der Kunde neue Investitionen tätigt.

Zu den Stärken von BlueArc zählt Gartner, dass BlueArc mit Titan und Mercury ein NAS System anbietet, welches hoch-performant und Modular ausbaubar ist. Zu den wichtigsten benötigten Verbesserungen gehören die ultraschnelle objektbasierte Replikation im Katastrophenfall. Als weitere Schwachstelle sieht Gartner zudem den kleinen Marktanteil.

\begin{center}
\includegraphics[, keepaspectratio = true]{media/magicquader_nas.png}
\mycaption{figure}{\label{abb:MagicQuaderNAS} Gartner Magic Quader März 2011}
\end{center}

\paragraph*{Modulare Disk Array Speicher}
Gartner hat die Anbieter von Modularen Disk Array im mittleren und oberen Bereich auf deren Marktchancen hin untersucht und hat diese in Marktführer, Herausforderer, Visionäre und Nichen-Anbieter unterteilt. Es wurden nur Anbieter berücksichtig die Modularen Disk Array Lösungen ab einem Preis von 25'000\$ anbieten und in den Märkten Nord Amerika, EMEA oder Japan und Asien Pazifik vertreten sind.

Als Marktführer wurden Anbieter, welche einen bedeutenden Marktanteil haben, ausreichend Marketing- und Verkaufs-Kapazitäten haben und technologisch führend und innovativ sind.

Als Herausforderer gelten Anbieter mit einem starken Produkt, welche einen namhaften Marktanteil besitzen und über Ressourcen verfügen, diesen ausbauen zu können. Die Anbieter sind jedoch zu wenig Visionär, um sich als Marktführer zu qualifizieren.

Als Visionäre gelten Anbieter, die ein einzigartiges, innovatives Produkte anbieten, welches operationale oder finanziell wichtige End-Benuzter Probleme anspricht, jedoch noch nicht bewiesen haben, einen substantiellen Marktanteil gewinnen zu können.

Als Nischen-Anbieter gelten Hersteller, welche clevere Produkte vermarkten, die auf kundenspezifische Bedürfnisse oder Marktsegmente ausgerichtet sind.

Wie aus der \refabb{abb:MagicQuaderModularDiskarrays} zu entnehmen ist, stuft Gartner EMC, NetApp, HP, Dell und Hitachi Data System zu den Marktführern ein. Oracle und Fujitsu sieht Gartner als Herausforderer und XIO wird als visionär bezeichnet.
 
\begin{center}
\includegraphics[, keepaspectratio = true]{media/magicquader_modulardiskarrays.png}
\mycaption{figure}{\label{abb:MagicQuaderModularDiskarrays} Gartner Magic Quader Modular Disk ArrayMärz 2011}
\end{center}


\paragraph*{Distributed Filesystem Cluster Speicherlösungen}
Zu den bekannten Vertreter der Distributed Filesystems gehören Hadoop HDFS, Gluster, Lustre. Alle drei haben gemeinsam, dass es sich bei den Lösungen um Open-Source Software handelt.

Hadoop basiert auf dem Design-Konzept von Google Filesystem und Google Mapreduce. The Guardian hat Apache Hadoop 2011 als Erfinder des Jahres ausgezeichnet. InfoWorld hat Hadoop für den InfoWorld 2012 Technology Award gewählt und für Gartner zählt Hadoop zu den top 10 Technologie-Trends, welche Einfluss auf die Informatik Infrastruktur nehmen. 
Zu dem prominentesten Unternehmen die Hadoop einsetzen und mitentwickeln zählen Yahoo und Facebook. Neben den beiden genannten gibt es mtllerweile viele weitere namhafte Unternehmen wie IBM, AOL, Twitter die Hadoop einsetzen. \cite{Guardian}\cite{Wayner2012}\cite{Casonato2012}\cite{Hadoop2012}

GlusterFS wurde von der Firma Gluster Inc. als Opensource Projekt entwickelt. Im Jahr 2011 wurde GlusterInc von der Firma Red Hat Inc. übernommen, um Lösungen für den Big Data Bereich anbieten zu können. Red Hat wurde mit der Übernahme von Gluster Inc zum Hauptunterstützer von GlusterFS.

\paragraph*{Online Speicher}
Einer der ersten grossen und wohl der bekannteste Online Speicher Anbieter zählt Amazone mit ihrem S3 Produkt. Amazone veröffentlicht zwar keine Finanzdaten über ihre Cloud Produkte, hingegen veröffentlichte Amazon Daten bezüglich dem Wachstum der Anzahl gespeicherten Objekten. Wie in \refabb{abb:s3_growth} zu sehen ist, speicherte Amazon im Jahr 2006 ca. 2.9 Milliarden Objekte, im Jahr 2010 wahren es bereits 269 Milliarden Objekte. Dieses Ergebnis könnte Amazon im Jahr 2011 mit 762 Milliarden Objekten mehr als verdoppeln. \cite{Barr2012}

\begin{center}
\includegraphics[, keepaspectratio = true]{media/s3_growth_2011_3.png}
\mycaption{figure}{\label{abb:s3_growth} 
Anzal gespeicherte Objekte in Amazon S3 \cite{Barr2012}}
\end{center}

Neben Amazon zählt RackSpace zu den führenden Cloud Anbietern. Wie Amazon bietet auch RackSpace den Online Speicher für jedermann an. Ihr Online Speicher wird unter der Produkt Bezeichnung Cloud File vermarktet. Hinter Cloud File steckt ein selbst entwickelter Speicher, OpenStack Object Storage mit Code Name Swift genannt. RackSpace hat an OpenStack Object Storage ein Jahr lang entwickelt und diese wie ihre anderen Cloud Eigenentwicklungen als Quelloffenes Projekt unter OpenStack veröffentlicht. Zu OpenStack tragen neben Rackspace weitere nahmhafte Unternehmen wie \gls{Dell}, \gls{HP}, Citrix, AMD, NetApp, Suse, AT\&T, NASA und andere bei.
RackSpace setzt ferner den selbst entwickelten OpenStack Object Storage als Online Speicher ein. \cite{OpenStack}

In der Schweiz ist die Entwicklung von Cloud Storage noch nicht so weit fortgeschritten wie in Amerika. Zu den wenigen Anbietern gehört unter anderem die Swisscom.


\section{Trend}
Für Gartner zählen Modular Disk Array Speicher und NAS zu den etablierten Speicherlösungen. Online Speicher (Cloud Storage) sieht Gartner eher als Speicherlösungen der Zukunft, welche sie laufend beobachtet und einen festen Platz in ihren Marktanalysen bekommen hat. 


%weshalb wiso warum

% wenig anbieter welche grosse datenanbieter 

% statistic massendaten
%!TEX root=../documentation-bachlorthesis-speicherarchitektur-lstucker.tex
\cleardoublepage
\chapter{Evaluation}

\section{Soll-Kriterien festlegen}
Die gewählten Kriterien für die Evaluation wurden Zusammen mit dem Auftraggeber im Meeting festgelegt. In einen weiteren schritt, sind die einzelnen Kriterien nach ihren Logischen Zugehörigkeit hierarchisch geordnet und verfeinert (\refabb{abb:AHPKriterienbaum}). Die Überarbeitete Kriterien Auswahl wurde bei einen weiteren Meeting mit dem Auftraggeber besprochen und fixiert. 

Die Kriterien sind für die spätere Verweise die Verweise Hierarchisch-Nummeriert.

\begin{center}
\includegraphics[width=\linewidth, keepaspectratio = true]{media/ahp_kirterienbaum.png}
\mycaption{figure}{\label{abb:AHPKriterienbaum} Optimales Speichersystem Kriterien}
\end{center}

\subsection{Haupt Soll-Kriterien festlegen}
Die Haupt-Kriterien sind auf der obersten Hierarchie Ebene und werden durch Ihre unter Kriterien definiert.
\setcounter{paragraph}{0}
\renewcommand\theparagraph{Soll-\arabic{paragraph}}
\paragraph{Kosten}\label{Soll-1}
Die Kosten sollen das Kostendach von ??? nicht überschreiten. 

\paragraph{Verfügbarkeit}\label{Soll-2}
Die Verfügbarkeit der Daten soll in Bezug auf die Szenerien die Verfügbarkeit erfüllen können. 

\paragraph{Datenzugriffe}\label{Soll-3}
Der Datenzugriff soll Skalieren können, dass heisst es soll möglich sein von mehren Webserver auf den Datenspeicher zugreifen zu können. Der Datenzugriff soll über POSIX IO oder über eine Dokumentiertes API erfolgen können.

\paragraph{Speicherkapazität}\label{Soll-4}
Die Speicherkapazität soll die Speicheranforderungen der Szenerien erfüllen können. Zudem sollen die Speicherung von Grossen Dateien bis mindestens 2 Gigabyte möglich sein.

\paragraph{Datenschutz}\label{Soll-5}
Der Datenschutz der Daten soll gewährleistet werden, dabei gilt das Haupt Augenmerk auf die Unveränderbarkeit der Gespeicherten Daten. Wichtig für den Auftraggeber ist ebenfalls das die Datengesichert werden können.

\paragraph{Technologie}\label{Soll-6}
asdf

\subsection{Unter Soll-Kriterien festlegen}
Die unter Kriterien mit den gleichen Nummer-Ebene gehören zusammen zur gleichen oben-Kriterien und werden später bei der Gewichtung der Kriterien nur untereinander direkt Verglichen.

\setcounter{paragraph}{0}
\renewcommand\theparagraph{Soll-1-\arabic{paragraph}}

\paragraph{Anschaffungskosten}\label{Soll-1-1}

\paragraph{Unterhaltskosten}\label{Soll-1-2}

\paragraph{Nachhaltigkeit}\label{Soll-1-3}

\setcounter{paragraph}{0}
\renewcommand\theparagraph{Soll-2-\arabic{paragraph}}

\paragraph{Redundanz}\label{Soll-2-1}
Die redundante Haltung der aktiven Daten, dass heisst der Daten die Aktive zugegriffen und manipuliert werden können, Dient zur Erhöhung der Verfügbarkeit der Daten bei einen Komponenten Ausfall. Die doppelt Haltung der Daten zur Sicherungszwecken, wie Archivierung wird nicht als redundant erachtet, da diese nicht direkt zu einer Erhöhung der System Verfügbarkeit ohne Menschliche Unterstützung führt. Die Speicherlösung sollte mindestens die Daten doppelt, wenn möglich dreifach Redundanz halten. 


\paragraph{Systemverfügbarkeit}\label{Soll-2-2}
Die Systemverfügbarkeit, wird durch Software oder Hardware Redundanz erreicht. Das System soll möglichst Redundant ausgelegt sein um eine Verfügbarkeit nach AEC-3 Standard zu erreichen.

\paragraph{Standort-übergreifend}\label{Soll-2-3}
Die aktive Daten sollen nach Möglichkeit an mindesten zwei Standorte verfügbar sein, um den Dienst bei einen Ausfall eines Rechenzentrums aufrecht zu erhalten können.

\setcounter{paragraph}{0}
\renewcommand\theparagraph{Soll-3-\arabic{paragraph}}

\paragraph{Skalierbarkeit}\label{Soll-3-1}
Die Speicherlösung sollte bei bedarf Ihren Speicher an bis zu 30 Serversysteme zur Verfügung stellen können.

\paragraph{Performance}\label{Soll-3-2}
Die Speicherlösung soll eine IO Performance von mindesten 27.31 MBit pro Sekunde haben.

\paragraph{POSIX IO}\label{Soll-3-3}
Die POSIX IO (inoffizielle Bezeichnung) ist eine Teil des POSIX Standards welche die IO Schnittstelle für POSIX Kompatible Applikationen definiert. Der Standard definiert unteranderem die Funktionen read(), write(), open(), close() inklisive deren Fehler Behandlung. Die Speicherlösung soll, für eine einfache Implementierung, nach Möglichkeit diesen Standard unterstützen. 

\paragraph{Simultaner Lesezugriff auf Objekte}\label{Soll-3-4}
Das gleichzeitige Lesen auf das selbe Objekte von zwei oder mehrere Serversysteme soll möglich sein.

\paragraph{Simulataner Schreibzugriff auf Objekte}\label{Soll-3-5}
Das gleichzeitige schreiben auf das selbe Objekte von zwei oder mehrere Serversysteme soll optional möglich sein.

\setcounter{paragraph}{0}
\renewcommand\theparagraph{Soll-4-\arabic{paragraph}}

\paragraph{Skalierbarkeit}\label{Soll-4-1}

\paragraph{Max Anzahl speicherbare Objekte}\label{Soll-4-2}
Das Speicherlösung soll die Anzahl Speicherbaren Objekten der Soll-Zenarien unterstützen. 

\paragraph{Max Objekte Speichergrösse grösser als 2 GB}\label{Soll-4-3}
Das Speichersystem muss die Speicherung von Objekten mit einer Speichergrösse von mindestenz 2 GigaByte unterstützen.

\setcounter{paragraph}{0}
\renewcommand\theparagraph{Soll-5-\arabic{paragraph}}

\paragraph{Datenintegrität}\label{Soll-5-1}
Die Datenintegrität der gespeicherten Daten soll gewährleistet sein.

\paragraph{Selbstheilung von Objekten}\label{Soll-5-2}
Die Selbstheilung von nicht mehr integer Daten soll nach Möglichkeit unterstützt werden. Diese Funktion ist bei der Verwaltung von grossen Datenmengen eine unterstützende Funktion.

\paragraph{Datensicherung}\label{Soll-5-3}
Die gespeicherten Daten soll mit einen Sicherungsverfahren gesichert werden können. Wenn die aktiven Daten nicht an zwei Standorten zur Verfügung gestellt werden können, wie in (\refsoll{Soll-2-3}) definiert, ist es zwingen erforderlich das die Sicherung an einen zweiten Standort erfolgen kann.

\paragraph{Datensicherheit}\label{Soll-5-4}
Die Datenberechtigung wird in der Applikation Implementiert, die Speicherlösung soll nach Möglichkeit Sicherstellen, das die Daten nicht von dritten Zugegriffen werden kann.

\setcounter{paragraph}{0}
\renewcommand\theparagraph{Soll-6-\arabic{paragraph}}

\paragraph{Marktverbreitung / Marktchancen}\label{Soll-6-1}
Die Speicherlösung soll im Markt verbreitet sein oder Tendenzen aufweisen die auf eine Verbreiterung im Markt in den nächsten fünf Jahren schliessen lässt.

\paragraph{Weiterentwicklung}\label{Soll-6-2}
Die Speicher Technologien welche aktive Weiterentwickelt werden soll hoher bewertet werden als Speicherlösung Technologie welche nicht mehr weiter gepflegt werden.

\paragraph{Verfügbarkeit von Experten}\label{Soll-6-3}
Die Verfügbarkeit von Experten Speicherlösungstechnologie soll gegeben sein. Dabei ist das Experten Wissen auch nach Verfügbarkeit nach Lokalität zu bewerten. Die Verfügbarkeit von Experten im Raum Schweiz ist höher zu werten als im Raum Europa.

\paragraph{Verwaltungskomfort}\label{Soll-6-4}
Die Speichertechnologie soll die geforderte Datenmenge mit möglichst geringen Aufwand komfortable verwalten lassen.

\paragraph{Kinderschuhe / Ausgereift}\label{Soll-6-5}
Die Speicher Technologie soll Ausgereift und Stabil laufen. Eine Implementierung von Beta Technologie ist nicht erwünscht.
Die Technologie soll Ausgereift sein, bzw. keine erst Implementierung.

\section{KO-Kriterien}
Die KO-Kriterien sind muss Kriterien die von einen Speichersystem erfüllt werden müssen. Speichersysteme welche die KO-Kriterien nicht erfüllen werden bei der AHP-Evaluation ausgeschlossen. Die KO-Kriterien wurde zusammen mit dem Auftraggeber besprochen und vereinbart.

\setcounter{paragraph}{0}
\renewcommand\theparagraph{KO-\arabic{paragraph}}

\paragraph{Dateigrösse bis 2 Gigibyte}\label{KO-1}
Die Speicherung von Dateien die eine Speichergrösse von 2 Gibibyte muss unterstützt werden.

\paragraph{Speicherkapazität Szenarien}\label{KO-2}
Die geforderten Speicherkapazitäten der Szenerien inklusive die benötigte Kapazität für die Redundanz muss von der Speicherlösung unterstützt werden.

\paragraph{Kosten Spannweite}\label{KO-2}
Die Kosten der teuersten Speicherlösung, darf nicht dreimal teuerer sein als die Günstigste Lösung.

\section{Auswahl der Alternativen / Vertreter}
Mit dem Auftraggeber wurde definiert, dass aus den Speicherlösung, SAN, NAS, Distributed Filesystem Cluster, Online Speicher und   Dedizierter Server Webserver einen Vertreter für die Evaluation ausgewählt werden soll. Die Lösungen wurden nach Marktverbreitung, Technologie Leader ausgewählt.

Die Alternativen/ Vertreter sind für die spätere Verweise die Verweise Nummeriert.

\setcounter{paragraph}{0}
\renewcommand\theparagraph{Al-\arabic{paragraph}}
\paragraph{Hetzner}\label{Al-1}
Dedizierte Webserver zählen im engeren Sinn nicht als reine Speicherlösungen, der Deutsche Hosting Anbieter Hetzner bietet jedoch Dedizierte Webserver mit einer Speicherkapazität welche die Anforderung an die Kapazität des Szenario 1 (siehe \refsec{Szenario1}) erfüllen. Hetzner als Dedizierter Webserver Anbieter wurde ausgewählt, weil die bestehende Lösungen ebenfalls mit Hetzer Webserver realisiert ist. 

\paragraph{NetApp}\label{Al-2}
Als Vertreter für die NAS Speicherlösung wurde Netapp xxx ausgewählt. Die Firma Netapp gehört mit ihren NAS Speicherlösung zu den Markführer (siehe \refsec{MartkEtabliert}) und gelten als 

\paragraph{NetApp}\label{Al-3}


\paragraph{Open-Stack Cloud Storage}\label{Al-4}
Open-Stack 
Als Verteilte Speicherlösung wurde ursprünglich zu beginn der Arbeit Hadoop HDFS ausgewählt, jedoch haben eigene Recherchen ergeben, dass Hadoop HDFS eher für die Speicherung von grossen Daten welche mittels einen MapReduce Algorithmus verarbeitet werden. Als Alternative zu Hadoop HDFS, kristalisierte sich GlusterFS und OpenStack Cloud Storage. Die Entscheidung viel am Ende auf Open-Stack Cloud Storage, da diese bereits erfolgreich bei RackSpace einen grösseren Online Speicheranbieter betrieben wird und es eine vergleichbares API hat wie Amazon S3. Zudem wird OpenStack von namhaften Informatik Unternehmen unterstützt-


\paragraph{Amazon S3}\label{Al-5}
Amazon S3 wurde als Represetator für Online Speicherlösung gewählt. Amazon S3 ist gemäss \refsec{MartkEtabliert} einer der Etabliertesten wenn nicht die erfolgreichste Online Speicherlösung. Amazon betreibt mehre Rechenzentrum in der gesamten Welt, als Speicherlösung Standort wird Europa ausgewählt.


\subsection{Gewichtung der Soll-Kriterien mit AHP}

% Vergleich mit Kosten
\paragraph*{\refsoll{Soll-1} verglichen mit \refsoll{Soll-2} (\ref{Soll-1}/\ref{Soll-2})} 
Mit steigenden Anforderungen an die Verfügbarkeit steigen auch die Kosten. Der Betrieb einer Infrastruktur eines Service Anbieters muss Kostendecken sein. Einen Ausfall des System während definierten Zeiten in welches das System Online sein muss, hat ebenfalls Auswirkungen auf das Unternehmen. So kann es zum Image Verlust, zu Kunden Abgänge führen, oder die Zahlung von Entschädigung erfordern. Einen Datenverlust kann trotzt Sicherungskopie bei Grossen Datenmengen zeitintensive und kostspielig werden. Aus diesen Gründen ist eine gute Balance zwischen Kosten und Verfügbarkeit zu finden. Die Kosten sind deshalb im Vergleich zur Verfügbarkeit etwas grösser zu Gewichten.

\textbf{Gewichtung: 3}

\paragraph*{\refsoll{Soll-1} verglichen mit \refsoll{Soll-3} (\ref{Soll-1}/\ref{Soll-3})}

\paragraph*{\refsoll{Soll-1} verglichen mit \refsoll{Soll-4}}

\paragraph*{\refsoll{Soll-1} verglichen mit \refsoll{Soll-5}}
Die Kosten werden als 
\paragraph*{\refsoll{Soll-1} verglichen mit \refsoll{Soll-6}}

% Vergleich mit Verfügbarkeit
\paragraph*{\refsoll{Soll-2} verglichen mit \refsoll{Soll-3}}
Aus Sicht des Anwenders ist die Verfügbarkeit der Daten höher zu Gewichten als möglichst schnellen Zugriff oder die Art und weise wie zugegriffen wird. Allerdings kann eine langes Zeitüberschreitung beim ausliefern der Daten für den Anwender ebenfalls als nicht Verfügbarkeit empfunden werden. Die Verfügbarkeit ist zwischen etwas Grösser bis erheblich Grösser zu Gewichten als der Datenzugriff.

\textbf{Gewichtung: 4}

\paragraph*{\refsoll{Soll-2} verglichen mit \refsoll{Soll-4}}
Zum Geschäftsmodell des Auftraggeber gehört unter anderem die Bereitstellung von Speicherkapazität für die Speicherung der Bilddaten seiner Kunden. Sind die Speicherkapazitäten ausgeschöpft ist es dem Auftraggeber nicht mehr möglich in diesen Geschäftsbereich weiter zu Wachsen. Aus diesen Grund ist die Verfügbarkeit etwas geringere zu Gewichten als die Speicherkapazität.

\textbf{Gewichtung: 1/3}


\paragraph*{\refsoll{Soll-2} verglichen mit \refsoll{Soll-5}}

\paragraph*{\refsoll{Soll-2} verglichen mit \refsoll{Soll-6}}


% Vergleich mit Datenzugriffe
\paragraph*{\refsoll{Soll-3} verglichen mit \refsoll{Soll-4}}

\paragraph*{\refsoll{Soll-3} verglichen mit \refsoll{Soll-5}}

\paragraph*{\refsoll{Soll-3} verglichen mit \refsoll{Soll-6}}

% Vergleich mit Speicherkapazität
\paragraph*{\refsoll{Soll-4} verglichen mit \refsoll{Soll-5}}

\paragraph*{\refsoll{Soll-4} verglichen mit \refsoll{Soll-6}}

% Vergleich mit Datenschutz
\paragraph*{\refsoll{Soll-5} verglichen mit \refsoll{Soll-6}}




\begin{table}[htbp]
\caption{AHP Gewichtung Top Kriterien}
\begin{tabular}{|l|c|c|c|c|c|r|r|}
\hline
\multicolumn{ 1}{|c|}{} & \multicolumn{ 5}{c|}{Evalutations Matrix} & \multicolumn{1}{l|}{} & \multicolumn{1}{l|}{Gewicht} \\ \cline{ 2- 8}
\multicolumn{ 1}{|c|}{} & K & V & P & S & D & \multicolumn{1}{c|}{r} & \multicolumn{1}{c|}{w} \\ \hline
Kosten (K) & \textbf{1} & 2 & 3 & 1 & 1 & 1.131 & 0.226 \\ \hline
Verfügbarkeit (V) &  1/2 & \textbf{1} & 5 &  1/3 &  1/8 & 0.569 & 0.114 \\ \hline
Performance (P) &  1/3 &  1/5 & \textbf{1} &  1/4 &  1/7 & 0.257 & 0.051 \\ \hline
Speicherkapazität (S) & 1 & 3 & 4 & \textbf{1} &  1/2 & 1.071 & 0.214 \\ \hline
Datenqualität (D) & 1 & 8 & 7 & 2 & \textbf{1} & 1.972 & 0.394 \\ \hline  \hline
Ci & \multicolumn{1}{r|}{3.833} & \multicolumn{1}{r|}{14.200} & \multicolumn{1}{r|}{20.000} & \multicolumn{1}{r|}{4.583} & \multicolumn{1}{r|}{2.768} & 5 & 1 \\ \hline 
\end{tabular}
\label{AHPTop}
\end{table}

\paragraph*{\refsoll{Soll-1-1} verglichen mit \refsoll{Soll-1-2} (\ref{Soll-1-1}/\ref{Soll-1-2})}
Die Betriebskosten sind der Hauptkostenfakor in der Lebenszeit eines Informations-Systems. Gemäss Gartner vielen die Weltweiten IT-Kosten im Jahr 2011 zu 20\% für die Computer Hardware und zu 43\% IT-Service an. Zudem Steigen die Kosten zum Beispiel von Disk Array Speicher nach Ablauf der ordentlichen von Hersteller gewährleisteten Wartung, wegen teuren weiterführenden Wartungsverträge stark an.
Aus diesen Grund sind die Anschaffung Kosten im vergleich zur Unterhaltskosten erheblich geringer zu Gewichten.

\textbf{Gewichtung: 1/5}

\paragraph*{\refsoll{Soll-1-1} verglichen mit \refsoll{Soll-1-3} (\ref{Soll-1-1}/\ref{Soll-1-3})}
Fällt die Langlebigkeit eines Systems, weil es technologisch veraltet ist oder weil die Kosten für Wartungsverträge nach Ablauf der ordentlichen Wartung im Vergleich zur neu Anschaffung unrentabel sind, kurz aus. Sind erneut Kosten in der Anschaffung und Migration der Daten notwendig. Aus diesen Grund sind die Anschaffungskosten im Vergleich zur Langlebigkeit etwas geringer zu Gewichten.

\textbf{Gewichtung: 1/3}


\paragraph*{\refsoll{Soll-1-2} verglichen mit \refsoll{Soll-1-3} (\ref{Soll-1-2}/\ref{Soll-1-3})}
Der Hauptkostenfakor in der Lebenzeit eines Information System sind die Betreibskosten, steigen diese Kosten Aufgrund hoher Wartungverträgekosten mit der Lebenspanne des Systems an, kann sich der Betrieb als unrentable herausstellen. Aus diesen Grund sind die Betriebskosten im Vergleich zu Langlebigkeit etwas grösser zu Gewichten.

\textbf{Gewichtung: 3}

\begin{table}[htbp]
\caption{AHP Kosten}
\begin{tabular}{|l|c|c|c|r|r|}
\hline
\multicolumn{ 1}{|c|}{Wirtschaftlichkeit} & \multicolumn{ 3}{c|}{Evalutations Matrix} & \multicolumn{1}{l|}{} & \multicolumn{1}{l|}{Gewicht} \\ \cline{ 2- 6}
\multicolumn{ 1}{|c|}{} & A & U & L & \multicolumn{1}{c|}{r} & \multicolumn{1}{c|}{w} \\ \hline
Anschaffung (A) & \textbf{1} &  1/4 &  1/3 & 0.368 & 0.123 \\ \hline
Unterhaltskosten (U)  & 4 & \textbf{1} & 2 & 1.671 & 0.557 \\ \hline
Langlebigkeit (L) & 3 &  1/2 & \textbf{1} & 0.961 & 0.320 \\ \hline \hline 
Ci & \multicolumn{1}{r|}{8.00} & \multicolumn{1}{r|}{1.750} & \multicolumn{1}{r|}{3.333} & 3 & 1 \\ \hline
\end{tabular}
\label{AHPKosten}
\end{table}


\paragraph*{\refsoll{Soll-2-1} verglichen mit \refsoll{Soll-2-2} (\ref{Soll-2-1}/\ref{Soll-2-2})}
Die Daten eines Informationssystem sind dessen wertvollstes Gut, mit höherer Redundanz der Daten steigt auch die Verfügbarkeit der Daten.
Die Gesamt Verfügbarkeit hängt jedoch auch von der Verfügbarkeit der System Komponenten zusammen. Aus diesen Grund sollten die Daten System übergreifend Redundant sein um eine hohe Verfügbarkeit zu erreichen. Gemäss eigener Erfahrungen, ist die Zahl der Datenträger ausfälle höher als die restlichen Komponenten ausfälle eines Systems. Die Datenredundanz ist deshalb erheblich grösser zu Gewichten als die System Redundanz.

\textbf{Gewichtung: 5}

\paragraph*{\refsoll{Soll-2-1} verglichen mit \refsoll{Soll-2-3} (\ref{Soll-2-1}/\ref{Soll-2-3})}
Die Standortübergreifende Verfügbarkeit der Daten kann nur mit Redundanz der Daten erreicht werden, aus diesen Grund ist die Datenredundanz absolut höher zu Gewichten als die Standortübergreifende Verfügbarkeit.

\textbf{Gewichtung: 9}

\paragraph*{\refsoll{Soll-2-2} verglichen mit \refsoll{Soll-2-3} (\ref{Soll-2-2}/\ref{Soll-2-3})}
Die Standortübergreifende Verfügbarkeit des System kann nur mit höherer Systemverfügbarkeit erreicht werden. Aus diesen Grund ist die Redundanz der sehr viel höher zu Gewichten als die Standortübergreifende Verfügbarkeit.

\textbf{Gewichtung: 7}

\begin{table}[htbp]
\caption{AHP Verfügbarkeit}
\begin{tabular}{|p{7.1cm}|c|c|c|r|r|}
\hline
\multicolumn{ 1}{|c|}{Verfügbarkeit} & \multicolumn{ 3}{c|}{Evalutations Matrix} & \multicolumn{1}{l|}{} & \multicolumn{1}{l|}{Gewicht} \\ \cline{ 2- 6}
\multicolumn{ 1}{|c|}{} & R & SV & St & \multicolumn{1}{c|}{r} & \multicolumn{1}{c|}{w} \\ \hline
Redundanz (R) & \textbf{1    } & 8     & 3     & 1.971 & 0.657 \\ \hline
Systemverfügbarkeit (SV) &  1/8 & \textbf{1} &  1/5 & 0.205 & 0.068 \\ \hline
Standortübergreifend (St) &  1/3 & 5     & \textbf{1} & 0.824 & 0.275 \\ \hline \hline
Ci & \multicolumn{1}{r|}{1.458} & \multicolumn{1}{r|}{14.00} & \multicolumn{1}{r|}{4.200} & 3 & 1 \\ \hline
\end{tabular}
\label{AHPVerfügbarkeit}
\end{table}

\paragraph*{\refsoll{Soll-3-1} verglichen mit \refsoll{Soll-3-2} (\ref{Soll-3-1}/\ref{Soll-3-2})}
Die Skalierung der Datenzugriff, dass heisst der Zugriff von mehren Systemen, ermöglicht es die Web-Applikation höher Redundant zu betreiben und die Verarbeitung der Bilddaten auf mehre Server zu Verteilen. Der maximale Datendurchsatz ist daher weniger bedeutend als dessen kostanz bei der Verteilung der Zugriffe auf mehere Server. Aus diesen Grund. Ein Speicherlösung welche schlechte Performance aufweist, skaliert jedoch in der Regel ebenfalls nicht. Aus diesen Grund ist die Skalierung der Datenzugriffe etwas grosser Gewichten als die Performance 

\textbf{Gewichtung: 3}

\paragraph*{\refsoll{Soll-3-1} verglichen mit \refsoll{Soll-3-3} (\ref{Soll-3-1}/\ref{Soll-3-2})}
Ist eine Zugriff auf die Daten über POSIX IO möglich, fällt allenfalls die Anpassung der Entwickelte Web-Applikation geringer aus als wenn er auf die Daten per API zugreifen muss. Für den Betrieb der Web-Applikation ist die Skalierung des Datenzugriffs sehr viel bis absolut bedeutender als die Methode des Datenzugriffs.

\textbf{Gewichtung: 8}


\paragraph*{\refsoll{Soll-3-1} verglichen mit \refsoll{Soll-3-4} (\ref{Soll-3-1}/\ref{Soll-3-4})}
Der Simultaner Lese Zugriff auf ein Objekt ermöglicht es, dass eine Bilddatei von mehren Server simultan gelesen werden kann und den Website Besuchern dargestellt werden kann. Wird diese nicht unterstützt ist es möglich das dem Webseiten Besucher die Bilddatei nicht dargestellt werden kann, wenn ein anderer Besucher die selbe Bilddatei betrachtet. Aus diesen Grund ist die Skalierung und der Simultane Lese Zugriff auf Objekten gleich bedeutend.

\textbf{Gewichtung: 1}


\paragraph*{\refsoll{Soll-3-1} verglichen mit \refsoll{Soll-3-5} (\ref{Soll-3-1}/\ref{Soll-3-5})}
Der Simultaner Schreibzugriff auf ein Objekt erlaubt es, dass eine Objekt simultan von zwei oder mehreren Systemen modifiziert werden kann. Die Web-Applikation des Auftraggeber führt jedoch keine Änderungen an einer Original Bilddatei durch sondern erstellt modifizierte Kopien. Aus diesen Grund ist Möglichkeit der gleichzeitigen Schreibzugriff auf ein Objekt absolut geringer zu Gewichten als die Skalierung des Datenzugriffs

\textbf{Gewichtung: 9}

\paragraph*{\refsoll{Soll-3-2} verglichen mit \refsoll{Soll-3-3} (\ref{Soll-3-2}/\ref{Soll-3-3})}
Für den Betrieb der Web-Applikation ist die Performance der Datenzugriffe sehr viel bedeutender als die Methode des Datenzugriffs.

\textbf{Gewichtung 7}

\paragraph*{\refsoll{Soll-3-2} verglichen mit \refsoll{Soll-3-4} (\ref{Soll-3-2}/\ref{Soll-3-4})}
Die Performance ist erheblich geringer bedeutend als der simultane Lesezugriff auf Objekte. Grund dafür ist das das gleiche Objekte von mehren Web-Server gelesen werden können um diese den Website Besuchern darstellen können 

\textbf{Gewichtung: 1/5}

\paragraph*{\refsoll{Soll-3-2} verglichen mit \refsoll{Soll-3-5} (\ref{Soll-3-2}/\ref{Soll-3-5})}
Aufgrund das keine Manipulationen an der Original Bilddatei durchgeführt wird, ist die Performance der Datenzugriff erheblich höher zu Gewichten als der Simultane Schreib zugriff auf Objekte 

\textbf{Gewichtung: 5}


\paragraph*{\refsoll{Soll-3-3} verglichen mit \refsoll{Soll-3-4} (\ref{Soll-3-3}/\ref{Soll-3-4})}
Die Zugriff Methode der Web-Applikation kann bei bedarf durch die Entwickler angepasst werden. Für den Betrieb der Web-Applikation ist daher die Zugriffs Methode auf die Bilddaten erheblich geringer bedeutend als der Simultane Lese Zugriff.

\textbf{Gewichtung: 1/5}


\paragraph*{\refsoll{Soll-3-3} verglichen mit \refsoll{Soll-3-5} (\ref{Soll-3-3}/\ref{Soll-3-5})}
 Aufgrund das keine Änderungen an Original Bilddateien durchgeführt werden, ist es für den Auftraggeber bedeutender, dass eine Zugriff über POSIX IO möglich ist.

\textbf{Gewichtung: 3}


\paragraph*{\refsoll{Soll-3-4} verglichen mit \refsoll{Soll-3-5} (\ref{Soll-3-4}/\ref{Soll-3-5})}
Die Webapplikation des Auftraggeber führt keine Änderungen an der Original Bilddatei durch, was den Simultaner Schreibzugriff auf Objekten für den Betrieb der Web-Applikation von geringer Bedeutung ist. Der Simultane Lesezugriff auf Objekten muss für den Betrieb der Web-Applikation möglich sein, damit die Bilddaten mehren Websitzungen dargestellt werden können. Aus diesen Grund ist es absolut grosser von Gewicht das ein Simultaner Lesezugriff möglich ist als der Simultaner Schreibzugriff.

\textbf{Gewichtung: 9}

\begin{table}[htbp]
\caption{AHP Gewichtung Datenzugriff}
\begin{tabular}{|p{4.5cm}|c|c|c|c|c|r|r|}
\hline
\multicolumn{ 1}{|c|}{Datenzugriff} & \multicolumn{ 5}{c|}{Evalutations Matrix} & \multicolumn{1}{l|}{} & \multicolumn{1}{l|}{Gewicht} \\ \cline{ 2- 8}
\multicolumn{ 1}{|c|}{} & Sk & P & PIO  & L & S & \multicolumn{1}{c|}{r} & \multicolumn{1}{c|}{w} \\ \hline
Skalierbarkeit (Sk) & \textbf{1} & 2 & 8 & 4 & 5 & 2.193 & 0.439 \\ \hline
Performance (P) &  1/2 & \textbf{1} & 5 &  1/2 &  1/3 & 0.649 & 0.130 \\ \hline
POSIX IO (PIO) &  1/8 &  1/5 & \textbf{1} &  1/9 &  1/7 & 0.152 & 0.030 \\ \hline
Simultaner Lesezufgriff 
auf Objekte (L) &  1/4 & 2 & 9 & \textbf{1} & 3 & 1.149 & 0.230 \\ \hline
Simultaner Schreibzugriff
 auf Objekte (S) &  1/5 & 3 & 7 &  1/3 & \textbf{1} & 0.857 & 0.171 \\ \hline \hline
Ci & \multicolumn{1}{r|}{2.075} & \multicolumn{1}{r|}{8.200} & \multicolumn{1}{r|}{30.00} & \multicolumn{1}{r|}{5.944} & \multicolumn{1}{r|}{9.476} & 5 & 1 \\ \hline
\end{tabular}
\label{AHPDatenzugriff}
\end{table}

\paragraph*{\refsoll{Soll-4-1} verglichen mit \refsoll{Soll-4-2} (\ref{Soll-4-1}/\ref{Soll-4-2})}
Die Skalierung der Speicherkapazität ist gleich zu Gewichten wie die maximal Anzahl Speicherbare Objekt. Beides sind limitierende Faktoren, die bei erreichen der Grenze die Speicherung von neuen Objekten verunmöglichen. Erreicht man die Grenze der Speicherbaren Objekten,  kann der Vorhandene frei Speicherkapazität nicht für neue Objekte verwendet werden. Umgekehrt kann die Speicherkapazität nicht ausgebaut werden, können die frei Kapazität an Speicherbaren Objekten nicht dazu verwendet werden neue Objekte zu Speichern.

\textbf{Gewichtung: 1}

\paragraph*{\refsoll{Soll-4-1} verglichen mit \refsoll{Soll-4-3} (\ref{Soll-4-1}/\ref{Soll-4-3})}
Die Skalierbarkeit der Speicherkapazität, ist für den Betrieb sehr viel grösser zu Gewichten als die Maximale Speichergrösse für Objekte die grösser als 2 Gigibyte sind.

\textbf{Gewichtung: 7}

\paragraph*{\refsoll{Soll-4-2} verglichen mit \refsoll{Soll-4-3} (\ref{Soll-4-2}/\ref{Soll-4-3})}
Die maximale Anzahl speicherbare Objekte, ist für den Betrieb sehr viel grösser zu Gewichten als die Maximale Speichergrösse für Objekte die grösser als 2 Gigibyte sind.

\textbf{Gewichtung: 7}

\begin{table}[htbp]
\caption{AHP Gewichtung Speicherkapazität}
\begin{tabular}{|p{7.1cm}|c|c|c|r|r|}
\hline
\multicolumn{ 1}{|c|}{Speicherkapazität } & \multicolumn{ 3}{c|}{Evalutations Matrix} & \multicolumn{1}{l|}{} & \multicolumn{1}{l|}{Gewicht} \\ \cline{ 2- 6}
\multicolumn{ 1}{|c|}{} & S & A & G & \multicolumn{1}{c|}{r} & \multicolumn{1}{c|}{w} \\ \hline
Skalierbarkeit (S) & \textbf{1} & 1 & 9 & 1.498 & 0.499 \\ \hline
Max Anzahl speicherbare Objekte (A) & 1 & \textbf{1} & 6 & 1.310 & 0.437 \\ \hline
Max Objekt Speichergrösse grösser 2GB (G) &  1/9 &  1/6 & \textbf{1} & 0.192 & 0.064 \\ \hline \hline
Ci & \multicolumn{1}{r|}{2.111} & \multicolumn{1}{r|}{2.167} & \multicolumn{1}{r|}{16.00} & 3 & 1 \\ \hline
\end{tabular}
\label{AHPSpeicherkapazität}
\end{table}

\paragraph*{\refsoll{Soll-5-1} verglichen mit \refsoll{Soll-5-2} (\ref{Soll-5-1}/\ref{Soll-5-2})}
Damit das Speichersystem korrupte Objekte selbständig heilen kann, ist es erforderlich das die Objekte Redundant gespeichert sind und die Integrität der Gespeicherten Objekte geprüft werden kann. Die Integrität der Objekte wird dabei mit einen zuvor erstellten und gespeicherten Hash Prüfsumme verglichen. Aus diesen Grund ist die Datenintegrität erheblich höher zu Gewichten als die Selbstheilung von Objekten.

\textbf{Gewichtung: 5}

\paragraph*{\refsoll{Soll-5-1} verglichen mit \refsoll{Soll-5-3} (\ref{Soll-5-1}/\ref{Soll-5-3})}
Verliert man alle \gls{Primären-Daten} durch einen Hard-, Software Fehler oder durch einwirken von dritten, ist es unerlässlich das eine Sicherungskopie bereit steht um nicht den Total Datenverlust zu erleiden.
Eine weiter Gefahr von Datenverlust besteht, wenn die Datenintegrität nicht sichergestellt ist. Ist ein Objekt nicht mehr integer bzw. korrupt und wird diese vor der Sicherung nicht festgestellt, besteht die Gefahr eines Datenverlustes.
Der Verlust aller Daten ist jedoch schwerwiegender als der Verlust einzelner Daten.
Aus diesen Grund ist die Datenintegrität geringer zu Gewichten als die Datensicherung.

\textbf{Gewichtung: 1/3}

\paragraph*{\refsoll{Soll-5-1} verglichen mit \refsoll{Soll-5-4} (\ref{Soll-5-1}/\ref{Soll-5-4})}
Primär muss bei einer Webapplikation die Sicherheit der Daten innerhalb der Webapplikation sicher gestellt werden. Aus diesen Grund ist die Datenintegrität sehr viel grösser zu Gewichten als Sicherheit.

\textbf{Gewichtung: 7}

\paragraph*{\refsoll{Soll-5-2} verglichen mit \refsoll{Soll-5-3} (\ref{Soll-5-2}/\ref{Soll-5-3})}
Die Selbstheilung von Daten stellt sicher das alle Redundanten gespeicherten \gls{Primären-Daten} integer sind. Durch die Selbstheilung verringert sich das Risiko der eines Datenverlust zu erleiden. 
Der Verlust aller Daten ist jedoch schwerwiegender als der Verlust einzelner Daten.
Aus diesen Grund ist die Selbstheilung der Daten erheblich geringer zu Gewichten als die Datensicherung.

\textbf{Gewichtung: 1/5}

\paragraph*{\refsoll{Soll-5-2} verglichen mit \refsoll{Soll-5-4} (\ref{Soll-5-2}/\ref{Soll-5-4})}
Primär muss bei einer Webapplikation die Sicherheit der Daten innerhalb der Webapplikation sicher gestellt werden. Aus diesen Grund ist die Selbsheilung der Daten erheblich grosser zu Gewichten als Sicherheit.

\textbf{Gewichtung: 5}

\paragraph*{\refsoll{Soll-5-3} verglichen mit \refsoll{Soll-5-4} (\ref{Soll-5-3}/\ref{Soll-5-4})}
Die Datensicherheit ist Primär auf der Web-Applikation Schicht zu gewährleisten und zu realisieren. Erfährt man einen Datenverlust durch einwirken von dritten, ist sicherzustellen das eine Sicherungskopie der Daten existiert. Aus diesen Grund ist die Sicherung der Daten absolut höher zu Gewichten als die Sicherheit.  

\textbf{Gewichtung: 5}

\begin{table}[htbp]
\caption{AHP Gewichtung Datenschutz}
\begin{tabular}{|l|c|c|c|c|r|r|}
\hline
\multicolumn{ 1}{|c|}{Datenqualität} & \multicolumn{ 4}{c|}{Evalutations Matrix} & \multicolumn{1}{l|}{} & \multicolumn{1}{l|}{Gewicht} \\ \cline{ 2- 7}
\multicolumn{ 1}{|c|}{} & I & H & B & S & \multicolumn{1}{c|}{r} & \multicolumn{1}{c|}{w} \\ \hline
Datenintegrität (I) & \textbf{1} & 9 & 3 & 7 & 2.316 & 0.579 \\ \hline
Selbstheilung von Objekten (H) &  1/9 & \textbf{1} &  1/6 &  1/3 & 0.186 & 0.046 \\ \hline
Sicherung (B) &  1/3 & 6 & \textbf{1} & 5 & 1.130 & 0.282 \\ \hline
Sicherheit (S) &  1/7 & \multicolumn{1}{r|}{3    } &  1/5 & \textbf{1} & 0.369 & 0.092 \\ \hline  \hline
Ci & \multicolumn{1}{r|}{1.587} & \multicolumn{1}{r|}{19.00} & \multicolumn{1}{r|}{4.367} & \multicolumn{1}{r|}{13.333} & 4 & 1 \\ \hline
\end{tabular}
\label{AHPDatenqualität}
\end{table}

\paragraph*{\refsoll{Soll-6-1} verglichen mit \refsoll{Soll-6-2} (\ref{Soll-6-1}/\ref{Soll-6-2})}Eine Technologie die nicht mehr weiter Entwickelt wird, wird trotzt allfähiger aktuellen grosser Marktverbreitung, über kurz oder längere Zeit durch neuere Technologie aus dem Markt verdrängt. Beim entscheid einer neuen Lösung ist es deshalb wichtig, dass man sich für eine Technologie Entscheidet die nicht am Ende Ihrer Lebenszyklus steht. Aus diesen Grund ist die Marktverbreitung/Marktchance erheblich geringer zu Gewichten als die Weiterentwicklung.

\textbf{Gewichtung: 1/5}


\paragraph*{\refsoll{Soll-6-1} verglichen mit \refsoll{Soll-6-3} (\ref{Soll-6-1}/\ref{Soll-6-3})}
Durch eine hohe Marktverbreitung ist in der Regeln die Verfügbarkeit von Experten ebenfalls gegeben. Aus diesen Grund sind die Marktverbreitung und die Verfügbarkeit von Experten gleich zu Gewichten

\textbf{Gewichtung: 1}

\paragraph*{\refsoll{Soll-6-1} verglichen mit \refsoll{Soll-6-4} (\ref{Soll-6-1}/\ref{Soll-6-4})}
Ein Produkt welches hohen Verwaltungskomfort aufweist, jedoch wegen anderen Faktoren eine schlechte Marktverbreitung oder Marktchancen aufweist, ist für den längeren betrieb schlechter geeignet als eine Produkt mit weniger hohen Verwaltungskomfort. Die Marktverbreitung ist deshalb etwas hoher zu Gewichten als der Verwaltungskomfort.

\textbf{Gewichtung: 3}

\paragraph*{\refsoll{Soll-6-1} verglichen mit \refsoll{Soll-6-5} (\ref{Soll-6-1}/\ref{Soll-6-5})}
Für den Betrieb der Web-Applikation ist es wichtig das die Speicherlösung technisch und Betrieblich ausgereift sind. Die Marktverbreitung ist deshalb absolut geringer zu Gewichten als die Ausgereiftheit der Speicherlösung.

\textbf{Gewichtung: 1/9}

\paragraph*{\refsoll{Soll-6-2} verglichen mit \refsoll{Soll-6-3} (\ref{Soll-6-2}/\ref{Soll-6-3})}
Eine Technologie welche nicht mehr weiter entwickelt wird, verschwindet gänzlich oder 

\paragraph*{\refsoll{Soll-6-2} verglichen mit \refsoll{Soll-6-4} (\ref{Soll-6-2}/\ref{Soll-6-4})}

\paragraph*{\refsoll{Soll-6-2} verglichen mit \refsoll{Soll-6-5} (\ref{Soll-6-2}/\ref{Soll-6-5})}
Für den Betrieb der Web-Applikation ist es wichtig das die Speicherlösung technisch und Betrieblich ausgereift sind. Die Weiterentwicklung ist deshalb absolut geringer zu Gewichten als die Ausgereiftheit der Speicherlösung.

\textbf{Gewichtung: 1/9}


\paragraph*{\refsoll{Soll-6-3} verglichen mit \refsoll{Soll-6-4} (\ref{Soll-6-3}/\ref{Soll-6-4})}
Biete eine Speicherlösung einen hohen Verwaltungskomfort, lassen sich Experten für den regulären Betrieb einsparen. Bei der Implementierung und Optimierung von Speicherlösungen kommt man in der Regel jedoch nicht ohne Experten Wissen aus, deshalb ist es wichtig, dass der Zugriff auf Experten gegeben ist. Aus diesen Grund ist die  Verfügbarkeit von Experten etwas bis erheblich höher zu Gewichten als der Verwaltungskomfort. 

\textbf{Gewichtung: 4}

\paragraph*{\refsoll{Soll-6-3} verglichen mit \refsoll{Soll-6-5} (\ref{Soll-6-3}/\ref{Soll-6-5})}
Für den Betrieb der Web-Applikation ist es wichtig das die Speicherlösung technisch und Betrieblich ausgereift sind. Die Verfügbarkeit von Experten ist deshalb absolut geringer zu Gewichten als die Ausgereiftheit der Speicherlösung.

\textbf{Gewichtung: 1/9}

\paragraph*{\refsoll{Soll-6-4} verglichen mit \refsoll{Soll-6-4} (\ref{Soll-6-3}/\ref{Soll-6-5})}
Für den Betrieb der Web-Applikation ist es wichtig das die Speicherlösung technisch und Betrieblich ausgereift sind. Der Verwaltungskomfort ist deshalb absolut geringer zu Gewichten als die Ausgereiftheit der Speicherlösung.

\textbf{Gewichtung: 1/9}

\begin{table}[htbp]
\caption{AHP Gewichtung Technologie}
\begin{tabular}{|p{3.9cm}|c|c|c|c|c|r|r|}
\hline
\multicolumn{ 1}{|c|}{Technologie} & \multicolumn{ 5}{c|}{Evalutations Matrix} & \multicolumn{1}{l|}{} & \multicolumn{1}{l|}{Gewicht} \\ \cline{ 2- 8}
\multicolumn{ 1}{|c|}{} & M & W & E & V & K & \multicolumn{1}{c|}{r} & \multicolumn{1}{c|}{w} \\ \hline
Marktverbreitung / Marktchancen (M) & \textbf{1} &  1/7 & 1 & 3 &  1/9 & 0.363 & 0.073 \\ \hline
Weiterentwicklung (W) & 7 & \textbf{1} & 5 & 4 &  1/9 & 1.067 & 0.213 \\ \hline
Verfügbarkeit von Experten (E) & 1 &  1/5 & \textbf{1} & 2 &  1/9 & 0.316 & 0.063 \\ \hline
Verwaltungskomfort (V) &  1/3 & \multicolumn{1}{r|}{ 1/4} &  1/2 & \textbf{1} &  1/9 & 0.202 & 0.040 \\ \hline
Kinderschuhe Ausgereift (K) & 9 & 9 & 9 & 9 & \textbf{1} & 3.052 & 0.610 \\ \hline  \hline
Ci & \multicolumn{1}{r|}{18.333} & \multicolumn{1}{r|}{10.593} & \multicolumn{1}{r|}{16.500} & \multicolumn{1}{r|}{19.00} & \multicolumn{1}{r|}{1.444} & 5 & 1 \\ \hline
\end{tabular}
\label{AHPTechnologie}
\end{table}


\section{Analyse Alternativen auf KO-Kriterien}

\section{Daten Sammeln}
\subsection{\ref{Al-1}: Hetzner Server}
Hetzner ist einer der grössten Hosting Anbieter in Deutschland und bietet seit 1997 für Unternehmen und Privat Personen Hosting-Produkte an. In den Vergangenen Jahren hat Hetzer diverse Computer-Magazin Auszeichnungen bekommen. Hetzner betreibt in Deutschland mehre Rechenzentren die mehrfach Redundant an das Internet Angeschlossen sind.

\paragraph{Speicherkapazität}
Der Grösste Dezidierter Server welche Hetzner in seinen Produkte Katalog führt, ist der Root Server XS 29. Der XS 29 ist mit 15 mal 3 Terabyte SATA Festplatten ausgerüstet. Mit dem zusätzlich eingebauten Hardware RAID Kontroller lassen, sich die 15 Festplatten zu einen RAID zusammen fügen.
Als CPU hat der Server einen Intel Xeon E3-1245 Quad-Core eingebaut und Verfügt über 16 Gigabyte Hauptspeicher. 

Die Max Grösse einer Datei ist von Dateisystem abhängig. ext3 oder "third extended filesystem" genannt, ist bei den meisten bekannten Linux Distributoren das Standard Dateisystem. Die maximale grösse einer Datei hängt von der verwendeten Blockgrösse ab. Nach eigener Test ist die Standard Blockgrösse bei Debian, Ubuntu, Red Hat und Suse 4 Kibibyte, gross. Bei einer Blockgrösse von 4 Kibibyte, kann eine Datei maximal 2 Tebibyte und das Dateisystem 16 Tebibyte gross sein. \cite{Card1993}

Die Maximale Anzahl an Objekte, hängt von der grösse des Dateisystems ab. Bei einen 16 Tebibyte ist mit der Standard Konfiguration maximal 17'592'186'044'416 Objekte. 

\paragraph*{Verfügbarkeit}
Die RAID-5 Konfiguration mit dem 15 Festplatten im RAID bietet mit 38,192 Tebibyte gemäss \refeqlb{eqn:MaxSpeicherkapazitätHeztner} bei einfacher Redundanz die grösste mögliche Speicherkapazität. Wegen der möglicherweise langen Wiederherstellungs-Zeit (MTTR) dieser Konfiguration, ist man darauf angewiesen, dass der Ausfall durch die eigene Überwachungssystem erkannt und nach Benachrichtigung des Hetzner Support die defekte Festplatte rasch ausgetauscht wird. Die Festplatten selber sind während des Betriebs austauschbar, es ist somit kein Unterbruch des Betriebs erforderlich.

Der eingesetzte LSI RAID Kontrolle lässt auch die Konfiguration einer Hot-Spare Festplatte zu. Eine Host-Spare Festplatte ist eine leere ungenutzte Festplatte, die bei einen Ausfall einer Festplatte im RAID automatisch die defekte Festplatte ersetzt. 

Da es sich nur um ein Server System handelt, ist eine Standort-übergreifende Verfügbarkeit der Daten nicht möglich.


\begin{equation}
\mbox{Anzahl Server} = (15 -1)* 2,728 \mathrm{\ TiB}=  38,192 \mathrm{\ TiB}
\label{eqn:MaxSpeicherkapazitätHeztner}
\end{equation}

\begin{equation}
\mbox{Anzahl Server} = (15 -1-1)* 2,728 \mathrm{\ TiB}=  35,464 \mathrm{\ TiB}
\label{eqn:MaxSpeicherkapazitätHeztnerHotspare}
\end{equation}

\paragraph*{Datenzugriff}
Der Datenzugriff findet lokal über POSIX IO statt, dass heisst das die Web-Applikation auf dem selben Server betrieben wird. 

Der Lese Zugriff und Schreibzugriff ist auf den Server begrenzt. Theoretisch könnte der Speicher mit Protokollen wie NFS zwischen, mehreren Systemen geteilt werden. Es handelt sich jedoch beim System um einen Miete Server, es kann deshalb nicht davon ausgegangen werden, das weitere Miete Server am selben Netzwerk Knoten angeschlossen sind, oder sich sogar im gleichen Rechenzentrum befindet. Es muss aus diesen Gründen mit einer schlechten Bandbreite mit hoher Latenz zwischen den Systemen gerechnet werden, was die Teilung des Speichers aus Performance gründen unbefriedigend macht.

\subsubsection{Datenschutz}
Die Integrität der Daten wird nur auf RAID Ebene sichergestellt und nicht auf Objekt ebene. Die Selbstheilung von Objekten wird aus diesen Grund nicht unterstützt.

Die Daten können mit einen RSYNC Job in Gesichert werden, gegen zusätzliche Gebühre von 79 € für 10 GB ist die Sicherung auch eine Sicherung bei Hetzner möglich.

Der Datenzugriff lässt sich über die Dateiberechtigung im Dateisystem auf Benutzer und Gruppen ebene steuern. Vor den Physischen Datenzugriff, lässt sich durch Verschlüsselung der Disk schützen.

\subsubsection{Technologie}
Die Konfiguration und Betrieb des Server inklusive RAID ist dem Mieter überlassen. Die RAID Technologie ist eine viele eingesetzte und bewehrte Technologie.

Für die Konfiguration und Betrieb des Server, reichen Standard Linux Administration Wissen aus.

Die Verwaltung des Speicher erfolgt über Kommandozeile bzw. über \gls{SSH}.

\subsubsection{Kosten}
Abgesehen von den einmaligen Einrichtung Kosten, welche 499 € Betragen, und die Installation der Servers, gibt es keine weitere Investitionskosten. 

Für die Installation und Betrieb des gemieteten Server ist der Mieter selber Verantwortlich. Die Mietkosten betragen pro Monat 299 €.

Die Kosten für die Miete des Server bleiben während der gesamten Mietdauer konstant. Die Kündigung ist jeweils auf 30 Tage zum Monatsende möglich. 

\paragraph*{Kosten Szenario-1}
Die Kosten für Szenario-1 betragen gemäss Zusammenstellung der Tabelle (10.9) Total € 22'027.00 das sind zum aktuellen Tageskurs (13 April 2012) 26'464.31 CHF.

\begin{table}[htbp]
\caption{Kosten OpenStack S1}
\begin{small}
\begin{tabular}{|l|r|r|r|}
\hline
\textbf{Beschreibung} & \multicolumn{1}{l|}{\textbf{Kosten pro Stk/M.}} & \multicolumn{1}{l|}{\textbf{Anzahl}} & \multicolumn{1}{l|}{\textbf{Total}} \\ \hline
  \multicolumn{ 4}{c}{} \\  \hline
\multicolumn{ 4}{|c|}{\textbf{Investitionskosten}} \\ \hline
Einrichtung Root Server XS 29 & € 499.00 & 1 & € 499.00 \\ \hline \hline
  \multicolumn{ 3}{r|}{\textbf{Total:}}  & \textbf{€ 499.50} \\ 
  \cline{4-4}
\multicolumn{ 4}{c}{} \\   \hline
\multicolumn{ 4}{|c|}{\textbf{Fortlaufende Kosten}} \\ \hline
Root Server XS 29  & € 299.00 & 1 & € 299.00 \\ \hline \hline
  \multicolumn{ 3}{r|}{\textbf{Total pro Monat:}} & € 299.00 \\
\cline{4-4}
  \multicolumn{ 3}{r|}{\textbf{Total 36 Monate:}} & \textbf{€ 21'528.00} \\ \cline{4-4}
  \multicolumn{ 4}{c}{} \\  \cline{4-4}
  \multicolumn{ 3}{r|}{\textbf{Total Gesamt:}} & \textbf{€ 22'027.00} \\ \cline{4-4}
\end{tabular}
\end{small}
\label{KostenOpenStackS1}
\end{table}


\paragraph*{Kosten Szenario-2}
Kein Produkt erhältlich welche die Anforderungen der Speicherkapazität von Szenario-2 erfüllen.

\subsection{\ref{Al-2}: Netapp}

Wie in der Markt Analyse erwähnt ist NetApp der Führende Anbieter in midtrange und Highen

Die Firma NetApp beschäftigt in  der Schweiz an ihren drei Standorten, Wallisellen, Lausanne und Bern ca. 80 Mitarbeiter. Zudem verfügt NetApp Schweiz über ein gut ausgebautes Partner Netzwerk, mit welche Sie die ganze Schweiz gut verankert sind. Mit FastLane und QSkills sind zwei Schulungspartner im Raum Zürich verfügbar, welche offizielle Kurse und Zertifizierenden für NetApp Produkte anbieten. 

Neben der Allgemeinen Garantie Gewährleistungen, bietet NetApp weiteren kostenpflichtige Support Leistungen an. So steht mit dem SupportEdge Premium, den Remote Support mit einer Reaktionszeit von minimal 30 Minuten rund um die Uhr zur Verfügung, oder den die Installation von Ersatzteilen mit einer Reaktionszeit von minimal 2 Stunden rund um die Uhr zur Verfügung. 

Für den Support und Wartungen stehen mehre Produkte bereit von Verlängerung der Hardware-Garantie, Auto-Support mit oder ohne Austausch der Hardware durch NetApp.


Für die Verwaltung der NetApp stehen, neben eines Kommandzeilen, Web-Interface weitere teilweise Kostenpflichtige Lösungen von NetApp zu Verfügung. 

Die Netapp hat mehre Mechanismen für die Sicherstellung der Daten Verfügbarkeit. So werden die Festplatten mittels RAID-DB System zusammen gefasst. RAID-DP ist eines  auf Basis von RAID-4 von NetApp weiter entwickeltes RAID. Bei RAID-DP wird jedoch im Unterschied zu RAID-4 eine weiter Paritäts-Festplatte eingesetzt. Wie im \refabb{abb:RAID-DP} ersichtlich, gibt es eine Horizontale Parität und eine Diagonale Parität, die diagonale Parität ist durch die Farben dargestellt und auf der Festplatte DP gespeichert, die horizontale ist auf der Festplatte P gespeichert, die Festplatten mit D sind normale RAID Daten Festplatten. Die Doppelte Parität hat den Vorteil, dass die Verfügbarkeit des RAID erhöht wird. So kann bei einen RAID-DP gleichzeitig zwei Festplatten im RAID zur gleichen Zeit ausfallen, ohne das diese zu einen Datenverlust führt. Im Vergleich dazu kann RAID-4 oder RAID-5 nur den Ausfall einer Festplatte kompensieren.\cite{White2010}

\begin{center}
\includegraphics[ keepaspectratio = true]{media/raid-dp.png}
\mycaption{figure}{\label{abb:RAID-DP}NetApp RAID-DP doppelte Parität\cite{White2010}}
\end{center}


Als weiteren Schutzmassnahme vor Datenverluste, empfiehlt NetApp den Einsatz von Spare-Festplatten. Durch den Einsatz von Spare-Festplatten kann die Wiederherstellungszeit MTTR verkleinert werden, da die Wiederherstellung automatisch starten kann. Die Anzahl an Spare-Festplatten ist abhängig von der Anzahl Festplatten und Enclosure.

Die Datenverfügbarkeit kann mittels weiteren NetApp und der Snapmirror Funktion weiter erhöht werden. Dabei werden alle Änderungen an den Daten auf an eine weitere NetApp Synchronisiert. 

NetApp Speichert die Daten auf die Disk in 4 Kilo Byte Blocks. Zu jeden 4 Kilo Byte Block berechnet NetApp eine Prüfsumme und speichert diese in die Block Metadaten. Wenn der Block zu einen späteren Zeitpunkt wieder gelesen ist, berechnet die NetApp die Prüfsumme erneut und vergleicht diese mit der Gespeicherten Prüfsumme. Wenn die Prüfsumme nicht übereinstimmt, wird der Block mittels den Paritäts-Daten neu geschrieben, erneut gelesen und geprüft. Um die Integrität von Daten zu gewährleisten auf welche über eine langen Zeitraum nicht zugegriffen wird, wie diese zum Beispiel bei Archive Daten der Fall ist, hat NetApp eine konfigurierbare RAID durchkämmen (engl. RAID scrub) Funktion. Die Funktion geht durch alle Daten durch, wodurch die Prüfsumme von allen gespeicherten Daten geprüft wird. Die Funktion kann Zeitgesteuert oder wenn das System im Leerlauf befindet.\cite{Sundaram2006}

Zu beachten ist, dass die Prüfsumme auf dem Block nur auf der Speicherebene Wirkung hat. Fehler die in einer höheren Ebene stattfindet, wie zum Beispiel beim Einsatz von iSCSI zusammen mit einen Dateisystem, könnte bereits einen Fehler im Dateisystem durch einen Software Fehler oder Memory Fehler statt finden. 

NetApp bietet mehre Möglichkeiten eine Sicherungskopie der Daten herzustellen. Beim Einsatz von NFS oder iSCSI können die Daten über die Web-Servern mit einer Handels üblichen Sicherungs-Software gesichert werden. Bei NFS wird dazu die angefügten NFS Freigaben gesichert, bei iSCSI werden die Daten wie bei allen anderen Dateisystemen gesichert. Zubeachten ist jedoch, bei diesen verfahren, dass die Daten von der NetApp über den Server transferiert werden müssen, und dadurch der Web-Server mit dem Sicherungslast belastet wird. Neben der Sicherung der NetApp über den Server lässt sich die NetApp auch direkt sichern, dabei gibt es zwei Verfahren, mittels  Network Data Management Protocol (kurz NDMP) oder Mittels Snapshots. 

NDMP ist eines von NetApp mitentwickeltes Protokoll, welches für die Sicherung von NAS Geräte entwickelt wurde. Das Problem bei NAS Geräte ist, dass auf Ihnen ein dezidierte Betriebsystem läuft auf welche keine Installation eines Sicherungs-Agenten erlaubt, aus diesen Grund wurde das NDMP Protokoll entwickelt um eine allgemeines Agenten-Loses Sicherungsverfahren für NAS Geräte zu ermöglichen. NDMP wurde dem IETF im Jahr 2000 von der NDMP Initianden als Entwurf eingereicht, bis anhin ist jedoch noch kein RFC für NDMP Standardisiert. Trotzdem wird es von vielen NAS-Geräte- und Sicherungs-Software- Herstellern unterstützt. Eine Liste mit Sicherungs-Software Produkten, welche das NDMP Verfahren unterstützen gibt es auf der Webseite der NDMP Initianden\footnote{\url{http://www.ndmp.org/products/index.shtml#backup}} zu finden.\cite{NDMP.orga}\cite{NDMP.org}

Beim Snapshots Sicherungsverfahren werden zeitbezogene Sicherung des Dateisystem  erstellt. Bei den Snapshots wird nicht eine eigentliche Kopie der Daten bzw. Blöcke erstellt sondern die Referenz auf die Blöcke gespeichert. Wird nach dem erstellen eines Snapshots ein Block geändert wird die Änderung nicht im Original Block vorgenommen sondern in einen neuen Block und diesen Referenziert. Dadurch benötigt ein Snapshot minimalen Speicher, zudem lassen sich die Snapshots und damit die Sicherung in Sekunden erstellen unabhängig von der Anzahl Dateien oder der Verwendete Speicherkapazität. Durch den Einsatz von zwei NetApp Systemen und der SnapMirror Funktion kann, das ganze Dateisystem bzw. Volume inklusive aller Snapshots auf der zweiten NetApp gesichert werden. Die Sicherung mit Snapshots hat bei grossen Datenmengen entscheidende Vorteile, die Sicherung benötigt minimalen Speicherplatz, die Sicherung und Wiederherstellung ist innert Sekunden erstellt bzw. wiederhergestellt. Nachteile sind, dass der Speicherplatz mit der Sicherung bzw. Snapshots geteilt werden muss und das löschen der Daten zusätzlichen Speicherplatz verbraucht.

Sicherheit NFS IPSEC, intern



\subsection{\ref{Al-4}: OpenStack Objekt Storage}
OpenStack Object Storage ist eine Quelloffene frei verfügbare Verteilter Speicherlösung. OpenStack Object Storage selbst ist kein Dateisystem sondern eher eine Speicher Cluster, welche primär für die Speicherung von Virtual Maschine Images, Bilder, E-Mail und Sicherungskopien vorgesehen ist. Für einen OpenStack Cluster kommen vergleichbar mit dem Google Filesystem, kosten günstige Konventionelle Computer Hardware zum Einsatz. 


\paragraph{Verfügbarkeit}
OpenStack Object Storage wurde so konzipiert das einen Ausfall einer Hardware Komponente, der weiter Betrieb weiter aufrecht erhalten werden kann ohne dabei einen Datenverlust oder die Verfügbarkeit des System zu gefährden. Die Daten bzw. Objekte werden Redundant auf mehren Zonen gespeichert. Zonen sind Gruppierungen von Servern. Die Anzahl redundanten Replikate für eine Objekt kann konfiguriert werden, gewöhnlich werden drei Kopien verwendet.

Für die Architekt wird von Joe Arnolds CEO bei SwiftStack bei dreifacher Redundanz, die Implementierung von 5 Zone empfohlen. OpenStack Object Storage, speichert die Redundante Objekte jeweils in einer separaten Zonen. Eine Zone kann bei kleinen Implementationen auch nur aus Gruppen von Festplatten bestehen, also innerhalb eines einzigen Server existieren, es wird aber empfohlen eine Zone auf Hierarchie von Server-Gruppen zu implementieren. Das bedeutet jede Redundante Kopie auf einen eigenen Server gespeichert. \cite{Arnold} 

Zusätzlich zu den Speicherserver benötigt es noch zwei oder mehrere Proxy Server um die hohe Verfügbarkeit zu gewährleisten. Die Proxy Server, werden bei Rackspace mit  10 Gb Ethernet an den Netzwerk-Switch angeschlossen werden. \cite{OpenStack2011}

Möchte man den Speicher Standort übergreifend betreiben, ist es erforderlich, dass pro Standort nicht mehr als $(Redundanz -1)$ Zonen betrieben werden, da ansonsten die Gefahr besteht das alle Replikate am gleichen Standort gespeichert werden.

\paragraph*{Speicherkapazität}
Der OpenStack Object Storage Cluster lässt sich im laufenden Betreib ohne Unterbruch, vergrössern. In der Regeln können auch Software Update und Patches ohne Unterbruch des Clusters eingespielt werden, wenn darauf geachtet wird nie mehr als eine Zone gleichzeitig zu aktualisieren.

Grundsätzlich ist Anzahl an Speicherbaren Objekten nicht begrenzt. Der Container Server von OpenStack Object Storage, welcher primärer Aufgabe ist die Auflistung der gespeicherten Objekten in einen Container zu handhaben, speichert die Informationen welche Objekte im Container befinden pro Container in eine SQLite Datenbank. Die Performance der Datenbank, welche mit der Anzahl Einträge abnimmt, kann sich zum Limitierenden Faktor entwickeln wie viele Objekte in einen Container gespeichert werden können. Der Performance Einbruch kann, je nach Leistung der Hardware und die generelle Last auf dem Cluster, bei einer Million Objekte pro Container eintreten. Es können jedoch pro Account mehre Container erstellt werden, wodurch sich das Problem mit der Limitierung durch Datenbank entschärfen lässt. Zu beachten ist hier jedoch, das pro Account ebenfalls eine SQLite Datenbank geführt wird mit allen Container und hier die selben Limitierung zum tragen kommt.\cite{OpenStack2012}\cite{A2011}

\paragraph*{Datenzugriff}
Der Zugriff auf die gespeicherten Objekten erfolgt vergleichbar wie bei Amazon S3 über ein API. Das API verwendet für den Zugriff das Standard HTTP Request. Mehre Speicherlösung Hersteller haben angekündigt, das Swift API ebenfalls für den Zugriff auf Ihren Speicher unterstützen zu wollen. Glusterfs ebenfalls eine Verteile Speicherlösung, bietet in Ihrer Entwickler Version bereits die Unterstützung für Swift an. Applikationen welche, das Swift API unterstützen können Ihren Speicherlösung zukünftig einfacher austauschen ohne dabei die Applikation anpassen zu müssen. Neben den Zugriff mit dem Swift API, bietet OpenStack eine Amazon S3 kompatibles API an. Der Zugriff über POSIX IO auf den Speicher ist nicht möglich.

Der Zugriff auf den Speicher erfolgt über einen Proxy welche, den Speicherort des Objektes prüft und die Anfrage weiterleitet. Pro OpenStack Object Storage Cluster können mehre Proxy eingesetzt werden. Mit Hilfe eines Load Balancers (Last Verteiler) können die Anfragen über die vorhanden Proxy verteilt werden.

Die Maximale Upload Grösse beträgt 5 GB, die Download Grösse selber ist jedoch unbeschränkt. Wenn ein Objekt welches grösser als 5 GB hochgeladen wird, wird das Objekt in Teilen hochgeladen und gespeichert, die Abfrage auf das Objekte erfolgt jedoch wie bei einen ganzen Objekt, es müssen also keine Teile durch das API Adressiert werden.\cite{OpenStack2012a}


\paragraph*{Datenschutz}
OpenStack Object Storage stellt die Datenintegrität beim Speichern von Objekten in den Speicher und beim auslesen der Daten sicher. Somit ist gewährleistet, dass die Daten im Original gespeichert werden und bei der Speicherung kein Datenverlust entsteht. Die Selbstheilung von korrupten Objekten, stellt OpenStack Objekt Storage mit dem Auditors Dienst sicher. Der Auditors Dienst läuft auf jeden Server und prüft dort die Integrität von Objekten, wenn ein korruptes Objekt festgestellt wird, wird die Datei unter Quarantäne gestellt und durch ein korrektes reblikations Objekt ersetzt.
Um die Last die durch den Auditors möglichst gering zu halten, kann die Anzahl Objekte oder die Anzahl Bits die pro Sekunde geprüft werden konfiguriert werden.\cite{OpenStack2012}

OpenStack Objekt Storage bietet neben der Redundanz der Objekte keine Möglichkeit, die Daten zu Sichern. Dadurch ist keine Versionierung der Objekte möglich. Zwar lässt sich die Cluster Nodes mit einen konventionellen Sicherungssoftware direkt sichern, jedoch wurde sich nur der gesamte Cluster wiederherstellen und nicht einzelne Objekte. Zudem wäre die Speicherkapazität pro Sicherung, so gross wie es für die redundante Speicherung der Objekte Speicherkapazität benötigt wird.
\cite{AndyBrezinsky2011}


Die Sicherheit der Daten wird mit sogenannten Accounts Sichergestellt, dass heist das ein Account nur auf seine gespeicherten Objekten zugreifen kann. Die Daten werden auf dem OpenStack Cluster Unverschlüsselt in Binären form gespeichert. Durch die Verteilung der Daten müsste jedoch eine Angreifer Zugriff auf alle Cluster Node erlangen um an alle Daten eines Account zu gelangen.

\paragraph*{Technologie}
OpenStack Objekt Storage wurde von der Firma RackSpace unter dem Code Name Swift entwickelt. Zusammen mit National Aeronautics and Space Administration (kurz. NASA) hat RackSpace die Softwareprojekt-Gemeinschaft OpenStack gegründet und Swift zusammen mit den Cloud Lösungen Nova für die Verwaltung von Virutellen Maschinen und Glace für die Verwaltung von Image als Quelloffene Software veröffentlicht. Die OpenStack Gemeinschaft ist in Vergleich zu anderen Quelloffenen Gemeinschaften noch eine relative jung Gemeinschaft, es gelangt Ihr dennoch neben den beiden Gründer Unternehmen, weiter bedeutende Unternehmen aus der Informatik Branche, wie Hewlett Packard, Dell, Citrix, Intel, AMD, NetApp und weitere zu gewinnen. Eine vollständige Liste aller Unterstützende  Unternehmen ist unter \url{http://openstack.org/community/companies/} zu finden. Die Aktivität auf der Projekt Seite von OpenStack Object Storage, lässt erkennen, dass die Entwicklung aktive vorangetrieben wird und viele Beiträge zur neuen Funktionen oder Verbesserungen von Anwender und Entwickler bei gesteuert werden. \cite{Ohloh2012}

Die Internet-Recherche auf OpenStack spezialisierte Firmen mit Standort Schweiz ergaben keine Ergebnisse. Einige
Gemäss Internet Recherchen gibt es noch keine spezialisierte Firma für OpenStack Lösungen. Einige wenige Experten für OpenStack mit Domizil in der Schweiz konnten über die Profile bei LinkedIn einen Soziales Netzwerk für Geschäft gefunden werden. 

Die Verwaltung des Speichers bietet OpenStack ein Kommandozeilen Tools an. Ein Web-GUI für Objekt Storage von OpenStack gibt es nicht. Die Firma SwiftStack bietet hierfür jedoch Kostenpflichtige Lösungen an. Die Überwachung des Speichers, zum Beispiel die Verfügbarkeit einzelner Cluster Nodes, muss jedoch mit zusätzlichen Fremdlösungen erfolgen. 

\subsubsection{Kosten}
Grundsätzlich unterscheidet sich die Kosten für die beiden Szenario nur in der Anzahl Server, den Verwendeten Server Typen, in der Anzahl Festplatten und den benötigten Rack Platz. Die restlichen Faktoren sind für beide Szenerien die selben und werden in diesen Abschnitt behandelt.

Um die Kosten der Rechenzentrumskosten möglichst tief zu halten, welche pro Rack verrechnet werden, soll möglichst viel Speicherkapazität in eine Rackeinheit gepackt werden. Dass bedeutet, dass die Richtigen Server für die Anzahl zur Verfügung gestellten Speicherkapazität verwendet werden soll.

Um die Preise für die Festplatten möglichst tief zu halten werden diese separate beschaffen. Als Festplatte soll eine SATA-3 3.5 Zoll 3 Terabyte Festplatte eingesetzt werden, diese entspricht 2.728 Tebibyte. Die Kosten für Festplatten sind stark Marktabhängig, für die Arbeit wird deshalb der einfachhalber mit ca. 300 CHF pro Festplatte gerechnet.

Für beide Szenerien werden die gleichen Proxy-Server verwendet. Als Proxy Server wurden die Transtec Server-Systeme "'Lynx CALLEO Application Server 2260S'" ausgewählt. Die Proxy Server stammen von selben Hersteller wie die in den Szenerien aufgeführten Datenserver, diese hat den Vorteil, dass nur ein Ansprechpartner für die Server Hardware gibt. Es ist jedoch kein muss Kriterium und kann bei bedarf auch durch eine besseres Angebot eines anderen Hersteller ersetzt werden. Die Proxy sind mit einen Intel Xeon Prozessor E5-2665 mit 64 Gigabyte Hauptspeicher ausgerüstet. Die Server haben eine Rack-Höheneinheit von je 2 Einheiten. Wie bei den Datenserver wurde ein 365 mal 24 Stunden vor Ort Service ausgewählt. Der Preis pro Server beträgt 9'954 CHF.

Neben der Server Infrastruktur sind noch zwei Netzwerk-Switches notwendig. Der Dell PowerConnect 8024 bietet 24 Port mit 10GbE. Die Switch haben eine Rack-Höheneinheit von je 1 Einheiten und benötigen gemäss Hersteller maximal 237,77 Watt. Dell gewährt einen 3 Jahres Support für den Switch. Der Preis pro Switch beträgt 14'732 CHF.

Als Rechenzentrum Betreiber wurde Nine Internet Solution ausgewählt. Nine Betreibt seine Rechenzentrum in Zürich und ist auf die Hosting Spezialisiert. Mit seinen Rechenzentrum Standort in der Stadt Zürich, befindet sich Nine Geographisch in der Nähe des Auftraggeber. Die Anfahrtszeit und Anfahrtsweg kann so für den Auftraggeber kurz gehalten werden. Die Verschiedenen Rack Produkte welche Nine anbietet, unterscheiden sich von der Anzahl beinhalteten Rack-Höheneinheiten und deren beinhalteten Watt Leistung. In den Produkten ist zusätzlich die Anbindung ans  Internet Anbindung mit 100 Mbps beinhaltet. Bei Zusätzlichen Watt bedarf, bietet Nine die Möglichkeit die Zusätliche Watt in 500 Pakete zu 165 CHF pro Monat an.

Für den den Betrieb der Speicherinfrastruktur wird mit einer Mann Stelle gerechnet. Organisatorisch kann diese auch auf mehre Prozentual Stellen aufgeteilt werden, somit ist auch der Betreib während Ferien Abwesenheit geregelt. Pro Monat wird mit Personal Kosten von 15'000 CHF gerechnet.


\paragraph*{Kosten Szenario-1}

Für die dreifache Redundanz sind für die aus der Soll-Analyse bestimmten 306 Tebibyte gemäss \refeqlb{eqn:SpeicherkapazitätS1} 918 Tebibyte an Speicherkapazität notwendig.

\begin{equation}
\mbox{Speicherkapazität}_{Redundanz} = 306  \mathrm{\ TiB} * 3 \mbox{\ Redundanz} =  918 \mathrm{\ TiB}
\label{eqn:SpeicherkapazitätS1}
\end{equation}


Als Server System wurde der Transtec Lynx CALLEO Application Server 1260 gewählt. Der Server basiert auf Server Hardware des Herstellers  SuperMicro. Der Server benötigt 1 Rack-Höcheneinheiten Rack platz und bieten Platz für den Einbau von 4 3.5 Zoll Festplatten, welche auch während des Betriebs ausgetauscht werden können. Als Prozessoreinheit kommen ein Intel Xeon Prozessor E5-2620 zum Einsatz, zudem verfügt der Server über 32 Gigabyte Hauptspeicher. Der Server wurde mit einen 3 Jahres, 24 Stunden, 365 Tage im Jahr Vor-Ort Service ausgestattet. Somit ist sichergestellt, das im Störungsfall der Unterbruch möglichst gering gehalten werden kann. Der Preis pro Server beträgt ohne Festplatten 1'260 CHF.


Ein Server welche voll Ausgerüstet mit Festplatten, weist gemäss der  \refeqlb{eqn:SpeicherkapazitätServerS1} eine Speicherkapazität von 10,912 Tebibyte aus. Damit sind in Anbetracht der notwendigen Speicherkapazität von 48,3 Tebibyte und der 5 Zonen insgesamt 5 Server notwendig (siehe \refeqlb{eqn:AnzahlServerS1}). Für die Erreichung der Speicherkapazität von 48,3 Tebibyte bei gleich ausgerüsteten Server sind jedoch insgesamt nur 20 Festplatten notwendig.

\begin{equation}
\mbox{Speicherkapazität}_{Server} = 2,728 \mathrm{\ TiB} * 4 \mbox{\ Einschub} =  10,912 \mathrm{\ TiB}
\label{eqn:SpeicherkapazitätServerS1}
\end{equation}

\begin{equation}
\mbox{Anzahl Server} = \frac{48,3 \mathrm{\ TiB}}{10,912 \mathrm{\ TiB}} \approx  5 \mbox{\ (Auf 5 Zonen gerundet)}
\label{eqn:AnzahlServerS1}
\end{equation}

\begin{equation}
\mbox{Anzahl Festplatten} = \frac{48,3 \mathrm{\ TiB}}{2,728 \mathrm{\ TiB}} \approx  20 \mbox{\ (Auf 5 Server gerundet)}
\label{eqn:AnzahlServerS1}
\end{equation}

Gemäss \refeqlb{eqn:AnzahlRackS1} wird ein viertel Rack für den Einbau aller Komponenten. Der Benötigte Rack Platz entspricht den Quarter Rack Angebot von Nine, welches 11 Rack-Höheneinheiten platz bietet. Der Preis pro Rack beträgt 450 CHF pro Monat, plus 650 CHF für die einmalige Einrichtung des Racks durch Nine.
Für die Berechnung der Gesamt Watt Leistung welcher der Server im Betrieb benötigt, fehlen die Angaben des Hersteller. Aus diesen Grund wird angenommen, dass der Daten-Server ca. 300 Watt und der Proxy-Server 500 Watt bezieht. Mit diesen Annahmen benötigt es gemäss  \refeqlb{eqn:AnzahlWattPaketeS1}, den 500 Watt welche im Rack beinhaltet sind, zusätzliche 5 Pakete a 500 Watt pro Monat.

\begin{equation}
\mbox{Anzahl Racks} = \frac{5 * 1 \mathrm{\ U} + 2 * 2 \mathrm{\ U} + 2 * 2 \mathrm{\ U}}{11\mathrm{\ U}} \approx  1 \mbox{\ (Auf viertel Rack gerundet)}
\label{eqn:AnzahlRackS1}
\end{equation}

\begin{equation}
\mbox{Anzahl 500 \mathrm{W} Pakete} = \frac{5 * 300 \mathrm{\ W} + 2 * 500 \mathrm{\ W} +2 * 237,77 \mathrm{\ W} - 2000 \mathrm{\ W} }{500\mathrm{\ W}} \approx  5
\label{eqn:AnzahlWattPaketeS1}
\end{equation}

Wie aus der \reftab{tab:KostenOpenStackS1} ersichtlich ist, betragen die Gesamtkosten 653'646.50 CHF.


\begin{table}[htbp]
\caption{Kosten OpenStack S1}
\begin{small}
\begin{tabular}{|l|r|r|r|}
\hline
\textbf{Beschreibung} & \multicolumn{1}{l|}{\textbf{Kosten pro Stk/M.}} & \multicolumn{1}{l|}{\textbf{Anzahl}} & \multicolumn{1}{l|}{\textbf{Total}} \\ \hline
  \multicolumn{ 4}{c}{} \\  \hline
\multicolumn{ 4}{|c|}{\textbf{Investitionskosten}} \\ \hline
Dell PowerConnect 8024 & CHF 14'732.00 & 2 & CHF 29'464.00 \\ \hline
Lynx CALLEO App. Server 1260 & CHF 5'451.50 & 2 & CHF 10'903.00 \\ \hline
Lynx CALLEO App. Server 1260 & CHF 2'831.30 & 5 & CHF 14'156.50 \\ \hline
Festplatte 3,5 3 TB & CHF 300.00 & 20 & CHF 6'000.00 \\ \hline
Rack Einrichtung & CHF 650.00 & 1 & CHF 650.00 \\ \hline \hline
  \multicolumn{ 3}{r|}{\textbf{Total:}}  & \textbf{CHF 61'173.50} \\ 
  \cline{4-4}
\multicolumn{ 4}{c}{} \\   \hline
\multicolumn{ 4}{|c|}{\textbf{Fortlaufende Kosten}} \\ \hline
Dell Service  & CHF 0.00 & 2 & CHF 0.00 \\ \hline
Transtec Service (Server 1260) & CHF 26.08 & 2 & CHF 52.17 \\ \hline
Transtec Service (Server 1260) & CHF 26.08 & 5 & CHF 130.42 \\ \hline
Rackkosten 11 U (500 W) & CHF 450.00 & 1 & CHF 450.00 \\ \hline
Strom 500 W & CHF 165.00 & 5 & CHF 825.00 \\ \hline
Personal & CHF 15'000.00 & 1 & CHF 15'000.00 \\ \hline \hline
  \multicolumn{ 3}{r|}{\textbf{Total pro Monat:}} & CHF 16'457.58 \\
\cline{4-4}
  \multicolumn{ 3}{r|}{\textbf{Total 36 Monate:}} & \textbf{CHF 592'473.00} \\ \cline{4-4}
  \multicolumn{ 4}{c}{} \\  \cline{4-4}
  \multicolumn{ 3}{r|}{\textbf{Total Gesamt:}} & \textbf{CHF 653'646.50} \\ \cline{4-4}
\end{tabular}
\end{small}
\label{KostenOpenStackS1}
\end{table}



\subsubsection{Kosten Szenario-2}

Für die dreifache Redundanz sind für die aus der Soll-Analyse bestimmten 306 Tebibyte gemäss \refeqlb{eqn:SpeicherkapazitätS2} 918 Tebibyte an Speicherkapazität notwendig.

\begin{equation}
\mbox{Speicherkapazität}_{Redundanz} = 306  \mathrm{\ TiB} * 3 \mbox{\ Redundanz} =  918 \mathrm{\ TiB}
\label{eqn:SpeicherkapazitätS2}
\end{equation}


Als Server System wurde der Transtec Lynx CALLEO Application Server 4260 gewählt. Der Server basiert auf Server Hardware des Herstellers  SuperMicro, welche gemäss eigenen Recherchen auch bei anderen OpenStack Lösungen verwendet wird. Der Server benötigt 4 Rack-Höcheneinheiten Rack platz und bieten Platz für den Einbau von 36 3.5 Zoll Festplatten, welche auch während des Betriebs ausgetauscht werden können. Als Prozessoreinheit kommen zwei Intel Xeon Prozessor E5-2620 zum Einsatz, zudem verfügt der Server über 64 Gigabyte Hauptspeicher. Der Server wurde mit einen 3 Jahres, 24 Stunden, 365 Tage im Jahr Vor-Ort Service ausgestattet. Somit ist sichergestellt, das im Störungsfall der Unterbruch möglichst gering gehalten werden kann. Der Preis pro Server beträgt ohne Festplatten 8'672 CHF.


Ein Server welche voll Ausgerüstet mit Festplatten, weist gemäss der  \refeqlb{eqn:SpeicherkapazitätServerS2} eine Speicherkapazität von 98.208 Tebibyte aus. Damit sind in Anbetracht der notwendigen Speicherkapazität von 918 Tebibyte und der 5 Zonen insgesamt 10 Server notwendig (siehe \refeqlb{eqn:AnzahlServerS2}). Für die Erreichung der Speicherkapazität von 918 Tebibyte bei gleich ausgerüsteten Server sind jedoch insgesamt nur 340 Festplatten notwendig, dass heisst pro Server sind 34 Einschub mit Festplatten belegt. Sie könnten also bei Bedarf noch weiter aufgerüstet werden.

\begin{equation}
\mbox{Speicherkapazität}_{Server} = 2,728 \mathrm{\ TiB} * 36 \mbox{\ Einschub} =  98.208 \mathrm{\ TiB}
\label{eqn:SpeicherkapazitätServerS2}
\end{equation}

\begin{equation}
\mbox{Anzahl Server} = \frac{918 \mathrm{\ TiB}}{98.208 \mathrm{\ TiB}} \approx  10 \mbox{\ (Auf 5 Zonen gerundet)}
\label{eqn:AnzahlServerS2}
\end{equation}

\begin{equation}
\mbox{Anzahl Festplatten} = \frac{918 \mathrm{\ TiB}}{2,728 \mathrm{\ TiB}} \approx  340 \mbox{\ (Auf 10 Server gerundet)}
\label{eqn:AnzahlServerS2}
\end{equation}

Gemäss \refeqlb{eqn:AnzahlRackS2} wird ein ganzes Rack für den Einbau aller Komponenten. Der Benötigte Rack Platz entspricht den Full Rack Angebot von Nine, welches 47 Rack-Höheneinheiten platz bietet. Der Preis pro Rack beträgt 1450 CHF pro Monat, plus 1500 CHF für die einmalige Einrichtung des Racks durch Nine.
Für die Berechnung der Gesamt Watt Leistung welcher der Server im Betrieb benötigt, fehlen die Angaben des Hersteller. Aus diesen Grund wird angenommen, dass der Daten-Server ca. 700 Watt und der Proxy-Server 500 Watt bezieht. Mit diesen Annahmen benötigt es gemäss  \refeqlb{eqn:AnzahlWattPaketeS2}, den 2000 Watt welche im Rack beinhaltet sind, zusätliche 13 Pakete a 500 Watt pro Monat.

\begin{equation}
\mbox{Anzahl Racks} = \frac{10 * 4 \mathrm{\ U} + 2 * 2 \mathrm{\ U} + 2 * 2 \mathrm{\ U}}{47\mathrm{\ U}} \approx  1 \mbox{\ (Auf ganze Rack gerundet)}
\label{eqn:AnzahlRackS2}
\end{equation}

\begin{equation}
\mbox{Anzahl 500 \mathrm{W} Pakete} = \frac{10 * 700 \mathrm{\ W} + 2 * 500 \mathrm{\ W} +2 * 237,77 \mathrm{\ W} - 2000 \mathrm{\ W} }{500\mathrm{\ W}} \approx  13
\label{eqn:AnzahlWattPaketeS2}
\end{equation}

Wie aus der \reftab{tab:KostenOpenStackS1} ersichtlich ist, betragen die Gesamtkosten 893'965 CHF.

\begin{table}[htbp]
\caption{Kosten OpenStack}
\begin{small}
\begin{tabular}{|l|r|r|r|}
\hline
\textbf{Beschreibung} & \multicolumn{1}{l|}{\textbf{Kosten pro Stk/M.}} & \multicolumn{1}{l|}{\textbf{Anzahl}} & \multicolumn{1}{l|}{\textbf{Total}} \\ \hline
  \multicolumn{ 4}{c}{} \\  \hline
\multicolumn{ 4}{|c|}{\textbf{Investitionskosten}} \\ \hline
Dell PowerConnect 8024 & CHF 14'732.00 & 2 & CHF 29'464.00 \\ \hline
Lynx CALLEO App. Server 1260 & CHF 5'451.50 & 2 & CHF 10'903.00 \\ \hline
Lynx CALLEO App. Server 4260H & CHF 6'111.00 & 10 & CHF 61'110.00 \\ \hline
Festplatte 3,5 3 TB & CHF 300.00 & 340 & CHF 102'000.00 \\ \hline \hline
  \multicolumn{ 3}{r|}{\textbf{Total:}} & \textbf{CHF 203'477.00} \\ \cline{4-4}
\multicolumn{ 4}{c}{} \\   \hline
\multicolumn{ 4}{|c|}{\textbf{Fortlaufende Kosten}} \\ \hline
Dell Service  & CHF 0.00 & 2 & CHF 0.00 \\ \hline
Transtec Service (Server 1260) & CHF 26.08 & 2 & CHF 52.17 \\ \hline
Transtec Service (Server 4260H) & CHF 53.31 & 10 & CHF 533.06 \\ \hline
 Rackkosten 47 U (2000 W) & CHF 1'450.00 & 1 & CHF 1'450.00 \\ \hline
Strom 500 W  & CHF 165.00 & 13 & CHF 2'145.00 \\ \hline
Personal & CHF 15'000.00 & 1 & CHF 15'000.00 \\ \hline \hline
  \multicolumn{ 3}{r|}{\textbf{Total pro Monat:}} & CHF 19'180.22 \\ \cline{4-4}
  \multicolumn{ 3}{r|}{\textbf{Total 36 Monate:}} & \textbf{CHF 690'488.00} \\ \cline{4-4}
  \multicolumn{ 4}{c}{} \\  \cline{4-4}
  \multicolumn{ 3}{r|}{\textbf{Total Gesamt:}} & \textbf{CHF 893'965.00} \\ \cline{4-4}
\end{tabular}
\end{small}
\label{tab:KostenOpenStackS2}
\end{table}


\subsection{\ref{Al-5}: Amazon S3}
Amazon zählt zu den Grössten, wenn nicht der grösste Online Speicheranbieter weltweit. Genaue Anzahlen über die Anzahl Kunden und Speicherkapazität veröffentlich Amazon nicht. Seit 2006 bietet Amazon Ihren Kunden unter den Produktname S3, einen Online Speicher an. Die genaueren Technischen Eigenschaften und Architektur des Online Speicher ist bis anhin von Amazon nicht veröffentlicht worden. Nach eigenen Angaben von Amazon basiert die Dienstleistung auf gewöhnlichen Computer Hardware.






Bei Online Speicher wie Amazon S3, kann der Kunde gleich nach Anmeldung an den Dienst, die Speicherressourcen verwenden. Es fallen also keine Installation von für
die Speicherinfrastruktur an. 

\paragraph*{Verfügbarkeit}
Amazon S3 Speichert die Daten auf mehren Geräten in Verschiedenen Rechenzentren in der selben Region ab. Bei speichern werden die Daten Synchron in Verschiedene Rechenzentren gespeichert bevor die Speicherung als erfolgreich gemeldet ist, damit ist sicher gestellt, dass die Daten von beginn weg Redundant gespeichert sind. Mit diesen Massnahmen gewährleistet Amazon gemäss Dienstgütevereinbarung eine Zuverlässigkeit von 99.999999999\%. Bei bedarf kann die Zuverlässigkeit für Daten welche wenig Schutz benötigen, wie zum Beispiel Thumbnails von Bilder, mit einer geringeren Zuverlässigkeit von 99.99\% zu einen günstigeren Preis gespeichert werden.\cite{Amazon2007}

Durch die Redundante Speicherung der Daten auf mehre Geräten gewährleistet Amazone gemäss Dienstgütevereinbarung von 99.99\% pro Monat

\paragraph*{Datenzugriff}
Der Zugriff auf die Daten findet über eine API oder über die Verwaltungskonsole von Amazon statt. Für den Zugriff mit einen API bietet Amazon ein \gls{REST} und ein \gls{SOAP} Schnittstelle an. Der \gls{REST} Zugriff findet über HTTP statt, dabei werden Standard HTTP requests verwendet. Für den Zugriff kann auch einen gewöhnlichen Web-Browser verwendet werden solange die Objekte Öffentlich sind.

Einen Offiziellen POSIX IO Zugriff von Amazon steht nicht zur Verfügung, es existiert jedoch eine Quelloffenes Projekt Namens s3fs\footnote{\url{https://code.google.com/p/s3fs/}} welche es ermöglicht eine Amazon S3 Speicher unter Linux zu mounten. \cite{S3fs}

Da die Infrastruktur von Amazon S3 nicht bekannt ist, können keine detailierte Angaben wie Amazon die Skalierung der Datenzugriffe handhabt. Es kann jedoch von einen hochskalierbaren System ausgegangen werden, da Amazon die Zugriffe von tausenden von Kunden gleichzeitig handhaben muss.

Amazon S3 unterstützt das Lesen und Schreiben von mehren Systemen. Beim Lesen einen Objekts welche von mehren Systemen Zugegriffen wird und beschrieben, stellt Amazon S3 die Lese konsistent sicher.\cite{Amazon2012a}

Für das Einlesen von Grossen Datenmengen bietet Amazon einen kostenpflichtigen Import/Export Dienstleistung an. Bei dieser Dienstleistung sendet der Kunde die Daten gespeichert auf  einen tragbares Speichergeräte an Amazon, welches die Daten direkt in Ihre Cloud ohne Umwege über das Internet einliesst.

Die API von Amazon ist gut Dokumentiert, weshalb für einen erfahrenen Entwickler die Anbindung der Web-Applikation an Amazon S3 umsetzbar sein sollte.

\paragraph*{Speicherkapazität}
Für die Skalierung der Speicherkapazität kümmert sich Amazon.

Im Amazon S3 können eine unbegrenzte Zahl von Objekten gespeichert werden, welche eine Speichergrösse von 1 Byte bis zu 5 Terrabyte haben können, wobei aber in einen PUT maximal 5 Gigabyte hochgeladen werden können. Für das Hochladen von Grösseren Objekten, muss die Multipart Funktion verwendet werden, welche das Objekte in mehren Teilen hochladet.\cite{Amazon2012b}

\paragraph*{Datenschutz}
Die Daten können bei Amazon in mehren Regionen gespeichert werden. Zu den verfügbaren Regionen gehören, US Standard, US West (Oregon), US West (Northern California), EU (Ireland), Asia Pacific (Singapore), Asia Pacific (Tokyo), South America (Sao Paulo). Gemäss Amazon verlassen die Gespeicherten Daten eine Region nicht, ausser für die Erfüllung von Gesetzen oder auf Anforderung von Regierung Anweisung. Als US Amerikanisches Unternehmen steht Amazon wegen dem Patrot Act, unter der Pflicht, den USA Behörden Zugang zu Informationen zu gewährleisten auf behördlichen Anforderungen, auch wenn diese Informationen ausserhalb der USA gespeichert sind. Gemäss EU Recht darf jedoch keine gespeicherten Informationen, an dritten Zugänglich gemacht werden oder ausserhalb der EU gespeichert werden, ohne Einverständnis. Bei Microsoft ebenfalls eine Grosser Cloud Anbieter bestätigt diese Annahme, das die Daten nicht vor den Zugriff der USA geschützt sind.\cite{Amazon2012}\cite{Ostler}

Die Integrität der Daten wird mit einer Prüfsumme sichergestellt. Die gespeicherten Daten werden von Amazon regelmässig auf Ihre Integrität geprüft und bei bedarf von einer Integer Kopie der Daten ersetzt. 

Die Daten können Verschlüsselt per SSL Zugegriffen bzw. gespeichert werden, somit ist gewährleistet, das dritte die Daten beim Transport nicht lesen können. Seit 2011 bietet Amazon kostenlos ebenfalls die Verschlüsselung mit AES-256 von Objekten im Speicher an. Dabei wird jedes Objekt mit einen eigenen Schluss ver- und entschlüsselt. Die erzeugten Schlüssel werden mit einen Master-Schlüssel ebenfalls Verschlüsselt und auf den Amazon-Servern gespeichert. Das Schlüsselmanagment bleibt jedoch bei Amazon, weshalb man abhängig von den Massnahmen zum schuzt des Schlusselmanagment die Amazon trifft. Wird diese Kompromittiert oder fällt aus ist der Zugriff auf die Verschlüsselten Daten nicht mehr möglich.\cite{RobertLippert2011}

Die Berechtigung auf gespeicherte Objekte oder Ordner können mit einen Rechte-Management verwaltet werden. Objekte können öffentlich zugänglich gemacht werden oder nur an bestimmte Authentifizierte Benutzer zur Verfügung gestellt werden. Zudem lassen Sich Zugriffe auf Objekte Protokollieren.\cite{Amazon2012b}

Für die Berechtigung und Verwaltung stellt Amazon ein Web-GUI zur Verfügung. Bis auf das eröffnen eines Speichers und die Initial Berechtigung sollten, wahrscheinlich nicht viel mehr Verwaltungsaufgaben anfallen.

Amazon bietet neben der Redundanz der Objekte kein weiteres Sicherungsverfahren an. Beim Herunterladen der Daten fallen für den Transfer weiter Kosten an.

\paragraph*{Technologie}
Der Online Speicher von Amazon S3 gilt als ausgereiftes und etabliertes Produkt im Markt, weshalb davon ausgegangen werden kann, dass die Dienstleistung über die nächsten vier Jahre drübenraus noch bestehen wird und für den Kunden keine Migration notwendig ist. Amazon S3 wird ebenfalls von SmugMug ebenfalls eine Photodienstleister oder andere bekannte Web-Applikationen wie Dropbox verwendet.\cite{SmugMug}\cite{Dropbox2011}

Der Online Speicher Markt ist im Verhältnis zu anderen Speicher Märkte noch relativ jung, Analysten wie Jeff Boles gehen davon aus, dass der Online Speicher Markt in den nächsten Jahren weiter stark wächst. Wie in der Marktstudie beschrieben konnte Amazon seit dem Start des Amazon S3 Produktes die Anzahl gespeicherte Objekte jeweils pro Jahr mehr als verdoppeln (sie dazu \refabb{abb:AnzahlObjekteAmazonS3}). Die Konkurrenz von grossen Anbietern wie Google, Microsoft oder Rackspace nimmt für Amazon bereits zu. Für den Kunden bedeutet diese, dass die Anbieter Ihre Preise noch stärker kalkulieren müssen um ein Konkurrenz fähiges Angebot zu haben. Amazon hat die Preise für S3 Speicher erst kürzlich angepasst.\cite{Boles2011}\cite{Barr2012a}


\subsubsection{Kosten}
Durch den Bezug des Speichersressourcen als Dienstleistung, fallen für den Kunden keine Investitionskosten für die Speicher Infrastruktur an.
Dafür muss in den meisten Fällen, die Applikation für den Zugriff mittels Amazon S3 API vorbereitet werden, wodurch Entwicklungskosten anfallen könnten.

Beim Amazon S3 wird jeweils nur der Effektive verwendete Speicherplatz pro Monat verrechnet, im Preis von Amazon ist jeweils die dreifache Redundanz inbegriffen. Der Speicherplatz Kostet ab 1 Terabyte bis 49 Terabyte \$ 0.110 per Gigiabyte. Ab 49 Terabyte bis 450 Terabyte reduzieren sich die Kosten pro Gigabyte auf \$ 0.095. Anders als beim betrieb seiner eigenen Speicherinfrastruktur muss der Kunden keine zusätzliche Speicherkapazitätsreserven für unvorhergesehenes einrechnen, da Ihm diese bei Bedarf jederzeit von Amazon zur Verfügung gestellt wird, was wiederum die Kosten für den Kunden senkt, da er keine Reserve zu Verfügung stellen muss.

Bei Verwendung von Amazon S3 fallen für den Kunden zudem keine Wartungskosten für den Betrieb der Speicherinfrastruktur an. 

Im vergleich zu betrieb der eignen Speicherinfrastruktur, fallen dem Kunden, stattdessen Kosten für den Transfer und Abfragen der Daten an, welche bei der eignenden Infrastruktur generell durch die Investition in die Infrastruktur beinhaltet ist. So Kostet der Transfer der Daten aus der Amazon S3 Speicher aus, für jedes GB \$ 0.12 bei einen Transfervolumen bis 10 TB.  Für die Abfrage wird unterschieden zwischen Schreibenden und Lesenden Abfragen. Für den Lese Zugriff auf die Daten wird pro 10'000 Abfragen \$ 0.01 und für jede Schreibzugriff pro 1'000 Abfragen\$ 0.01 verrechnet. Gerade diese Kosten sind jedoch schwer vorher zubestimmen und könnten zu Überraschungen sorgen. Grund dafür ist, dass diese jeweils stark von jeweiligen Anzahl Benutzer und deren Benutzerverhalten abhängig sind und diese schwer zu vorhersagen ist.

\paragraph*{Kosten Szenario-1}
Die Kosten für Szenario-1 betragen gemäss Zusammenstellung der\reftab{tab:KostenAmazonS3S1} Total  \$ 42'646.32 das sind zum aktuellen Tageskurs (13 April 2012)  39'10.52 CHF. 


\subsubsection*{Kosten Szenario-2}
Die Kosten für Szenario-1 betragen gemäss Zusammenstellung der \reftab{tab:KostenAmazonS3S2} Total  \$ 493'902.84 das sind zum aktuellen Tageskurs (13 April 2012)  450'637.17 CHF. 

\begin{table}
\caption{Kosten Amazon S3 Szenario-1}
\begin{center}
\begin{tabular}{|l|r|r|r|}
\hline
\multicolumn{1}{|l|}{\textbf{Bezeichnung}} &\multicolumn{1}{|l|}{\textbf{Monat}} & \multicolumn{1}{l|}
{\textbf{Speicherkapazität}} & \multicolumn{1}{l|}{\textbf{Kosten}} \\ \hline
\multicolumn{4}{c}{} \\  \hline
\multicolumn{ 4}{|c|}{\textit{Investitionskosten}} \\ \hline 
- &  &  &    \\ \hline
\multicolumn{3}{r|}{\textbf{Total:}} & CHF 0.00 \\ \cline{4-4}
\multicolumn{4}{c}{} \\  \hline
\multicolumn{ 4}{|c|}{\textit{Fortlaufende Kosten}} \\ \hline
Amazon S3 & 1 & 2,75 &  \$691.82  \\ \hline
Amazon S3 & 2 & 3 &  \$719.98  \\ \hline
Amazon S3 & 3 & 3,25 &  \$748.14  \\ \hline
Amazon S3 & 4 & 3,5 &  \$776.30  \\ \hline
Amazon S3 & 5 & 3,75 &  \$804.46  \\ \hline
Amazon S3 & 6 & 4 &  \$832.62  \\ \hline
Amazon S3 & 7 & 4,25 &  \$860.78  \\ \hline
Amazon S3 & 8 & 4,5 &  \$888.94  \\ \hline
Amazon S3 & 9 & 4,75 &  \$917.10  \\ \hline
Amazon S3 & 10 & 5 &  \$945.26  \\ \hline
Amazon S3 & 11 & 5,25 &  \$973.42  \\ \hline
Amazon S3 & 12 & 5,5 &  \$1'001.58  \\ \hline
Amazon S3 & 13 & 5,75 &  \$1'029.74  \\ \hline
Amazon S3 & 14 & 6 &  \$1'057.90  \\ \hline
Amazon S3 & 15 & 6,25 &  \$1'086.06  \\ \hline
Amazon S3 & 16 & 6,5 &  \$1'114.22  \\ \hline
Amazon S3 & 17 & 6,75 &  \$1'142.38  \\ \hline
Amazon S3 & 18 & 7 &  \$1'170.54  \\ \hline
Amazon S3 & 19 & 7,25 &  \$1'198.70  \\ \hline
Amazon S3 & 20 & 7,5 &  \$1'226.86  \\ \hline
Amazon S3 & 21 & 7,75 &  \$1'255.02  \\ \hline
Amazon S3 & 22 & 8 &  \$1'283.18  \\ \hline
Amazon S3 & 23 & 8,25 &  \$1'311.34  \\ \hline
Amazon S3 & 24 & 8,5 &  \$1'339.50  \\ \hline
Amazon S3 & 25 & 8,75 &  \$1'367.66  \\ \hline
Amazon S3 & 26 & 9 &  \$1'395.82  \\ \hline
Amazon S3 & 27 & 9,25 &  \$1'423.98  \\ \hline
Amazon S3 & 28 & 9,5 &  \$1'452.14  \\ \hline
Amazon S3 & 29 & 9,75 &  \$1'480.30  \\ \hline
Amazon S3 & 30 & 10 &  \$1'508.46  \\ \hline
Amazon S3 & 31 & 10,25 &  \$1'536.62  \\ \hline
Amazon S3 & 32 & 10,5 &  \$1'564.78  \\ \hline
Amazon S3 & 33 & 10,75 &  \$1'592.94  \\ \hline
Amazon S3 & 34 & 11 &  \$1'621.10  \\ \hline
Amazon S3 & 35 & 11,25 &  \$1'649.26  \\ \hline
Amazon S3 & 36 & 11,5 &  \$1'677.42  \\ \hline
\multicolumn{3}{r|}{\textbf{Total:}} & \textbf{\$ 42'646.32} \\ \cline{4-4}
\multicolumn{4}{c}{} \\  \cline{4-4}
\multicolumn{3}{r|}{\textbf{Total Gesamt:}} & \textbf{ \$ 42'646.32} \\ \cline{4-4}
\end{tabular}
\end{center}
\label{tab:KostenAmazonS3S1}
\end{table}

\begin{table}
\caption{Kosten Amazon S3 Szenario-2}
\begin{center}
\begin{tabular}{|l|r|r|r|}
\hline
\multicolumn{1}{|l|}{\textbf{Bezeichnung}} &\multicolumn{1}{|l|}{\textbf{Monat}} & \multicolumn{1}{l|}
{\textbf{Speicherkapazität}} & \multicolumn{1}{l|}{\textbf{Kosten}} \\ \hline
\multicolumn{4}{c}{} \\  \hline
\multicolumn{ 4}{|c|}{\textit{Investitionskosten}} \\ \hline 
- &  &  &    \\ \hline
\multicolumn{3}{r|}{\textbf{Total:}} & CHF 0.00 \\ \cline{4-4}
\multicolumn{4}{c}{} \\  \hline
\multicolumn{ 4}{|c|}{\textit{Fortlaufende Kosten}} \\ \hline
Amazon S3 & 1 & 8.5 &  \$2'937.87  \\ \hline
Amazon S3 & 2 & 14.5 &  \$3'613.71  \\ \hline
Amazon S3 & 3 & 20.5 &  \$4'289.55  \\ \hline
Amazon S3 & 4 & 26.5 &  \$4'965.39  \\ \hline
Amazon S3 & 5 & 32.5 &  \$5'641.23  \\ \hline
Amazon S3 & 6 & 38.5 &  \$6'317.07  \\ \hline
Amazon S3 & 7 & 44.5 &  \$6'992.91  \\ \hline
Amazon S3 & 8 & 50.5 &  \$7'661.07  \\ \hline
Amazon S3 & 9 & 56.5 &  \$8'244.75  \\ \hline
Amazon S3 & 10 & 62.5 &  \$8'828.43  \\ \hline
Amazon S3 & 11 & 68.5 &  \$9'412.11  \\ \hline
Amazon S3 & 12 & 74.5 &  \$9'995.79  \\ \hline
Amazon S3 & 13 & 80.5 &  \$10'579.47  \\ \hline
Amazon S3 & 14 & 86.5 &  \$11'163.15  \\ \hline
Amazon S3 & 15 & 92.5 &  \$11'746.83  \\ \hline
Amazon S3 & 16 & 98.5 &  \$12'330.51  \\ \hline
Amazon S3 & 17 & 104.5 &  \$12'914.19  \\ \hline
Amazon S3 & 18 & 110.5 &  \$13'497.87  \\ \hline
Amazon S3 & 19 & 116.5 &  \$14'081.55  \\ \hline
Amazon S3 & 20 & 122.5 &  \$14'665.23  \\ \hline
Amazon S3 & 21 & 128.5 &  \$15'248.91  \\ \hline
Amazon S3 & 22 & 134.5 &  \$15'832.59  \\ \hline
Amazon S3 & 23 & 140.5 &  \$16'416.27  \\ \hline
Amazon S3 & 24 & 146.5 &  \$16'999.95  \\ \hline
Amazon S3 & 25 & 152.5 &  \$17'583.63  \\ \hline
Amazon S3 & 26 & 158.5 &  \$18'167.31  \\ \hline
Amazon S3 & 27 & 164.5 &  \$18'750.99  \\ \hline
Amazon S3 & 28 & 170.5 &  \$19'334.67  \\ \hline
Amazon S3 & 29 & 176.5 &  \$19'918.35  \\ \hline
Amazon S3 & 30 & 182.5 &  \$20'502.03  \\ \hline
Amazon S3 & 31 & 188.5 &  \$21'085.71  \\ \hline
Amazon S3 & 32 & 194.5 &  \$21'669.39  \\ \hline
Amazon S3 & 33 & 200.5 &  \$22'253.07  \\ \hline
Amazon S3 & 34 & 206.5 &  \$22'836.75  \\ \hline
Amazon S3 & 35 & 212.5 &  \$23'420.43  \\ \hline
Amazon S3 & 36 & 218.5 &  \$24'004.11  \\ \hline
\multicolumn{3}{r|}{\textbf{Total:}} & \textbf{\$ 493'902.84}
 \\ \cline{4-4}
\multicolumn{4}{c}{} \\  \cline{4-4}
\multicolumn{3}{r|}{\textbf{Total Gesamt:}} & \textbf{\$ 493'902.84}
 \\ \cline{4-4}
\end{tabular}
\end{center}
\label{tab:KostenAmazonS3S2}
\end{table}




\subsection{Evaluation KO Kriteren}
\subsubsection{Evaluation KO Kriteren Szenario 1}

Alle Alternativen \refbf{Al-1} bis \refbf{Al-5} unterstützen die Speicherung von Dateien bis 2 Gibibyte Speichergrösse (\refbf{KO-1}).

Alle Alternativen \refbf{Al-1} bis \refbf{Al-5} aus dem Szenario-1 erfüllen die geforderten Speicherkapazität (\refbf{KO-1}).

Mit sdf
Die günstigste Lösung ist die Alternative Al-1 mit Gesamtkosten von 26’464.31 CHF

%!TEX root=../documentation-bachlorthesis-speicherarchitektur-lstucker.tex
\cleardoublepage
\chapter{Empfehlung für den Auftraggeber}
\cleardoublepage
\chapter{Machbarkeitsnachweis}

Der selbe Hosting Provider des bestehenden Servers bietet eine Produkt an welches mit 15 SATA Festplatten mit einer Speicherkapazität von je 3 Terabyte an. Die Maximale Raid-5 Speicherkapazität \ref{eqn:maxRaid5-15disk} ist aufgrund der Anzahl Disk und der Speicherkapazität nicht die Ideale Konfiguration, da die Gefahr eines Doppelten Disk Ausfall durch die geringere MTTF \ref{eqn:MTBF15Disk}, welche durch die Zunahme der Anzahl Festplatten sinkt \ref{eqn:MTBF1Disk} und der höheren Rebuild Zeit steigt. Speichersystem Hersteller wie NetApp setze bei 
dieser Konfiguration auf Raid-6 welche doppelte Parität bietet und somit zwei Festplattenausfälle kompensieren können.

Festplattenkapazität in Tebibyte:
\begin{equation}
3   \, \mathrm{TB} =  3 * \frac{1000^4}{1024^4} = 2.7285  \, \mathrm{TiB}
\label{eqn:3TerrabyteTebibyte}
\end{equation}

Maximale Raid-5 Speicherkapazität:
\begin{equation}
(15 -1) * 2.7285  \, \mathrm{TiB} =  38.199 \, \mathrm{TiB}
\label{eqn:maxRaid5-15disk}
\end{equation}

MTBF 1 Festplatten (ST33000650SS): 
\begin{equation}1'200'000  \, \mathrm{h}
\label{eqn:MTBF1Disk}
\end{equation}

MTBF 15 Festplatten (Raid-5) (ST33000650SS): 
\begin{equation}
\frac{1'200'000  \, \mathrm{h}}{15}= 8'000  \, \mathrm{h}
\label{eqn:MTBF15Disk}
\end{equation}
\cleardoublepage

% \usepackage{fancyhdr}
%\pagenumbering{roman}
%\setcounter{page}{1}
% \input{chapter/Listings}
% %!TEX root=../documentation-bachlorthesis-speicherarchitektur-lstucker.tex

\newglossaryentry{RFC}{ name={RFC}, description={Request for Comments (RFC) sind Dokumente, über Internet, inklusive der technischen Spezifikation und Richtlinien, welche von der Organisation Internet Engineering Task Force entwickelt wurde. "'Das RFC wird erst nach erfolgter Diskussion unter der Aussicht des Internet Architecture Board (IAB) herausgegeben und fungiert als Quasistandard. Jedes RFC enthält eine eindeutige, vorlaufende Nummer, die kein zweites Mal zu gewiesen wird."' \cite{MicrosoftComputerLex}  \url{http://www.rfc-editor.org/}}}

\newglossaryentry{UDP}{ name={UDP}, description={  adfajsdfjadslkfjaödjfölaksdjfajsklfj }}

\newglossaryentry{IBM}{ name={IBM}, description={International Business Machines Corporation (kurz IBM) ist ein führendes unternehmen in Software, Hardware und IT-Dienstleistung Bereich. }}

\newglossaryentry{IBM}{ name={CIFS}, description={Common Internet File System (kurz CIFS) wurde 1996 von Microsoft eingeführt und beschreibt eine erweiterte Version von SMB. CIFS und SMB sind eine  Netzwerkdateisystem vergleichbar mit NFS und wird vorwiegend im MS Windows Bereich eingesetzt. }}


\newglossaryentry{XDR}{ name={XDR}, description={Die eXternal Data Representation (kurz XDR) Spezifikation stellt ein Standardisierte Verfahren zur Präsentation von gebräuchlichsten Daten Typen über das Netzwerk zur Verfügung. Dies löst das Problem der verschiedenen Byte-Reihenfolge (Big Endian), Speicherausrichtung auf unterschiedlichen Kommunikations Partner}}

\newglossaryentry{POSIX}{ name={POSIX}, description={Portable Operating System Interface (kurz POSIX) ist eine von IEEE entwickelter Standard, welche die Schnittstelle zwischen Applikation und Betriebsystem darstellt. Die aktuelle Version des Standards ist IEEE Std 1003.1-2008 \url{http://www.opengroup.org/austin/papers/posix_faq.html} }}

\newglossaryentry{FUSE}{ name={FUSE}, description={Filesystem in Userspace (kurz FUSE), ermöglicht die Implementierung eines voll Funktionsfähigen Dateisystem in Userspace. Normaler weise laufen 
FUSE wurde urspünglich Entwickelt um AVFS zu unterstützen, ist jedoch heute ein seperates Projekt. \url{http://fuse.sourceforge.net/}}} 

\newglossaryentry{RPC}{File locking erlaubt es einen Prozess den exklusiven Zugriff auf eine Datei oder teile einer Datei und zwingt ander Prozesse die auf die selbe Ressource zugreifen wollen zu warten bis das Locking aufgehoben wurde.} 

\newglossaryentry{FileLocking}{File locking erlaubt es einen Prozess den exklusiven Zugriff auf eine Datei oder teile einer Datei und zwingt ander Prozesse die auf die selbe Ressource zugreifen wollen zu warten bis das Locking aufgehoben wurde.} 

\newglossaryentry{API}{ name={API}, description={Application Programming Interface (kurz API) auch Anwendungsprogrammierschnittstelle genannt. "'Ein Satz an Routinen, die vom Betriebsystem des Computers für die Verwendung aus Anwendungsprogrammen heraus angeboten werden und diverse Dienste zur Verfügung stellen."' \cite{MicrosoftComputerLex} }}

\newglossaryentry{MIT{ name={MIT}, description={Die MIT Lizenz stammt von Massachusetts Institute of Technology und erlaub die die Verwendung von Software welche Quelloffen ist als auch software welche nicht Quell geschlossene ist. Die genauen Lizenz Bestimmungen sind unter folgenden URL zu finden \url{http://www.opensource.org/licenses/mit-license.php}}}

\newglossaryentry{GNU GPL{ name={GNU GPL}, description={Die GNU General Public License Lizenz auch GPL genannt stammt von der Free Software Foundation und regelt die Lizenzierung von Freie Software. Es gibt drei Versionen der GPL welche unter folgenden URL beschrieben sind \url{http://www.gnu.org/licenses/}}}

\newglossaryentry{Ruby{ name={Ruby}, description={Ruby ist eine interpretierte und objektorientierte Programmiersprache und beinhaltet einige bewährte Prinzipien wie z.B. "'DuckTyping"' und "'Principle of Least Suprice"'. Die Entwickler von Ruby stellen sich selber den Anspruch eine Programmiersprache zu schaffen, die durch Ihre Natürlichkeit einfach erlernbar ist und es den Programmierern ermöglicht, einfachen und übersichtlichen Code zu schreiben, welcher aber nicht seine Mächtigkeit und innere Komplexität verliert.
Ruby hat sich in den letzten Jahren von einer kaum beachteten Programmiersprache zu einem Publikums-Magneten entwickelt. Es gibt eine stetig wachsende offene Community "'Gemeinschaft"', welche sich und die Sprache durch Austausch von Erfahrungen und Ideen weiterbringen möchte.
Ein Grund für die hohe Bereitschaft der Community die Sprache Ruby weiter zu bringen ist der Umstand, dass die Programmiersprache vollständig OpenSource ist und unter der Lizenz der Ruby-License und GPL steht. Zudem ist die Sprache fast beliebig erweiterbar und bestehende Funktionen können einfach durch eigene Funktionen ausgetauscht werden.}}
% \input{chapter/Akronyme}
% \input{chapter/literatur}

\pagenumbering{roman}
% Glossar ausgeben
\printglossary[style=altlist,title=Glossar]
% Abkürzungen ausgeben
% \deftranslation[to=German]{Acronyms}{Abkürzungsverzeichnis}
% \printglossary[type=\acronymtype,style=long]
%Symbole ausgeben
% \printglossary[type=symbolslist,style=long]

% Literatur
\bibliographystyle{gerplain}
\bibliography{library.bib}
% \input{Index}

%\appendix
\clearpage
\renewcommand{\appendixtocname}{Anhang}
\renewcommand{\appendixpagename}{Anhang}
\appendix
%\addappheadtotoc
\appendixpage

%!TEX root=../documentation-bachlorthesis-speicherarchitektur-lstucker.tex
\cleardoublepage
\chapter{Details zur Evaluation}
\section{Evaluation Soll Kriteren}
\subsection{Evaluation Soll Kriteren Szenario-1}

\subsubsection{Kosten}

\paragraph*{\refsoll{Soll-1-1} \refsoll{Al-1} verglichen mit \refsoll{Al-5} (\ref{Al-1}/\ref{Al-5})}
Bei \ref{Al-1} Hetzner entstehen Einrichtungs-Kosten für den Root Server von 499.00 € in 600,02 CHF (Kurs 13 April 2012), weitere Kosten kommen nicht hinzu. Bei Amazon S3 \ref{Al-5} entstehen keine Einrichtung bzw. Anschaffungskosten. In Vergleich zu \ref{Al-5} sind die Anschaffungskosten von \ref{Al-1} höher, sind aber in Vergleich zu den Gesamtkosten von \ref{Al-1} oder \ref{Al-5} gering, aus diesem Grund wird die \ref{Al-1} etwas bis erheblich Geringer bewertet als \ref{Al-5}.

\textbf{Bewertet: 1/5}

\paragraph*{\refsoll{Soll-1-2} \refsoll{Al-1} verglichen mit \refsoll{Al-5} (\ref{Soll-1-2} \ref{Al-1}/\ref{Al-5})}
Die Unterhaltskosten von Hetzner \ref{Al-1} sind mit 13'422.95 CHF in Vergleich zu den Unterhaltskosten von Amazon S3 \ref{Al-5} mit 39'10.52 CHF erheblich tiefer. Aus diesen Grund wird \ref{Al-1} sehr viel besser bewertet als \ref{Al-5}.

\textbf{Bewertet: 7}

\paragraph*{\refsoll{Soll-1-3} \refsoll{Al-1} verglichen mit \refsoll{Al-5} (\ref{Soll-1-3} \ref{Al-1}/\ref{Al-5})}
Beide Alternativen sind von Dienstleister abgängig. Sowohl Hetzer als auch Amazon sind grössere Anbieter in Ihrem Marktumfeld. Bei Hetzer ist nach drei Jahren zu Prüfen ob die gemietete Hardware durch neuere günstigere und besser ausgebaute Produkte von Hetzner ersetztet werden soll. Im Marktsegment des Webspeichers ist hingegen mit weiteren Veränderungen zu Rechnen was ebenfalls eine regelmässige Überprüfung erfordert. Aus diesen Grund sind beide gleich zu bewerten.

\textbf{Bewertet: 1}


\subsubsection{Verfügbarkeit}

\paragraph*{\refsoll{Soll-2-1} \refsoll{Al-1} verglichen mit \refsoll{Al-5} (\ref{Soll-2-1} \ref{Al-1}/\ref{Al-5})}
Bei Hetzner \ref{Al-1} sind die Daten einfach redundant gespeichert, wobei die Redundanz durch Parität sichergestellt ist und die Daten nicht 1:1 doppelt gespeichert sind. Bei Amazon S3 sind die Daten mindestens in dreifacher Redundanz gespeichert. Die redundante Speicherung von Amazon S3 hat folgende Vorteile:

\begin{itemize}
\item höhere Redundanz über mehrere Medien
\item bei doppeltem Ausfall, kein Datenverlust
\item Keine Berechnung aus Parität notwendig
\end{itemize}

Wegen der genannten Vorteile von \ref{Al-5 }ist die \refsoll{Soll-2-1} der Aktiven Daten von \ref{Al-1} erheblich bis sehr viel geringer zu bewerten als \ref{Al-5}.

\textbf{Bewertet: 1/6}

\paragraph*{\refsoll{Soll-2-2} \refsoll{Al-1} verglichen mit \refsoll{Al-5} (\ref{Soll-2-2} \ref{Al-1}/\ref{Al-5})}
Bei Hetzner \ref{Al-1} sind die Daten nur auf einem System gespeichert, welche keine besonderen Massnahmen hat, um die Systemverfügbarkeit zu erhöhen. Bei Amazon S3 sind die Daten mindestens auf drei unterschiedlichen Servern verteilt, über die restlichen Massnahmen die Amazon für die System-Redundanz trifft sind nicht bekannt, es kann davon ausgegangen werden, dass das System von Amazon S3 die Verfügbarkeit des Systems Harvard Research Group AEC-4 erfüllt.

Die \refsoll{Soll-2-2} von \ref{Al-1} sehr viel Geringer zu bewerten als \ref{Al-5}.

\textbf{Bewertet: 1/8}

\paragraph*{\refsoll{Soll-2-3} \refsoll{Al-1} verglichen mit \refsoll{Al-5} (\ref{Soll-2-3} \ref{Al-1}/\ref{Al-5})}
Bei Hetzner \ref{Al-1} sind die Aktiven Daten nur auf einem System am einen Standort gespeichert. Bei Amazon S3 \ref{Al-5} sind die Daten in mehreren Rechenzentren gespeichert.

Die \refsoll{Soll-2-3} Verfügbarkeit der aktiven Daten von \ref{Al-1} sind absolut Geringer zu bewerten als \ref{Al-5}.

\textbf{Bewertet: 1/9}

\subsubsection{Datenzugriffe}

\paragraph*{\refsoll{Soll-3-1} \refsoll{Al-1} verglichen mit \refsoll{Al-5} (\ref{Soll-3-1} \ref{Al-1}/\ref{Al-5})}
Bei Hetzner\ref{Al-1} handelt es sich nur um einen Server welche nur von sich selber zugegriffen werden kann. Bei Amazon S3 handelt es sich um ein hochskalierbares System, welches von Hunderttausenden von Benutzer bzw. Systeme zugegriffen wird. Nachteil dabei ist, dass durch die hohe Anzahl an Benutzer die auf Amazon S3 zugreifen, Schwankungen in der Antwort und Auslieferung der Daten durch den Tag verteilt geben kann.
Die \refsoll{Soll-3-1} ist bei \ref{Al-1} erheblich bis sehr geringer zu bewerten als bei \ref{Al-5}.

\textbf{Bewertet: 1/6}

\paragraph*{\refsoll{Soll-3-2} \refsoll{Al-1} verglichen mit \refsoll{Al-5} (\ref{Soll-3-2} \ref{Al-1}/\ref{Al-5})}
Die Performance bei Amazon S3 ist von der Internet Anbindung abhängig. Bei Hetzner gibt es diese Beschränkung nicht, da alles auf einen Server stattfindet. Somit ist die Performance bei \refsoll{Al-1} \ref{Al-1} sehr viel grosser zu bewerten als \refsoll{Al-5} \ref{Al-5}.

\textbf{Bewertet: 7} 

\paragraph*{\refsoll{Soll-3-3} \refsoll{Al-1} verglichen mit \refsoll{Al-5} (\ref{Soll-3-3} \ref{Al-1}/\ref{Al-5})}
Der Zugriff bei Hetzner \ref{Al-1} erfolgt über POSIX-IO, bei Amazon S3 hingegen existiert keine offizielle POSIX-IO Schnittstelle. Aus diesen Grund ist der \refsoll{Soll-3-3} bei \refsoll{Al-1} \ref{Al-1} sehr viel bis absolut höher zu bewerten als bei \refsoll{Al-5} \ref{Al-5}.

\textbf{Bewertet: 8} 


\paragraph*{\refsoll{Soll-3-4} \refsoll{Al-1} verglichen mit \refsoll{Al-5} (\ref{Soll-3-4} \ref{Al-1}/\ref{Al-5})}
Bei Hetzner kann keinen simultaner Lesezugriff von mehreren Server auf ein Objekt erfolgen, da der Speicher nur einem System zur Verfügung steht. Bei Amazon S3 ist der Zugriff auf ein Objekt von Mehren Server-System möglich. Aus diesen Grund ist \refsoll{Al-1} \ref{Al-1} absolut schlechter zu bewerten als \refsoll{Al-5} \ref{Al-5}.

\textbf{Bewertet: 1/9} 

\paragraph*{\refsoll{Soll-3-4} \refsoll{Al-1} verglichen mit \refsoll{Al-5} (\ref{Soll-3-4} \ref{Al-1}/\ref{Al-5})}
Bei Hetzner kann keinen simultaner Schreibzugriff von mehreren Server auf ein Objekt erfolgen, da der Speicher nur einem System zur Verfügung steht. Bei Amazon S3 ist das Simulateschreiben auf ein Objekt von Mehrern Server möglich, es gewinnt aber nur die aktuellste Version.
Beider Alternativen sind gleich zu bewerten.

\textbf{Bewertet: 1}

\subsubsection{Speicherkapazität}

\paragraph*{\refsoll{Soll-4-1} \refsoll{Al-1} verglichen mit \refsoll{Al-5} (\ref{Soll-4-1} \ref{Al-1}/\ref{Al-5})} 
Die Speicherkapazität von \ref{Al-1} bei Hetzner ist auf maximal mit geringster Redundanz auf 38,192 TiB begrenzt. Durch die Begrenzung von ext3 können jedoch nicht die ganzen 38,192 TiB in einem einzigen Dateisystem genutzt werden, sondern müssen auf mehre Dateisysteme aufgeteilt werden. Bei Amazon S3 \ref{Al-5} existiert eine solche Begrenzung für den Kunden nicht. Die maximale Speicherkapazität von \ref{Al-1} ist dennoch mehr als doppelt so gross wie die geforderte Speicherkapazität von Szenario-1 aus diesen Grund ist die \refsoll{Soll-4-1} von \ref{Al-1} erheblich Geringer zu bewerten als \ref{Al-5}.

\textbf{Bewertet: 1/5}

\paragraph*{\refsoll{Soll-4-2} \refsoll{Al-1} verglichen mit \refsoll{Al-5} (\ref{Soll-4-2} \ref{Al-1}/\ref{Al-5})} 
Die \ref{Al-1} kann durch die Begrenzung von Dateisystem ext3 maximal 17'592'186'044'416 Objekte in einem Dateisystem Speichern. Bei Amazon S3 sind keine Informationen bekannt über eine mögliche Begrenzung eines Amazon S3 Bucket, über die Bucket grenze hinaus gibt es keine Begrenzung für den Kunden. Es werden jedoch hauptsächlich grössere Objekte gespeichert weshalb im Speicher gespeichert wo durch die Begrenzung von Alternativen \ref{Al-1} ausreichend Platz für die Speicherung von Objekten hat. Aus diesen Grund ist die \refsoll{Soll-4-2} bei \ref{Al-1} etwas schlechter zu bewerten als bei \ref{Al-5}.

\textbf{Bewertet: 1/3}

\paragraph*{\refsoll{Soll-4-3} \refsoll{Al-1} verglichen mit \refsoll{Al-5} (\ref{Soll-4-3} \ref{Al-1}/\ref{Al-5})} 
Die \ref{Al-1} kann durch die Begrenzung von Dateisystem ext3 maximal Objekte mit einer Speicherkapazität von 2 TiB gespeichert werden. Bei \ref{Al-5} ist die Begrenzung bei 5 TB, wobei hier die Objekte maximal in 5 GB Stücke hochgeladen werden können, bei späterem Zugriff ist jedoch einen Zugriff auf das ganze Objekt möglich. Aus diesen Grund ist \ref{Al-1} etwas tiefer zu bewerten als \ref{Al-5}.

\textbf{Bewertet: 1/3}

\subsubsection{Datenschutz}

\paragraph*{\refsoll{Soll-5-1} \refsoll{Al-1} verglichen mit \refsoll{Al-5} (\ref{Soll-5-1} \ref{Al-1}/\ref{Al-5})} 
Die Alternative \ref{Al-1} kann die Integrität der Daten nur auf RAID-Ebene Sicherstellen nicht aber auf Objekte. Amazon S3 \ref{Al-5} stellt mittels Hash Prüfsumme die Integrität beim Übermitteln und im Speicher sicher. Aus diesen Grund ist die \refsoll{Soll-5-1} von \ref{Al-1} sehr viel tiefer zu bewerten als \ref{Al-5}.

\textbf{Bewertet: 1/7}


\paragraph*{\refsoll{Soll-5-2} \refsoll{Al-1} verglichen mit \refsoll{Al-5} (\ref{Soll-5-2} \ref{Al-1}/\ref{Al-5})} 
Die Alternative \ref{Al-1} bietet keine Selbstheilung von Objekten. Amazon S3\ref{Al-5} prüft regelmässig alle gespeicherten Kopien eines Objekte auf deren Integrität auf Ihren Server. Wird festgestellt, dass eingespeicherte Kopie eines Objekts nicht mehr Integer ist, wird es für den Zugriff gesperrt und von einer intakten Kopie wiederhergestellt. Aus diesen Grund ist die \refsoll{Soll-5-1} von \ref{Al-1} sehr viel bis absolut tiefer zu bewerten als \ref{Al-5}.

\textbf{Bewertet: 1/8}

\paragraph*{\refsoll{Soll-5-3} \refsoll{Al-1} verglichen mit \refsoll{Al-5} (\ref{Soll-5-3} \ref{Al-1}/\ref{Al-5})} 
Die Alternative \ref{Al-1} kann über RSYNC gesichert werden oder auf einem kostenpflichtigen Sicherungsspeicherplatz von Heztner. Bei Amazon S3 gibt es keine integrierte Sicherungsmöglichkeit, da Daten werden jedoch dreifach redundant gehalten. Eine Sicherung der Daten ausserhalb Amazon S3 währen mit hohen Kosten für den Datentransfer verbunden. Aus diesen Grund ist die \refsoll{Soll-5-1} von \ref{Al-1} erheblich bis viel besser zu bewerten als \ref{Al-5}.

\textbf{Bewertet: 6}

\paragraph*{\refsoll{Soll-5-4} \refsoll{Al-1} verglichen mit \refsoll{Al-5} (\ref{Soll-5-4} \ref{Al-1}/\ref{Al-5})} 
Die Haltung der Daten auf einen Server, welche selber betreut, wie es bei \ref{Al-1} der Fall ist, kann man höheren Einfluss nehmen auf die Sicherheit. Die Sicherheit ist jedoch nur so gut, wie man selber erfahren ist in die sichere Konfiguration des Servers. Vor den physischen Zugriff auf die Daten lassen sich diese durch eine Festplattenverschlüsselung schützen. Auf dem Server selber lassen sich durch eine Berechtigungsverwaltung den Zugriff von anderem Benutzer schützen. Bei Amazon S3 ist man auf die Vertrauenswürdigkeit des Anbieters angewiesen. Zwar ermöglicht es Amazon die Daten ebenfalls zu Verschlüsseln der Hauptschlüssel bleibt jedoch bei Amazon. Zudem handelt sich bei Amazon um ein US-amerikanisches Unternehmen das den Patriot Act unterstellt ist.
Aus diesen Grund ist die \refsoll{Soll-5-4} viel besser zu bewerten bei \ref{Al-1} als bei \ref{Al-5}.

\textbf{Bewertet: 7}


\subsubsection{Technologie}

\paragraph*{\refsoll{Soll-6-1} \refsoll{Al-1} verglichen mit \refsoll{Al-5} (\ref{Soll-6-1} \ref{Al-1}/\ref{Al-5})} 
Die Alternative \ref{Al-1} kann die Integrität der Daten nur auf RAID-Ebene Sicherstellen nicht aber auf Objekte. Amazon S3 \ref{Al-5} stellt mittels Hash Prüfsumme die Integrität beim Übermitteln und im Speicher sicher. Aus diesen Grund ist die \refsoll{Soll-5-1} von \ref{Al-1} sehr viel tiefer zu bewerten als \ref{Al-5}.

\textbf{Bewertet: 1/7}

\paragraph*{\refsoll{Soll-6-2} \refsoll{Al-1} verglichen mit \refsoll{Al-5} (\ref{Soll-6-2} \ref{Al-1}/\ref{Al-5})} 
Die Technologie von \ref{Al-1} ist eine ausgereift viel verwendete Technologie, an der Basis Technologie hat sich in den letzten fünf oder mehr Jahren nichts geändert. Die Technologie von \ref{Al-5} ist dagegen im Verhältnis zur \ref{Al-1} noch eine junge Technologie, die trotzdem ihre stark zunehmende Verbreitung noch am Anfang ihrer potenzielle Entwicklung steht. Die Weiterentwicklungs-Möglichkeiten sind bei \ref{Al-5} sehr viel höher zu Gewichten als bei \ref{Al-1}.

\textbf{Bewertet: 1/7}

\paragraph*{\refsoll{Soll-6-3} \refsoll{Al-1} verglichen mit \refsoll{Al-5} (\ref{Soll-6-3} \ref{Al-1}/\ref{Al-5})} 
Die Technologie von \ref{Al-1} ist eine ausgereift viel verwendete Technologie, an der Basis Technologie hat sich in den letzten fünf oder mehr Jahren nichts geändert. Die Technologie von \ref{Al-5} ist dagegen im Verhältnis zur \ref{Al-1} noch eine junge Technologie, die trotzdem ihre stark zunehmende Verbreitung noch am Anfang ihrer potenzielle Entwicklung steht. Die Weiterentwicklungs-Möglichkeiten sind bei \ref{Al-5} sehr viel höher zu Gewichten als bei \ref{Al-1}.

\textbf{Bewertet: 1/7}

\paragraph*{\refsoll{Soll-6-3} \refsoll{Al-1} verglichen mit \refsoll{Al-5} (\ref{Soll-6-3} \ref{Al-1}/\ref{Al-5})} 
Die Verfügbarkeit von Experten welche sich mit \ref{Al-1} auskennen ist sehr viel grosser als bei \ref{Al-5}. 

\textbf{Bewertet: 8}


\paragraph*{\refsoll{Soll-6-3} \refsoll{Al-1} verglichen mit \refsoll{Al-5} (\ref{Soll-6-3} \ref{Al-1}/\ref{Al-5})} 
Der Verwaltungsaufwand ist durch den Bezug der Speicherkapazität bei Amazon S3 \ref{Al-5} sehr viel geringer als bei Hetzner \ref{Al-1}. Für die wenigen Aufgaben, die für die Verwaltung notwendig sind, stellt Amazon zudem ein übersichtliches Webinterface zur Verfügung.
Aus diesen Grund ist der Verwaltungskomfort bei \ref{Al-1} erheblich geringer zu bewerten als bei \ref{Al-5}.

\textbf{Bewertet: 1/5}

\paragraph*{\refsoll{Soll-6-3} \refsoll{Al-1} verglichen mit \refsoll{Al-5} (\ref{Soll-6-3} \ref{Al-1}/\ref{Al-5})} 
Durch das lange Bestehen der Technologie von \ref{Al-1} ist die Technologie als ausgereifter zu Betrachten als bei \ref{Al-5}.

\textbf{Bewertet: 3}


\subsection{Evaluation Soll Kriteren Szenario-2}

\subsubsection{Kosten}

%Anschaffungskosten
\paragraph*{\refsoll{Soll-1-1} \refsoll{Al-2} verglichen mit \refsoll{Al-3} (\ref{Soll-1-1} \ref{Al-2}/\ref{Al-3})}
Die Anschaffungskosten für NetApp NFS und NetApp iSCSI sind gleich hoch, dass es sich um dieselbe Speicher-Infrastruktur handelt.

\textbf{Bewertet: 1}

\paragraph*{\refsoll{Soll-1-1} \refsoll{Al-2} verglichen mit \refsoll{Al-4} (\ref{Soll-1-1} \ref{Al-2}/\ref{Al-4})}
Die Anschaffungskosten mit 550'152.44 CHF sind bei NetApp NFS mehr als doppelt so hoch wie bei OpenStack Object Storage, welche 203'477.00 CHF betragen. Aus diesen Grund ist \refsoll{Al-2} \ref{Al-2} erheblich bis sehr viel geringer zu bewerten als \refsoll{Al-4} \ref{Al-4}.

\textbf{Bewertet: 1/6}

\paragraph*{\refsoll{Soll-1-1} \refsoll{Al-2} verglichen mit \refsoll{Al-5} (\ref{Soll-1-1} \ref{Al-2}/\ref{Al-5})}
Die Anschaffungskosten mit 550'152.44 CHF sind NetApp NFS im Vergleich zu Amazon S3 wo keine Anschaffungskosten anfallen, sehr viel höher. Aus diesen Grund ist \refsoll{Al-3} \ref{Al-3} absolut geringer zu bewerten als \refsoll{Al-5} \ref{Al-5}.

\textbf{Bewertet: 1/9}


\paragraph*{\refsoll{Soll-1-1} \refsoll{Al-3} verglichen mit \refsoll{Al-4} (\ref{Soll-1-1} \ref{Al-3}/\ref{Al-4})}
Die Anschaffungskosten mit 550'152.44 CHF sind bei NetApp iSCSI mehr als doppelt so hoch wie bei OpenStack Object Storage, welche 203'477.00 CHF betragen. Aus diesen Grund ist \refsoll{Al-3} \ref{Al-3} erheblich bis sehr viel geringer zu bewerten als \refsoll{Al-4} \ref{Al-4}.

\textbf{Bewertet: 1/6}

\paragraph*{\refsoll{Soll-1-1} \refsoll{Al-3} verglichen mit \refsoll{Al-5} (\ref{Soll-1-1} \ref{Al-3}/\ref{Al-5})}
Die Anschaffungskosten mit 550'152.44 CHF sind bei NetApp iSCSI im Vergleich zu Amazon S3 wo keine Anschaffungskosten anfallen, sehr viel höher. Aus diesen Grund ist \refsoll{Al-3} \ref{Al-3} absolut geringer zu bewerten als \refsoll{Al-4} \ref{Al-4}.

\textbf{Bewertet: 1/9}


\paragraph*{\refsoll{Soll-1-1} \refsoll{Al-4} verglichen mit \refsoll{Al-5} (\ref{Soll-1-1} \ref{Al-4}/\ref{Al-5})}
Die Anschaffungskosten für Openstack Object Storage \ref{Al-4} von 203'477.00 CHF sind in Vergleich zu Amazon S3 \ref{Al-5} erheblich viel höher.
Aus diesen Grund ist \ref{Al-4} sehr viel tiefer zu bewerten als \ref{Al-5}.

\textbf{Bewertet: 1/7}

%Unterhaltskosten
\paragraph*{\refsoll{Soll-1-2} \refsoll{Al-2} verglichen mit \refsoll{Al-3} (\ref{Soll-1-2} \ref{Al-2}/\ref{Al-3})}
Die Unterhaltskosten sind bei NetApp NFS und NetApp iSCSI gleich hoch. Aus diesen Grund sind beide gleich zu bewerten.

\textbf{Bewertet: 1}

\paragraph*{\refsoll{Soll-1-2} \refsoll{Al-2} verglichen mit \refsoll{Al-4} (\ref{Soll-1-2} \ref{Al-2}/\ref{Al-4})}
Die Unterhaltskosten sind bei NetApp NFS mit 334'620.00 CHF im Vergleich zu den Unterhaltskosten von OpenStack Object Storage mit 710'288.00 CHF auf drei Jahre gerechnet erheblich günstiger. Grund dafür sind die höheren Personalkosten die bei NetApp wesentlich geringer sind als bei OpenStack Object Storage sind. Aus diesen Grund ist \refsoll{Al-2} \ref{Al-2} sehr viel bis absolut besser zu bewerten als \refsoll{Al-4} \ref{Al-4}.

\textbf{Bewertet: 8}

\paragraph*{\refsoll{Soll-1-2} \refsoll{Al-2} verglichen mit \refsoll{Al-5} (\ref{Soll-1-2} \ref{Al-2}/\ref{Al-5})}
Die Unterhaltskosten für drei Jahre sind mit 334'620.00 CHF in Vergleich zu Amazon S3 mit 450'637.17 CHF um mehr als 100'000 CHF günstiger. Grund dafür ist, dass bei Amazon S3 alle Kosten als Unterhaltskosten anfallen.
Aus diesen Grund ist \refsoll{Al-2} \ref{Al-2} etwas besser zu bewerten als \refsoll{Al-5} \ref{Al-5}.

\textbf{Bewertet: 3}

\paragraph*{\refsoll{Soll-1-2} \refsoll{Al-3} verglichen mit \refsoll{Al-4} (\ref{Soll-1-2} \ref{Al-3}/\ref{Al-4})}
Die Unterhaltskosten sind bei NetApp iSCSI mit 334'620.00 CHF im Vergleich zu den Unterhaltskosten von OpenStack Object Storage mit 710'288.00 CHF auf drei Jahre gerechnet erheblich günstiger. Grund dafür sind die höheren Personalkosten die bei NetApp wesentlich geringer sind als bei OpenStack Object Storage sind. Aus diesen Grund ist \refsoll{Al-3} \ref{Al-3} sehr viel bis absolut besser zu bewerten als \refsoll{Al-4} \ref{Al-4}.

\textbf{Bewertet: 8}

\paragraph*{\refsoll{Soll-1-2} \refsoll{Al-3} verglichen mit \refsoll{Al-5} (\ref{Soll-1-2} \ref{Al-3}/\ref{Al-5})}
Die Unterhaltskosten für drei Jahre sind mit 334'620.00 CHF in Vergleich zu Amazon S3 mit 450'637.17 CHF um mehr als 100'000 CHF günstiger. Grund dafür ist, dass bei Amazon S3 alle Kosten als Unterhaltskosten anfallen.
Aus diesen Grund ist \refsoll{Al-3} \ref{Al-3} etwas besser zu bewerten als \refsoll{Al-5} \ref{Al-5}.

\textbf{Bewertet: 3}


\paragraph*{\refsoll{Soll-1-2} \refsoll{Al-4} verglichen mit \refsoll{Al-5} (\ref{Soll-1-2} \ref{Al-4}/\ref{Al-5})}
Die Unterhaltskosten auf 3 Jahre gerechnet sind mit 710'288.00 CHF in von OpenStack Object Storage \ref{Al-4} Vergleich zu Amazon S3 \ref{Al-5} mit 450'637.17 CHF um mehr als 250'000 CHF teurer. Diese hat vor allen mit den hohen Personalkosten zu tun, die alleine für die drei Jahre 540'000 CHF betragen. Aus diesen Grund sind die \refsoll{Soll-1-2} bei \ref{Al-4} erheblich tiefer zu bewerten als bei \ref{Al-5}

\textbf{Bewertet: 1/5}


%Langlebigkeit
\paragraph*{\refsoll{Soll-1-3} \refsoll{Al-2} verglichen mit \refsoll{Al-3} (\ref{Soll-1-3} \ref{Al-2}/\ref{Al-3})}
Die Langlebigkeit ist bei beiden Alternativen gleich einzustufen.

\textbf{Bewertet: 1}

\paragraph*{\refsoll{Soll-1-3} \refsoll{Al-2} verglichen mit \refsoll{Al-4} (\ref{Soll-1-3} \ref{Al-2}/\ref{Al-4})}
Bei der gewählten NetApp NFS Konfiguration handelt sich um eine maximal ausgebaute Variante des Modells. Eine Erweiterung des Modells ist aus diesem Grund nicht. Zudem steigen die Wartungsverträge bei Speicherprodukten mit den Betrieb Jahren zum Teil stark an. Bei OpenStack Object Storage ist die Erweiterung durch weitere Server Systeme jederzeit möglich. Zudem ist man nicht von einen einzigen Hardware Lieferanten und dessen Dienstleitung und Wartungsverträge abhängig. Aus diesen Grund ist \refsoll{Al-2} \ref{Al-2} erheblich geringer zu bewerten als \refsoll{Al-4} \ref{Al-4}.

\textbf{Bewertet: 1/5}

\paragraph*{\refsoll{Soll-1-3} \refsoll{Al-2} verglichen mit \refsoll{Al-5} (\ref{Soll-1-3} \ref{Al-2}/\ref{Al-5})}
Im Vergleich zur NetApp NFS wird der Speicher bei Amazon S3 als Dienstleitung bezogen. Deshalb gibt es keinen verschleiss der Hardware, noch existiert eine Begrenzung der maximalen Speicherkapazität oder die Kosten für den notwendigen Support erhöht sich derart, dass ein Ersatz die bessere Alternative ist.
Aus diesen Grund ist die Langlebigkeit bei \refsoll{Al-2} \ref{Al-2} sehr viel geringer zu Bewerten als bei \refsoll{Al-5} \ref{Al-5}.

\textbf{Bewertet: 1/7}

\paragraph*{\refsoll{Soll-1-3} \refsoll{Al-3} verglichen mit \refsoll{Al-4} (\ref{Soll-1-3} \ref{Al-3}/\ref{Al-4})}
Bei der gewählten NetApp iSCSI Konfiguration handelt sich um eine maximal ausgebaute Variante des Modells. Eine Erweiterung des Modells ist aus diesem Grund nicht. Zudem steigen die Wartungsverträge bei Speicherprodukten mit den Betrieb Jahren zum Teil stark an. Bei OpenStack Object Storage ist die Erweiterung durch weitere Server Systeme jederzeit möglich. Zudem ist man nicht von einen einzigen Hardware Lieferanten und dessen Dienstleitung und Wartungsverträge abhängig. Aus diesen Grund ist \refsoll{Al-3} \ref{Al-3} erheblich geringer zu bewerten als \refsoll{Al-4} \ref{Al-4}.

\textbf{Bewertet: 1/5}

\paragraph*{\refsoll{Soll-1-3} \refsoll{Al-3} verglichen mit \refsoll{Al-5} (\ref{Soll-1-3} \ref{Al-3}/\ref{Al-5})}
Im Vergleich zur NetApp iSCSI wird der Speicher bei Amazon S3 als Dienstleitung bezogen. Deshalb gibt es keinen verschleiss der Hardware, noch existiert eine Begrenzung der maximalen Speicherkapazität oder die Kosten für den notwendigen Support erhöht sich derart, dass ein Ersatz die bessere Alternative ist.
Aus diesen Grund ist die Langlebigkeit bei \refsoll{Al-3} \ref{Al-3} sehr viel geringer zu Bewerten als bei \refsoll{Al-5} \ref{Al-5}

\textbf{Bewertet: 1/7}


\paragraph*{\refsoll{Soll-1-3} \refsoll{Al-4} verglichen mit \refsoll{Al-5} (\ref{Soll-1-3} \ref{Al-4}/\ref{Al-5})}
Dadurch, dass es sich bei Amazon S3 \ref{Al-5} um eine Dienstleitung handelt, muss sich der Kunde nicht um die Erneuerung der Speicherinfrastruktur kümmern. Es ist auch nicht absehbar, dass diese Dienstleistung in den nächsten 5 oder mehr Jahre von Markt verschwinden wird. Bei der Alternative OpenStack Object Storage \ref{Al-5} müssen mit den Jahren gewisse Komponenten, welche nicht mehr den aktuellen Anforderungen entsprechen oder Gebrauchserscheinungen, wie diese bei Festplatten der Fall sein kann, aufweisen ersetzt werden. Aus diesen Grund ist die \refsoll{Soll-1-3} erheblich tiefer zu bewerten als bei \ref{Al-5}

\textbf{Bewertet: 1/5}


\subsubsection{Verfügbarkeit}

%Redundanz
\paragraph*{\refsoll{Soll-2-1} \refsoll{Al-2} verglichen mit \refsoll{Al-3} (\ref{Soll-2-1} \ref{Al-2}/\ref{Al-3})}
NetApp NFS und NetApp iSCSI haben die Redundanz über zwei Stufen gelöst. In der ersten Stufe unterscheiden sich die beiden Alternativen nicht. Die erste Stufe wird mit RAID innerhalb der NetApp gelöst, in welcher zwei Festplatten ausfallen können, ohne einen Datenverlust erleiden zu müssen. In der zweiten Stufe unterscheiden sich die Alternativen, bei NetApp NFS wird die Redundanz zwischen zwei identischen NetApp Cluster mit SnapMirror durch die NetApp selber gelöst. Bei NetApp iSCSI wird die Redundanz zwischen zwei identischen NetApp Cluster über den Volume Manager des Applikations-Servers gelöst. Beide Alternativen weisen jedoch dieselbe Redundanz auf. Aus diesen Grund sind beide gleich zu bewerten.

\textbf{Bewertet: 1}


\paragraph*{\refsoll{Soll-2-1} \refsoll{Al-2} verglichen mit \refsoll{Al-4} (\ref{Soll-2-1} \ref{Al-2}/\ref{Al-4})}
Bei der Alternative NetApp NFS wird die Redundanz in zwei Stufen gelöst. Die erste Redundanz wird mit RAID gelöst die zweite Redundanz mit SnapMirror auf einem identischen NetApp-Cluster mit ebenfalls RAID Schutz. Bei OpenStack Object Storage werden die Objekte auf drei Cluster Nodes verteilt. Im Vergleich zu OpenStack Object Storage können bei NetApp mehr Datenträger ausfallen als bei OpenStack Object Storage. Die Ausfall Wahrscheinlichkeit ist bei beiden Alternativen aber ausreichend gering, zudem währe eine Erhöhung der Redundanz bei OpenStack Object Storage möglich. Aus diesen Grund ist \refsoll{Al-2} \ref{Al-2} etwas besser zu bewerten als \refsoll{Al-4} \ref{Al-4}.

\textbf{Bewertet: 3}

\paragraph*{\refsoll{Soll-2-1} \refsoll{Al-2} verglichen mit \refsoll{Al-5} (\ref{Soll-2-1} \ref{Al-2}/\ref{Al-5})}
Bei der Alternative NetApp NFS wird die Redundanz in zwei Stufen gelöst. Die erste Redundanz wird mit RAID gelöst die zweite Redundanz mit SnapMirror auf einem identischen NetApp-Cluster mit ebenfalls RAID Schutz. Bei Amazon S3 werden die Objekte auf drei Cluster-Nodes verteilt. Im Vergleich zu Amazon S3 können bei NetApp mehr Datenträger ausfallen als bei Amazon S3. Die Ausfall Wahrscheinlichkeit ist bei beiden Alternativen aber ausreichend gering. Aus diesen Grund ist \refsoll{Al-2} \ref{Al-2} etwas besser zu bewerten als \refsoll{Al-5} \ref{Al-5}.

\textbf{Bewertet: 3}

\paragraph*{\refsoll{Soll-2-1} \refsoll{Al-3} verglichen mit \refsoll{Al-4} (\ref{Soll-2-1} \ref{Al-3}/\ref{Al-4})}
Bei der Alternative NetApp iSCSI wird die Redundanz in zwei Stufen gelöst. Die erste Redundanz wird mit RAID gelöst die zweite Redundanz mit der Spiegelungsfunktion des Volume Managers auf eine identischen NetApp Cluster mit ebenfalls RAID Schutz. Bei OpenStack Object Storage werden die Objekte auf drei Cluster-Nodes verteilt. Im Vergleich zu OpenStack Objekt Storage können bei NetApp mehr Datenträger ausfallen als bei OpenStack Object Storage. Die Ausfall Wahrscheinlichkeit ist bei beiden Alternativen aber ausreichend gering, zudem währe eine Erhöhung der Redundanz bei OpenStack Object Storage möglich. Aus diesen Grund ist \refsoll{Al-3} \ref{Al-3} etwas besser zu bewerten als \refsoll{Al-4} \ref{Al-4}.

\textbf{Bewertet: 3}

\paragraph*{\refsoll{Soll-2-1} \refsoll{Al-3} verglichen mit \refsoll{Al-5} (\ref{Soll-2-1} \ref{Al-3}/\ref{Al-5})}
Bei der Alternative NetApp NFS wird die Redundanz in zwei Stufen gelöst. Die erste Redundanz wird mit RAID gelöst die zweite Redundanz mit der Spiegelungsfunktion des Volume Managers auf eine identischen NetApp Cluster mit ebenfalls RAID Schutz. Bei Amazon S3 werden die Objekte auf drei Cluster-Nodes verteilt. Im Vergleich zu Amazon S3 können bei NetApp mehr Datenträger ausfallen als bei Amazon S3. Die Ausfall Wahrscheinlichkeit ist bei beiden Alternativen aber ausreichend gering. Aus diesen Grund ist \refsoll{Al-3} \ref{Al-3} etwas besser zu bewerten als \refsoll{Al-5} \ref{Al-5}.

\textbf{Bewertet: 3}

\paragraph*{\refsoll{Soll-2-1} \refsoll{Al-4} verglichen mit \refsoll{Al-5} (\ref{Soll-2-1} \ref{Al-4}/\ref{Al-5})}
Beide Alternativen, OpenStack Object Storage \ref{Al-4} und Amazon S3 \ref{Al-5}, Speichern die Daten in dreifacher Redundanz. Sie sind deshalb gleich zu bewerten.

\textbf{Bewertet: 1}




%Systemverfügbarkeit
\paragraph*{\refsoll{Soll-2-2} \refsoll{Al-2} verglichen mit \refsoll{Al-3} (\ref{Soll-2-2} \ref{Al-2}/\ref{Al-3})}
Beider Alternativen NetApp NFS und NetApp iSCSI sind mit zwei NetApp Cluster, welche beide Cluster jeweils alle Komponenten redundant ausgelegt sind. Der Unterschied liegt bei den Alternativen ist der Spiegelung der Daten von einem NetApp Cluster zum anderen. Während diese bei NFS durch die NetApp selber erfolgt wird diese bei iSCSI durch den Volume Manager des Servers gelöst. Vorteil von der Lösung von iSCSI ist, dass bei einem Ausfall des eines NetApp-Clusters keinen Unterbruch gibt, während bei NFS eine manuelle Umschaltung der NFS Freigaben auf den zweiten NetApp Cluster erfordert muss. Aus diesen Grund ist \refsoll{Al-2} \ref{Al-2} erheblich schlechter zu bewerten als \refsoll{Al-3} \ref{Al-3}

\textbf{Bewertet: 1/5}

\paragraph*{\refsoll{Soll-2-2} \refsoll{Al-2} verglichen mit \refsoll{Al-4} (\ref{Soll-2-2} \ref{Al-2}/\ref{Al-4})}
OpenStack Object Storage ist so ausgelegt, dass alle Dienste und Daten mehrfach auf mehrere Serversysteme verteilt werden können. Bei NetApp NFS sind alle Komponenten redundant ausgelegt. Fällt jedoch der Primäre NetApp Cluster aus muss auf den Hot-Standby-Cluster manuelle umgeschaltet werden, indem die NFS Freigaben auf den Applikations-Server auf den zweiten NetApp Cluster zeigen. Bei OpenStack Object Storage ist dagegen kein manueller Eingriff erforderlich. Aus diesen Grund ist \refsoll{Al-2} \ref{Al-2} erheblich bis viel schlechter zu bewerten als \refsoll{Al-4} \ref{Al-4}.

\textbf{Bewertet: 1/6}

\paragraph*{\refsoll{Soll-2-2} \refsoll{Al-2} verglichen mit \refsoll{Al-5} (\ref{Soll-2-2} \ref{Al-2}/\ref{Al-5})}
Amazon S3 wird so ausgelegt sein, dass alle Dienste und Daten mehrfach auf mehrere Serversysteme verteilt sind. Bei NetApp NFS sind alle Komponenten redundant ausgelegt. Fällt jedoch der Primäre NetApp Cluster aus muss auf den Hot Standby Cluster manuelle umgeschaltet werden, indem die NFS Freigaben auf den Applikations-Server auf den zweiten NetApp Cluster zeigen. Bei Amazon S3 ist dagegen kein manueller Eingriff erforderlich. Die Nachteile bei Amazon S3 sind hingegen, die Abhängigkeit von der Internetverbindung. Ist die Internetverbindung nicht verfügbar oder zu stark ausgelastet, kann die Systemverfügbarkeit leiden.

Aus diesen Grund ist \refsoll{Al-2} \ref{Al-2} etwas schlechter zu bewerten als \refsoll{Al-5} \ref{Al-5}.

\textbf{Bewertet: 1/3}

\paragraph*{\refsoll{Soll-2-2} \refsoll{Al-3} verglichen mit \refsoll{Al-4} (\ref{Soll-2-2} \ref{Al-3}/\ref{Al-4})}
OpenStack Object Storage ist so ausgelegt, dass alle Dienste und Daten mehrfach auf mehrere Serversysteme verteilt werden können. Bei NetApp iSCSI sind alle Komponenten redundant ausgelegt. Durch die dreifache Isolierung der Daten und Dienste bei OpenStack Object Storage ist \refsoll{Al-3} \ref{Al-3} etwas geringer zu bewerten als \refsoll{Al-4} \ref{Al-4}.

\textbf{Bewertet: 1/3}

\paragraph*{\refsoll{Soll-2-2} \refsoll{Al-3} verglichen mit \refsoll{Al-5} (\ref{Soll-2-2} \ref{Al-3}/\ref{Al-5})}
Amazon S3 wird so ausgelegt sein, dass alle Dienste und Daten mehrfach auf mehrere Server-Systeme verteilt sind. Bei NetApp iSCSI sind alle Komponenten redundant ausgelegt. Die Nachteile bei Amazon S3 sind hingegen, die Abhängigkeit von der Internet Verbindung. Ist die Internetverbindung nicht verfügbar oder zu stark ausgelastet, kann die Systemverfügbarkeit leiden.
Aus diesen Grund ist \refsoll{Al-3} \ref{Al-3} besser zu bewerten als \refsoll{Al-5} \ref{Al-5}.

\textbf{Bewertet: 3}


\paragraph*{\refsoll{Soll-2-2} \refsoll{Al-4} verglichen mit \refsoll{Al-5} (\ref{Soll-2-2} \ref{Al-4}/\ref{Al-5})}
Beide Alternativen sind von der Architektur und System Verfügbarkeit gleich ausgelegt. Durch den möglichen Betrieb von Applikations-Server und Speicherlösung in derselben Infrastruktur kann die Ausfallsicherheit zwischen Applikations-Server und Speicherlösung redundanter gestaltet werden. Aus diesen Grund ist \refsoll{Al-4} \ref{Al-4} erheblich bis sehr viel besser zu bewerten als \refsoll{Al-5} \ref{Al-5}.

\textbf{Bewertet: 7}


%Stanndortübergreifend
\paragraph*{\refsoll{Soll-2-3} \refsoll{Al-2} verglichen mit \refsoll{Al-3} (\ref{Soll-2-3} \ref{Al-2}/\ref{Al-3})}
Beide Alternativen sind Standort übergreifen bei NetApp NFS erfolgt jedoch keinen Automatischen Umschaltung bei einem Ausfall eines Standorts. Aus diesen Grund ist \refsoll{Al-2} \ref{Al-2} sehr viel schlechter zu bewerten als \refsoll{Al-3} \ref{Al-3}.

\textbf{Bewertet: 1/7}

\paragraph*{\refsoll{Soll-2-3} \refsoll{Al-2} verglichen mit \refsoll{Al-4} (\ref{Soll-2-3} \ref{Al-2}/\ref{Al-4})}
Beide Alternativen sind Standort übergreifen bei NetApp NFS erfolgt jedoch keinen Automatischen Umschaltung bei einem Ausfall eines Standorts. Aus diesen Grund ist \refsoll{Al-2} \ref{Al-2} sehr viel schlechter zu bewerten als \refsoll{Al-4} \ref{Al-4}.

\textbf{Bewertet: 1/7}

\paragraph*{\refsoll{Soll-2-3} \refsoll{Al-2} verglichen mit \refsoll{Al-5} (\ref{Soll-2-3} \ref{Al-2}/\ref{Al-5})}
Beide Alternativen stellen die Verfügbarkeit der Daten über mindestens zwei Standorte zur Verfügung, bei NetApp NFS erfolgt jedoch keinen Automatischen Umschaltung bei einem Ausfall eines Standorts. Die genaue Standort übergreifende Architektur ist jedoch nicht bekannt. Mangels automatischer Umschaltung ist \refsoll{Al-2} \ref{Al-2} erheblich bis sehr viel schlechter zu bewerten als \refsoll{Al-5} \ref{Al-5}.

\textbf{Bewertet: 1/5}

\paragraph*{\refsoll{Soll-2-3} \refsoll{Al-3} verglichen mit \refsoll{Al-4} (\ref{Soll-2-3} \ref{Al-3}/\ref{Al-4})}
Beide Alternativen stellen die Verfügbarkeit der Daten über mindestens zwei Standorte und verfügen über eine automatische Umschaltung. Aus diesen Grund sind beide gleich zu bewerten.

\textbf{Bewertet: 1}

\paragraph*{\refsoll{Soll-2-3} \refsoll{Al-3} verglichen mit \refsoll{Al-5} (\ref{Soll-2-3} \ref{Al-3}/\ref{Al-5})}
Beide Alternativen stellen die Verfügbarkeit der Daten über mindestens zwei Standorte zur Verfügung. Die genaue Standort übergreifende Architektur ist jedoch nicht bekannt. Aus diesen Grund ist \refsoll{Al-3} \ref{Al-3} etwas besser zu bewerten als \refsoll{Al-5} \ref{Al-5}.

\textbf{Bewertet: 2}

\paragraph*{\refsoll{Soll-2-3} \refsoll{Al-4} verglichen mit \refsoll{Al-5} (\ref{Soll-2-3} \ref{Al-4}/\ref{Al-5})}
Beide Alternativen stellen die Verfügbarkeit der Daten über mindestens zwei Standorte zur Verfügung. Die genaue Standort übergreifende Architektur ist jedoch nicht bekannt.  Aus diesen Grund ist \refsoll{Al-3} \ref{Al-3} etwas besser zu bewerten als \refsoll{Al-5} \ref{Al-5}.

\textbf{Bewertet: 2}


\subsubsection{Datenzugriff}

%Skalierbarkeit
\paragraph*{\refsoll{Soll-3-1} \refsoll{Al-2} verglichen mit \refsoll{Al-3} (\ref{Soll-3-1} \ref{Al-2}/\ref{Al-3})}
Gemäss Red Hat Skaliert iSCSI zusammen mit GFS beim Zugriff von Mehren Server besser als NFS. Einen direkten Vergleich auf NetApp hat dabei nicht stattgefunden. 
\cite{O'Keefe2005}

Aus diesen Grund ist die Skalierbarkeit von NetApp NFS \refsoll{Al-2} in Vergleich zu \refsoll{Al-3} etwas bis erheblich tiefer zu bewerten.

\textbf{Bewertet: 1/4}

\paragraph*{\refsoll{Soll-3-1} \refsoll{Al-2} verglichen mit \refsoll{Al-4} (\ref{Soll-3-1} \ref{Al-2}/\ref{Al-4})}
Bei NetApp NFS \ref{Al-2} ist der Datenzugriff mittels NFS stark optimiert, es kann davon ausgegangen werden, dass NetApp besser skaliert als eine gewöhnliche NAS Lösung. NFS wurde jedoch nicht für hoch skalierte Lösung entwickelt. Reicht eine NetApp NFS nicht aus für die Bewältigung der Datenzugriffe kann diese nicht durch eine weitere NetApp erweitert werden. Bei OpenStack Object Storage \ref{Al-5} handelt sich eine Lösung, welche in Hinblick auf die Skalierung entwickelt wurde. Durch Erweiterung von Daten-Proxy Servern und Daten-Server kann die Speicherlösung skaliert werden. So werden die Datenzugriffe auf mehr Server verteilt. Wie höher die Anzahl der Datenzugriffe sind, desto besser sollten die Skalierung von OpenStack Object Storage in Vergleich zur NetApp sein.
Aus diesen Grund ist \ref{Al-2} sehr viel tiefer zu bewerten als \ref{Al-4}

\textbf{Bewertet: 1/7}

\paragraph*{\refsoll{Soll-3-1} \refsoll{Al-2} verglichen mit \refsoll{Al-5} (\ref{Soll-3-1} \ref{Al-2}/\ref{Al-5})}
Bei der direkten Auslieferung der Bilddaten von Speichersystem zum Client, was bei Amazon S3 möglich ist, ist die Skalierung erheblich besser als bei NetApp NFS wo alle Bilddaten von Speichersystem über die Applikations-Server erfolgen muss. 

Nachteil von Amazon S3 ist jedoch, dass der Speicher über eine Internetverbindung bereitgestellt wird und nicht über Ethernet, welche die schlechtere Bandbreite besitzt. Dieser Nachteil kommt jedoch nur bei der Bearbeitung der Bilder für den Druck zum Tragen, wo die Bilder von Speicher auf den Applikations-Server übertragen werden müssen. Bei der Auslieferung der Bilder an den Endanwender kann der Zugriff hingegen direkt auf den Speicher erfolgen und entlastet somit die Internet Verbindung des Applikations-Servers.

Aus diesen Grund ist die Skalierung von NetApp NFS \ref{Al-2} gleich bis etwas höher zu Gewichten als \ref{Al-5}.

\textbf{Bewertet: 2}

\paragraph*{\refsoll{Soll-3-1} \refsoll{Al-3} verglichen mit \refsoll{Al-4} (\ref{Soll-3-1} \ref{Al-3}/\ref{Al-4})}
Bei NetApp iSCSI \ref{Al-3} ist der Datenzugriff mittels iSCSI optimiert, es kann davon ausgegangen werden, dass NetApp besser Skaliert als eine gewöhnliche iSCSI Lösung. Bei OpenStack Object Storage \ref{Al-5} handelt sich eine Lösung, welche in Hinblick auf die Skalierung entwickelt wurde. Durch Erweiterung von Daten-Proxy Servern und Daten-Server kann die Speicherlösung skaliert werden. So werden die Datenzugriffe auf mehr Server verteilt. Wie höher die Anzahl der Datenzugriffe sind, desto besser sollten die Skalierung von OpenStack Object Storage in Vergleich zur NetApp sein.
Aus diesen Grund ist \refsoll{Al-3} \ref{Al-3} erheblich tiefer zu bewerten als \refsoll{Al-4} \ref{Al-4}.

\textbf{Bewertet: 1/5}

\paragraph*{\refsoll{Soll-3-1} \refsoll{Al-3} verglichen mit \refsoll{Al-5} (\ref{Soll-3-1} \ref{Al-3}/\ref{Al-5})}
Bei der direkten Auslieferung der Bilddaten von Speichersystem zum Client, was bei Amazon S3 möglich ist, ist die Skalierung erheblich besser als bei NetApp iSCSI wo alle Bilddaten von Speichersystem über die Applikations-Server erfolgen muss. 

Nachteil von Amazon S3 ist jedoch, dass der Speicher über eine Internetverbindung bereitgestellt wird und nicht über Ethernet, welche die schlechtere Bandbreite besitzt. Dieser Nachteil kommt jedoch nur bei der Bearbeitung der Bilder für den Druck zum Tragen, wo die Bilder von Speicher auf den Applikations-Server übertragen werden müssen. Bei der Auslieferung der Bilder an den Endanwender kann der Zugriff hingegen direkt auf den Speicher erfolgen und entlastet somit die Internet Verbindung des Applikations-Servers.

Aus diesen Grund ist die Skalierung von \refsoll{Al-3} \ref{Al-3} erheblich höher zu Gewichten als \refsoll{Al-5} \ref{Al-5}.

\textbf{Bewertet: 5}


\paragraph*{\refsoll{Soll-3-1} \refsoll{Al-4} verglichen mit \refsoll{Al-5} (\ref{Soll-3-1} \ref{Al-4}/\ref{Al-5})}
Die Alternative OpenStack Object Storage\ref{Al-4} kann durch den Ausbau der Daten-Proxy-Server und Daten-Server im Datenzugriff skalieren. Bei Amazon S3 ist der Dienstleister für die Skalierung verantwortlich von der Architektur her sollte Amazon S3 gleich skalierbar sein wie OpenStack Object Storage.

Nachteil von Amazon S3 ist jedoch, dass der Speicher über eine Internetverbindung bereitgestellt wird und nicht über Ethernet, welche die schlechtere Bandbreite besitzt. Dieser Nachteil kommt jedoch nur bei der Bearbeitung der Bilder für den Druck zum Tragen, wo die Bilder von Speicher auf den Applikations-Server übertragen werden müssen. Bei der Auslieferung der Bilder an den Endanwender kann der Zugriff hingegen direkt auf den Speicher erfolgen und entlastet somit die Internet Verbindung des Applikations-Servers.
Deshalb ist \refsoll{Al-4} \ref{Al-4} sehr viel besser zu bewerten als \refsoll{Al-4} \ref{Al-5}

\textbf{Bewertet: 7}


%Performance
\paragraph*{\refsoll{Soll-3-2} \refsoll{Al-2} verglichen mit \refsoll{Al-3} (\ref{Soll-3-2} \ref{Al-2}/\ref{Al-3})}
Die Studie von NetApp wie die \refabb{abb:NetappIOPS} aus \refsec{DurchsatzIO} zeigt, unterscheidet sich die Performance der beiden Protokolle NFS und iSCSI im 1Gb und 10Gb mit einer NetApp Speichersystem kaum. Aus diesen Grund sind \ref{Al-2} und \ref{Al-3} gleich zu bewerten.

\textbf{Bewertet: 1}

\paragraph*{\refsoll{Soll-3-2} \refsoll{Al-2} verglichen mit \refsoll{Al-4} (\ref{Soll-3-2} \ref{Al-2}/\ref{Al-4})}
Die reine Übertragung Performance von Netapp NFS \ref{Al-2} wird bei wenig Server zugreifen in Vergleich zu OpenStack Object Storage \ref{Al-5} besser sein. Aus diesen Grund ist \ref{Al-2} erheblich besser zu bewerten als \ref{Al-4}.
 
\textbf{Bewertet: 5}

\paragraph*{\refsoll{Soll-3-2} \refsoll{Al-2} verglichen mit \refsoll{Al-5} (\ref{Soll-3-2} \ref{Al-2}/\ref{Al-5})}
Die reine Übertragung Performance von Netapp NFS \ref{Al-2} sehr viel besser als bei Amazon S3 \ref{Al-5}, da die Übertragung im selben Netzwerk stattfindet. Aus diesen Grund ist \ref{Al-2} sehr viel besser bis absolut besser zu bewerten als \ref{Al-5}.

\textbf{Bewertet: 8}

\paragraph*{\refsoll{Soll-3-2} \refsoll{Al-3} verglichen mit \refsoll{Al-4} (\ref{Soll-3-2} \ref{Al-3}/\ref{Al-4})}
Die reine Übertragung Performance von Netapp iSCSI \ref{Al-3} wird bei wenig Server zugreifen in Vergleich zu OpenStack Object Storage \ref{Al-5} besser sein. Aus diesen Grund ist \ref{Al-3} erheblich besser zu bewerten als \ref{Al-4}.

\textbf{Bewertet: 5}

\paragraph*{\refsoll{Soll-3-2} \refsoll{Al-3} verglichen mit \refsoll{Al-5} (\ref{Soll-3-2} \ref{Al-3}/\ref{Al-5})}
Die reine Übertragung Performance von Netapp iSCSI \ref{Al-3} sehr viel besser als bei Amazon S3 \ref{Al-5}, da die Übertragung im selben Netzwerk stattfindet. Aus diesen Grund ist \ref{Al-3} sehr viel besser bis absolut besser zu bewerten als \ref{Al-5}.

\textbf{Bewertet: 8}

\paragraph*{\refsoll{Soll-3-2} \refsoll{Al-4} verglichen mit \refsoll{Al-5} (\ref{Soll-3-2} \ref{Al-4}/\ref{Al-5})}
Dadurch, dass bei OpenStack Object Storage \ref{Al-4} die Speicherinfrastruktur und die Applikations-Server (Web-Server) im selben Netzwerk betrieben werden können und die Speicherinfrastuktur nicht mit anderen Kunden geteilt werden muss, ist bei \ref{Al-4} mit einer besseren und konstanteren Performance zu rechnen als bei Amazon S3 \ref{Al-5}. Bei Amazon S3 muss die Kommunikation und der Datenaustausch zwischen Applikations-Server und Speichersystem über die Internet Verbindung erfolgen. Aus diesen Grund ist die Alternative \ref{Al-4} etwas, bis viel besser zu bewerten als \ref{Al-5}.

\textbf{Bewertet: 6}


%POSIX
\paragraph*{\refsoll{Soll-3-3} \refsoll{Al-2} verglichen mit \refsoll{Al-3} (\ref{Soll-3-3} \ref{Al-2}/\ref{Al-3})}
Die Alternative NetApp iSCSI \refsoll{Al-3} hat dem Cluster Dateisystem GFS volle POSIX Unterstützung. Die Alternative NetApp NFS \refsoll{Al-2} hat aufgrund NFS eine teilweise POSIX Unterstützung. Aus Performance Gründen wird jedoch nicht alle POSIX Funktionen unterstützt, so ist zum Beispiel bei NFS nicht garantiert, dass wenn ein Prozess in eine Datei Schreibt, dass ein weiterer Prozess welche die selbe Datei liest die Änderung sieht. \cite{O'Keefe2005}

Aus diesen Grund ist die \refsoll{Soll-3-3} bei Alternative erheblich \refsoll{Al-2} geringer zu bewerten als bei \refsoll{Al-3}.

\textbf{Bewertet: 1/5}

\paragraph*{\refsoll{Soll-3-3} \refsoll{Al-2} verglichen mit \refsoll{Al-4} (\ref{Soll-3-3} \ref{Al-2}/\ref{Al-4})}
OpenStack Object Storage \ref{Al-4} unterstützt in Vergleich zu NetApp NFS \ref{Al-2} keine POSIX-IO. NetApp NFS unterstützt jedoch nicht die volle POSIX IO. Deshalb ist \ref{Al-2} erheblich besser zu bewerten als \ref{Al-4}.

\textbf{Bewertet: 5}

\paragraph*{\refsoll{Soll-3-3} \refsoll{Al-2} verglichen mit \refsoll{Al-5} (\ref{Soll-3-3} \ref{Al-2}/\ref{Al-5})}
Amazon S3 \ref{Al-5} unterstützt in Vergleich zu NetApp NFS \ref{Al-2} keine POSIX IO. NetApp NFS unterstützt jedoch nicht die volle POSIX-IO. Deshalb ist \ref{Al-2} erheblich besser zu bewerten als \ref{Al-5}.

\textbf{Bewertet: 5}

\paragraph*{\refsoll{Soll-3-3} \refsoll{Al-3} verglichen mit \refsoll{Al-4} (\ref{Soll-3-3} \ref{Al-3}/\ref{Al-4})}
OpenStack Object Storage \ref{Al-4} unterstützt in Vergleich zu NetApp iSCSI \ref{Al-3} keine oder teilweise Unterstützung POSIX-IO. Deshalb ist \ref{Al-3} absolut besser zu bewerten als \ref{Al-4}.

\textbf{Bewertet: 9}

\paragraph*{\refsoll{Soll-3-3} \refsoll{Al-3} verglichen mit \refsoll{Al-5} (\ref{Soll-3-3} \ref{Al-3}/\ref{Al-5})}
Amazon S3 \ref{Al-5} unterstützt in Vergleich zu NetApp iSCSI \ref{Al-2} keine oder teilweise Unterstützung POSIX-IO. Deshalb ist \ref{Al-3} absolut besser zu bewerten als \ref{Al-5}.

\textbf{Bewertet: 9}


\paragraph*{\refsoll{Soll-3-3} \refsoll{Al-4} verglichen mit \refsoll{Al-5} (\ref{Soll-3-3} \ref{Al-4}/\ref{Al-5})}
Beide Alternativen OpenStack Object Storage\ref{Al-4} und Amazon S3 \ref{Al-5} bieten kein POSIX-IO an. Aus diesen Grund sind beide gleich zu Werten.

\textbf{Bewertet: 1}

%Simulatner Lese Zugriff auf Objekte
\paragraph*{\refsoll{Soll-3-4} \refsoll{Al-2} verglichen mit \refsoll{Al-3} (\ref{Soll-3-4} \ref{Al-2}/\ref{Al-3})}
Beide Alternativen NetApp NFS \refsoll{Al-2} und NetApp iSCSI \refsoll{Al-3} ermöglichen es, dass die Objekte von mehreren Systemen simultan gelesen werden können. Aus diesen Grund sind beide gleich zu bewerten.

\textbf{Bewertet: 1}

\paragraph*{\refsoll{Soll-3-4} \refsoll{Al-2} verglichen mit \refsoll{Al-4} (\ref{Soll-3-4} \ref{Al-2}/\ref{Al-4})}
Beide Alternativen NetApp NFS \ref{Al-2} und OpenStack Object Storage \ref{Al-4} unterstützen den simultanen Lesezugriff auf Objekte. Aus diesen Grund sind beide gleich zu Werten.

\textbf{Bewertet: 1}

\paragraph*{\refsoll{Soll-3-4} \refsoll{Al-2} verglichen mit \refsoll{Al-5} (\ref{Soll-3-4} \ref{Al-2}/\ref{Al-5})}
Beide Alternativen NetApp NFS \ref{Al-2} und Amazon S3 \ref{Al-5} unterstützen den simultanen Lesezugriff auf Objekte. Aus diesen Grund sind beide gleich zu Werten.

\textbf{Bewertet: 1}

\paragraph*{\refsoll{Soll-3-4} \refsoll{Al-3} verglichen mit \refsoll{Al-4} (\ref{Soll-3-4} \ref{Al-3}/\ref{Al-4})}
Beide Alternativen NetApp iSCSI \ref{Al-3} (abhängig von Dateisystem und Volume Manager) und OpenStack Object Storage \ref{Al-4} unterstützen den simultanen Lesezugriff auf Objekte. Aus diesen Grund sind beide gleich zu Werten.

\textbf{Bewertet: 1}

\paragraph*{\refsoll{Soll-3-4} \refsoll{Al-3} verglichen mit \refsoll{Al-5} (\ref{Soll-3-4} \ref{Al-3}/\ref{Al-5})}
Beide Alternativen NetApp iSCSI \ref{Al-3} (abhängig von Dateisystem und Volume Manager) und Amazon S3\ref{Al-5} unterstützen den simultane Lese Zugriff auf Objekte. Aus diesen Grund sind beide gleich zu Werten.

\textbf{Bewertet: 1}


\paragraph*{\refsoll{Soll-3-4} \refsoll{Al-4} verglichen mit \refsoll{Al-5} (\ref{Soll-3-4} \ref{Al-4}/\ref{Al-5})}
Beide Alternativen OpenStack Object Storage\ref{Al-4} und Amazon S3\ref{Al-5} unterstützen den simultane Lese Zugriff auf Objekte. Aus diesen Grund sind beide gleich zu Werten.


\textbf{Bewertet: 1}


%Simulatner Schreib Zugriff auf Objekte
\paragraph*{\refsoll{Soll-3-5} \refsoll{Al-2} verglichen mit \refsoll{Al-3} (\ref{Soll-3-5} \ref{Al-2}/\ref{Al-3})}
Beide Alternativen verhindern das gleichzeitige Schreiben auf eine Datei mit einem Locking verfahren. Dadurch wird sichergestellt, dass Dateien Konsistenz bleiben. Bei NFS bis Version 3 wird das Locking von NFS Client gehalten und beim nicht mehr gebrauch dem Server mitgeteilt, dass die Datei wieder zugänglich ist. Stürzt der Client während er das Locking hält ab, kann er dem Server die Freigabe der Datei nicht mitteilen und die Datei bleibt gesperrt. 
Diese Schwäche der Sperrung der Dateien ist bei NetApp iSCSI mit GFS nicht vorhanden. Aus diesen Grund ist der simultane Schreibzugriff bei \refsoll{Al-2} \ref{Al-2} erheblich tiefer zu bewerten als bei  \refsoll{Al-3} \ref{Al-3}.

\textbf{Bewertet: 1/3}

\paragraph*{\refsoll{Soll-3-5} \refsoll{Al-2} verglichen mit \refsoll{Al-4} (\ref{Soll-3-5} \ref{Al-2}/\ref{Al-4})}
OpenStack Object Storage \ref{Al-4} kenn kein Sperrverfahren um das gleichzeitige Schreiben auf ein Objekt zu verhindern. Schreiben zwei Server gleichzeitig auf ein Objekt gewinnt die neuere Version der Änderung die andere geht verloren. Aus diesen Grund ist \refsoll{Al-2} \ref{Al-2} erheblich besser zu Gewichten als \refsoll{Al-4} \ref{Al-4}.

\textbf{Bewertet: 5}

\paragraph*{\refsoll{Soll-3-5} \refsoll{Al-2} verglichen mit \refsoll{Al-5} (\ref{Soll-3-5} \ref{Al-2}/\ref{Al-5})}
Amazon S3 \ref{Al-5} kenn kein Sperrverfahren um das gleichzeitige Schreiben auf ein Objekt zu verhindern. Schreiben zwei Server gleichzeitig auf ein Objekt gewinnt die neuere Version der Änderung die andere geht verloren. Aus diesen Grund ist \refsoll{Al-2} \ref{Al-2} erheblich besser zu Gewichten als \refsoll{Al-5} \ref{Al-5}.

\textbf{Bewertet: 5}


\paragraph*{\refsoll{Soll-3-5} \refsoll{Al-3} verglichen mit \refsoll{Al-4} (\ref{Soll-3-5} \ref{Al-3}/\ref{Al-4})}
OpenStack Object Storage \ref{Al-4} kenn kein Sperrverfahren um das gleichzeitige Schreiben auf ein Objekt zu verhindern. Schreiben zwei Server gleichzeitig auf ein Objekt gewinnt die neuere Version der Änderung die andere geht verloren. Aus diesen Grund ist \refsoll{Al-3} \ref{Al-3} sehr viel besser zu Gewichten als \refsoll{Al-4} \ref{Al-4}.

\textbf{Bewertet: 7}

\paragraph*{\refsoll{Soll-3-5} \refsoll{Al-3} verglichen mit \refsoll{Al-5} (\ref{Soll-3-5} \ref{Al-3}/\ref{Al-5})}
Amazon S3 \ref{Al-5} kenn kein Sperrverfahren um das gleichzeitige Schreiben auf ein Objekt zu verhindern. Schreiben zwei Server gleichzeitig auf ein Objekt gewinnt die neuere Version der Änderung die andere geht verloren. Aus diesen Grund ist \refsoll{Al-3} \ref{Al-3} sehr viel besser zu Gewichten als \refsoll{Al-5} \ref{Al-5}.

\textbf{Bewertet: 7}

\paragraph*{\refsoll{Soll-3-5} \refsoll{Al-4} verglichen mit \refsoll{Al-5} (\ref{Soll-3-5} \ref{Al-4}/\ref{Al-5})}
Sowohl die Alternative von OpenStack Object Storage \ref{Al-4} als auch die Alternative von Amazon S3 \ref{Al-5} verhindern nicht das gleichzeitige Schreiben auf Objekt. Die beiden schreib Vorgänge konkurrenzieren sich jedoch gegeneinander und nur das aktuellere, welches als Letzteres abschliesst bleibt erhalten. Die Lese Konsistenz nach dem Schreibvorgang, dass die neuste Version, der drei vorhandenen Replikationskopien, gelesen wird, kann bei beiden mit einer zusätzlichen Option garantiert werden, ist jedoch mit einer höheren Latenz verbunden. Aus diesen Grund sind beide Alternativen gleich zu bewerten.

\textbf{Bewertet: 1}


\subsubsection{Speicherkapazität}

%Skalierbarkeit
\paragraph*{\refsoll{Soll-4-1} \refsoll{Al-2} verglichen mit \refsoll{Al-3} (\ref{Soll-5-1} \ref{Al-2}/\ref{Al-3})}
Vom Ausbau der physischen Speicherkapazität sind beiden Alternativen aufgrund desselben Speichersystems gleich limitiert. Bei beiden Alternativen hängt die maximale Volumegrösse zudem nicht von Protokoll ab. NetApp NFS wird durch das Dateisystem von NetApp beschränkt, ein NFS Freigabe kann nicht grösser als das darunter liegende Dateisystem sein. Bei NetApp iSCSI wird die maximale Volumegrösse von verwendeten Volume Manager oder von verwendeten Dateisystem welches im erstellten Volume Manger Volume installiert wird.

Seitens NetApp gilt für eine NFS Freigabe oder iSCSI LUN folgende Beschränkung:

Im 32-Bit-Betrieb der NetApp ist die Linierung eines Aggregats bzw. Volume auf 16 Terabyte beschränkt, im 64-Bit-Betrieb, mit der Betriebssystem Version ab 8 ist, die Beschränkung bei 100 Terabyte.

Die von der NetApp iSCSI LUN können auf dem Server mit dem Volume Manager LVM zusammengefasst werden. Durch die von Red Hat maximal unterstützten 100 Terabyte wird die maximale Volumegrösse von Dateisystem GFS beschränkt.

Um höhere Speicherkapazitäten zur Verfügung zu stellen, müssen mehre NFS Freigaben oder Dateisysteme erstellt werden.

Die beiden Alternativen \refsoll{Al-2} \ref{Al-2} und \refsoll{Al-3} \ref{Al-3} sind aus diesem Grund gleich zu bewerten.

\textbf{Bewertet: 1}


\paragraph*{\refsoll{Soll-4-1} \refsoll{Al-2} verglichen mit \refsoll{Al-4} (\ref{Soll-5-1} \ref{Al-2}/\ref{Al-4})}
Bei der Alternative OpenStack \ref{Al-4} kann der zur Verfügung gestellte Speicher durch Hinzufügen von weiteren Daten-Notes erweitert werden. Die Alternative NetApp NFS ist hingegen bereits voll ausgebaut eine Skalierung währe nur durch ein Produktwechsel zum Beispiel auf NetApp FAS3210 möglich.

Aus diesen Grund ist die Skalierbarkeit von \refsoll{Al-2} \ref{Al-2} erheblich bis sehr viel tiefer zu bewerten als \refsoll{Al-4} \ref{Al-4}.

\textbf{Bewertet: 1/6}

\paragraph*{\refsoll{Soll-4-1} \refsoll{Al-2} verglichen mit \refsoll{Al-5} (\ref{Soll-5-1} \ref{Al-2}/\ref{Al-5})}
Bei der Alternative Amazon S3 \ref{Al-5} gibt es durch den Bezug des Speichers als Dienstleistung keine Begrenzung in der Speicherkapazität, zudem steht der erforderliche Speicherplatz sofort zur Verfügung. Die Alternative NetApp NFS ist hingegen bereits voll ausgebaut eine Skalierung währe nur durch ein Produktwechsel zum Beispiel auf NetApp FAS3210 möglich.

Aus diesen Grund ist die Skalierbarkeit von \refsoll{Al-2} \ref{Al-2} absolut tiefer zu bewerten als \refsoll{Al-5} \ref{Al-5}.

\textbf{Bewertet: 1/9}

\paragraph*{\refsoll{Soll-4-1} \refsoll{Al-3} verglichen mit \refsoll{Al-4} (\ref{Soll-5-1} \ref{Al-3}/\ref{Al-4})}
Bei der Alternative OpenStack Object Storage \ref{Al-4} kann der zur Verfügung gestellte Speicher durch Hinzufügen von weiteren Daten-Notes erweitert werden. Die Alternative NetApp NFS ist hingegen bereits voll ausgebaut eine Skalierung währe nur durch ein Produktwechsel zum Beispiel auf NetApp FAS3210 möglich.

Aus diesen Grund ist die Skalierbarkeit von \refsoll{Al-3} \ref{Al-3} erheblich bis sehr viel tiefer zu bewerten als \refsoll{Al-4} \ref{Al-4}.

\textbf{Bewertet: 1/6}

\paragraph*{\refsoll{Soll-4-1} \refsoll{Al-3} verglichen mit \refsoll{Al-5} (\ref{Soll-5-1} \ref{Al-3}/\ref{Al-5})}
Bei der Alternative Amazon S3 \ref{Al-5} gibt es durch den Bezug des Speichers als Dienstleistung keine Begrenzung in der Speicherkapazität, zudem steht der erforderliche Speicherplatz sofort zur Verfügung. Die Alternative NetApp NFS ist hingegen bereits voll ausgebaut eine Skalierung währe nur durch ein Produktwechsel zum Beispiel auf NetApp FAS3210 möglich.

Aus diesen Grund ist die Skalierbarkeit von \refsoll{Al-3} \ref{Al-3} absolut tiefer zu bewerten als \refsoll{Al-5} \ref{Al-5}.

\textbf{Bewertet: 1/9}

\paragraph*{\refsoll{Soll-4-1} \refsoll{Al-4} verglichen mit \refsoll{Al-5} (\ref{Soll-5-1} \ref{Al-4}/\ref{Al-5})}
Von der Architektur her skalieren bei Alternativen OpenStack Object Storage \ref{Al-4} und Amazon S3 \ref{Al-5} gleich gut. Durch den Bezug des Speichers als Dienstleitung muss man sich bei Amazon S3 \ref{Al-5} nicht um einen Ausbau kümmern, sondern wird von Dienstleiter geplant und umgesetzt. Bei \ref{Al-4} muss man selber den Speicher überwachen und frühzeitig den Ausbau planen. Die eingesetzten Server können alle mit zwei weiteren Festplatten ausgerüstet werden, sollte der Speicherbedarf noch grösser sein, müssen weitere Server installiert werden. Aus diesen Grund ist die Skalierung bei \ref{A-4} etwas tiefer zu bewerten als \ref{Al-5}.

\textbf{Bewertet: 1/3}


%Max Anzahl Objekte
\paragraph*{\refsoll{Soll-4-2} \refsoll{Al-2} verglichen mit \refsoll{Al-3} (\ref{Soll-5-2} \ref{Al-2}/\ref{Al-3})}
Bei der Alternative NetApp NFS können maximal 3'355'443'200 Objekte erstellt werden. Bei der Alternative NetApp iSCSI ist die Maximale Anzahl von Dateisystem GFS abhängig, bei welcher die Maximale Anzahl dynamisch erstellt werden Aus diesen Grund sind \refsoll{Al-2} \ref{Al-2} etwas schlechter als \refsoll{Al-3} \ref{Al-3} zu bewerten.

\textbf{Bewertet: 1/3}

\paragraph*{\refsoll{Soll-4-2} \refsoll{Al-2} verglichen mit \refsoll{Al-4} (\ref{Soll-5-2} \ref{Al-2}/\ref{Al-4})}
Die Maximale Anzahl an Objekten mit 3'355'443'200 Objekten ist bei NetApp NFS mehr als ausreichen hoch. Im Vergleich dazu gibt es bei OpenStack Object Storage keine Begrenzung der Anzahl Objekte. Aus diesen Grund ist \refsoll{Al-2} \ref{Al-2} erheblich schlechter als \refsoll{Al-4} \ref{Al-4} zu bewerten.

\textbf{Bewertet: 1/5}

\paragraph*{\refsoll{Soll-4-2} \refsoll{Al-2} verglichen mit \refsoll{Al-5} (\ref{Soll-5-2} \ref{Al-2}/\ref{Al-5})}
Die maximale Anzahl an Objekten mit 3'355'443'200 Objekten ist bei NetApp NFS mehr als ausreichen hoch. Im Vergleich dazu gibt es bei Amazon S3 keine Begrenzung der Anzahl Objekte. Aus diesen Grund ist \refsoll{Al-2} \ref{Al-2} erheblich schlechter als \refsoll{Al-5} \ref{Al-5} zu bewerten.

\textbf{Bewertet: 1/5}

\paragraph*{\refsoll{Soll-4-2} \refsoll{Al-3} verglichen mit \refsoll{Al-4} (\ref{Soll-5-2} \ref{Al-3}/\ref{Al-4})}
Bei der Alternative NetApp iSCSI ist die maximale Anzahl speicherbare Objekte von Dateisystem GFS abhängig, bei welcher die maximale Anzahl Objekte dynamisch erstellt werden. Bei OpenStack Object Storage gibt es keine Begrenzung über die maximale Anzahl an Objekte. Aus diesen Grund ist \refsoll{Al-3} \ref{Al-3} etwas schlechter zu bewerten als \refsoll{Al-4} \ref{Al-4}.

\textbf{Bewertet: 1/3}

\paragraph*{\refsoll{Soll-4-2} \refsoll{Al-3} verglichen mit \refsoll{Al-5} (\ref{Soll-5-2} \ref{Al-3}/\ref{Al-5})}
Bei der Alternative NetApp iSCSI ist die maximale Anzahl speicherbare Objekte von Dateisystem GFS abhängig, bei welchem die maximale Anzahl Objekte dynamisch erstellt werden. Bei Amazon S3 gibt es keine Begrenzung über die maximale Anzahl an Objekte. Aus diesen Grund ist \refsoll{Al-3} \ref{Al-3} etwas schlechter als \refsoll{Al-5} \ref{Al-5} zu bewerten.

\textbf{Bewertet: 1/3}


\paragraph*{\refsoll{Soll-4-2} \refsoll{Al-4} verglichen mit \refsoll{Al-5} (\ref{Soll-5-2} \ref{Al-4}/\ref{Al-5})}
Beide Alternativen habe keine Begrenzung über die maximale Anzahl an Objekte die gespeichert werden können. Aus diesen Grund sind beide gleich zu bewerten.

\textbf{Bewertet: 1}


%Max Objekt Grösse
\paragraph*{\refsoll{Soll-4-3} \refsoll{Al-2} verglichen mit \refsoll{Al-3} (\ref{Soll-5-3} \ref{Al-2}/\ref{Al-3})}
Beide Alternativen NetApp NFS und NetApp iSCSI unterstützen eine Maximalgrösse von Objekten bis 100 TiB. Während diese bei NetApp NFS durch das Dateisystem der NetApp beschränkt ist, ist es bei NetApp iSCSI die Beschränkung des Dateisystems GFS. Beide Alternativen sind gleich zu bewerten.

\textbf{Bewertet: 1}

\paragraph*{\refsoll{Soll-4-3} \refsoll{Al-2} verglichen mit \refsoll{Al-4} (\ref{Soll-5-3} \ref{Al-2}/\ref{Al-4})}
Im Vergleich zu NetApp NFS hat OpenStack Object Storage keine Begrenzung der Objektgrösse. Bei OpenStack Object Storage werden Objekt welche grösser als 5 GiB sind in mehre Speichereinheiten aufgeteilt, für den Anwender erscheint das Objekt aber als ganzes. Aus diesen Grund ist \refsoll{Al-2} \ref{Al-2} etwas schlechter zu Bewerten als \refsoll{Al-4} \ref{Al-4}.

\textbf{Bewertet: 1/3}

\paragraph*{\refsoll{Soll-4-3} \refsoll{Al-2} verglichen mit \refsoll{Al-5} (\ref{Soll-5-3} \ref{Al-2}/\ref{Al-5})}
Bei NetApp NFS ist die maximale Objektgrösse auf 100 TiB beschränkt, bei Amazon S3 liegt die Beschränkung bei 5 GiB. Aus diesen Grund ist \refsoll{Al-2} \ref{Al-2} etwas besser zu bewerten als \refsoll{Al-5} \ref{Al-5}.

\textbf{Bewertet: 3}

\paragraph*{\refsoll{Soll-4-3} \refsoll{Al-3} verglichen mit \refsoll{Al-4} (\ref{Soll-5-3} \ref{Al-3}/\ref{Al-4})}
Im Vergleich zu NetApp NFS hat OpenStack Object Storage keine Begrenzung der Objektgrösse. Bei OpenStack Object Storage werden Objekt welche grösser als 5 GiB sind in mehre Speichereinheiten aufgeteilt, für den Anwender erscheint das Objekt aber als Ganzes. Aus diesen Grund ist \refsoll{Al-2} \ref{Al-2} etwas schlechter zu Bewerten als \refsoll{Al-4} \ref{Al-4}.

\textbf{Bewertet: 1/3}

\paragraph*{\refsoll{Soll-4-3} \refsoll{Al-3} verglichen mit \refsoll{Al-5} (\ref{Soll-5-3} \ref{Al-3}/\ref{Al-5})}
Bei NetApp iSCSI ist die maximale Objektgrösse auf 100 TiB beschränkt, bei Amazon S3 liegt die Beschränkung bei 5 GiB. Aus diesen Grund ist \refsoll{Al-3} \ref{Al-3} etwas besser zu bewerten als \refsoll{Al-5} \ref{Al-5}.

\textbf{Bewertet: 3}


\paragraph*{\refsoll{Soll-4-3} \refsoll{Al-4} verglichen mit \refsoll{Al-5} (\ref{Soll-5-3} \ref{Al-4}/\ref{Al-5})}
Bei OpenStack Object Storage \ref{Al-4} gibt es keine Begrenzung der Objektgrösse. Objekten welche grösser als 5 GiB betragen, werden im Speicher jedoch in Stücke gespeichert. Der Zugriff erfolgt jedoch auf das eine Objekt. Bei Amazon S3 \ref{Al-5} gilt dieselbe Einschränkung bezüglich der Aufteilung der Daten im Speicher. Amazon S3 beschränkt die maximal grösse eines Objektes jedoch auf 5 TiB. Aus diesen Grund ist \ref{Al-4} erheblich besser zu bewerten als \ref{Al-5}.

\textbf{Bewertet: 5}


\subsubsection{Datenschutz}

%Datenintegrität
\paragraph*{\refsoll{Soll-5-1} \refsoll{Al-2} verglichen mit \refsoll{Al-3} (\ref{Soll-5-1} \ref{Al-2}/\ref{Al-3})}
NetApp NFS und NetApp iSCSI stellt die Datenintegrität auf 4 Kilo Byte Block auf Speicher ebene sicher. Im Unterschied zu NFS werden bei iSCSI noch weitere Speicherschichten zwischen dem gespeicherten Objekt und dem Speichersystem erstellt. In den höheren Schichten kann iSCSI zusammen mit dem Dateisystem GFS die Integrität nicht sicherstellen. Eine allfällige Beschädigung der Integrität währe möglich. Aus diesen Grund ist \refsoll{Al-2} \ref{Al-2} erheblich besser zu Bewerten als \refsoll{Al-3} \ref{Al-3}.

\textbf{Bewertet: 5}

\paragraph*{\refsoll{Soll-5-1} \refsoll{Al-2} verglichen mit \refsoll{Al-4} (\ref{Soll-5-1} \ref{Al-2}/\ref{Al-4})}
Im Vergleich zur NetApp NFS wo die Integrität auf Speicherebene sichergestellt wird, wird bei OpenStack Object Storage die Integrität auf dem gespeicherten Objekt selbst sichergestellt. Zudem kann bei OpenStack Object Storage die Integrität ebenfalls bei Transfer der Daten sichergestellt werden. Aus diesen Grund ist die Datenintegrität von \refsoll{Al-2} \ref{Al-2} erheblich bis sehr viel schlechter zu bewerten als \refsoll{Al-4} \ref{Al-4}.

\textbf{Bewertet: 1/6}

\paragraph*{\refsoll{Soll-5-1} \refsoll{Al-2} verglichen mit \refsoll{Al-5} (\ref{Soll-5-1} \ref{Al-2}/\ref{Al-5})}
Im Vergleich zur NetApp NFS wo die Integrität auf Speicherebene sichergestellt wird, wird bei Amazon S3 die Integrität auf dem gespeicherten Objekt selbst sichergestellt. Zudem kann bei Amazon S3 die Integrität ebenfalls bei Transfer der Daten sichergestellt werden. Aus diesen Grund ist die Datenintegrität von \refsoll{Al-2} \ref{Al-2} erheblich bis sehr viel schlechter zu bewerten als \refsoll{Al-5} \ref{Al-5}.

\textbf{Bewertet: 1/6}


\paragraph*{\refsoll{Soll-5-1} \refsoll{Al-3} verglichen mit \refsoll{Al-4} (\ref{Soll-5-1} \ref{Al-3}/\ref{Al-4})}
Im Vergleich zur NetApp iSCSI wo die Integrität auf Speicherebene sichergestellt wird, wird bei OpenStack Object Storage die Integrität auf dem gespeicherten Objekt selbst sichergestellt. Zudem kann bei OpenStack Object Storage die Integrität ebenfalls bei Transfer der Daten sichergestellt werden. Aus diesen Grund ist die Datenintegrität von \refsoll{Al-3} \ref{Al-3} sehr viel bis absolut schlechter zu bewerten als \refsoll{Al-4} \ref{Al-4}.

\textbf{Bewertet: 1/8}


\paragraph*{\refsoll{Soll-5-1} \refsoll{Al-3} verglichen mit \refsoll{Al-5} (\ref{Soll-5-1} \ref{Al-3}/\ref{Al-5})}
Im Vergleich zur NetApp iSCSI wo die Integrität auf Speicherebene sichergestellt wird, wird bei Amazon S3 die Integrität auf dem gespeicherten Objekt selbst sichergestellt. Zudem kann bei Amazon S3 die Integrität ebenfalls bei Transfer der Daten sichergestellt werden. Aus diesen Grund ist die Datenintegrität von \refsoll{Al-3} \ref{Al-3} erheblich bis sehr viel schlechter zu bewerten als \refsoll{Al-5} \ref{Al-5}.

\textbf{Bewertet: 1/8}

\paragraph*{\refsoll{Soll-5-1} \refsoll{Al-4} verglichen mit \refsoll{Al-5} (\ref{Soll-5-1} \ref{Al-4}/\ref{Al-5})}
Beide Alternativen stellen die Integrität mittels Hash Prüfsumme beim übertragen und im Speicher sicher. Sie sind deshalb gleich zu bewerten.

\textbf{Bewertet: 1}


%Selbstheilung
\paragraph*{\refsoll{Soll-5-2} \refsoll{Al-2} verglichen mit \refsoll{Al-3} (\ref{Soll-5-2} \ref{Al-2}/\ref{Al-3})}
NetApp NFS und NetApp iSCSI können die Daten auf Speicherebene selbstheilen. Im Unterschied zu NFS werden bei iSCSI noch weitere Speicherschichten zwischen dem gespeicherten Objekt und dem Speichersystem erstellt. In den höheren Schichten kann iSCSI zusammen mit dem Dateisystem GFS die Integrität nicht sicherstellen. Eine allfällige Beschädigung der Integrität währe möglich. In diesen Fall könnten die Daten nicht selbst geheilt werden. Aus diesen Grund ist \refsoll{Al-2} \ref{Al-2} erheblich besser zu Bewerten als \refsoll{Al-3} \ref{Al-3}.

\textbf{Bewertet: 5}

\paragraph*{\refsoll{Soll-5-2} \refsoll{Al-2} verglichen mit \refsoll{Al-4} (\ref{Soll-5-2} \ref{Al-2}/\ref{Al-4})}
Beide Alternativen können die Daten selbstheilen. Im Unterschied zu NetApp NFS wo die Selbstheilung auf Speicherebene erfolgt, erfolgt bei OpenStack Object Storage die Selbstheilung auf Objekt ebene. Aus diesen Grund ist die Selbstheilung von \refsoll{Al-2} \ref{Al-2} erheblich bis sehr viel schlechter zu bewerten als \refsoll{Al-4} \ref{Al-4}.

\textbf{Bewertet: 1/6}


\paragraph*{\refsoll{Soll-5-2} \refsoll{Al-2} verglichen mit \refsoll{Al-5} (\ref{Soll-5-2} \ref{Al-2}/\ref{Al-5})}
Beide Alternativen können die Daten selbstheilen. Im Unterschied zu NetApp NFS wo die Selbstheilung auf Speicherebene erfolgt, erfolgt bei Amazon S3 die Selbstheilung auf Objekt ebene. Aus diesen Grund ist die Selbstheilung von \refsoll{Al-2} \ref{Al-2} erheblich bis sehr viel schlechter zu bewerten als \refsoll{Al-5} \ref{Al-5}.

\textbf{Bewertet: 1/6}

\paragraph*{\refsoll{Soll-5-2} \refsoll{Al-3} verglichen mit \refsoll{Al-4} (\ref{Soll-5-2} \ref{Al-3}/\ref{Al-4})}
Beide Alternativen können die Daten selbstheilen. Im Unterschied zu NetApp iSCSI wo die Selbstheilung auf Speicherebene erfolgt, erfolgt bei OpenStack Object Storage die Selbstheilung auf Objekt ebene. Zudem kann die Integrität bei iSCSI durch die Zusätzliche Speicherschichten auf einer höheren Ebene verletzt werden, in diesen Fall hätte eine Selbstheilung auf Speicherebene keine Wirkung. Aus diesen Grund ist die Selbstheilung von \refsoll{Al-3} \ref{Al-3} sehr viel bis absolut schlechter zu bewerten als \refsoll{Al-4} \ref{Al-4}.

\textbf{Bewertet: 1/8}

\paragraph*{\refsoll{Soll-5-2} \refsoll{Al-3} verglichen mit \refsoll{Al-5} (\ref{Soll-5-2} \ref{Al-3}/\ref{Al-5})}
Beide Alternativen können die Daten selbstheilen. Im Unterschied zu NetApp iSCSI wo die Selbstheilung auf Speicherebene erfolgt, erfolgt bei OpenStack Object Storage die Selbstheilung auf Objekt ebene. Zudem kann die Integrität bei iSCSI durch die Zusätzliche Speicherschichten auf einer höheren Ebene verletzt werden, in diesen Fall hätte eine Selbstheilung auf Speicherebene keine Wirkung. Aus diesen Grund ist die Selbstheilung von \refsoll{Al-3} \ref{Al-3} sehr viel bis absolut schlechter zu bewerten als \refsoll{Al-5} \ref{Al-5}.

\textbf{Bewertet: 1/8}


\paragraph*{\refsoll{Soll-5-2} \refsoll{Al-4} verglichen mit \refsoll{Al-5} (\ref{Soll-5-2} \ref{Al-4}/\ref{Al-5})}
Beide Alternativen untersuchen in regelmässigen abständen die gespeicherten Replikations-Kopien anhand der gespeicherten Hash Prüfsumme und stellen diese bei nicht mehr integren Kopien von einer integren Kopie wieder her. Beide Alternativen sind deshalb gleich zu bewerten.

\textbf{Bewertet: 1}


%Datensicherung
\paragraph*{\refsoll{Soll-5-3} \refsoll{Al-2} verglichen mit \refsoll{Al-3} (\ref{Soll-5-3} \ref{Al-2}/\ref{Al-3})}
Beide Alternativen können über SnapShot oder NDMP gesichert werden. Bei der Alternative NetApp iSCSI \ref{Al-3} wird dabei jedoch die ganze LUN Datei auf der NetApp gesichert, dadurch ist eine Wiederherstellung nur von ganzen LUN und nicht von einzelnen Dateien möglich. Die Sicherung von einzelnen Bilddaten ist bei \ref{Al-3} mit gewöhnlicher Sicherungssoftware über die Applikations-Server möglich.
Durch die direkte Sicherung der Bilddaten über das Speichersystem ist die Alternative \refsoll{Al-2} \ref{Al-2} erheblich höher zu Gewichten als \refsoll{Al-3} \ref{Al-3}.

\textbf{Bewertet: 5}

\paragraph*{\refsoll{Soll-5-3} \refsoll{Al-2} verglichen mit \refsoll{Al-4} (\ref{Soll-5-3} \ref{Al-2}/\ref{Al-4})}
Die Alternative OpentStack Object Storage \ref{Al-4} bietet neben der Redundanz kein weiteres Sicherungsverfahren an. Aus diesen Grund ist \refsoll{Al-2} \ref{Al-2} sehr viel bis absolut höher zu bewerten als \refsoll{Al-4} \ref{Al-4}.

\textbf{Bewertet: 8}

\paragraph*{\refsoll{Soll-5-3} \refsoll{Al-2} verglichen mit \refsoll{Al-5} (\ref{Soll-5-3} \ref{Al-2}/\ref{Al-5})}
Die Alternative Amazon S3 \ref{Al-5} bietet neben der Redundanz kein weiteres Sicherungsverfahren an. Aus diesen Grund ist \refsoll{Al-2} \ref{Al-2} sehr viel bis absolut höher zu bewerten als \refsoll{Al-5} \ref{Al-5}.

\textbf{Bewertet: 8}

\paragraph*{\refsoll{Soll-5-3} \refsoll{Al-3} verglichen mit \refsoll{Al-4} (\ref{Soll-5-3} \ref{Al-3}/\ref{Al-4})}
Die Alternative OpentStack Object Storage \ref{Al-4} bietet neben der Redundanz kein weiteres Sicherungsverfahren an. Aus diesen Grund ist \refsoll{Al-3} \ref{Al-3} erheblich bis sehr höher zu bewerten als \refsoll{Al-4} \ref{Al-4}.

\textbf{Bewertet: 6}

\paragraph*{\refsoll{Soll-5-3} \refsoll{Al-3} verglichen mit \refsoll{Al-5} (\ref{Soll-5-3} \ref{Al-3}/\ref{Al-5})}
Die Alternative Amazon S3 \ref{Al-5} bietet neben der Redundanz kein weiteres Sicherungsverfahren an. Aus diesen Grund ist \refsoll{Al-3} \ref{Al-3} erheblich bis sehr höher zu bewerten als \refsoll{Al-5} \ref{Al-5}.

\textbf{Bewertet: 6}


\paragraph*{\refsoll{Soll-5-3} \refsoll{Al-4} verglichen mit \refsoll{Al-5} (\ref{Soll-5-3} \ref{Al-4}/\ref{Al-5})}
Beide Alternativen bieten neben der Redundanz kein weiteres Sicherungsverfahren an. Aus diesen Grund sind beide gleich zu bewerten.

\textbf{Bewertet: 1}

%Datensicherheit
\paragraph*{\refsoll{Soll-5-4} \refsoll{Al-2} verglichen mit \refsoll{Al-3} (\ref{Soll-5-4} \ref{Al-2}/\ref{Al-3})}
Die Daten werden bei beiden Alternative NetApp NFS und NetApp iSCSI in der eigenen Infrastruktur betreiben. Zudem können bei beiden Alternativen dieselben Sicherheitsfunktionen seitens NetApp aktiviert werden.

Aus diesen Grund sind beide Alternativen \refsoll{Al-2} \ref{Al-2} und \refsoll{Al-3} \ref{Al-3} gleich zu bewerten.

\textbf{Bewertet: 1}

\paragraph*{\refsoll{Soll-5-4} \refsoll{Al-2} verglichen mit \refsoll{Al-4} (\ref{Soll-5-4} \ref{Al-2}/\ref{Al-4})}
Die Daten werden bei beiden Alternative NetApp NFS \ref{Al-2} und OpenStack Object Storage \ref{Al-4} in der eigenen Infrastruktur betreiben. Bei NetApp wird ein eigenes im Vergleich zum eingesetzten Linux Betriebssystem von OpenStack Object Storage eingesetzt, aus diesem Grund ist die Gefahr kleiner das eine Schwachstelle ausgenutzt werden kann.
Die Alternative \refsoll{Al-2} \ref{Al-2} ist im Vergleich zur Alternative \refsoll{Al-4} \ref{Al-4} etwas besser zu bewerten.

\textbf{Bewertet: 3}

\paragraph*{\refsoll{Soll-5-4} \refsoll{Al-2} verglichen mit \refsoll{Al-5} (\ref{Soll-5-4} \ref{Al-2}/\ref{Al-5})}
Bei der Alternative NetApp NFS \ref{Al-2} werden die Daten abgesehen von Rechenzentrum in der eignen Infrastruktur betrieben und müssen nicht einer Drittpartei anvertraut werden. Im Vergleich dazu vertraut man seine Daten bei der Alternative \ref{Al-5} an Amazon an. Da es sich bei Amazon um eine Amerikanisches unternehmen handelt, dass dem Patriot Act unterstellt ist, besteht die Gefahr, dass auf Verlanden von US Behörden diese, diesen ausgehändigt werden. Aus diesen Grund ist die Sicherheit der Daten bei \refsoll{Al-2}\ref{Al-2} sehr viel bis absolut höher zu bewerten als bei \refsoll{Al-5}\ref{Al-5}.

\textbf{Bewertet: 8}

\paragraph*{\refsoll{Soll-5-4} \refsoll{Al-3} verglichen mit \refsoll{Al-4} (\ref{Soll-5-4} \ref{Al-3}/\ref{Al-4})}
Die Daten werden bei beiden Alternative NetApp iSCSI \ref{Al-3} und OpenStack Object Storage \ref{Al-4} in der eigenen Infrastruktur betreiben. Bei NetApp wird ein eigenes im Vergleich zum eingesetzten Linux Betriebssystem von OpenStack Object Storage eingesetzt, aus diesem Grund ist die Gefahr kleiner das eine Schwachstelle ausgenutzt werden kann.

Die Alternative \refsoll{Al-3} \ref{Al-3} ist im Vergleich zur Alternative \refsoll{Al-4} \ref{Al-4} etwas besser zu bewerten.

\textbf{Bewertet: 3}

\paragraph*{\refsoll{Soll-5-4} \refsoll{Al-3} verglichen mit \refsoll{Al-5} (\ref{Soll-5-4} \ref{Al-3}/\ref{Al-5})}
Bei der Alternative NetApp iSCSI \ref{Al-3} werden die Daten abgesehen von Rechenzentrum in der eignen Infrastruktur betrieben und müssen nicht einer Drittpartei anvertraut werden. Im Vergleich dazu vertraut man seine Daten bei der Alternative \ref{Al-5} an Amazon an. Da es sich bei Amazon um eine Amerikanisches unternehmen handelt, dass dem Patriot Act unterstellt ist, besteht die Gefahr, dass auf Verlanden von US Behörden diese, diesen ausgehändigt werden. Aus diesen Grund ist die Sicherheit der Daten bei \refsoll{Al-3} \ref{Al-3} sehr viel bis absolut höher zu bewerten als bei \refsoll{Al-5}\ref{Al-5}.

\textbf{Bewertet: 8}

\paragraph*{\refsoll{Soll-5-4} \refsoll{Al-4} verglichen mit \refsoll{Al-5} (\ref{Soll-5-4} \ref{Al-4}/\ref{Al-5})}
Bei der Alternative OpenStack Object Storage \ref{Al-4} werden die Daten abgesehen von Rechenzentrum in der eignen Infrastruktur betrieben und müssen nicht einer Drittpartei anvertraut werden. Im Vergleich dazu vertraut man seine Daten bei der Alternative \ref{Al-5} an Amazon an. Da es sich bei Amazon um eine Amerikanisches unternehmen handelt, dass dem Patriot Act unterstellt ist, besteht die Gefahr, dass auf Verlanden von US Behörden diese, diesen ausgehändigt werden. Aus diesen Grund ist die Sicherheit der Daten bei \refsoll{Al-4} \ref{Al-4} sehr viel höher zu bewerten als bei \refsoll{Al-5} \ref{Al-5}

\textbf{Bewertet: 7}

\subsubsection{Technologie}

%Martverbreitung
\paragraph*{\refsoll{Soll-6-1} \refsoll{Al-2} verglichen mit \refsoll{Al-3} (\ref{Soll-6-1} \ref{Al-2}/\ref{Al-3})}
Bei beiden Alternativen kommt der selber Hersteller zum Einsatz. Installationen in welche der Speicher über NFS zur Verfügung gestellt wird sind aus eigenen Erfahrungen eher anzutreffen als solche die, die die Netapp für iSCSI verwenden. Durch die steigende Bandbreite im IP-Netzwerk könnte sich iSCSI vermehrt zugunsten Fibre Channel SAN verbreiten.

Die Alternative \refsoll{Al-2} \refsoll{Al-2} ist deshalb etwas höher zu bewerten als \refsoll{Al-3} \ref{Al-3}.

\textbf{Bewertet: 3}

\paragraph*{\refsoll{Soll-6-1} \refsoll{Al-2} verglichen mit \refsoll{Al-4} (\ref{Soll-6-1} \ref{Al-2}/\ref{Al-4})}
Bei beiden Alternative OpentStack Object Storage \ref{Al-4} handelt sich um eine Lösung für ein spezifischen Kunden Segment. Bei NetApp NFS hingegen handelt es sich um eine viel breiter aufgestellte Lösung, weshalb NetApp die viel grösser Marktverbreitung aufweist. Aus diesen Grund ist \refsoll{Al-2} \ref{Al-2} sehr viel besser zu bewerten als \refsoll{Al-4} \ref{Al-4}.

\textbf{Bewertet: 7}

\paragraph*{\refsoll{Soll-6-1} \refsoll{Al-2} verglichen mit \refsoll{Al-5} (\ref{Soll-6-1} \ref{Al-2}/\ref{Al-5})}
Beide Alternativen sind führend in Ihrem Marktsegment. Durch das breitere Kunden Segment hat NetApp einen grössere Marktverbreitung. Grössere Marktchancen sind jedoch eher im Markt Segment von Amazon S3 zu erwarten. Aus diesen Grund können die beiden Alternativen gleich bewertet werden.

Durch das breitere Marktsegment ist \refsoll{Al-2} \ref{Al-2} etwas besser zu bewerten als \refsoll{Al-5} \ref{Al-5}.

\textbf{Bewertet: 3}

\paragraph*{\refsoll{Soll-6-1} \refsoll{Al-3} verglichen mit \refsoll{Al-4} (\ref{Soll-6-1} \ref{Al-3}/\ref{Al-4})}
Bei beiden Alternative OpentStack Object Storage \ref{Al-4} handelt sich um eine Lösung für ein spezifischen Kunden Segment. Bei NetApp NFS hingegen handelt es sich um eine viel breiter aufgestellte Lösung, weshalb NetApp die viel grösser Marktverbreitung aufweist. Aus diesen Grund ist \refsoll{Al-3} \ref{Al-3} viel besser zu bewerten als \refsoll{Al-4} \ref{Al-4}.

\textbf{Bewertet: 5}

\paragraph*{\refsoll{Soll-6-1} \refsoll{Al-3} verglichen mit \refsoll{Al-5} (\ref{Soll-6-1} \ref{Al-3}/\ref{Al-5})}
Beide Alternativen sind führend in ihrem Marktsegment. Durch das breitere Kunden Segment hat NetApp einen grössere Marktverbreitung. Grössere Marktchancen sind jedoch eher im Markt Segment von Amazon S3 zu erwarten. Aus diesen Grund können die beiden Alternativen gleich bewertet werden.

\textbf{Bewertet: 1}


\paragraph*{\refsoll{Soll-6-1} \refsoll{Al-4} verglichen mit \refsoll{Al-5} (\ref{Soll-6-1} \ref{Al-4}/\ref{Al-5})}
Bei OpenStack Object Storage \ref{Al-4} handelt sich im Vergleich zur Amazon S3 \ref{Al-5} um eine jüngere Lösung, weshalb Amazon S3 die bekanntere Lösung der beiden sind. OpenStack Object Storage wird zunehmen von mehr und mehr Hersteller unterstützt, weshalb hier es sich zukünftig gut im Markt behaupten könnte. Da beiden Alternativen sich in einem Marktsegment befinden, welche sich stark am Entwickeln sind, ist schwer vorhersagbar, wie sich der Markt entwickeln wird. Zurzeit ist \ref{Al-5}noch etwas tiefer zu bewerten als \ref{Al-6}

\textbf{Bewertet: 1/3}


%Weiterentwicklung
\paragraph*{\refsoll{Soll-6-2} \refsoll{Al-2} verglichen mit \refsoll{Al-3} (\ref{Soll-6-2} \ref{Al-2}/\ref{Al-3})}
NetApp gilt als einer der innovativsten Hersteller in seinem Marktsegment. In Bezug auf Daten Verwaltung, wird bei einer Weiterentwicklung eher NFS profitieren als iSCSI, da dort die Daten in einer LUN Datei gekapselt sind. Auch ist NetApp bei der pNFS Entwicklung beteiligt.

Aus diesen Grund ist \refsoll{Al-2} \ref{Al-2} etwas höher zu bewerten als \refsoll{Al-3} \ref{Al-3}.
 
\textbf{Bewertet: 3}

\paragraph*{\refsoll{Soll-6-2} \refsoll{Al-2} verglichen mit \refsoll{Al-4} (\ref{Soll-6-2} \ref{Al-2}/\ref{Al-4})}
Da es sich bei OpenStack Object Storage um eine relative junge Speichertechnologie handelt welche aktuell relative viel Aufmerksamkeit geniest, hat es mehr potenzial in der Weiterentwicklung als NetApp NFS. Aus diesen Grund ist \refsoll{Al-2} \ref{Al-2} viel geringer zu bewerten als \refsoll{Al-4} \ref{Al-4}.

\textbf{Bewertet: 1/5}

\paragraph*{\refsoll{Soll-6-2} \refsoll{Al-2} verglichen mit \refsoll{Al-5} (\ref{Soll-6-2} \ref{Al-2}/\ref{Al-5})}
Wie OpenStack Object Storage handelt sich bei Amazon S3 \ref{Al-5} um eine junge Speichertechnologie die aktuell relative viel Aufmerksamkeit geniest, es hat ebenfalls ein höheres potenzial für Weiterentwicklung als NetApp NFS, anders als OpentStack Object Storage ist Amazon S3 alleine in der Weiterentwicklung von Amazon S3. Aus diesen ist \refsoll{Al-2} \ref{Al-2} etwas geringer zu bewerten als \refsoll{Al-5} \ref{Al-5}.

\textbf{Bewertet: 1/3}

\paragraph*{\refsoll{Soll-6-2} \refsoll{Al-3} verglichen mit \refsoll{Al-4} (\ref{Soll-6-2} \ref{Al-3}/\ref{Al-4})}
Da es sich bei OpenStack Object Storage um eine relative junge Speichertechnologie handelt, welche aktuell relative viel Aufmerksamkeit auf sich zieht, hat es mehr potenzial in der Weiterentwicklung als NetApp iSCSI. Aus diesen Grund ist \refsoll{Al-3} \ref{Al-3} sehr viel geringer zu bewerten als \refsoll{Al-4} \ref{Al-4}.

\textbf{Bewertet: 1/7}

\paragraph*{\refsoll{Soll-6-2} \refsoll{Al-3} verglichen mit \refsoll{Al-5} (\ref{Soll-6-2} \ref{Al-3}/\ref{Al-5})}
Wie OpenStack Object Storage handelt sich bei Amazon S3 \ref{Al-5} um eine junge Speichertechnologie die aktuell relative viel Aufmerksamkeit geniest, es hat ebenfalls ein höheres potenzial für Weiterentwicklung als NetApp NFS, anders als OpentStack Object Storage ist Amazon S3 alleine in der Weiterentwicklung von Amazon S3. Aus diesen ist \refsoll{Al-2} \ref{Al-2} geringer zu bewerten als \refsoll{Al-5} \ref{Al-5}.

\textbf{Bewertet: 1/5}

\paragraph*{\refsoll{Soll-6-2} \refsoll{Al-4} verglichen mit \refsoll{Al-5} (\ref{Soll-6-2} \ref{Al-4}/\ref{Al-5})}
Bis anhin hat sich OpenStack Object Storage \ref{Al-4} stark an Amazon S3 \ref{Al-5} orientiert. Bei Amazon S3 handelst sich um eine Lösung, die nur von Amazon eingesetzt wird, es macht zurzeit noch den Anschein als sei Amazon die weiterentwickelte Lösung. Bei OpenStack handelt sich um eine quelloffene Lösung, an welche sich mehr und mehr namhafte Hersteller am Projekt beteiligen. Langfristig wird wahrscheinlich die Weiterentwicklung bei OpenStack Object Storage schnellere vorschritte machen als Amazon S3, diese ist aber letztendlich auch abhängig davon ob OpenStack Object Storage sich als quasi Standard behaupten kann.
Aus diesen Grund ist \ref{Al-4} etwas besser zu bewerten als \ref{Al-5}.

\textbf{Bewertet: 3}


%Verfügbarkeit von Experten
\paragraph*{\refsoll{Soll-6-3} \refsoll{Al-2} verglichen mit \refsoll{Al-3} (\ref{Soll-6-3} \ref{Al-2}/\ref{Al-3})}
Beide Alternativen basieren auf demselben NetApp Speicher. Die NetApp Produkte sind in der Schweiz gut verbreitet, zudem unterhält NetApp ein gut ausgebautes Partnernetzwerk in der Schweiz. Bei der Mehrheit der Installationen in der Schweiz wird der Speicher mittels NFS oder CIFS Protokoll freigegeben. Aus diesen Grund ist \refsoll{Al-2} etwas höher zu Bewerten als \refsoll{Al-3} \ref{Al-3}

\textbf{Bewertet: 3}

\paragraph*{\refsoll{Soll-6-3} \refsoll{Al-2} verglichen mit \refsoll{Al-4} (\ref{Soll-6-3} \ref{Al-2}/\ref{Al-4})}
Gemäss eignenden Recherchen gibt es in der Schweiz kaum bis sehr wenige Experten die sich mit OpenStack Object Storage \ref{Al-4} auskennen. Die NetApp Produkte sind in der Schweiz viel stärker verbreitet als OpenStack Object Storage, zudem unterhält NetApp in der Schweiz ein gut ausgebautes Partnernetzwerk in der Schweiz. Aus Grund ist \refsoll{Al-2} \ref{Al-2} absolut hoher zu bewerten als \refsoll{Al-4} \ref{Al-4}.

\textbf{Bewertet: 9}

\paragraph*{\refsoll{Soll-6-3} \refsoll{Al-2} verglichen mit \refsoll{Al-5} (\ref{Soll-6-3} \ref{Al-2}/\ref{Al-5})}
NetApp unterhält in der Schweiz ein gut ausgebautes Partnernetzwerk mit geschulten Experten. Amazon unterhält in der Schweiz kein solches Partnernetzwerk im Vergleich zur NetApp ist aber bei Amazon S3 erheblich weniger Wissen für den Betrieb notwendig. Aus diesen Grund ist \refsoll{Al-2} \ref{Al-2} erheblich besser zu bewerten als \refsoll{Al-5} \ref{Al-5}.

\textbf{Bewertet: 5}

\paragraph*{\refsoll{Soll-6-3} \refsoll{Al-3} verglichen mit \refsoll{Al-4} (\ref{Soll-6-3} \ref{Al-3}/\ref{Al-4})}
Gemäss eigenen Recherchen gibt es in der Schweiz kaum bis sehr wenige Experten die sich mit OpenStack Object Storage \ref{Al-4} auskennen. Die NetApp Produkte sind in der Schweiz viel stärker verbreitet als OpenStack Object Storage, zudem unterhält NetApp in der Schweiz ein gut ausgebautes Partnernetzwerk in der Schweiz. Bei den mehr meisten Installationen in der Schweiz werden mehrheitlich den Speicher per NFS oder CIFS freigegeben. Aus diesen Grund ist \refsoll{Al-3} \ref{Al-3} sehr viel hoher zu bewerten als \refsoll{Al-3} \ref{Al-4}.

\textbf{Bewertet: 7}


\paragraph*{\refsoll{Soll-6-3} \refsoll{Al-3} verglichen mit \refsoll{Al-5} (\ref{Soll-6-3} \ref{Al-3}/\ref{Al-5})}
NetApp unterhält in der Schweiz ein gut ausgebautes Partnernetzwerk mit geschulten Experten. Amazon unterhält in der Schweiz kein solches Partnernetzwerk im Vergleich zur NetApp ist aber bei Amazon S3 erheblich weniger Wissen für den Betrieb notwendig. Aus diesen Grund ist \refsoll{Al-3} \ref{Al-3} besser zu bewerten als \refsoll{Al-5} \ref{Al-5}.

\textbf{Bewertet: 3}


\paragraph*{\refsoll{Soll-6-3} \refsoll{Al-4} verglichen mit \refsoll{Al-5} (\ref{Soll-6-3} \ref{Al-4}/\ref{Al-5})}
Die Anforderungen an Experten ist bei beiden Alternativen \ref{Al-4} und \ref{Al-5} stark unterschiedlich. Während es für Amazon S3 Experten für die Einbindung der Applikation an das API benötigt, sind bei OpenStack Object Storage ebenfalls Experten für die Implementierung und Betrieb der Infrastruktur notwendig. Zudem sind die Experten in der Schweiz für OpenStack noch sehr rar. Aus diesen Grund ist \ref{Al-5} erheblich bis sehr viel tiefer zu bewerten als \ref{Al-5}.

\textbf{Bewertet: 1/6}

%Verwaltungskomfort
\paragraph*{\refsoll{Soll-6-4} \refsoll{Al-2} verglichen mit \refsoll{Al-3} (\ref{Soll-6-4} \ref{Al-2}/\ref{Al-3})}
Mit Ausnahme dem Erstellen der iSCSI LUN auf der NetApp ist bei iSCSI der meisten Verwaltungsaufgaben auf den Applikation Server erforderliche. Bei NetApp NFS hingegen sind die Verwaltungsaufgaben des Speichers eher auf Seite der NetApp angesiedelt. Für die Bearbeitung von NFS stehen vonseiten NetApp mehr Werkzeuge zur Verfügung als bei iSCSI. Der Verwaltungskomfort ist deshalb bei \refsoll{Al-2} etwas besser zu bewerten als bei \refsoll{Al-3}

\textbf{Bewertet: 3}

\paragraph*{\refsoll{Soll-6-4} \refsoll{Al-2} verglichen mit \refsoll{Al-4} (\ref{Soll-6-4} \ref{Al-2}/\ref{Al-4})}
Bei NetApp NFS stehen gute mehr oder weniger einfach zu bedienend Verwaltungs-Werkzuge zur Verfügung. Die Schwächen von NetApp sind jedoch, das mehre Werkzeuge zur Verfügung stehen, die zum überschneidend Aufgaben erfüllen. Bei OpenStack Object Storage stehen mit Ausnahme von Dritthersteller nur Kommando Zeilen Werkzeuge zur Verfügung. 
Aus diesen Grund ist \refsoll{Al-2} \ref{Al-2} gegenüber \refsoll{Al-4} \ref{Al-4} erheblich besser zu bewerten.
 
\textbf{Bewertet: 5}

\paragraph*{\refsoll{Soll-6-4} \refsoll{Al-2} verglichen mit \refsoll{Al-5} (\ref{Soll-6-4} \ref{Al-2}/\ref{Al-5})}
Bei Amazon S3 handelt es sich im Speicher als Dienstleistung, der Betrieb des Speichersystems wird Amazon überlassen. Für den Kunden fallen nur wenig Verwaltungsaufgaben an die in einen Guten und übersichtlichen Webinterface erfolgen. Durch den eigenen Betrieb ist bei NetApp NFS mehr Verwaltungsaufgaben erforderlich. Aus diesen Grund ist \refsoll{Al-2} \ref{Al-2} erheblich geringer zu bewerten als \refsoll{Al-5} \ref{Al-5}.

\textbf{Bewertet: 1/5}


\paragraph*{\refsoll{Soll-6-4} \refsoll{Al-3} verglichen mit \refsoll{Al-4} (\ref{Soll-6-4} \ref{Al-3}/\ref{Al-4})}
Bei NetApp iSCSI stehen gute mehr oder weniger einfach zu bedienend Verwaltungs-Werkzuge zur Verfügung. Die Schwächen von NetApp sind jedoch, das mehre Werkzeuge zur Verfügung stehen, die zum überschneidend Aufgaben erfüllen. Bei OpenStack Object Storage stehen mit Ausnahme von Dritthersteller nur Kommando Zeilen Werkzeuge zur Verfügung. 
Aus diesen Grund ist \refsoll{Al-3} \ref{Al-3} gegenüber \refsoll{Al-4} \ref{Al-4} etwas bis erheblich besser zu bewerten.
 
\textbf{Bewertet: 4}


\paragraph*{\refsoll{Soll-6-4} \refsoll{Al-3} verglichen mit \refsoll{Al-5} (\ref{Soll-6-4} \ref{Al-3}/\ref{Al-5})}
Bei Amazon S3 handelt es sich im Speicher als Dienstleistung, der Betrieb des Speichersystems wird Amazon überlassen. Für den Kunden fallen nur wenig Verwaltungsaufgaben an die in einen Guten und übersichtlichen Webinterface erfolgen. Durch den eigenen Betrieb sind bei NetApp iSCSI mehr Verwaltungsaufgaben erforderlich. Aus diesen Grund ist \refsoll{Al-3} \ref{Al-3} erheblich bis sehr viel geringer zu bewerten als \refsoll{Al-5} \ref{Al-5}.

\textbf{Bewertet: 1/6}


\paragraph*{\refsoll{Soll-6-4} \refsoll{Al-4} verglichen mit \refsoll{Al-5} (\ref{Soll-6-4} \ref{Al-4}/\ref{Al-5})}
Bei Amazon S3 handelt es sich im Speicher als Dienstleistung, der Betrieb des Speichersystems wird Amazon überlassen. Für den Kunden fallen nur wenig Verwaltungsaufgaben an die in einen Guten und übersichtlichen Webinterface erfolgen. Bei OpenStack Object Storage müssen alle Verwaltungsaufgaben selber durchgeführt werden. Für die Verwaltung steht zurzeit mit ausnahmen von Dritthersteller Lösungen nur die Kommandozeilen Werkzeuge (engl. Tools) zur Verfügung. Der Verwaltungskomfort ist bei \refsoll{Al-4} \ref{Al-4} sehr viel bis absolut geringer als bei \ref{Al-5} \refsoll{Al-5}.

\textbf{Bewertet: 1/8}


%Ausgereift
\paragraph*{\refsoll{Soll-6-5} \refsoll{Al-2} verglichen mit \refsoll{Al-3} (\ref{Soll-6-5} \ref{Al-2}/\ref{Al-3})}
Sowohl iSCSI als auch NFS gelten als Stabile ausgereifte Protokolle. NetApp Systeme gelten ebenfalls als Stabil. Gemäss meiner Erfahrung hat sich iSCSI noch nicht so stark durchgesetzt wie NFS. Aus diesen Grund ist \refsoll{Al-2} \ref{Al-2} gleich, bis etwas besser zur bewerten als \refsoll{Al-3} \ref{Al-3}.

\textbf{Bewertet: 2}

\paragraph*{\refsoll{Soll-6-5} \refsoll{Al-2} verglichen mit \refsoll{Al-4} (\ref{Soll-6-5} \ref{Al-2}/\ref{Al-4})}
OpenStack ist eine relative junge Speicherlösung, zudem ist es eine Lösung die sich noch weiterentwickelt. Im Vergleich dazu ist NFS eine Technologie, die schon lange erhältlich ist und keine starke Weiterentwicklung erfahren hat. Sie gilt deshalb als ausgereift und stabil.
Aus diesen Grund ist \refsoll{Al-2} \ref{Al-2} sehr viel grosser zu bewerten als \refsoll{Al-4} \ref{Al-4}.

 \textbf{Bewertet: 7}

\paragraph*{\refsoll{Soll-6-5} \refsoll{Al-2} verglichen mit \refsoll{Al-5} (\ref{Soll-6-5} \ref{Al-2}/\ref{Al-5})}
Amazon S3 ist seit 2006 erhältlich und hat in dieser Zeit ein starkes Wachstum erhalten. Die Technologie die Amazon S3 verwendet kann man als ausgereift bezeichnen. Im Vergleich zu Amazon S3 ist jedoch NFS wesentlich länger erhältlich und wird bei den meisten Unix-Artigen Betriebssysteme unterstützt.
Aus diesen Grund ist \refsoll{Al-2} \ref{Al-2} besser zu bewerten als \refsoll{Al-5} \ref{Al-5}.

 \textbf{Bewertet: 3}

\paragraph*{\refsoll{Soll-6-5} \refsoll{Al-3} verglichen mit \refsoll{Al-4} (\ref{Soll-6-5} \ref{Al-3}/\ref{Al-4})}
OpenStack  Object Storage ist eine relative junge Speicherlösung, zudem ist es eine Lösung die sich noch weiterentwickelt. Im Vergleich dazu ist iSCSI eine Technologie die schon lange erhältlich ist die Weiterentwicklung für iSCSI erfahren vor allem in Netzwerkbereich. Sie gilt deshalb als ausgereift und stabil.
Aus diesen Grund ist \refsoll{Al-3} \ref{Al-3} etwas bis sehr viel grosser zu bewerten als \refsoll{Al-4} \ref{Al-4}.

 \textbf{Bewertet: 6}

\paragraph*{\refsoll{Soll-6-5} \refsoll{Al-3} verglichen mit \refsoll{Al-5} (\ref{Soll-6-5} \ref{Al-3}/\ref{Al-5})}
Amazon S3 ist seit 2006 erhältlich und hat in dieser Zeit ein starkes Wachstum erhalten. Die Technologie die Amazon S3 verwendet kann man als ausgereift bezeichnen. Im Vergleich zu Amazon S3 ist jedoch iSCSI wesentlich länger erhältlich und die gängigsten Betriebssysteme unterstützt standardmässig iSCSI. 
Aus diesen Grund ist \refsoll{Al-3} \ref{Al-3} gleich bis etwas besser zu als\refsoll{Al-5} \ref{Al-5}.

 \textbf{Bewertet: 2}

\paragraph*{\refsoll{Soll-6-5} \refsoll{Al-4} verglichen mit \refsoll{Al-5} (\ref{Soll-6-5} \ref{Al-4}/\ref{Al-5})}
Bei OpenStack Object Storage \ref{Al-4} handelt es sich um eine sehr junge Lösung, es wird jedoch bereits von RackSpace einen namhaften Webdienstleister eingesetzt. Durch das länger bestehen von Amazon S3 \ref{Al-5}, wird Amazon S3 mehr Erfahrung im Betrieb gesammelt haben als RackSpace und weshalb davon auszugehen ist das mehr Verbesserungen in Amazon S3 für die Stabilität eingeflossen sind als bei OpenStack Object Storage. Deshalb ist \ref{Al-4} erheblich tiefer zu bewerten als \ref{Al-5}.

\textbf{Bewertet: 1/5}






\cleardoublepage
\chapter{Installation OpenStack Object Storage}
\section{Generelle Konfiguration}

\begin{lstlisting}[label=paketegenerell, language=Bash, caption=Installation generelle Host Pakete ]
sudo apt-get install swift openssh-server  rsync memcached python-netifaces python-xattr python-memcache 
\end{lstlisting}

\begin{lstlisting}[label=swiftconfdir, language=Bash, caption=Erstell /etc/swift Ordner und setzt Berechtigung ]
sudo mkdir -p /etc/swift
sudo chown -R swift:swift /etc/swift/
\end{lstlisting}


\begin{lstlisting}[label=swift.conf language=Bash, caption=Swift in /etc/swift/swift.conf konfigurieren]
[swift-hash]
# random unique string that can never change (DO NOT LOSE)
swift_hash_path_suffix = KmwBretYgombitrL
\end{lstlisting}

\section{Storage Nodes Konfiguration}

\begin{lstlisting}[label=paketenode, language=Bash, caption=Installation Data-Node Pakete ]
sudo apt-get install swift-account swift-container swift-object xfsprogs
\end{lstlisting}

begin{lstlisting}[label=partition, language=Bash, caption=Partition auf zweiter Festplatte erstellen ]
sudo fdisk /dev/sdb 
Device contains neither a valid DOS partition table, nor Sun, SGI or OSF disklabel
Building a new DOS disklabel with disk identifier 0xd5f71779.
Changes will remain in memory only, until you decide to write them.
After that, of course, the previous content won't be recoverable.

Warning: invalid flag 0x0000 of partition table 4 will be corrected by w(rite)

Command (m for help): m
Command action
   a   toggle a bootable flag
   b   edit bsd disklabel
   c   toggle the dos compatibility flag
   d   delete a partition
   l   list known partition types
   m   print this menu
   n   add a new partition
   o   create a new empty DOS partition table
   p   print the partition table
   q   quit without saving changes
   s   create a new empty Sun disklabel
   t   change a partition's system id
   u   change display/entry units
   v   verify the partition table
   w   write table to disk and exit
   x   extra functionality (experts only)

Command (m for help): n
Partition type:
   p   primary (0 primary, 0 extended, 4 free)
   e   extended
Select (default p): p
Partition number (1-4, default 1): 1
First sector (2048-41943039, default 2048): 
Using default value 2048
Last sector, +sectors or +size{K,M,G} (2048-41943039, default 41943039): 
Using default value 41943039

Command (m for help): w
The partition table has been altered!

Calling ioctl() to re-read partition table.
Syncing disks.
\end{lstlisting}

\begin{lstlisting}[label=xfs, language=Bash, caption=XFS Dateisystem in der Partition anlegen ]
sudo mkfs.xfs -i size=1024 /dev/sdb1
meta-data=/dev/sdb1              isize=1024   agcount=4, agsize=1310656 blks
         =                       sectsz=512   attr=2, projid32bit=0
data     =                       bsize=4096   blocks=5242624, imaxpct=25
         =                       sunit=0      swidth=0 blks
naming   =version 2              bsize=4096   ascii-ci=0
log      =internal log           bsize=4096   blocks=2560, version=2
         =                       sectsz=512   sunit=0 blks, lazy-count=1
realtime =none                   extsz=4096   blocks=0, rtextents=0
\end{lstlisting}

\begin{lstlisting}[label=mountconfig, language=Bash, caption=Mount Konfiguration anlegen ]
sudo echo "/dev/sdb1 /srv/node/sdb1 xfs noatime,nodiratime,nobarrier,logbufs=8 0 0" >> /etc/fstab
\end{lstlisting}

\begin{lstlisting}[label=MountPointDisk, language=Bash, caption=Mount Anlegen, Disk Mounten und Berechtigung setzen]
sudo mkdir -p /srv/node/sdb1
sudo mount /srv/node/sdb1
sudo chown -R swift:swift /srv/node
\end{lstlisting}

\begin{lstlisting}[label=rsyncConf, language=Bash, caption=Rsyncd in /etc/rsyncd.conf konfigurieren]
uid = swift
gid = swift
log file = /var/log/rsyncd.log
pid file = /var/run/rsyncd.pid
address = 172.16.251.90 

[account]
max connections = 2
path = /srv/node/
read only = false
lock file = /var/lock/account.lock

[container]
max connections = 2
path = /srv/node/
read only = false
lock file = /var/lock/container.lock

[object]
max connections = 2
path = /srv/node/
read only = false
lock file = /var/lock/object.lock
\end{lstlisting}

\begin{lstlisting}[label=configRsyncDefault, language=Bash, caption=Rsync Aktivieren in /etc/default/rsync]
RSYNC_ENABLE = true
\end{lstlisting}

\begin{lstlisting}[label=startRsynx, language=Bash, caption=Rsync Starten]
sudo service rsync start
\end{lstlisting}

\begin{lstlisting}[label=account-server.conf, language=Bash, caption=Account Server in /etc/swift/account-server.conf konfigurieren]
[DEFAULT]
bind_ip = 172.16.251.90  # IP Addresse des Node Servers
workers = 2

[pipeline:main]
pipeline = account-server

[app:account-server]
use = egg:swift#account

[account-replicator]

[account-auditor]

[account-reaper]
\end{lstlisting}


\begin{lstlisting}[label=container-server.conf, language=Bash, caption=Container Server in /etc/swift/container-server.conf konfigurieren]
[DEFAULT]
bind_ip = 172.16.251.90 # IP Addresse des Node Servers
workers = 2

[pipeline:main]
pipeline = container-server

[app:container-server]
use = egg:swift#container

[container-replicator]

[container-updater]

[container-auditor]
\end{lstlisting}


\begin{lstlisting}[label=object-server.conf, language=Bash, caption=Object Server in /etc/swift/object-server.conf konfigurieren]
[DEFAULT]
bind_ip = 172.16.251.90 # IP Addresse des Node Servers
workers = 2

[pipeline:main]
pipeline = object-server

[app:object-server]
use = egg:swift#object

[object-replicator]

[object-updater]

[object-auditor]
\end{lstlisting}

\section{Proxy Node Konfiguration}
\begin{lstlisting}[label=paketeProxy, language=Bash, caption=Installation Proxy-Node Pakete ]
sudo apt-get install swift-proxy memcached
\end{lstlisting}

\begin{lstlisting}[label=df, language=Bash, caption=X.509 Zertifikats in /etc/swift erstellen ]
sudo openssl req -new -x509 -nodes -out cert.crt -keyout cert.key
Generating a 1024 bit RSA private key
......++++++
.......................................++++++
writing new private key to 'cert.key'
-----
You are about to be asked to enter information that will be incorporated
into your certificate request.
What you are about to enter is what is called a Distinguished Name or a DN.
There are quite a few fields but you can leave some blank
For some fields there will be a default value,
If you enter '.', the field will be left blank.
-----
Country Name (2 letter code) [AU]:CH
State or Province Name (full name) [Some-State]:Zuerich
Locality Name (eg, city) []:Wallisellen
Organization Name (eg, company) [Internet Widgits Pty Ltd]:Stuker.biz
Organizational Unit Name (eg, section) []:Cloud Storage
Common Name (e.g. server FQDN or YOUR name) []:swift-c1.swift.stuker.biz
Email Address []:it@stuker.biz
\end{lstlisting}

\begin{lstlisting}[label=memcachedconf, language=Bash, caption=MemCached in /etc/memcached.conf konfigurieren ]
-l 127.0.0.1
in ändern
-l 172.16.251.80
\end{lstlisting}

\begin{lstlisting}[label=startMemcached, language=Bash, caption=Memcached starten]
sudo service memcached restart
\end{lstlisting}

\begin{lstlisting}[label=ring, language=Bash, caption=Account Container und Object Ring erstellen]
sudo swift-ring-builder account.builder create 18 3 1
sudo swift-ring-builder container.builder create 18 3 1
sudo swift-ring-builder object.builder create 18 3 1
\end{lstlisting}

\begin{lstlisting}[label=ring, language=Bash, caption=Server bzw. Speicher den Ringen hinzufügen]
sudo swift-ring-builder account.builder add z1-172.16.251.90:6002/sdb1 100
 sudo swift-ring-builder account.builder add z2-172.16.251.91:6002/sdb1 100
sudo swift-ring-builder account.builder add z3-172.16.251.92:6002/sdb1 100
sudo swift-ring-builder account.builder add z4-172.16.251.93:6002/sdb1 100
sudo swift-ring-builder account.builder add z5-172.16.251.94:6002/sdb1 100
sudo swift-ring-builder container.builder add z1-172.16.251.90:6001/sdb1 100
sudo swift-ring-builder container.builder add z2-172.16.251.91:6001/sdb1 100
sudo swift-ring-builder container.builder add z3-172.16.251.92:6001/sdb1 100
sudo swift-ring-builder container.builder add z4-172.16.251.93:6001/sdb1 100
sudo swift-ring-builder container.builder add z5-172.16.251.94:6001/sdb1 100
sudo swift-ring-builder object.builder add z1-172.16.251.90:6000/sdb1 100
sudo swift-ring-builder object.builder add z2-172.16.251.91:6000/sdb1 100
sudo swift-ring-builder object.builder add z3-172.16.251.92:6000/sdb1 100
sudo swift-ring-builder object.builder add z4-172.16.251.93:6000/sdb1 100
sudo swift-ring-builder object.builder add z5-172.16.251.94:6000/sdb1 100
sudo swift-ring-builder account.builder
sudo swift-ring-builder container.builder
sudo swift-ring-builder object.builder
\end{lstlisting}

\begin{lstlisting}[label=ring, language=Bash, caption=Ring rebalance]
sudo swift-ring-builder account.builder rebalance
sudo swift-ring-builder container.builder rebalance
sudo swift-ring-builder object.builder rebalance
\end{lstlisting}

\begin{lstlisting}[label=ring, language=Bash, caption=Dateien account.ring.gz, container.ring.gz, und object.ring.gz an alle Nodes verteilen und Berechtigung setzen]
  scp *.gz data01-c1-d1:/etc/swift
  scp *.gz data02-c1-d1:/etc/swift  
  scp *.gz data03-c1-d1:/etc/swift
  scp *.gz data04-c1-d2:/etc/swift
  scp *.gz data05-c1-d2:/etc/swift
  sudo chown -R swift:swift /etc/swift
\end{lstlisting}

\begin{lstlisting}[label=ring, language=Bash, caption=Proxy Dienst starten]
swift-init proxy start
\end{lstlisting}

\begin{lstlisting}[label=ring, language=Bash, caption=Auf allen Nodes die Dienste starten]
swift-init all start
\end{lstlisting}

\end{document}