%!TEX root=documentation-bachlorthesis-speicherarchitektur-lstucker.tex



\documentclass[
oneside, % einseitiges Dokument
a4paper, % Papierformat
pdftex,
fontsize=11pt,
%headsepline, % use headinclude also! (see M. Kohm)
% footsepline, % use footinclude also! (see M. Kohm)
% headinclude, % count head to text body (not to margin)
% footinclude, % count foot to text body (not to margin)
% BCOR8mm, % set extra margin for book fixation
openany,
titlepage, % es wird eine Titelseite verwendet
draft=false, % Status des Dokuments (final/draft)
ngerman % für Umlaute, Silbentrennung etc.
]{scrbook}

%\usepackage{helvet}
% Umlaute ----------------------------------------------------------------------
%   Umlaute/Sonderzeichen wie ‰¸ˆfl direkt im Quelltext verwenden (CodePage).
%   Erlaubt automatische Trennung von Worten mit Umlauten.
% ------------------------------------------------------------------------------
\usepackage[utf8]{inputenc}
\usepackage[T1]{fontenc}
\usepackage{textcomp} % Euro-Zeichen etc.

% Anpassung an Landessprache
\usepackage[ngerman]{babel} %Thomas hatte noch English drin

%!TEX root=documentation-bachlorthesis-speicherarchitektur-lstucker.tex

% meta-information -----------------------------------------------------------
%   Definition von globalen Parametern, die im gesamten Dokument verwendet
%   werden können (z.B auf dem Deckblatt etc.).
%
%   ACHTUNG: Wenn die Texte Umlaute oder ein Esszet enthalten, muss der folgende
%            Befehl bereits an dieser Stelle aktiviert werden:
%            \usepackage[latin1]{inputenc}
% ------------------------------------------------------------------------------
\newcommand{\titel}{Speicherarchitektur für Massendaten einer Webapplikation}
\newcommand{\untertitel}{Webapplikation}
\newcommand{\art}{Bachlorthesis}
\newcommand{\fachgebiet}{Betriebsysteme}
\newcommand{\autor}{Lucien Stucker}
\newcommand{\autoremail}{lstucker@hsz-t.ch}
\newcommand{\studienbereich}{Informatik}
\newcommand{\matrikelnr}{06-557-540}
\newcommand{\dozent}{Beat Seeliger}
\newcommand{\dozentemail}{bseliger@hsz-t.ch}
\newcommand{\jahr}{2011}
\newcommand{\ort}{Zürich}
\newcommand{\logo}{logo_hszt.jpg}


\usepackage{geometry}
\usepackage{fancyhdr}
\fancyhead[L]{\leftmark}
\fancyhead[R]{}

\usepackage{graphicx} % Grafiken
\graphicspath{{media/}} % hier liegen die Bilder des Dokuments

% sorgt daf¸r, dass Leerzeichen hinter parameterlosen Makros nicht als Makroendezeichen interpretiert werden
\usepackage{xspace}

\usepackage{enumerate}

% Schrift
\usepackage{helvet}
% \usepackage{lmodern} % bessere Fonts
\usepackage{relsize} % Schriftgröfle relativ festlegen
\usepackage{ascii}

\usepackage{setspace} % Einfache Definition der Zeilenabstände 
\onehalfspacing % Zeilenabstand 1,5 Zeilen



% zum Einbinden von Programmcode
\usepackage{listings}
\usepackage[parfill]{parskip}
\usepackage{color}
\usepackage[table]{xcolor}
\usepackage[T1]{fontenc}
\usepackage[utf8]{inputenc}
\usepackage[toc,page]{appendix}
\usepackage{multirow} 
\usepackage{tabularx}
\usepackage{longtable}

% Farben
\definecolor{hsztblue}{cmyk}{0.946,0.452,0,0.349}
\definecolor{hellgelb}{rgb}{1,1,0.9}
\definecolor{light-gray}{gray}{0.95}

% URL verlinken, lange URLs umbrechen etc. -------------------------------------
\usepackage{url}
%% Define a new 'leo' style for the package that will use a smaller font.
\makeatletter
\def\url@leostyle{%
  \@ifundefined{selectfont}{\def\UrlFont{\sf}}{\def\UrlFont{\small\ttfamily}}}
\makeatother
\urlstyle{leo} % Now actually use the newly defined style.


\usepackage{minitoc} % ToC for chapters
\minitoc[n] % No caption for mini ToCs



%\fancyfoot[RO, LE] {\thepage}

% http://stackoverflow.com/questions/586572/make-code-in-latex-look-nice
%\lstset{breaklines=true,frame=single,basicstyle=\ttfamily}

\lstset{
basicstyle=\small\ttfamily,
numbers=left,
numberstyle=\tiny,
frame=b,
columns=fullflexible,
showstringspaces=true,
breaklines=true
}

\lstdefinelanguage{JavaScript}{
     keywords={attributes, class, classend, do, empty, endif, endwhile, fail, function, functionend, if, implements, in, inherit, inout, not, of, operations, out, return, set, then, types, while, use},
     keywordstyle=\color{blue}\bfseries,
     ndkeywords={},
     ndkeywordstyle=\color{black}\bfseries,
     identifierstyle=\color{black},
     sensitive=false,
     comment=[l]{//},
     commentstyle=\color{black}\ttfamily,
     stringstyle=\color{red}\ttfamily
  }


% \renewcommand{\familydefault}{\sfdefault} % Standardschriftart Helvet

\newcommand*\oldurlbreaks{} % Handle long urls
\let\oldurlbreaks=\UrlBreaks
\renewcommand{\UrlBreaks}{\oldurlbreaks\do\a\do\b\do\c\do\d\do\e%
  \do\f\do\g\do\h\do\i\do\j\do\k\do\l\do\m\do\n\do\o\do\p\do\q%
  \do\r\do\s\do\t\do\u\do\v\do\w\do\x\do\y\do\z\do\?\do\&}



\usepackage{titlepic} % http://typethinker.blogspot.com/2008/08/picture-on-title-page-in-latex.html

\usepackage{hyperref}
\hypersetup{
    unicode=false, % non-Latin characters in Acrobat’s bookmarks
    pdftoolbar=true, % show Acrobat’s toolbar?
    pdfmenubar=true, % show Acrobat’s menu?
    pdffitwindow=false, % window fit to page when opened
    pdfstartview={FitH}, % fits the width of the page to the window
    pdftitle={\art - \titel}, % title
    pdfauthor={\autor}, % author
    pdfsubject={\titel \untertitel}, % subject of the document
    pdfcreator={\autor}, % creator of the document
    pdfproducer={\autor}, % producer of the document
    pdfkeywords={solr} {nutch} {information} {retrieval}, % list of keywords
    pdfnewwindow=true, % links in new window
    colorlinks=true, % false: boxed links; true: colored links
    linkcolor=black, % color of internal links
    citecolor=black, % color of links to bibliography
    filecolor=black, % color of file links
    urlcolor=black % color of external links
}

% http://en.wikibooks.org/wiki/LaTeX/Glossary
%\usepackage[toc,xindy,acronym]{glossaries}

\usepackage{acronym} % [printonlyused]


%\usepackage[toc,xindy]{glossaries}
%\makeglossaries
%\input{Include/Glosar_Terme}

% für Index-Ausgabe mi
\usepackage{makeidx}
\setcounter{secnumdepth}{4} %nunmbers
\setcounter{tocdepth}{3} %inhaltsverzeichnis
\makeindex

% Glossar
\usepackage[
acronym,      %ein Abkürzungsverzeichnis erstellen
toc]          %Einträge im Inhaltsverzeichnis]{glossaries}
{glossaries}
\makeglossaries

% Seitenraender -----------------------------------------------------------------
\setlength{\headheight}{26pt}
\geometry{a4paper, top=25mm, left=40mm, right=25mm, bottom=30mm,
headsep=10mm, footskip=12mm}
% Uni Freiburg entfield links 2,5 cm; rechts 2,5 cm; oben 2,5 cm; unten 2 cm

\usepackage[font=small,labelfont=bf]{caption} % Caption auch in non-float tabellen/bildern setzen. Benutzung: \mycaption{table|figure}{Titel}

\DeclareCaptionFont{white}{\color{white}}
\DeclareCaptionFormat{listing}{\colorbox{gray}{\parbox{\textwidth}{#1#2#3}}}
\captionsetup[lstlisting]{format=listing,labelfont=white,textfont=white}

\newcommand{\mycaption}[2]{
\begin{minipage}[c]{0.8\linewidth}
\renewcommand{\figurename}{Abbildung}
\captionsetup{type=figure}
\captionof{#1}{#2}
\end{minipage}
}

\newcounter{qcounter}
\newcounter{para} \setcounter{para}{0}
\newcommand{\newpara}{%
  \refstepcounter{para}%
  \noindent\llap{\thepar. }\quad}
\newcommand{\oldpara}[1]{%
  \noindent\llap{\ref{#1}. }\quad}

\newcommand{\refbf}[1]{\textbf{\ref{#1}}}
\newcommand{\refsoll}[1]{\textbf{\nameref{#1}}}
\newcommand{\refko}[1]{\textbf{\nameref{#1}}}
\newcommand{\reftab}[1]{\textbf{Tabelle (\ref{#1})}}
\newcommand{\reflisting}[1]{\textbf{Listing \ref{#1}}}
\newcommand{\refabb}[1]{\textbf{Abbildung \ref{#1}}}
\newcommand{\refchap}[1]{\textbf{Kapitel \ref{#1}}}
\newcommand{\refsec}[1]{\textbf{Absatz \ref{#1}}}
\newcommand{\refeql}[1]{\textbf{Gleichung \ref{#1}}}
\newcommand{\refeqlb}[1]{\textbf{Berechnung \ref{#1}}}

\usepackage[fleqn]{amsmath}

\usepackage{tabularx}
\newcolumntype{L}[1]{>{\raggedright\arraybackslash}p{#1}} % linksbündig mit Breitenangabe
\newcolumntype{C}[1]{>{\centering\arraybackslash}p{#1}} % zentriert mit Breitenangabe
\newcolumntype{R}[1]{>{\raggedleft\arraybackslash}p{#1}} % rechtsbündig mit Breitenangabe
\newcommand{\ltab}{\raggedright\arraybackslash} % Tabellenabschnitt linksbündig
\newcommand{\ctab}{\centering\arraybackslash} % Tabellenabschnitt zentriert
\newcommand{\rtab}{\raggedleft\arraybackslash} % Tabellenabschnitt rechtsbündig

\setlength{\baselineskip}{16.5pt} % 16 pt usual spacing between lines


%Footnote
\newcommand{\footnoteremember}[2{\footnote{#2}\newcounter{#1}\setcounter{#1}%{\value{footnote}}}
\newcommand{\footnoterecall}[1]{\footnotemark[\value{#1}]}

% Meta-Informationen -----------------------------------------------------------
%   Informationen über das Dokument, wie z.B. Titel, Autor, Matrikelnr. etc
%   werden in der Datei meta.tex definiert und können danach global
%   verwendet werden.
% ------------------------------------------------------------------------------
%!TEX root=documentation-bachlorthesis-speicherarchitektur-lstucker.tex

% meta-information -----------------------------------------------------------
%   Definition von globalen Parametern, die im gesamten Dokument verwendet
%   werden können (z.B auf dem Deckblatt etc.).
%
%   ACHTUNG: Wenn die Texte Umlaute oder ein Esszet enthalten, muss der folgende
%            Befehl bereits an dieser Stelle aktiviert werden:
%            \usepackage[latin1]{inputenc}
% ------------------------------------------------------------------------------
\newcommand{\titel}{Speicherarchitektur für Massendaten einer Webapplikation}
\newcommand{\untertitel}{Webapplikation}
\newcommand{\art}{Bachlorthesis}
\newcommand{\fachgebiet}{Betriebsysteme}
\newcommand{\autor}{Lucien Stucker}
\newcommand{\autoremail}{lstucker@hsz-t.ch}
\newcommand{\studienbereich}{Informatik}
\newcommand{\matrikelnr}{06-557-540}
\newcommand{\dozent}{Beat Seeliger}
\newcommand{\dozentemail}{bseliger@hsz-t.ch}
\newcommand{\jahr}{2011}
\newcommand{\ort}{Zürich}
\newcommand{\logo}{logo_hszt.jpg}


\begin{document}

\frontmatter
\pagestyle{empty}
\pagenumbering{alph}
%!TEX root=documentation-bachlorthesis-speicherarchitektur-lstucker.tex

\thispagestyle{plain}
\begin{titlepage}
\sffamily
	\renewcommand{\headheight}{4.5cm}
	% \renewcommand{\familydefault}{\sfdefault} % Standardschriftart Helvet 
    \begin{center}

	\huge{\textbf{\titel}}\\
%	 \vspace{1.5cm}
%	\LARGE{\untertitel}\\
	 \vspace{1.5 cm}
	\LARGE{\textbf{\art}}\\
	 \vspace{2cm}
	
    	\includegraphics[width=80mm, keepaspectratio = true]{\logo}

    \normalsize
    \vspace{2cm}
    \vspace{2.5cm}

 \normalsize{
    \begin{tabular}{ll}
     Abgabe: & 22. Mai 2012\\
     Präsentation: & 6. Juli 2011\\\\
     Student: &\autor, \autoremail\\
     Dozent: & \dozent, \dozentemail \\
     Studienbereich: & \studienbereich\\
    \end{tabular}\\
    }
\end{center}
\rmfamily
\end{titlepage} %title
\cleardoublepage

\pagestyle{fancy}

\pagenumbering{roman}
\setcounter{page}{1}

\phantomsection
 \renewcommand{\contentsname}{Inhaltsverzeichnis}
\renewcommand{\chaptername}{Kapitel}
% \renewcommand{\listfigurename}{Abbildungsverzeichnis}


\tableofcontents
\cleardoublepage

\listoffigures % Abbildungsverzeichnis
\mtcaddchapter\addcontentsline{toc}{chapter}{\listfigurename}
\cleardoublepage

% Tabellenverzeichnis
\listoftables
\mtcaddchapter\addcontentsline{toc}{chapter}{\listtablename}
\cleardoublepage

\lstlistoflistings  % Listings-Verzeichnis
\mtcaddchapter\addcontentsline{toc}{chapter}{Source Code Verzeichnis}
\cleardoublepage %tableofcontents
\cleardoublepage

\mainmatter
\pagenumbering{arabic}
\setcounter{page}{1}
% Load at the beginngin that we can use it
%!TEX root=../documentation-bachlorthesis-speicherarchitektur-lstucker.tex

\newglossaryentry{RFC}{ name={RFC}, description={Request for Comments (RFC) sind Dokumente, über Internet, inklusive der technischen Spezifikation und Richtlinien, welche von der Organisation Internet Engineering Task Force entwickelt wurde. "'Das RFC wird erst nach erfolgter Diskussion unter der Aussicht des Internet Architecture Board (IAB) herausgegeben und fungiert als Quasistandard. Jedes RFC enthält eine eindeutige, vorlaufende Nummer, die kein zweites Mal zu gewiesen wird."' \cite{MicrosoftComputerLex}  \url{http://www.rfc-editor.org/}}}

\newglossaryentry{UDP}{ name={UDP}, description={  adfajsdfjadslkfjaödjfölaksdjfajsklfj }}

\newglossaryentry{IBM}{ name={IBM}, description={International Business Machines Corporation (kurz IBM) ist ein führendes unternehmen in Software, Hardware und IT-Dienstleistung Bereich. }}

\newglossaryentry{IBM}{ name={CIFS}, description={Common Internet File System (kurz CIFS) wurde 1996 von Microsoft eingeführt und beschreibt eine erweiterte Version von SMB. CIFS und SMB sind eine  Netzwerkdateisystem vergleichbar mit NFS und wird vorwiegend im MS Windows Bereich eingesetzt. }}


\newglossaryentry{XDR}{ name={XDR}, description={Die eXternal Data Representation (kurz XDR) Spezifikation stellt ein Standardisierte Verfahren zur Präsentation von gebräuchlichsten Daten Typen über das Netzwerk zur Verfügung. Dies löst das Problem der verschiedenen Byte-Reihenfolge (Big Endian), Speicherausrichtung auf unterschiedlichen Kommunikations Partner}}

\newglossaryentry{POSIX}{ name={POSIX}, description={Portable Operating System Interface (kurz POSIX) ist eine von IEEE entwickelter Standard, welche die Schnittstelle zwischen Applikation und Betriebsystem darstellt. Die aktuelle Version des Standards ist IEEE Std 1003.1-2008 \url{http://www.opengroup.org/austin/papers/posix_faq.html} }}

\newglossaryentry{FUSE}{ name={FUSE}, description={Filesystem in Userspace (kurz FUSE), ermöglicht die Implementierung eines voll Funktionsfähigen Dateisystem in Userspace. Normaler weise laufen 
FUSE wurde urspünglich Entwickelt um AVFS zu unterstützen, ist jedoch heute ein seperates Projekt. \url{http://fuse.sourceforge.net/}}} 

\newglossaryentry{RPC}{File locking erlaubt es einen Prozess den exklusiven Zugriff auf eine Datei oder teile einer Datei und zwingt ander Prozesse die auf die selbe Ressource zugreifen wollen zu warten bis das Locking aufgehoben wurde.} 

\newglossaryentry{FileLocking}{File locking erlaubt es einen Prozess den exklusiven Zugriff auf eine Datei oder teile einer Datei und zwingt ander Prozesse die auf die selbe Ressource zugreifen wollen zu warten bis das Locking aufgehoben wurde.} 

\newglossaryentry{API}{ name={API}, description={Application Programming Interface (kurz API) auch Anwendungsprogrammierschnittstelle genannt. "'Ein Satz an Routinen, die vom Betriebsystem des Computers für die Verwendung aus Anwendungsprogrammen heraus angeboten werden und diverse Dienste zur Verfügung stellen."' \cite{MicrosoftComputerLex} }}

\newglossaryentry{MIT{ name={MIT}, description={Die MIT Lizenz stammt von Massachusetts Institute of Technology und erlaub die die Verwendung von Software welche Quelloffen ist als auch software welche nicht Quell geschlossene ist. Die genauen Lizenz Bestimmungen sind unter folgenden URL zu finden \url{http://www.opensource.org/licenses/mit-license.php}}}

\newglossaryentry{GNU GPL{ name={GNU GPL}, description={Die GNU General Public License Lizenz auch GPL genannt stammt von der Free Software Foundation und regelt die Lizenzierung von Freie Software. Es gibt drei Versionen der GPL welche unter folgenden URL beschrieben sind \url{http://www.gnu.org/licenses/}}}

\newglossaryentry{Ruby{ name={Ruby}, description={Ruby ist eine interpretierte und objektorientierte Programmiersprache und beinhaltet einige bewährte Prinzipien wie z.B. "'DuckTyping"' und "'Principle of Least Suprice"'. Die Entwickler von Ruby stellen sich selber den Anspruch eine Programmiersprache zu schaffen, die durch Ihre Natürlichkeit einfach erlernbar ist und es den Programmierern ermöglicht, einfachen und übersichtlichen Code zu schreiben, welcher aber nicht seine Mächtigkeit und innere Komplexität verliert.
Ruby hat sich in den letzten Jahren von einer kaum beachteten Programmiersprache zu einem Publikums-Magneten entwickelt. Es gibt eine stetig wachsende offene Community "'Gemeinschaft"', welche sich und die Sprache durch Austausch von Erfahrungen und Ideen weiterbringen möchte.
Ein Grund für die hohe Bereitschaft der Community die Sprache Ruby weiter zu bringen ist der Umstand, dass die Programmiersprache vollständig OpenSource ist und unter der Lizenz der Ruby-License und GPL steht. Zudem ist die Sprache fast beliebig erweiterbar und bestehende Funktionen können einfach durch eigene Funktionen ausgetauscht werden.}}

\input{chapter/vorwort}
%!TEX root=../documentation-bachlorthesis-speicherarchitektur-lstucker.tex
\cleardoublepage
\chapter{Zusammenfassung}

Der Auftraggeber, ein Zürcher Startup Unternehmen, betreibt und entwickelt eine Bild-Archiv Webapplikation für Kunden mit hohen Ansprüchen an die Qualität ihrer Bilder. Die Anwendung erlaubt das Speichern, Archivieren, Verwalten und die Druckaufbereitung von qualitativ hochauflösenden digitalen Bildern. Ein einzelnes Bild kann eine Speichergrösse von bis zu 2 Gigabyte beanspruchen. Das Archiv belegt schon heute mehr als 2.5 Terrabyte Diskplatz.

Die Aufgabe dieser Arbeit besteht darin, ein geeignetes Speichersystem zu evaluieren, welches die Anforderungen für eine Speichergrösse von 11.5 Tebibyte in Szenario 1 und für 218 Tebibyte in Szenario 2 am besten erfüllt. In der Arbeit sollen die Anforderungen bezüglich Skalierbarkeit der Speicherkapazität und Anzahl Datenabfragen, der Datendurchsatz, die Datenverfügbarkeit, die Datenintegrität und Wirtschaftlichkeit des Gesamtsystems untersucht werden.

Bis anhin werden die Daten auf einem einzelnen gehosteten Server gespeichert, auf welchem auch die Webapplikation betrieben wird. Um die Daten vor Verlust zu schützen, sind die Festplatten des Servers mit einem RAID-5 System konfiguriert. Die verfügbare Speicherkapazität ist bereits zu über 50\% ausgelastet und kann nicht ohne Upgrade auf ein neues System erweitert werden. Bei gleichbleibendem Datenwachstum ist die freie Kapazität innert sieben Monaten ausgeschöpft. Das bestehende System kann die Anforderungen bezüglich Verfügbarkeit und Skalierbarkeit nicht erfüllen. Das Überdenken der künftigen Speicherarchitektur drängt sich deshalb auf.
 
Die heutigen am Markt erhältlichen Speicherarchitekturen lassen sich in der obersten Kategorie in Block-, Datei- und Objekt-basierte Systeme unterteilen. Die Block-basierende Speicherarchitektur ist wohl die traditionellste und die am weitverbreitetste Form von allen. Block-basierende Speicher können mittels FibreChannel oder iSCSI über ein Speichernetzwerk (SAN) an mehrere Server-Systeme zur Verfügung gestellt werden. Mit dem Aufkommen von Desktop-Computern, wuchs der Bedarf die Daten zentral für alle Benutzer ablegen und dezentral auf diese zugreifen zu können. Aus dieser Anforderung heraus wurden die Datei-basierenden Speicherarchitekturen entwickelt. Dazu gehört Network File System (NFS). Objekt-basierende Speichersysteme werden zunehmend bei sehr hohem Speicherkapazitätsbedarf oder wo die Daten für Datenanalyse aus Performanceoptimierung auf verschiedene Systeme gespeichert werden sollen.

Für die Evaluation wurde das Analytic Hierarchy Process (AHP) Verfahren gegenüber der Nutzwertanalyse bevorzugt. Bei AHP wird durch den hierarchischen Analyseprozess die Evaluation stark strukturiert. Der Analyseprozess unterstützt mit einer Softwareanwendung erlaubt die neutrale Berechnung der zu untersuchenden Lösungsvarianten. Bei vielen zu vergleichenden Kriterien erwächst zwar schnell ein etwas höherer Arbeitsaufwand, das Ergebnis der Analyse ist aber jederzeit gut dokumentiert und nachvollziehbar.

Für die Evaluation wurde jeweils ein Produktevertreter für jede der zu untersuchenden Speicherarchitekturen gewählt. Es wurden die heute am Markt angebotenen Architekturlösungen berücksichtigt, wie die Block-basierenden, Datei-basierenden und Objekt-basierenden Produkte. Ferner wurden ein Vertreter von Online Speicher (Amazon S3) und der bisherige Anbieter der bestehenden Lösung einbezogen. Anbieter von Cloud-Speicherplatz wie Google Drive, iCloud, myDrive oder Dropbox bieten zwar Speicherlösungen für Endanwender (consumer frontend services) an, konnten aber wegen der fehlenden Möglichkeit die bestehende Anwendung des Auftraggebers einzubinden, nicht berücksichtigt werden. 

Untersucht wurde zudem die Eignung von einer Private Cloud. Private Cloud beinhaltet den Aufbau eines eigenen Datencenters basierend auf der OpenStack Object Storage Lösung, währendem bei Public Cloud Hosting-Anbieter gemeint sind, welche die gesamte IT-Infrastruktur mit einer bestimmten Speicherarchitektur samt Unterhalt und Verwaltung als Service anbieten. Es hat sich gezeigt, dass der Betrieb eines eigenen Datencenters mit eigenen Mitarbeitern zwar interessant ist, sich finanziell im Vergleich zu Hosting-Lösungen jedoch nicht rechnet.

Als Vertreter für Block-basierende Speicher wurde das Speichersystem des Herstellers NetApp gewählt, welches den Speicher über iSCSI den Applikations-Servern zur Verfügung stellt. Als Vertreter für Datei-basierende Speicher wurde dieselbe NetApp gewählt, welche den Speicher auch über NFS den Applikations-Servern zur Verfügung stellt. Als Vertreter von Objekt-basierenden Speichern wurde OpenStack Object Storage gewählt, die ebenfalls für Online Speicher eingesetzt wird. OpenStack Object Storage verwendet für die Speicherung gewöhnliche Computer-Systeme und erreicht durch redundante Verteilung der Daten eine hohe Verfügbarkeit. Dabei sind alle Daten auf mindestens drei Computer-Systemen gespeichert. Als Vertreter von Online Speicher wurde Amazon S3 berücksichtigt.
Das Evaluationsresultat empfiehlt als Gewinner für beide der oben beschriebenen Speicherbedarfsszenerien Amazon S3. Amazon S3 erhielt wegen den niedrigen Gesamtkosten, der beliebig skalierbaren Speicherkapazität, der hohen Redundanz und für die Sicherstellung der Datenintegrität die meisten Punkte. An zweiter Stelle bei Szenario zwei platzierte sich NetApp iSCSI dicht gefolgt von OpenStack Object Storage, welche beide aufgrund der hohen Personalkosten sich hinter Amazon S3 klassierten.

Für die OpenStack Object Storage wurde ein geeignetes System für die Machbarkeitsstudie aufgebaut und erfolgreich getestet. Aufgrund dass sich Amazon S3 und OpenStack Object Storage ändeln, kann abgeleitet werden, dass auch die empfohlene Lösung mit Amazon S3 umsetzbar ist. 
%!TEX root=../documentation-bachlorthesis-speicherarchitektur-lstucker.tex

\cleardoublepage
\chapter{Ist-Analyse}
Ziel der Ist-Analyse ist es, mit der Situationsanalyse den aktuellen Stand des bestehenden Systems bzw. der Speicherinfrastruktur zu beschreiben und zu bewerten.
Die Angaben für die Ist-Analyse wurden in diversen Interviews mit dem Auftraggeber und aus den von ihm gelieferten schriftlichen Dokumenten erhoben und zusammengestellt.

\section{Bestandesaufnahme}
\subsection{Anwendungs-Architektur}
Die genaue Anwendungs-Architektur wird hier nicht beschrieben, da dies die Geschäftsgeheimnisse des Auftraggebers verletzen würde. Es werden deshalb nur die Teile davon beschrieben, welche die Geschäftsgeheimnisse des Auftragsgebers nicht verletzen.

Die Applikation ist eine Web-Applikation, welche auf dem Web-Framework Ruby On Rails aufsetzt.
Wie dem Namen "Ruby On Rails" zu entnehmen ist, basiert das Framework auf der Programmiersprache Ruby. Ruby ist eine dynamische, Objekt-orientierte, interpretierbare Programmiersprache, von Yukihiro Matsumoto 1995 hervorgebracht. Ein Ruby Programm benötigt für deren Ausführung keine Kompilation.

Ruby On Rails auch Rails oder RoR genannt, implementiert eine Model-View-Controller Architektur. Die drei Sub-Frameworks, spielen dabei eine signifikante Rolle in der Separierung des Codes: Active Record, Action View, und Action Controller. Die drei Sub-Frameworks sind auch im refabb{abb:Rails-Architektur} dargestellt (nach \cite{Bachle2007}).

\begin{center}
\includegraphics[width=\linewidth, keepaspectratio = true]{media/rails-architecure.png}
\mycaption{figure}{\label{abb:Rails-architektur}Rails Architektur \textit{RAID-5} \cite{Wikipedia2006}}
\end{center}

Beides, Ruby On Rails (auch nur "Rails" genannt) und Ruby sind Open-Source Programme. Die Quellprogramme sind für jeden Programmierer offen. Ruby wird unter der Lizenz "'Ruby License"' und GPL veröffentlicht, während Rails unter der "MIT License" veröffentlicht ist.

\subsection{System-Architektur}\label{System-Architektur}
Die bestehende System-Architektur besteht aus einem einzelnen Server, welcher von einem bekannten deutschen Webhosting-Provider gemietet wird. Beim gemieteten System handelt es sich um einen x64 Mikroarchitektur.

\subsection{Speicher-Architektur}
Die Speicher-Architektur besteht aus einem internen Block-basierendem RAID-5 Speicher. Die im Server verbauten Festplatten (RAID-5) werden mittels dem Linux Software RAID Kernel\footnote{\url{https://raid.wiki.kernel.org/}} und dem Verwaltungstool mdadm\footnote{\url{http://neil.brown.name/blog/mdadm}} zu einer logischen Einheit zusammengefasst. Für das RAID-5 werden drei  "'SATA 2"' Festplatten verwendet, welche jeweils eine Speicherkapazität von 2 Terabyte bzw. 1.818 Tebibyte aufweisen.

Im RAID Speicher ist ein root Dateisystem für das Betriebsystem, die Webapplikation, Datenbank, Logdateien und Bilddaten installiert.

\subsection{Speicherkapazität}
Die von RAID-5 zur Verfügung gestellte Speicherkapazität beträgt gemäss \refeqlb{eqn:RAID-5-3disk} 3.636 Tebibyte, davon sind nach Installation des Dateisystems $\approx 3.5$ Tebibyte bzw. $\ca. 3.8 $ Terabyte effektiv verfügbar. Davon sind anhand des \textit{df} Befehls,  wie im\reflisting{df} ersichtlich, $\ca. 2.5$ Terabyte für das Betriebsystem, die Webapplikation, Datenbank, Logdateien und Bilddaten reserviert. Als freier Speicherplatz verbleiben $\ca. 1.3$ Terabyte.

\begin{equation}
\mbox{Speicherkapazität RAID-5}= (3 - 1) * 1.818  \, \mathrm{TiB} =  3.636 \, \mathrm{TiB}
\label{eqn:RAID-5-3disk}
\end{equation}

\begin{lstlisting}[label=df, language=Bash, caption=Report Dateisystem Speicherplatz Belegung in Dezimal Prefix ]
root@www1:~# df -h
Filesystem            Size  Used Avail Use% Mounted on
simfs                      3.8T  2.5T  1.3T   67% /
\end{lstlisting}
\footnote{\url{http://www.debianadmin.com/manpages/dfmanpage.htm}}

\subsection{Datenwachstum}
Das \refabb{abb:disk-usage-by-year} zeigt den Speicherzuwachs von Mitte Juni bis Ende November.

\begin{center}
\includegraphics[width=\linewidth, keepaspectratio = true]{media/disk-usage-by-year.png}
\mycaption{figure}{\label{abb:disk-usage-by-year}Verhältnis von Speicheplatzverbrauch zur Speicherkapazität in Prozent in einer Zeitspanne von einen Jahr}
\end{center}

\subsection{Zugriffswachstum}
Für die Ist-Aufnahme konnten keine historischen Messdaten zur Verfügung gestellt werden.

\subsection{Skalierbarkeit Datenvolumen}
Während dem Betrieb ist ein Ausbau der Speicherkapazität (Skalierbarkeit) nicht möglich. Eine Vergrösserung der Speicherkapazität ist nur durch einen Wechsel auf ein anderes Hostingprodukt möglich. Dies erfordert eine Migration auf eine neue Server-Plattform. 

\subsection{Skalierbarkeit Datenzugriffe}
Keine Skalierbarung möglich. 

\subsection{Daten Durchsatz I/O}
Für die Ist-Aufnahme konnten keine Messdaten zur Verfügung gestellt werden.

\subsection{Daten Redundanz}

Die Daten-Redundanz wird durch das RAID-5 gewährleistet.

\subsection{Datenverfügbarkeit}
Stromausfälle und Netzwerkstörungen im Rechenzentrum des Hosting Providers führten zu Störungen und Ausfällen in der Webapplikation und in der Datenauslieferung. Die Störungen wurden jedoch nicht dokumentiert, weshalb keine statistische messbare Aussage über die erreichte Datenverfügbarkeit gemacht werden kann.

\subsection{Daten Sicherheit}
Die Daten sind durch physischen Fremdzugriff mittels Verschlüsselung geschützt. Für die Verschlüsselung wird das Device Mapper Module dm-crypt eingesetzt. dm-crypt verschlüsselt die Daten bereits auf Blockebene und ist somit für das Dateisystem tranRAID-5nt.

\subsection{Daten Integrität}
Zu jedem Bild wird eine Hash Prüfsumme gespeichert, die bei jedem Sicherung geprüft wird. Die Daten-Integrität ist somit sichergestellt.

\subsection{Sicherung}
Es wird täglich mittels dem Tool ccollect\footnote{\url{http://www.nico.schottelius.org/software/ccollect/}} ein R-Sync Sicherung der Daten erstellt. Das Sicherung wird ausser Hause gelagert.

\subsection{Wirtschaftlichkeit}
Die Kosten für die Speicherung der Daten inklusive Webinfrastruktur sind vergleichsweise zu alternativen Lösungen tief.

\subsection{Lokalität}

Die Web-Applikation inklusive dessen \gls{Primären-Daten} werden am selben Standort betrieben. Die Sicherungsdaten werden an einen seperaten Standort gelagert. Der Zugriff auf Sicherungs-Daten ist innerhalb 30 Minuten möglich.

\section{Analyse-Ergebnisse}

\subsection{System-Architektur}

\subsection{Datenwachstum und Speicherkapazität}
Die Messdaten aus dem  \refabb{abb:disk-usage-by-year} zeigen, dass das Datenvolumen seit  Juni 2011 bis Ende November 2011 mit Ausnahme einer unregelmässigen Spitze und einem grösseren Wachstumsschub Ende Oktober kontinuierlich zugenommen hat. Die etwas erhöhte Spitze lässt sich durch eine vorübergehende technische Änderung am System erklären und hat somit für die Auswertung keine Relevanz. Während der erwähnten Zeitspanne hat das Datenvolumen von 1,634 Terabyte auf 2,546 Terabyte zugenommen. Dies entspricht einer Datenzuwachsrate von 0,912 Terabyte bzw. 55,8\% in fünf Monate. Das Durchschnittliche Datenwachstum beträgt 0,182 Terabyte bzw. 11,1\% pro Monat.

Setzt sich der bisherige Trend fort, so ist die verfügbare Speicherkapazität von 3,8 Terabyte in weniger als 7 Monate erschöpft. Berücksichtigt man bei diesem durchschnittlichen Datenwachstum mögliche Wachstumsschübe für Neukunden, welche durch das Einlesen von deren bestehende Bilddatenbanken entstehen könnten, so reduziert sich die Kapazitätsreserve nach unten und würde eine vorzeitige Umstellung auf ein neues System bedingen.

Gemäss des Auftraggebers ist ferner ein neues Feature für die Web-Applikation geplant, welches den Speicherplatzbedarf verdoppeln würde. Die aktuelle Speichersituation lässt momentan den produktiven Einsatz des neuen Feature nicht zu. Das durchschnittliche Datenwachstum würde voraussichtlich auf 0,364 Terabyte pro Monat steigen.

Die jetzige Speicherkapazität erfüllt schon heute die Anforderungen nicht mehr.

\begin{equation}
\mbox{Datenwachstum}_{Monate5} = 2,546  \mathrm{\ TB} - 1,634 \mathrm{\ TB} =  0,912 \mathrm{\ TB}
\label{eqn:Verfügbarkeit_5Monate}
\end{equation}

\begin{equation}
\mbox{Datenwachstum}_{Monate5} = \frac{0,912 \mathrm{\ TB}}{\frac{1,634 \mathrm{\ TB}}{100 \ \%}} =  55,813 \ \%
\label{eqn:Verfügbarkeit_5Monate_in_Prozent}
\end{equation}

\begin{equation}
\mbox{Durchschnittliches Datenwachstum}_{Monate} = \frac{0,912 \mathrm{\ TB}}{5\mathrm{m}} =  0,1824 \mathrm{\ TB}
\label{eqn:Verfügbarkeit_1Monate}
\end{equation}

\begin{equation}
\mbox{Durchschnittliches Datenwachstum}_{Monate} = \frac{55,813\ \%}{5 \mathrm{m}} = 11,162 \ \%
\label{eqn:Verfügbarkeit_1Monate_in_Prozent}
\end{equation}

\subsection{Skalierbarkeit Datenvolumen}\label{AnalyseSkalierbarkeitDatenvolumen}
Die Eingesetzte Speicherarchitektur lässt eine Skalierung des Datenvolumen mittels hinzufügen von weiteren Festplatten oder durch Migration auf grösseren Festplatten zu. Die maximale Anzahl von Festplatten wird dabei durch die Anzahl vorhandenen SATA Anschlüsse im Server bestimmt. Beim eingesetzten Server handelt sich um eine Hosting Produkt welche meist begrenzt ausbaubar ist. Bei Hosting Produkte erfolgt eine Skalierung meist durch eine Wechsel auf eine besser ausgebautes Hosting Produkt. 

\subsection{Skalierbarkeit Datenzugriffe}
Eine mögliche Skalierung bezüglich des Datendurchsatzes könnte durch schnellere Festplatten, durch eine optimierte Verteilung der I/O-Operationen auf verschiedene Festplatten oder durch die Verteilung der Daten auf weitere Systeme erreicht werden. Zu beachten sind auch hier dieselben Prämissen, wonach wie beim Ausbau des Datenvolumens das bestehende System nur sehr begrenzt ausgebaut werden kann bzw. auf ein anderes Hostingprodukt migriert werden müsste.

\textbf{Skalierung durch schnellere Festplatten:}
Moderne Festplatten, mit Ausnahme von den teuren Solid State Disk (SSD), bieten zwar eine grössere Speicherdichte aber aufgrund der erreichten physikalischen Grenzen keine bedeutend effizientere IOPS versprechen. Eine Festplatte mit 7200 RPM erreicht durchschnittlich ein IOPS zwischen 75 und 100 IOPS. Bei SSD Disks wurden schon 1'190'000 IOPS in einem einzelnen PCI Geräte gemessen.\cite{Symantec2011} \cite{Fusionio} 
Ein Wechsel auf die genannten SSD Festplatten, ist aktuell noch mit sehr hohen Kosten pro Gigabyte verbunden. Gartner geht davon aus, dass 2012 die Preise pro Gigabyte bei SSD auf durchschnittliche 1\$ sinken werden. Gegenüber den heutigen Preisen bei konventionellen Festplatten von 30 Cents pro Gigabyte ist dies immerhin 3x teurer. Aus diesem Grund werden SSD Disks bei grossen Datenvolumen meist noch nicht eingesetzt, ausser spezielle Anforderungen an die Performance rechtfertigen die höhere Investition. \cite{AgamShah2011}

\textbf{Skalierung duch Verteilung der I/O Operationen:}
Ein RAID oder Volume Manager bietet die Möglichkeit, die I/O Operationen auf mehrere Festplatten zu verteilen. In einem RAID-5 verkleinert sich wie in Kapitel \refchap{AnalyseSkalierbarkeitDatenvolumen} beschrieben, die MTTF mit jeder weiteren Festplatte. Zudem sind die verfügbaren Anschlüsse in einem Server ebenfall ein zu berücksichtigen limitierender Faktor.

\textbf{Skalierung durch Verteilung der Daten:}
Nicht alle Kategorien von Daten haben die gleichen Anforderungen an den Durchsatz. Datenbanken z.B. benötigen in der Regeln einen höheren IOPS als statische Daten wie z.B. Bild-Daten, welche weniger oft abgefragt werden. Für die Verteilung der Daten ist eine Änderung in der System-Architektur und allenfalls in der Anwendungs-Architektur notwendig. Ein Beispiel für eine solche Anpassung der System-Architektur wäre die Auslagerung der Datenbank auf ein separates System, welches mit schnellen Solid State Disk ausgerüstet ist. 

\subsection{Redundanz}
Eine RAID-5 System bietet wie im \refsec{RAID-5} beschrieben, keine 1:1 Redundanz. Die Daten sind im RAID-5 nicht doppelt gespeichert, sondern werden bei einen Datenverlust mittels XOR Operation aus den Parität Stripes und den vorhandenen Daten Stripe-Einheiten berechnet. 

Bei einen Zugriff auf einen verlorenen Daten Stripe werden die Daten online berechnet. 

Treten bei einer Festplatte Fehler auf, werden die verlorenen Daten bei einem Zugriff aus dem Parität Stripe Einheit.??? Der Zugriff auf die Daten ist durch die Berechnung der Daten aus dem Parität Stripe Einheit Online möglich. Wird die ausgefallen Festplatte durch eine neu intakte Festplatte ersetzt, können die Daten durch einen Rebuild während des Betriebs wieder hergestellt werden. Durch die Berechnung und Schreib/Lese Operationen im RAID während des Wiederherstellungprozesses verschlechtert sich der Datendurchsatz und I/O Rate für den Datenzugriff wesentlich.

Einen Ausfall einer weiteren Festplatten kann das RAID-5 System nicht mehr kompensieren und führt zu einem physischen Datenverlust aller Online-Daten. Die Daten müssen in der Folge mittels des Sicherung eingelesen und wiederhergestellt werden, was jedoch nur im Offline-Betrieb möglich ist. 

Bei Festplatten welchen aus dem gleichen Produktionszyklus stammen, kann die Wahrscheinlichkeit eines weiteren Ausfall höher sein, im vergleich zu Festplatten die aus unterschiedlichen Produktionszyklen stammen.

\subsection{Service- und Daten-Verfügbarkeit}
Die bisherige System-Architektur gemäss dem Ist-Bestand ermöglichen keine AEC-2 Hochverfügbarkeit. Grund dafür ist die Fokussierung auf einen einzelnen Server in der System-Architektur. Der Server stellt eine "'Single Point of Failure"' (SPOF) dar. Tritt beim Server eine Störung auf, wie Sie zum Beispiel Aufgrund eines Gerätedefekts oder Softwarefehler auftreten können, kann der Service nicht von einem redundantem System übernommen oder weitergeführt werden. Eine Störung kann also die Verfügbarkeit während den Betriebszeiten starl gefährden und entspricht nicht den heutigen Anforderungen.

Die Netzwerkverfügbarkeit wird vom Hosting Anbieter gemäss den Allgemeinen Geschäftsbedingungen \cite{Ag2009} mit einer Verfügbarkeit von 99\% im Jahresmittel gewährleistet. Die Verfügbarkeit könnte somit bei einem 24x365 Stunden Betrieb während 87,6 Stunden nicht gewährleistet sein.

Das Speichersystem für sich selbst betrachtet kann die Verfügbarkeit und die Daten-Integrität gewährleisten, sofern keine Störungen durch einen Festplattendefekt eintreten. Das Speichersystem, welches ein interner Block-basierenden RAID-5 Speicher darstellt, ist ein Bestandteil des Serversystems und weist deshalb die selbe Verfügbarkeitsstufe wie das Serversystem auf. Die Service und Datenverfügbarkeit weist aus den genannten Gründen eine Verfügbarkeit Stufe von '"Highly Reliable"' AEC-2 auf.

Die im \refsec{System-Architektur}  beschriebene System-Architektur gewährleistet keine Hochverfügbarkeit der Daten. Einer der Gründe dafür ist, dass die Applikation nur auf einem einzelnen Server betrieben wird, dessen Hardware mit Ausnahme der Festplatten keine Redundanz aufweist und nicht für Hochverfügbar ausgelegt ist. 

Die Applikation wird auf einem einzelnen Server betrieben. Der Server und dessen Hardware mit Ausnahme der Festplatten stellen einen "'Single Point of Failure"' (SPOF) dar. Fällt der Server wegen eines Hardwaredefektes oder einem Softwarefehler aus, sind Daten und Anwendung für den Endbenutzer nicht mehr verfügbar.

Zudem werden die Daten und Applikation nur an einem Standort betrieben. Treten unvorhersehbare Ereignisse auf, wie z.B. ein lokaler Stromausfall, ein Brand oder eine Naturkatastrophe, kann der Betrieb nicht ohne die Wiederherstellung des ganzen Systems inklusive Daten von Sicherung wieder aufgenommen werden. Sofern nicht ein gemieteter Ersatzserver bereitsteht, muss ebenfalls die Beschaffung und Installation eines neuen Systems bei einem anderen Hosting Provider zur Ausfallzeit (Downtime) mit einberechnet werden. Der Service wäre während einer längeren Zeit nicht mehr verfügbar. Ein Kundenverlust und weitere Konsequenzen könnten die Folge sein.

\subsection{Daten Sicherheit}
Durch die Festplattenverschlüsselung sind die Daten bei abgeschaltetem Server vor unberechtigten Zugriffen geschützt. Befindet sich der Server jedoch im laufenden Betrieb, bietet die Verschlüsselung keinen Schutz vor Datendiebstahl. 

In einem Interview mit Ben Schwan im Technology Review hat Edward W. Felden, Professor für Informatik an der Princeton University den Grund dafür erklärt. 

\begin{quotation}\em
... ,die Dechiffrierungsschlüssel bei einer Festplattenverschlüsselung sitzen immer irgendwo im DRAM-Speicher. Um an sie heranzukommen, schaltet der Angreifer zunächst den Strom des Rechners aus und stellt die Energieversorgung dann gleich wieder her. Dann bootet er die Maschine in ein spezielles, böswilliges Betriebssystem hinein. Zu diesem Zeitpunkt enthält der Speicher noch immer die Originalinformationen, die verfügbar waren, als der Rechner noch nicht abgeschaltet wurde – die gewünschten Schlüssel natürlich auch. Das kurze Abschalten des Stroms hat daran rein gar nichts geändert. Der Angreifer kann dann die gewünschten Dechiffrierungsschlüssel aus dem Speicher auslesen und damit die geschützten Informationen auf der Festplatte jederzeit entschlüsseln\end{quotation}\cite{Schwan2008}

Des weiteren bietet eine Festplatten keinen Schutz bei Schwachstellen in der Software oder  der Konfiguration im laufenden Betrieb. Grund dafür ist, dass das Betriebsystem und Anwendungen über eine Verschlüsselungsschicht unverschlüsselt zugreifbar sind.


%!TEX root=../documentation-bachlorthesis-speicherarchitektur-lstucker.tex

\cleardoublepage
\chapter{Soll-Analyse}
Ziel ist es mit der Sollanalyse, den soll zustand Stand der System bzw. Speicherinfrastruktur, welche die verschiedenen Szenerien der Datenzuwachs und Datenzugriffe in den nächsten drei Jahren erfüllen soll  zu beschreiben. Der Auftraggeber geht von einen Starken Wachstum im Bereichen der Datenmengen und Datenzugriffe aus welche das Szenario 4 am ehesten wieder spiegelt.


\section{Szenario 1 - Schwaches Datenvolumen und Datenzugriff Wachstum}
Dieses Szenario beschreib den Soll-Zustand, wenn sich das Datenwachstum und Datenzugriffs Wachstum im gleichen umfang wie in der Ist-Analyse weiter entwickelt. Dieses Szenario würde eintreffen, wenn sich die Kunden Anzahlt und die gewünschten Aufbereitung der Daten für den Druck in den nächsten drei Jahre auf dem selben aktuellen Niveau befinden.


Datenwachstum in Prozent pro Monat: 5\% 

Datenwachstum in Tebibyte pro Monat: 0.19 Tebibyte

Speichervolumen in 36 Monaten: 9.34 Tebibyte

\subsection{Skalierbarkeit Datenzugriff}
Die bestehende Speicherarchitektur konnte die bisherigen Datenzugriff mit einen Web-Server zufriedenstellend erfüllen. Eine Skalierung der Zugriffe von mehreren Server soll deshalb nicht als Hauptkriterium sein sondern.

\subsection{Skalierbarkeit der Speicherkapazität}
Die Speicherkapazität soll exklusive der Datenredundanz bis mindestens 13 Tebibyte skalieren.

\subsection{Redundanz}
Dem Kunden wird einen Qualität Standard gewährleistet, die Original gespeicherten Daten, sollen vor Veränderung geschützt werden, aus diesen Grund soll die Daten Integrität beim Zugriff auf die Original Daten sichergestellt sein.



\subsection{Verfügbarkeit}
Die Verfügbarkeit soll mindestens dem AEC-2 Standard von Harvard Research entsprechen.

\subsection{Daten Integrität}
Dem Kunden wird einen Qualität Standard gewährleistet, die Original gespeicherten Daten, sollen vor Veränderung geschützt werden, aus diesen Grund soll die Daten Integrität beim Zugriff auf die Original Daten sichergestellt sein.

\subsection{Lokalität}
Die Daten sollen in einer minimal Version an einen weiteren Standort als Backup gehalten werden.



\section{Szenario 2 - Schwaches Wachstum Daten / mittleres Wachstum der Abfragen}
Dieses Szenario beschreib den Soll-Zustand, wenn sich das Datenwachstum im gleichen umfang wie in der Ist-Analyse weiter entwickelt, sich jedoch der Datenzugriff im vergleich zur Ist-Analyse stark steigert. Diese Szenario würde eintreffen, wenn sich die Kunden Anzahl in den nächsten drei Jahren auf dem gleichen Niveau hält, jedoch die Aufbereitung der Daten für den Druck stark zunimmt.

Datenwachstum in Prozent pro Monat: 5\% 

Datenwachstum in Tebibyte pro Monat: 0.19 Tebibyte

Speichervolumen in 36 Monaten: 9.34 Tebibyte

\subsection{Skalierbarkeit Datenzugriff}
Der Datenzugriff soll von mehreren Webserver welche die Bilddaten am Kunden ausliefern und Server welche die Original Bilder in das gewünschte Format umrechnen möglich sein. Die Anzahl Datenzugriffe soll von bis zu zwanzig Server möglich.

\subsection{Skalierbarkeit der Speicherkapazität}
Die Speicherkapazität soll exklusive der Datenredundanz bis mindestens 13 Tebibyte skalieren.

\subsection{Redundanz}
Die Daten sollen mindestens doppelt Redundanz haben.

\subsection{Verfügbarkeit}
Die Verfügbarkeit soll dem AEC-2 Standard von Harvard Research entsprechen.

\subsection{Daten Integrität}
Dem Kunden wird einen Qualität Standard gewährleistet, die Original gespeicherten Daten, sollen vor Veränderung geschützt werden, aus diesen Grund soll die Daten Integriät beim Zugriff auf die Original Daten sichergestellt sein.

\subsection{Lokalität}
Die Daten sollen in einer minimal Version an einen weiteren Standort als Backup gehalten werden.

\section{Szenario 3 - Starkes Wachstum Daten / schwaches Wachstum der Abfragen}
Dieses Szenario beschreib den Soll-Zustand, wenn sich das Datenwachstum im vergleich zum Ist-Zustand stark steigert, aber der Datenzugriff auf gleichen Niveau hält wie in der Ist-Analyse hält. Dieses Szenario würde eintreffen, wenn sich die Kunden Anzahl oder die Anzahl Bilddaten pro Kund stark steigert, jedoch die Aufbereitung der Daten für den Druck gleich bleiben würde.

Datenwachstum in Prozent pro Monat: 

Datenwachstum in Tebibyte pro Monat: 6 Tebibyte

Speichervolumen in 36 Monaten: 218,5 Tebibyte

Bilder mit einer Speichervolumen von 1 Gigibyte: 221'184

\subsection{Datenzugriff}
Durch den Geringen Zugriff auf die Daten, muss auf die Daten nicht parallel von vielen Systemen zugegriffen werden können.

\subsection{Redundanz}
Die Daten sollen in doppelte oder dreifacher Redundanz gespeichert werden.

\subsection{Speicherkapazität}
Es soll 300 Tebibyte an Speicherkapazität zur Verfügung stehen.

\subsection{Verfügbarkeit}
Die Verfügbarkeit soll dem AEC-2 Standard von Harvard Research entsprechen.

\subsection{Daten Integrität}
Dem Kunden wird einen Qualität Standard gewährleistet, die Original gespeicherten Daten, sollen vor Veränderung geschützt werden, aus diesen Grund soll die Daten Integriät beim Zugriff auf die Original Daten sichergestellt sein.

\subsection{Lokalität}
Die Daten sollen in einer minimal Version an einen weiteren Standort als Backup gehalten werden.

\section{Szenario 4 - Starkes Wachstum Daten / starkes Wachstum der Abfragen}
Dieses Szenario beschreib den Soll-Zustand, wenn sich das Datenwachstum und den Datenzugriff im vergleich zum Ist-Zustand stark steigert. Dieses Szenario würde eintreffen, wenn sich die Kunden Anzahl oder die Anzahl Bilddaten pro Kund stark steigert und somit auch die Anfragen zur Aufbereitung der Daten für den Druck.

Datenwachstum in Prozent pro Monat: 

Datenwachstum in Tebibyte pro Monat: 6 Tebibyte

Speichervolumen in 36 Monaten: 218,5 Tebibyte

Bilder mit einer Speichervolumen von 1 Gigibyte: 221'184

\subsection{Datenzugriff}
Auf die Daten müssen von verschiedenen Systemen gleichzeitig zugegriffen werden können.

\subsection{Redundanz}
Dem Kunden wird die Datensicherheit gewährleistet, aus diesen Grund sollen die Daten mindestens in doppelter oder dreifacher echter Redundanz gehalten werden.  Eine Redundanz durch Berechnung der Daten erfüllt diese Anforderung nicht sondern wird nur als Ergänzung für die Verfügbarkeit angesehen.

\subsection{Speicherkapazität}
Es soll 300 Tebibyte an Speicherkapazität zur Verfügung stehen.

\subsection{Verfügbarkeit}
Die Verfügbarkeit soll dem AEC-4 Standard von Harvard Research entsprechen. Bei hoher Kundenzahl und Speichervolumen, würde einen Unterbruch bei der Verfügbarkeit des Dienstes zu einen Repräsentation Schaden verursachen und Unsicherheit der Zuverlässigkeit bei den bestehenden Kunden in Bezug Ihrer Daten Verusachen.

\subsection{Daten Integrität}
Dem Kunden wird einen Qualität Standard gewährleistet, die Original gespeicherten Daten, sollen vor Veränderung geschützt werden, aus diesen Grund soll die Daten Integrität beim Zugriff auf die Original Daten sichergestellt sein.

\subsection{Lokalität}
Um eine Verfügbarkeit gemäss AEC-4 zu gewährleisten sollen die Daten Online über zwei Standorte zugreifbar sein. Eine Backup der Daten könnte an einen weiteren dritten Standort erfolgen.
%!TEX root=../documentation-bachlorthesis-speicherarchitektur-lstucker.tex
\cleardoublepage
\chapter{Speicherarchitekturen}

Sowohl Private Personen als auch Unternehmen, haben individuellen Anforderungen an die Speicherung Ihrer Daten. Während früher der Speicher der Daten in der Regeln aus internen Speicher eines Computersystems gehandelt haben, haben sich heute eine grosse Bandbreite an Speicherlösungen am Markt entwickelt. Ein Grund dafür ist die kontinuierliche steigende Anforderung an die Speicherkapazität aber auch Anforderungen an das Datenverwaltung, Datensicherung und Daten zur Verfügung zu stellen, machte es bis anhin laufend notwendig neue Lösungsansätze zu entwickeln oder bestehende Lösungsansätze zu erweitern. 

Die heutigen am Markt erhältlichen Speicherarchitekturen lassen sich in der Obersten Kategorie in  in Block- (Block-Based), Datei- (File-Based) und Objekt-Basierende adressierende Systeme unterteilt werden. Wobei man die Kategorisierung für die Einteilung nicht ganz exakt verallgemeinern kann, da einige Speicherlösungen Mischungen von mehren Kategorien sein können.

\section{Block-Basierend}
Die Block-Basierende Speicherarchitektur ist wohl die traditionellste und weit verbreitetste Form zum Speichern und Zuzugreifen von Daten. Die meisten Computersysteme, sei es Server, Desktop-PCs, Tablet-PC, Smartphones, Spielkonsole, speichern Ihre Daten in einen Blockbasierenden Speicher ab. Als Speicher werden in diesen Geräte meist magnetischen Festplatten, Solid State Disk oder Flash-Speicher eingesetzt.

Bei Block-Speicher werden Daten in Blöcke gelesen und gespeichert (adressiert), ein Block bildet sich aus einer Sequenz von Bits bzw. Byte. Die Grösse eines Blocks wird als Blocklänge bezeichnet, und ist bei allen Blöcken einer Einheit gleich gross. 

Experten wie Mike Mesnier, Greg Ganger und Erik Riedel, sehen jedoch bei zunehmender Speichergrösse und Komplexität von Systemen fundamentale Limitierungen von Block Schnittstellen.

\begin{quotation}
\em Since the first disk drive in 1956, disks have grown by over six orders of magnitude in density and over four orders in performance, yet the storage interface (i.e., blocks) has remained largely unchanged. Although the stability of the block-based interfaces of SCSI and ATA/IDE has benefited systems, it is now becoming a lim- iting factor for many storage architectures. As storage infrastructures increase in both size and complexity, the functions system designers want to perform are fundamentally limited by the block interface. \end{quotation}\cite{Mesnier2003}

Vergleicht man die erste Festplatte, welche von IBM produziert wurde, mit einer Seagate von 2011, hat sich die Speicherdichte von 2000 bit per Quadratzoll auf 625 Gigabyte und in der Geschwindigkeit von 8 Kilobyte auf 600 Megabyte verbessert\cite{Seagate2011}\cite{Seagate2011a}

Für den Zugriff auf Blockbasierende Speichersysteme werden meist Schnittstellen Protokolle wie Small Computer System Interface (SCSI) oder Advanced Technology Attachment (kurz \gls{ATA}) verwendet. Diese Protokolle wurden jedoch in einer Zeit entwickelt, wo man davon ausging, dass ein Block Speicher jeweils nur von einem Computersystem verwendet wurde und nicht mit mehreren Computersystemen geteilt wird. Dies Annahme stimmt in Consumer Elektronik Bereich meist noch heute, in Bereichen wo jedoch grosse Speicherkapazitäten oder eine grössere Verfügbarkeit gefordert ist, wie Sie im Geschäftsbereich vorkommen, stimmen diese Annahmen nicht mehr.

Blockbasierende Speicher, welche nicht aus Internen Speicher eines Server gebildet werden, unterscheidet man in Direct Attached Storage (kurz DAS) und Storage Area Network (kurz SAN). 

\subsection{Direct Attached Storage}
Bei DAS handelt es sich, wie es aus der englischen Bezeichnung zu entnehmen ist, um Speicher, welche direkt an ein Computersystem angeschlossen wird. Bei DAS Enclosure handelt sich um ein Gehäuse mit mehren verbauten Festplatten, welche üblich über einen Host-Bus-Adapter an ein Computersytem angeschlossen wird. Als Schnittstellen Protokoll werden \gls{ATA}, \gls{STA}, \gls{eSATA}, \gls{SCSI}, \gls{SAS} und Fibre Channel eingesetzt. DAS können mit mehreren Computersystemen geteilt werden, sofern genügen Schnittstellen zur Verfügung stehen.

\subsection{Storage Area Network}
Die Storage Networking Industry Association (kurz \gls{SNIA}) definiert ein Storage Area Network (kurz SAN) als ein Netzwerk, welch primärer Bestimmungszweck ist Daten zwischen Computersysteme und Speicherelemente und unter Storage Elemente zu transferieren. Ein SAN besteht aus einer Kommunikations-Infrastrukture, welches eine physische Verbindung und eine Management-Schicht beinhaltet, welches die Verbindungen, die Speichereinheiten und das Computersystem organisiert, sodass der Datentransfer sicher und robust erfolgen kann. Der Begriff SAN wird normalerweise (aber nicht notwendigerweise) mit dem Block I/O Service in Verbindung gebracht und weniger mit dem Datei-Zugriff-Service.\cite{SNIA2011}

Je nach SAN Implementierung kommen folgende Geräte bzw. Komponenten vor:
\begin{itemize}
\item Server
\item Host Bus Adapter
\item Gigabit Interface Converter
\item SAN-Switch
\item Speichersystem
\item Tape Library
\item Logical Unit
\end{itemize}

\paragraph*{Server} 
Der Server greift über das SAN auf Ressourcen von Speichersystem oder Tape Library. In einzelnen Fällen kann der Server selbst über SAN anderen Server Speicher zur Verfügung stellen.

\paragraph*{Host Bus Adapter}
Host Bus Adapter (kurz HBA) für das SAN sind intelligente Hardwareschnittstellen, welche für die Verbindung von Server in einem SAN verwendet werden. Sofern die Server nicht bereits mit einen Host Bus Adapter ausgerüstet sind, können die meisten durch Host Bus Adapter in form von Steckkarten erweitert werden. Der Host Bus Adapter selber hat pro Port ein Einschub in welche ein Gigabit Interface Converter eingebaut wird. \cite{Christopher2009}

\paragraph*{Gigabit Interface Converter}
Der Gigabit Interface Converter sind modulare Schnittstellen, welche Elektrische Signale in optische Signale umwandeln.\cite{SNIA2011}

\paragraph*{SAN-Switch}
Der SAN Switch ist ein Switch, welche dezidiert für das SAN Umgebung verwendet wird.

\paragraph*{Speichersystem}
Das Speichersystem stellt im SAN den geteilten Speicher zur Verfügung. Gemäss den \gls{IT} Marktforschung und Analyse Unternehmen Gartner gehören \gls{EMC}, \gls{IBM}, \gls{NetApp}, \gls{Dell}, \gls{HP}, \gls{Hitachi} zu den Marktführer\cite{RogerW.CoxPushanRinnenStanleyZaffos2011}.

\paragraph*{Tape Library}
Tape Library sind Bandbibliothek in welche ein oder mehre Bandlaufwerke und mehrere Magnetbänder befinden und der automatische Bandwechsel mittels eines Roboter realisiert wird. Die Tape Library werden für die Sicherung von Daten auf Band eingesetzt.

\paragraph*{Logical Unit}
Ein Logical Unit ist ein Gerät, welche über SCSI Protokoll andressiert mittel Logical Unit Number (kurz LUN) wird, weshalb oft auch wenn technisch nicht korrekt das Gerät als LUN bezeichnet wird. Im Speichersystem werden mehre Festplatten mittels RAID zu einer Einheit zusammengefasst, sofern keine weitere Virtualisierung von dem Speicherhersteller zum Einsatz kommt, wird die zusammengefasste Einheit wiederum in Speichereinheiten aufgeteilt und diese als LUN dem Server zugeteilt.\cite{SNIA2011}

\subsubsection{Fibre Channel}
SCSI ist zwar sehr populär, ist jedoch mit 80 Mbps Geschwindigkeit, maximal 25 Meter Bus länge, und mit maximal 32 Geräte pro Bus, ein limitierender Faktor für viele Anwendungen. Unteranderem wegen erwähnten Limitierungen von SCSI hat, das American national Standards Institute (ANSI) die Fibre Channel Technik entwickelt. Fibre Channel ist ein mehrschichtiges Netzwerk, welche die Charakteristische und Funktionen für die Übertragung von Daten über ein Netzwerk definiert. Der Standard beinhaltet von der physikalischen Schnittstelle, Daten Codierung, Übertragungssteuerung (Link Control), Fluss Kontrolle, bis hinzu den Protokoll Schnittstellen. In Vergleich zu anderen Netzwerken, beinhaltet die Fibre Channel Architektur einen signifikanten Anteil von Hardware Prozesse um eine hohe Performance zu erreichen.\cite{Gupta2002}\cite{Christopher2009}

Beim Design von Fibre Channel hat man darauf geachtet die Besten charakteristischen Eigenschaften von I/O Bus Kommunikation (Channel) zwischen zwei Geräte und der Netzwerk Kommunikation zwischen Mehren Geräte zu kombinieren. Die Channel Kommunikation ist im Vergleich zur Netzwerk-Kommunikation, Hardware-Intensive, schnell und produziert wenig Overhead. Netzwerk Kommunikation ist hingegen, abhängig von der Software Implementierung genannt Protokoll, unterstützt aber die Kommunikation von einer grossen Anzahl Geräten.

Anders wie es der Namen von Fibre Channel vermuten lässt, ist Fibre Channel nicht auf Fiberoptik-Kabel als Kommunikations-Medium beschränkt, sondern lässt sich auch auf Kupferkabel betreiben. Aufgrund von physikalischen Eigenschaften ist hier Fiberoptik-Kabel in Geschwindigkeit kombiniert mit Distanz dem Kupferkabel überlegen. So liegt die maximale Distanz bei Kupferkabel bei 30 Metern bei einer Geschwindigkeit von 1 Gbps, bei höheren Geschwindigkeiten wird die maximale Distanz noch weiter reduziert. Bei Fiberoptic-Kabel wird die maximale Distanz von der Qualität der Installation, des Fiberoptic-Kabel-Typ, der Kern-Durchmesser, der Lichtwellenlänge Rundreise Latenz und eingesetzter Hardware bestimmt. Je weiter das Licht innerhalb des Kabels übertragen werden muss, desto grösser ist der Verlust des Ursprünglichen Signal stärke. Mit spezieller Hardware können auch Distanzen von bis zu 600 Kilometer \footnote{\url{http://www.enterprisestorageforum.com/industrynews/article.php/2171801/Synchronous-SAN-Sets-Fibre-Channel-Distance-Record.htm}} erreicht werden.

Es gibt drei Fibre Channel Topolgien:
\begin{itemize}
\item Point-to-Point
\item Arbitrated-Loop
\item Switched-Fabric
\end{itemize}

\paragraph*{Point-to-Point-Topologie}
Die Point-to-Point-Topologie ist die direkte Verbindung von zwei Fibre Channel Geräte, meistens handelt es sich bei der Verbindung von einem Server und einen Speichersystem, wie Sie im Direct Attached Storage (kurz DAS) Umfeld vorkommt. \cite{Christopher2009}

\paragraph*{Arbitrated-Loop-Topologie}
Bei der Arbitrated-Loop-Topologie können bis zu 126 Knoten (NL\_Ports) an einen geteilten Bus Ring zusammengeschlossen werden. In diesen Ring kann eine Verbindung zwischen zwei Ports aktive sein, alle anderen Ports fungieren währende diese Verbindung aktive ist als Repeater und leiten das Signal weiter. Die Arbitrated-Loop-Topologie ist deshalb von der Architektur ähnlich wie dem Token Ring.\cite{Gupta2002}\cite{Christopher2009}

\paragraph*{Switched-Fabric-Topologie}
Die Klassischen SAN Topologie ist die Switched-Fabric-Topologie. Eine Switched-Fabric-Topologie besteht aus einer oder mehreren Switches die zu einer Fibre Channel Fabric zusammengeschlossen werden. Die einzelnen FC-Geräte, wie Server bzw. Storagesystem, werden über eine oder mehre Ports an eine Switched Fabric angeschlossen. In eine Fabric können bis zu $2^{24}$ Ports angeschlossen werden.\cite{Gupta2002}\cite{Christopher2009}

Mit der Switched-Fabic-Topologie lassen sich verschiedene Fabric Topologien bilden.
Die einfachste Topologie, welche das Design Ziel die Eliminierung von Single "'Point of Failure"' erfüllt, ist die Dual Switch Topologie, wie in der \refabb{abb:DualSwitchTopologie} dargestellt dient jeder Switch als eigenständige Fabric. Die FC-Geräte wie Server und Storagesystem werden jeweils pro Fabric bzw. Switch mit mindestens einen FC-Port angeschlossen. Durch den Einsatz von Path Management Software auf dem Server, kann eine vom Speichersystem zugeteiltes Logical Unit über mehre Path angesprochen werden. Diese Implementierung bietet gleich mehre Vorteile. Wenn ein Path oder eine ganze Fabric ausfällt übernimmt der andere Path automatisch für den ausgefallen Path. Bei Wartungsarbeiten an Komponenten einer Fabric kann der Service ohne Downtime weiter betrieben werden. Moderne Path Management Software und Speichersystem unterstützen zudem die Lastverteilung (Loadbalance) des I/O Last über alle Pathe.\cite{Christopher2009}

\begin{center}
\includegraphics[width=\linewidth, keepaspectratio = true]{media/}
\mycaption{figure}{\label{abb:DualSwitchTopologie}Fibre Channel SAN mit Dual Switch Topologie}
\end{center}

Die Meshed Fabric Topologie erhöht die Ausfallsicherheit zusätzlich innerhalb den einzelne Fabric. Für die Meshed Fabric sind pro Fabric mindestens vier Fibre Channel SAN Switches erforderlich. Jeder Switch wird wie in \refabb{abb} ersichtlich ist mit mindestens einen Path, den sogenannten Inter Switch Link (kurz ISL), zu allen anderen Switches in der Fabric Verbunden. Die Meshed Fabric kann den Ausfall von Mehren Kabel und Switch verkraften ohne das die ganze Fabric ausfällt.\cite{Christopher2009}

\begin{center}
\includegraphics[width=\linewidth, keepaspectratio = true]{media/}
\mycaption{figure}{\label{abb:MashedFabricTopologie}Fibre Channel SAN mit Mashed Fabric Topologie}
\end{center}

\subsubsection{iSCSI}
Das SCSI-Protokoll ist ein populäres Protokoll für die Kommunikation mit I/O Geräten, spezielle für Speicher Geräte. SCSI weist die Client-Server Architektur auf, wobei der Clients bei SCSI Interface als "initiators" bezeichnet wird und die logische Einheit vom Server als "target".

SCSI Protokoll wurde schon über Protokolle transportiert, jedoch waren all die Transport Protokolle limitiert in der Distanz. \gls{IBM} startete 1996 mit der Forschung für die Übertragung von SCSI über das Ethernet, dabei untersuchte \gls{IBM} ob sich der Transport mittels IP oder \gls{TCP/IP} besser eignen würde. Messungen zeigte da zumal, dass in einem lokalen Netzwerk, der Transport mittels IP besser eignet, anstelle von TCP/IP, mit der Extrapolation in die Zukunft, und den Transport über die lokale Netzwerkgrenze hinweg war aber der Weg mittels TCP/IP die bessere Wahl. 1999 hatten sich \gls{IBM} und \gls{Cisco} geeinigt "SCSI over TCP/IP" gemeinsam in eine Internet Engineering Task Force (kurz \gls{IETF}) Standart weiter zu entwickeln. \cite{JohnL.202} Die fertige Spezifikation von SCSI over \gls{TCP/IP} ist im \gls{RFC} 3720 mit dem Namen iSCSI in April 2004 fertiggestellt worden.\cite{J.Satran2004}

% \paragraph*{Kosten}
Für den Betrieb eines Fibre Channel SAN sind spezielle Hardware und Fibre Channel Kenntnis notwenig. Aufgrund, dass bei iSCSI die selbe Technik wie im Computernetzwerk verwendet wird, benötigt es für den Betrieb keine zusätzliche Ausbildung, Netzwerk Infrastruktur und Management Software Lösungen, was die Gesamtbetriebskosten (kurz TCO) senkt.

Grundsätzlich kann jeder Computer, welcher mit einem Netzwerkanschluss ausgerüstet ist und einen iSCSI Software Treiber hat, iSCSI nutzen. Computer welche Genügen Prozessor Leistung haben können die zusätzliche Last für die Verarbeitung von iSCSI mit konventionellen Netzwerkkarten lösen. Bei Computersystem, welche die Verarbeitungsgeschwindigkeit kritisch ist, wie bei Server kann diese zusätzliche Last negativ sein. Vergleichbar wie für Fibre Channel gibt es für iSCSI spezielle Netzwerkkarten bzw. Host Bus Adapter, welche mittels TCP/IP Offload Engine (kurz TOE) und volle iSCSI Offload Engine im eignen Chip die \gls{TCP/IP} bzw. iSCSI Pakete verarbeiten. Solche Netzwerkkarten entlasten durch die Verarbeitung der \gls{TCP/IP} und iSCSI Pakete im eignen Chip die Central Processing Unit (kurz \gls{CPU}) des Servers und weisen bessere Werte in der Latenz auf.

% \paragraph*{Netzwerk}
In einem Ethernet Netzwerk verwaltet sich jeder Switch mehr oder weniger autonome und führt eine eigene Weiterleitungstabelle, mit welcher der Switch entscheidet, über welchen Port eine Ethernet Packe ausgeliefert werden muss. Dazu enthält die Weiterleitungstabelle pro MAC Adresse den dazugehörigen Port. Trifft eine Ethernet Paket mit noch unbekannter MAC Adresse ein, leitet der Switch das Paket über alle Ports weiter. Durch die Rückantwort des Zielsystems lernt der Switch, über welchen Port, das System erreichbar ist. Werden in einen Switch Netzwerk benachbarte Switches untereinander über mehre Pfade Verbunden, kann es bei in einen solchem Szenario eintreffen, dass das Paket wieder am ursprünglichen Switch ankommt, wenn der benachbarte Switch die MAC-Adresse ebenfalls nicht kennt. Es entsteht dadurch eine Verdoppelung, der Ethernet Paket im Netzwerk bzw. es kommt zu einer Schleifenbildung, was wiederum zu Netzwerkstörungen führt. Mittels Spanning Tree Protokoll (STP) sollen solche Schleifen vermieden werden. Das Spannung Tree Protokoll erstellt eine Baum Topologie mit jeweils einer aktiven Path zwischen zwei Switches. Diese Topologie hat mehre Nachteile: 

\begin{itemize}
\item Beim Topologie-Wechsel wird im Netzwerk der Spanning Tree neu ausgehandelt, während dieser Neuaushandlung kommt es zu einem mindestens 15-Sekunden-Unterbruch. In dieser Zeit werden keine Ethernet Paket weiter geleitet. Eine Topologie Wechsel kann, zum Beispiel durch einen Pfad Ausfall zwischen zwei Switches hervorgerufen werden.

\item In einer Baum-Hierarchie, müssen die Pakete innerhalb des Baumes weitergeleitet werden und können nicht über einen theoretisch direkteren Pfad weitergeleitet werden. Befindet sich das Ziel auf der anderen Baum Seite, muss das Paket die ganze Hierarchie hinauf weitergeleitet werden und auf der anderen Seite hinunter bis zum Ziel System. Könnten die Switches über mehre Pfade miteinander Kommunizieren, könnte ein direkter Weg gewählt werden und in der Kommunikation währen weniger Switches belastet.
\end{itemize}

Hersteller wie Brocade haben diese Problematik für den Betrieb von iSCSI SAN erkannt und haben Lösungen entwickelt welche das Prinzip von Fibre Channel Fabrics für Ethernet Netzwerke umsetzen. Bislang ist darauf jedoch noch keinen allgemeinen Standard entstanden, weshalb es bei solchen Lösungen um porträtiere Lösungen handelt.

%\paragraph*{Sicherheit}
Wie beim Fibre Channel SAN sollte im Geschäftsumfeld iSCSI über ein dediziertes Netzwerk laufen. Die Abgrenzung erhöht die Sicherheit, das Storage Netzwerk ist somit klar abgeschottet von restlichen Netzwerken. Fehlerhafte Firewall Regeln im Computer Netzwerk haben keinen direkten Einfluss auf die Sicherheit des Datennetzwerkes. Störungen oder Überlast im Computer Netzwerk haben beeinflussen nicht die iSCSI Verbindungen. Mittels IPsec kann die Sicherheit durch eine sichere Authentifizierung und optionaler Verschlüsselung der Verbindung weiter erhöht werden. 

%\paragraph*{Integität}
iSCSI stellt die Integrität des übermittelten Paketes mit dem CRC-32c Digests sicher. Die Integrität auf anderen Ebenen, wie Speichersystem, Computer Bit Fehler auf Speichersystem-Ebene oder Memory des Computersystems können. 

\paragraph*{Datenverfügbarkeit / Redundanz}


\paragraph*{Skalierbarkeit Datenvolumen}
Einen Server können mehre Logical Unit zugeteilt werden. Durch den Einsatz eines Volume Manager können mehre Logical Unit zu einer grossen Logischen Volume zusammengefasst werden. Sollte die Kapazität einer Speichersystems nicht ausreichen kann ein weiteres Speichersystem ans SAN angeschlossen werden.

\paragraph*{Durchsatz I/O}
Die Firma Netapp zählt zu den Marktführen in Bereich Unternehmens NAS Speicherlösungen. Neben dem "Network File System" (NFS), beherrschen die Speicherlösungen von Netapp auch das zur Verfügungsstellen von Logical Units über iSCSI als auch über Fibre Channel. Saad Jafri und Chris Lemmons von Netapp haben die Bereitstellung von Speicher über die verschiedenen Verfahren, bezüglich Performance für eine VMWare vSphere Umgebung untersucht. Netapp weisst in Ihren Report nicht die konkreten Messzahlen aus, sondern die Werte in Vergleich zu einen 4Gb Fibre Channel.

Wie im \refabb{abb:NetappIOPS} von Netapp zu entnehmen ist sind die I/O pro Sekunden Werte von iSCSI in einen 1Gb Ethernet Netzwerk in Vergleich zu 4Gb Fibre Channel rund 8\%tiefer. Wobei höhere Werte bei I/0 pro Sekunden besser sind. Im 10 Gb Ethernet Netzwerk erreicht iSCSI dieselben Werte wie Fibre Channel in einen 8Gb Netzwerk.\cite{Jafri2011}

\begin{center}
\includegraphics[width=\linewidth, keepaspectratio = true]{media/netapp_iops.png}
\mycaption{figure}{\label{abb:NetappIOPS}Netapp IOPS durchsatz für alle unterstützte Protokolle in Vergleich zu 4Gb FC mit 8K Block grösse\cite{Jafri2011}}
\end{center}

Der Report von Netapp hat ebenfalls die Latenz verglichen. Bei Latenz möchte man möglichst einen tiefen Wert erreichen. Gemäss \refabb{abb:NetappLatenz} ist die Latenz von iSCSI in einen 1 Gb Ethernet Netzwerk rund 9\% höher als bei Fibre Channel in einen 4Gb Netzwerk. Bei iSCSI in 10Gb Ethernet waren die Werte gleich wie Fibre Channel in 4Gb Netzwerk. Fibre Channel in einen 8Gb Netzwerk hatte jedoch rund 1\% tiefere Werte.\cite{Jafri2011}

\begin{center}
\includegraphics[width=\linewidth, keepaspectratio = true]{media/netapp_latence.png}
\mycaption{figure}{\label{abb:NetappLatenz}Netapp Latenz für alle unterstützte Protokolle in Vergleich zu 4Gb FC mit 8K Block grösse \cite{Jafri2011}}
\end{center}
 
\subsection{Logical Volume Manager}
Logical Volume Manager oder Dateisysteme welche die Logik eines Logical Volume Manager Kombinieren, ermöglichen es mehre Block-Geräte bzw. Logical Units zu einer grossen Logischen Volume zusammen zufassen. Diese hat den Vorteil, das die maximale Grösse eines Block-Gerätes nicht der limitierende Faktor des darauf installierten Dateisystems ist. Neben dem erstellen von Logischen Volume, können einige Logical Volume Manager den Last-Zugriff auf die verschiedenen Block-Geräte durch Striping optimaler verteilen und sorgen damit für eine besser Performance beim Datenzugriff. Eine weitere Aufgabe, die Logical Volume Manager übernehmen, ist die redundante Haltung der Daten durch Spiegelung. Mittel serverseitige Spiegeln (Host-Based Mirroring) können die Daten mittels der Spiegelungsfunktion des Logical Volume Manager auf zwei Standorte zur Verfügung gemacht werden. Dazu werden von zwei Speichersystemen, welche an unterschiedlichen Standorten installiert sind, gleich viele und gleich grosse Logical Units über das SAN dem Computersystem zugeteilt werden.

Klassische Server Linux Distributoren, wie Red Hat, Suse und Debian inkl. Ubuntu liefern den Quelloffenen Logical Volume Manager 2 (LVM2) in Ihrer Distribution mit. LVM2 kann theoretisch auf einen 64Bit System eine Logical Volume von 8 Exabyte bilden und adressieren.\cite{Levine2009}

Das ursprünglich von Oracle als ZFS Ersatz entwickelte Btrfs Dateisystem, könnte sich zukünftig als Standard Dateisystem vieler Server Distributionen entwickeln. Das für Solaris entwickelte ZFS als auch Btrfs kombinieren Dateisystem und Logical Volume Manager. Btrfs selbst wurde jedoch noch nicht als stabile Version veröffentlicht und ist deshalb für den produktiven Betrieb noch nicht zu empfehlen.\cite{Redler2011}


\subsection{Datei System}
Betriebssysteme adressieren die Daten auf einem Block-Geräte nicht direkt an, sondern greift über ein Dateisystem zu. Das Dateisystem organisiert wie und wo Dateien auf dem Block-Geräte abgelegt werden und Verwalten den freien Speicher. Einige Datei Systeme regeln auch die Zugriffsberechtigungen auf Dateien. Ein Block-Geräte (Logical Unit) können im SAN oder DAS an Mehren Computersysteme gleichzeitig zugeteilt werden, jedoch ist es die Aufgabe des Dateisystems sicherzustellen, dass mehre Computersysteme gleichzeitig auf das gleiche Dateisystem lesen bzw. schreiben können. Konventionelle Dateisysteme wie Ext3 unter Linux gehen von einer exklusiven Nutzung des Speichers aus, weshalb diese Dateisysteme keine Funktionen implementiert haben, die den gleichzeitigen Zugriff regeln. Problematik beim gleichzeitigen Zugriff ist die Sicherstellung der Konsistenz. Schreiben zum Beispiel zwei Computersysteme gleichzeitig in dieselbe Datei, kann nicht sichergestellt werden, welche Änderung gültig ist, und führt zu Inkonsistenz. Dateisysteme, welche den gleichzeitigen Zugriff von Mehren Computersysteme unterstützen, regeln den Zugriff auf Dateien mit Sperrmechanismen (locking). Schreibt ein Computersystem in eine Datei, wird die Datei vom Dateisystem vor der Änderung anderen Computersystem gesperrt. Cluster Dateisysteme wie Red Hat Global Filesystem (GFS) und Red Hat Global Filesystem 2 (GFS2) unterstützen dieses Sperren. Das Dateisystem Red Hat GFS Version 2 unterstützt bei einem 64-bit-System theoretisch Dateisysteme bis 8 Exabyte, Red Hat gewährleistet jedoch nur einen Support von maximal 100 Terabyte.\cite{Levine2011}

Dateisysteme wie ZFS und Btrfs stellen die Integrität der Daten vor Veränderung, wie Sie zum Beispiel durch ein Bit Fehler auf dem Block Geräte oder in Memory entstehen können, mittels Prüfsumme sicher. Dabei wir zur jeder Datei eine Hash Prüfsumme abgespeichert, wird die Datei gelesen wird die Prüfsumme aus der Datei neu berechnet und mit der abgespeicherten Prüfsumme verglichen, sofern das Dateisystem ebenfalls gespiegelt ist, korrigiert das Dateisystem die fehlerhafte Datei aus der intakten Kopie. Dieses Verfahren wird auch als Selbst heilend (Self-healing) bezeichnet. \cite{Bonwick2005}\cite{Oracle}

Dateisysteme können mittels Sicherungssoftware auf Bandlaufwerke oder andere Speichermedien gesichert werden.


\subsection{Zusammenfassung}

\begin{table}[htbp]
\caption{Umgekehrte Relationen der Bewertungsskala}
\begin{tabular}{|L{3.5cm}|L{10cm}|}
\hline
Redundanz & Logical Volume Manager oder Raid (Hard bzw. Software) \\ \hline
Standortübergreifend & Einsatz von SAN und Logical Volume Manager notwendig \\ \hline
Skalierbarkeit Datenzugriffe &  \\ \hline
Performance &  \\ \hline
POSIX IO & Bei Unix Dateisystem \\ \hline
Simultaner Lesezugriff & Nur mit einen Cluster fähigen Dateisystem möglich \\ \hline
Simultaner Schreibzugriff & Nur mit einen Cluster fähigen Dateisystem möglich \\ \hline
Skalierbarkeit Speicherkapazität & Mit einen Logical Volume Manager, können mehre Blockgeräte zu einen grossen Zusammen gefasst werden \\ \hline
Max Anzahl speicherbare Objekte &  \\ \hline
Max Objekte Speichergrosse &  \\ \hline
Datenintegrität & Dateisysteme wie Btrfs und ZFS speichern für jede Datei einen Hashwert anhand dieser kann die Integrität sichergestellt werden, jedoch ist Btrfs noch nicht Produktive einsetzbar und ZFS läuft unter Linux nur im Userspace (FUSE) und nicht im Kernel. \\ \hline
Selbstheilung von Objekten & Dateisysteme wie Btrfs und ZFS haben diese Funktion integriert, jedoch ist Btrfs noch nicht Produktive einsetzbar und ZFS läuft unter Linux nur im Userspace (FUSE) und nicht im Kernel. \\ \hline
Sicherung & Normales Sicherungsverfahren \\ \hline
Sicherheit & Wird von Dateisystem bestimmt \\ \hline
\end{tabular}
\label{UmgekehrteBewertungsskala}
\end{table}

\begin{table}[htbp]
\caption{Umgekehrte Relationen der Bewertungsskala}
\begin{tabular}{|L{3.5cm}|L{10cm}|}
\hline
Redundanz & Mit Switched-Fabric-Topologie können redundante SAN Netzwerke erstellt werden. Zudem können mehre Speichersysteme im SAN Verfügbar gemacht werden. \\ \hline
Standortübergreifend & Eine Fibre-Channel SAN kann Standortübergreifend über eine ISL Verbindung Verfügbar gemacht werden. Beim Betrieb von zwei Standorte kann pro Standort je eine Speichersystem und eine Cluster Node platziert werden. \\ \hline
Skalierbarkeit Datenzugriffe & Durch hinzufügen von mehren Verbindungen durch gehend von Server, über die Switches bis hin zu den Speichersystem, kann der Datenzugriff skaliert werden. \\ \hline
Performance &  \\ \hline
POSIX IO & wird durch Dateisystem sichergestellt \\ \hline
Simultaner Lesezugriff & hängt von Dateisystem ab \\ \hline
Simultaner Schreibzugriff & hängt von Dateisystem ab \\ \hline
Skalierbarkeit Speicherkapazität & Einen Server kann druch zur Verfügungstellen weiterer Logical Units (Blockgeräte), mehr Speicher zur Verfügung gestellt werden. Die Speicherkapazität im ganzen SAN kann durch hinzufügen von weiteren Speichersystem oder grosseren Speichersystem skaliert werden. \\ \hline
Max Anzahl speicherbare Objekte & hängt von Dateisystem ab \\ \hline
Max Objekte Speichergrosse & hängt von Dateisystem ab \\ \hline
Datenintegrität &  \\ \hline
Selbstheilung von Objekten & hängt von Dateisystem ab \\ \hline
Sicherung & - \\ \hline
Sicherheit &  \\ \hline
\end{tabular}
\label{UmgekehrteBewertungsskala}
\end{table}

\begin{table}[htbp]
\caption{Umgekehrte Relationen der Bewertungsskala}
\begin{tabular}{|L{3.5cm}|L{10cm}|}
\hline
Redundanz & Durch hinzufügen von mehren Verbindungen durch gehend von Server, über die Switches bis hin zu den Speichersystem, und den einsatzt von Link Aggregation. Bei einen Ausfall eines Switch bzw. Topologie wechesel kann es beim einsatz von Spanning-Tree zu einen mindestens 15-Sekundigen Unterbruch führen. Alternative Ethernet-Fabric, ist jedoch nicht Standardisiert sondern Hersteller Spezifisch. \\ \hline
Standortübergreifend & Eine ISCSI-SAN kann Standortübergreifend Verfügbar gemacht werden. Beim Betrieb von zwei Standorte kann pro Standort je eine Speichersystem und eine Cluster Node platziert werden. \\ \hline
Skalierbarkeit Datenzugriffe & . \\ \hline
Performance & Bei 10 Gigabit Vergleichbar mit 8 Gigabite Fibre-Channel \\ \hline
POSIX IO & wird durch Dateisystem sichergestellt \\ \hline
Simultaner Lesezugriff & hängt von Dateisystem ab \\ \hline
Simultaner Schreibzugriff & hängt von Dateisystem ab \\ \hline
Skalierbarkeit Speicherkapazität & Einen Server kann druch zur Verfügungstellen weiterer Logical Units (Blockgeräte), mehr Speicher zur Verfügung gestellt werden. Die Speicherkapazität im ganzen SAN kann durch hinzufügen von weiteren Speichersystem oder grosseren Speichersystem skaliert werden. \\ \hline
Max Anzahl speicherbare Objekte & hängt von Dateisystem ab \\ \hline
Max Objekte Speichergrosse & hängt von Dateisystem ab \\ \hline
Datenintegrität & ISCSI Pakete sind zusätzlich zur 16bit TCP Prüfsumme durch eine CRC-32c Prüfsumme vor Bit-Fehler oder manipulationen gesichert. \\ \hline
Selbstheilung von Objekten & hängt von Dateisystem ab \\ \hline
Sicherung & - \\ \hline
Sicherheit & Kann durch IP-Sec gesichert werden \\ \hline
\end{tabular}
\label{UmgekehrteBewertungsskala}
\end{table}


\section{Datei-Basierend}
Bei Datei basierte Speicherarchitekturen werden Daten nicht wie bei Block basierten Speicherarchitekturen über Blocke adressiert, sondern über Dateien.

Mit dem Aufkommen von Desktop-Computer, wurden die Daten nicht mehr Zentral auf einen Mainframe bearbeitet, sondern vermehrt verteilt auf den einzelnen Desktop-Computer. Ohne Vernetzung der Computer mussten die Daten mittels portablen Speichermedien ausgetauscht werden. Diese mag noch in kleinen Umgebungen praktikabel gewesen, sobald jedoch die Anzahl Teilnehmer steigt, wird es schwierig, den Überblick über seine Daten zu haben. Die reine Vernetzung der Computer und den direkten Austausch zwischen den einzelnen Desktop-Computer über eine Netzwerk bietet hier ebenfalls keine Lösung. Die bessere Lösung dieses Problems ist eine Zentraler Speicher in welcher die Dokumente (Dateien) gespeichert sind und mittels Netzwerk Protokoll den Desktop-Computer zur Verfügung gestellt werden. Somit ist es für den Anwender klar welche Dateien existieren und hat zugriff auf auf die aktuelle Version. Unternehmen wie Sun Microsystem, IBM, Microsoft und Apple erkannten ebenfalls dieses Bedarf und entwickelten für Ihre Betriebsysteme Software und dazugehörige Protokolle für den geteilten Datenzugriff.

Zu den bekanntesten und weitverbreitesten Lösungen zählen NFS und \gls{CIFS} (SMB).


\subsection{Network File System}
Das ursprünglich rein von der Firma SUN Microsystems (heute Oracle) 1984 entwickeltes Network-File-System, ermöglicht den gemeinsamen Zugriff von mehreren Computersystemen auf das Dateisystem eines anderen Host (Server), als ob er zugriff auf ein lokales Dateisystem stattfindet. Die zweite Version von NFS erschien 1989 und war die erste Version welche von Internet Standard Request for Comments (kurz \gls{RFC}) Standardisiert wurde und unter der \gls{RFC} Nummer 1094 \footnote{\url{http://tools.ietf.org/html/rfc1094}} veröffentlicht wurde. Für den Transport des NFS Protokoll wurde bis Version zwei ausschliesslich das \gls{UDP} Transportprotokoll unterstützt.
Die Version 3 von NFS (\gls{RFC} 1813)\footnote{\url{http://tools.ietf.org/html/rfc1813}}, die 1995 veröffentlicht wurde, war die erste Version, welche Maschinen, Betriebsystem und Netzwerk Architektur, und Transport-Protokoll unabhängig ist. Die Unabhängigkeit wird mit der Verwendung von Remote Procedure Call (\gls{RPC}), welches wiederum ein eXternal Data Representation (kurz \gls{XDR}) verwendet, erreicht. \cite{Stern2001}

Wie im \refabb{} zu entnehmen ist, ist NFS eine weitere Schicht, welche auf dem Dateisystem und dessen Block-Geräte des Computersystems bzw. Speichersystem aufbaut. So ist es zum Beispiel die Aufgabe des Dateisystems bzw. des Block-Gerätes sich um die Redundanz und Integrität der gespeicherten Daten zu kümmern. 

Die Konsistenz der Daten, bei gleichzeitigem Zugriff von Mehren Computersysteme, stellt NFS mit einem separaten Protokoll, genannt Network Lock Manager (kurz NLM) sicher. Der Network-Lock-Manager sorgt dafür, dass eine Datei, welche von einem Computersystem geändert wird, vor der Änderung anderen Computer Systeme gesichert ist. Wenn ein Client eine Sperrung angefordert hat, muss der Client, nach nicht mehr gebrauch der Sperrung, dem Server die Entsperrung mitteilen. Dieser Implementierung der Sperrung führt jedoch zu Problemen, wenn der Client ein System Absturz erleidet, in diesen Fall kann der Client die Entsperrung dem Server nicht mitteilen und die Datei bleibt für gesperrt.
NFS setzt mit NLM das Advisory Locking Sperr-Verfahren ein, dass bedeutet, dass ein weiter Client beim Zugriff auf eine gesperrte Datei nur hingewiesen wird, dass die Datei gesperrt ist, aber nicht den Client zwingt, keine Änderung vorzunehmen.\cite{Stern2001}


Seit Version 4 von NFS Protokoll (\gls{RFC} 3530) ist das Sperre-Verfahren (Locking) im Protokoll selber implementiert, dadurch entfällt der zusätzliche Einsatz von Network Lock Manager. NFS Version 4 beherrscht das Sperren von Bytes-Bereich in einer Datei. Zudem erhält der Client von Server nur einen Leasing-Zeitraum für eine Sperrung (Lock), welcher er vor Ablauf wieder erneuern muss, um die Sperrung aufrecht zu halten. NFS in der Version 4 unterstützt das Sperr-Verfahren Mandatory Locking, dass bedeutet ein weiter Client kann, sich über die Sperrung nicht hinwegsetzen.\cite{Callaghan2003}

Dadurch, dass NFS auf TCP/IP als Kommunikation Protokoll aufbaut, kann eine NFS Freigabe, Standort übergreifend verfügbar gemacht werden, jedoch gilt auch hier das die Latenz und die Bandbreite der Verbindung zwischen den Standorten der limitierende Faktor ist.

NFS selbst hat keine eigene Implementierung für die Sicherstellung der Integrität der übermittelten Daten, stattdessen verlässt sich NFS seit Version 2 auf TCP und Ethernet Fehler Erkennung. TCP prüft im Standard verfahren die Integrität mit einer 16-Bit-Integer Prüfsumme. Die 16-Bit-Integer Prüfsumme erkennt Fehler im pseudo IP-Header, TCP-Header und Daten. Das Verfahren hat jedoch schwächen bei Einzel Bit-Fehlern Erkennung. Ethernet verwendet für die Fehlererkennung einen CRC32 Prüfsumme, diese gilt als effektive in der Erkennung Behebung von Bit Fehler, bietet aber keine durchgehend (End-to-End) Schutz. Grund dafür ist, dass beim Wechseln des Pakets der Kollisionsdomäne, wie es bei einen Swtich oder Router der Fall ist, jedes Mal eine neu CRC32 Prüfsumme erstellt wird.\cite{JohnL.202}. Bei NFS ab Version 4 kann der Datentransfer zusätzlich mit Kerberos abgesichert werden. Kerberos hat einen strengen Schutz gegen Manipulationen am Datenpaket und stellt somit die Integrität der Daten sicher. Nachteil ist aber das Kerberos zusätzlich eingerichtet werden muss. 

Bis und mit Version 4, wahren die Verarbeitung der Metadaten und die Verarbeitung der Daten in einem Protokoll und Server implementiert. NFS skalierte deshalb bis anhin bei der Verarbeitung von Dateien mit grosser Speicherplatz bedarf nicht ausreichend. Fragt einen NFS Client einen NFS Server für eine Datei an prüft der Server die Metadaten, die Metadaten enthalten den Speicherort, die Grösse, das Erstellungsdatum und das Änderungsdatum einer Datei, und wandelt die Anfrage in einen Disk I/O um, die Daten der Datei werden gesammelt und über das Netzwerk übertragen. Bei kleinen Dateien verwendet der Server die meiste Zeit für das sammeln der Daten, bei Grössen Dateien ist der limitierende Faktor der Transfer der Daten über das Netzwerk selbst.
Mit der Entwicklung von pNFS wurde der Transfer der Daten parallelisiert. Die Architektur von NFS wurde dazu in mehre Komponenten aufgeteilt. Der NFS Server besteht neu aus einen getrennten Metadaten Server und eine oder mehre Daten-Servern. Die Aufgabe des Metaservers ist die Verwaltung der Daten wo und wie die Daten gespeichert sind. Die Daten Servern, wo die Dateien gespeichert sind, kümmern sich um lese und schreib Anfragen von den Clients.
Bei einer anfrage an einer grosse Datei können Mehrere Daten-Server parallel Teile von der Datei dem Client ausliefern, der Client kann dann die Verschiedenen teile der Datei wieder zusammen setzen zu einer ganzen Datei. \cite{Shepler2010}\cite{Group2010}

Mit der NFS Version 4.1 wurde pNFS Bestandteil von NFS und ist seit 2010 im \gls{RFC} 5661 standardisiert. Server Linux Distributoren, wie Red Hat haben allerdings NFS 4.1, erst als Vorschau in Ihrer Distribution integriert.\cite{EastJacquelynnMichaelHidep-Smith2011}




\subsection{NAS Appliance}

Network Attached Storage (kurz NAS) sind Speichersystem mit angepassten Datei System für den gemeinsamer Dateizugriff in einen Hetrogenen Computer Netzwerk welche über ein LAN angeschlossen sind. Als Speicher verwenden NAS je nach Typ interne Festplatten, Direct Attached Storage oder über eine SAN angefügten Speicher.
An Clients stellen NAS Ihren Speicher über NFS, CIFS, ISCSI zur Verfügung. High-End NAS können Ihren Speicher wiederum über Fibre-Channel zur Verfügung stellen.

Gemäss Gartner gehören die Anbieter \gls{IBM,} \gls{EMC} und \gls{NetAPP} zu den führenden NAS Anbieter in Midrange und High-End bereich. Wobei gemäss Garnter Magic Quadrant Netapp zusammen mit EMC zu den innovativsten Anbieter.

\begin{quotation}
\em 
\textbf{Strengths}
\begin{itemize}
\item NetApp remains one of the few truly unified storage providers among all top-tier vendors, with its software features continuing to be industry benchmarks. The company was able to regain some of the NAS revenue market share that it had lost in 2009. Its fast revenue growth in 2010 was driven by its successful campaign targeted at midsize enterprises with the value propositions of NFS supporting VMware and unified storage in consolidating Windows application storage.

\item In 2010, NetApp increased its aggregate up to 100TB with Data ONTAP 8.0.1 and introduced compression to complement its popular deduplication capability. It added a RESTful object storage interface (based on its acquisition of Bycast) to its unified storage, targeting global content repositories. On the hardware side, it launched new systems with better performance and denser disk shelves.

\item NetApp's new software bundles have simplified the procurement process and made software pricing more affordable. For customers seeking converged infrastructure, NetApp launched FlexPod for VMware with its partners Cisco and VMware, offering packages including servers, storage and switches.
\end{itemize}
\textbf{Cautions} 
\begin{itemize}
\item The vast majority of the Data ONTAP 8.0 adoption was on the 7 mode (instead of the cluster mode) for larger aggregates, while the early adoption of the cluster mode focuses on high- performance NFS file services. The cluster mode is not ready for mainstream enterprise customers who require those 7-mode features that are still missing in the cluster mode. The ONTAP 8.1 scheduled for release later this year will likely continue to support the two modes: clustered and nonclustered modes of operation.
\item While NetApp continues to enjoy its leading edge in unified storage, it's facing fiercer competition in the high-end NAS market, where file systems larger than 100TB are required and where high performance without the expensive Flash Cache is desired.
NetApp is also challenged in the low-end NAS and unified storage market with new products from both major and emerging competitors.
\end{itemize}
\end{quotation}\cite{IEEE2003}



\subsection{Zusammenfassung}


\section{Objekt-Basierend}

\subsection{Verteilte Dateisysteme}
Unternehmen wie Google, welche Web-Applikationen mit Millionen von Anwendern Betreiben und der Speicherbedarf von hunderten Terrabyte bis Petabyte an Daten,  haben hohe Anforderungen an Ihr Speichersystem. Google hat für Ihren Bedarf eine eigenes Verteiltes Dateisystem genannt Google Filesystem entwickelt. Google hat beim Designe und des Dateisystem angenommen, das es auf gewöhnlichen und günsige Hardware laufen kann, welche aber öfter Komponenten Fehler habe. Daher ist der Ausfall von Komponenten nicht eine Sonderfalls sondern gehört zur Normalität. Zudem handelt sich bei den gespeicherten Dateien eher um grössere Dateien mit 100 Megabyte bis Multi Gigabyte an grösse. Die Auslastung ist primär durch zwei Arten von lese Vorgänge verursacht. Das lesen eines ganzen Datenstroms und das regellos Lesen. Die Schreibbelastung wird durch grosse Sequentiele schreib Vorgänge verursacht und Dateien werden erweitert als modifiziert. Als Architektur hat Google einen Cluster gewählt bestehend aus einen einzigen Metadaten Server und mehren Chunksserver. Die Daten werden bei Google in Einheiten genannt Chunks unterteilt und jeder Chunk bekommt eine eindeutige Identifizierung. Diese Chunks werden über mehre Chunkserver repliziert um die Ausfallsicherheit zu gewährleisten. Der Metadaten Server speichert in seinen Arbeitsspeicher die ganze metadaten bestehend aus Namensraum, Berechtigung Informationen, die Zuordnung der Datei zu den Chunks, und die Speicherort der Chunks. Google hat Ihre Dateisystem bis anhin nicht veröffentlich, hat jedoch eine  Studie über die Desinge und Architektur Prinzipen veröffentlicht. Einige erhältliche Verteileten Dateisysteme, wie Hadoop Distributed Filesystem (kurz HDFS), CloudStore und GLORY-FS beruhen auf den selben Architektur Prinzipen aus der Studie.


 Diese Studie galt für eine Verteilte Dateisysteme, wie Hadoop Distributed Filesystem (kurz HDFS), CloudStore, GLORY-FS,
 Das Google Filesystem 

welche die


 Google stellte an sein Dateisystem folgende Anforderungen: 
\begin{itemize}
\item Komponenten Fehler sind die Norm
\item Multi-Gigabyte Dateien sind häufig
\item Dateien werden werden hauptsächlich erweitert als überschrieben
\end{itemize}

Für diese Hauptanforderungen hat Google eine eigenes Dateisystem Entworfen und Entwickelt.




\cleardoublepage
\chapter{Machbarkeitsnachweis}

Der selbe Hosting Provider des bestehenden Servers bietet eine Produkt an welches mit 15 SATA Festplatten mit einer Speicherkapazität von je 3 Terabyte an. Die Maximale Raid-5 Speicherkapazität \ref{eqn:maxRaid5-15disk} ist aufgrund der Anzahl Disk und der Speicherkapazität nicht die Ideale Konfiguration, da die Gefahr eines Doppelten Disk Ausfall durch die geringere MTTF \ref{eqn:MTBF15Disk}, welche durch die Zunahme der Anzahl Festplatten sinkt \ref{eqn:MTBF1Disk} und der höheren Rebuild Zeit steigt. Speichersystem Hersteller wie NetApp setze bei 
dieser Konfiguration auf Raid-6 welche doppelte Parität bietet und somit zwei Festplattenausfälle kompensieren können.

Festplattenkapazität in Tebibyte:
\begin{equation}
3   \, \mathrm{TB} =  3 * \frac{1000^4}{1024^4} = 2.7285  \, \mathrm{TiB}
\label{eqn:3TerrabyteTebibyte}
\end{equation}

Maximale Raid-5 Speicherkapazität:
\begin{equation}
(15 -1) * 2.7285  \, \mathrm{TiB} =  38.199 \, \mathrm{TiB}
\label{eqn:maxRaid5-15disk}
\end{equation}

MTBF 1 Festplatten (ST33000650SS): 
\begin{equation}1'200'000  \, \mathrm{h}
\label{eqn:MTBF1Disk}
\end{equation}

MTBF 15 Festplatten (Raid-5) (ST33000650SS): 
\begin{equation}
\frac{1'200'000  \, \mathrm{h}}{15}= 8'000  \, \mathrm{h}
\label{eqn:MTBF15Disk}
\end{equation}
\cleardoublepage



%\pagenumbering{roman}
%\setcounter{page}{1}
% \input{chapter/Listings}
% %!TEX root=../documentation-bachlorthesis-speicherarchitektur-lstucker.tex

\newglossaryentry{RFC}{ name={RFC}, description={Request for Comments (RFC) sind Dokumente, über Internet, inklusive der technischen Spezifikation und Richtlinien, welche von der Organisation Internet Engineering Task Force entwickelt wurde. "'Das RFC wird erst nach erfolgter Diskussion unter der Aussicht des Internet Architecture Board (IAB) herausgegeben und fungiert als Quasistandard. Jedes RFC enthält eine eindeutige, vorlaufende Nummer, die kein zweites Mal zu gewiesen wird."' \cite{MicrosoftComputerLex}  \url{http://www.rfc-editor.org/}}}

\newglossaryentry{UDP}{ name={UDP}, description={  adfajsdfjadslkfjaödjfölaksdjfajsklfj }}

\newglossaryentry{IBM}{ name={IBM}, description={International Business Machines Corporation (kurz IBM) ist ein führendes unternehmen in Software, Hardware und IT-Dienstleistung Bereich. }}

\newglossaryentry{IBM}{ name={CIFS}, description={Common Internet File System (kurz CIFS) wurde 1996 von Microsoft eingeführt und beschreibt eine erweiterte Version von SMB. CIFS und SMB sind eine  Netzwerkdateisystem vergleichbar mit NFS und wird vorwiegend im MS Windows Bereich eingesetzt. }}


\newglossaryentry{XDR}{ name={XDR}, description={Die eXternal Data Representation (kurz XDR) Spezifikation stellt ein Standardisierte Verfahren zur Präsentation von gebräuchlichsten Daten Typen über das Netzwerk zur Verfügung. Dies löst das Problem der verschiedenen Byte-Reihenfolge (Big Endian), Speicherausrichtung auf unterschiedlichen Kommunikations Partner}}

\newglossaryentry{POSIX}{ name={POSIX}, description={Portable Operating System Interface (kurz POSIX) ist eine von IEEE entwickelter Standard, welche die Schnittstelle zwischen Applikation und Betriebsystem darstellt. Die aktuelle Version des Standards ist IEEE Std 1003.1-2008 \url{http://www.opengroup.org/austin/papers/posix_faq.html} }}

\newglossaryentry{FUSE}{ name={FUSE}, description={Filesystem in Userspace (kurz FUSE), ermöglicht die Implementierung eines voll Funktionsfähigen Dateisystem in Userspace. Normaler weise laufen 
FUSE wurde urspünglich Entwickelt um AVFS zu unterstützen, ist jedoch heute ein seperates Projekt. \url{http://fuse.sourceforge.net/}}} 

\newglossaryentry{RPC}{File locking erlaubt es einen Prozess den exklusiven Zugriff auf eine Datei oder teile einer Datei und zwingt ander Prozesse die auf die selbe Ressource zugreifen wollen zu warten bis das Locking aufgehoben wurde.} 

\newglossaryentry{FileLocking}{File locking erlaubt es einen Prozess den exklusiven Zugriff auf eine Datei oder teile einer Datei und zwingt ander Prozesse die auf die selbe Ressource zugreifen wollen zu warten bis das Locking aufgehoben wurde.} 

\newglossaryentry{API}{ name={API}, description={Application Programming Interface (kurz API) auch Anwendungsprogrammierschnittstelle genannt. "'Ein Satz an Routinen, die vom Betriebsystem des Computers für die Verwendung aus Anwendungsprogrammen heraus angeboten werden und diverse Dienste zur Verfügung stellen."' \cite{MicrosoftComputerLex} }}

\newglossaryentry{MIT{ name={MIT}, description={Die MIT Lizenz stammt von Massachusetts Institute of Technology und erlaub die die Verwendung von Software welche Quelloffen ist als auch software welche nicht Quell geschlossene ist. Die genauen Lizenz Bestimmungen sind unter folgenden URL zu finden \url{http://www.opensource.org/licenses/mit-license.php}}}

\newglossaryentry{GNU GPL{ name={GNU GPL}, description={Die GNU General Public License Lizenz auch GPL genannt stammt von der Free Software Foundation und regelt die Lizenzierung von Freie Software. Es gibt drei Versionen der GPL welche unter folgenden URL beschrieben sind \url{http://www.gnu.org/licenses/}}}

\newglossaryentry{Ruby{ name={Ruby}, description={Ruby ist eine interpretierte und objektorientierte Programmiersprache und beinhaltet einige bewährte Prinzipien wie z.B. "'DuckTyping"' und "'Principle of Least Suprice"'. Die Entwickler von Ruby stellen sich selber den Anspruch eine Programmiersprache zu schaffen, die durch Ihre Natürlichkeit einfach erlernbar ist und es den Programmierern ermöglicht, einfachen und übersichtlichen Code zu schreiben, welcher aber nicht seine Mächtigkeit und innere Komplexität verliert.
Ruby hat sich in den letzten Jahren von einer kaum beachteten Programmiersprache zu einem Publikums-Magneten entwickelt. Es gibt eine stetig wachsende offene Community "'Gemeinschaft"', welche sich und die Sprache durch Austausch von Erfahrungen und Ideen weiterbringen möchte.
Ein Grund für die hohe Bereitschaft der Community die Sprache Ruby weiter zu bringen ist der Umstand, dass die Programmiersprache vollständig OpenSource ist und unter der Lizenz der Ruby-License und GPL steht. Zudem ist die Sprache fast beliebig erweiterbar und bestehende Funktionen können einfach durch eigene Funktionen ausgetauscht werden.}}
% \input{chapter/Akronyme}
% \input{chapter/literatur}

\pagenumbering{roman}
%Glossar ausgeben
\printglossary[style=altlist,title=Glossar]
%Abkürzungen ausgeben
\deftranslation[to=German]{Acronyms}{Abkürzungsverzeichnis}
\printglossary[type=\acronymtype,style=long]
%Symbole ausgeben
\printglossary[type=symbolslist,style=long]

% Literatur
\bibliographystyle{gerplain}
\bibliography{library.bib}
% \input{Index}

%\appendix
\clearpage
\renewcommand{\appendixtocname}{Anhang}
\renewcommand{\appendixpagename}{Anhang}
\appendix
%\addappheadtotoc
\appendixpage

\end{document}