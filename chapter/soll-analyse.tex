%!TEX root=../documentation-bachlorthesis-speicherarchitektur-lstucker.tex

\cleardoublepage
\chapter{Soll-Analyse}
Ziel ist es mit der Sollanalyse, den soll Zustandstand der Speicherinfrastruktur, welche die verschiedenen Szenerien der Datenwachstum und Datenzugriffe in den nächsten drei Jahren erfüllen soll zu beschreiben. 

Der Auftraggeber Selbst geht von einen Starken Wachstum im Bereichen der Datenmengen und Datenzugriffe aus welche das Szenario 3 am ehesten wieder spiegelt.

\section{Defintion}
\subsection{Daten Redundanz}


\section{Szenario 1 - Schwaches Datenvolumen und Datenzugriff Wachstum}
Dieses Szenario beschreib den Soll-Zustand, wenn sich das Datenwachstum und Datenzugriffs Wachstum im gleichen umfang wie in der Ist-Analyse weiter entwickelt. Dieses Szenario würde eintreffen, wenn sich die Kunden Anzahlt und die gewünschten Aufbereitung der Daten für den Druck in den nächsten drei Jahre auf dem selben aktuellen Niveau befinden.

Anzahl Kunden: 10 (Annahme)

Datenwachstum in Prozent pro Monat: 5\% 

Datenwachstum in Tebibyte pro Monat: 0.19 Tebibyte

Speichervolumen in 36 Monaten: 9.34 Tebibyte

\subsection{Skalierbarkeit Datenzugriff}
Die bestehende Speicherarchitektur konnte die bisherigen Datenzugriff mit einen Web-Server zufriedenstellend erfüllen. Eine Skalierung der Zugriffe von mehreren Server soll deshalb nicht als Hauptkriterium sein sondern.

\subsection{Skalierbarkeit der Speicherkapazität}
Die Speicherkapazität soll exklusive der Datenredundanz bis mindestens 13 Tebibyte skalieren.

\subsection{Daten Redundanz}




\subsection{Verfügbarkeit}
Die Verfügbarkeit soll mindestens dem AEC-2 Standard von Harvard Research entsprechen.

\subsection{Daten Integrität}
Dem Kunden wird einen Qualität Standard gewährleistet, die Original gespeicherten Daten, sollen vor Veränderung geschützt werden, aus diesen Grund soll die Daten Integrität beim Zugriff auf die Original Daten sichergestellt sein.

\subsection{Lokalität}
Die Daten sollen in einer minimal Version an einen weiteren Standort als Backup gehalten werden.



\subsection{AHP Wer berechnung}

\begin{table}[htbp]
\tiny
\caption{}
\begin{longtable}{|l|c|c|c|c|c|c|c|c|c|c|r|r|}
\hline
\multicolumn{ 1}{|c|}{Technologie} & \multicolumn{ 5}{c|}{Evalutations Matrix} & \multicolumn{ 5}{c|}{Normalisierung} & \multicolumn{1}{l|}{} & \multicolumn{1}{l|}{Gewicht} \\ \cline{ 2- 13}
\multicolumn{ 1}{|c|}{} & M & W & E & V & K & M & W & E & V & K & \multicolumn{1}{l|}{} & \multicolumn{1}{l|}{} \\ \hline
Marktverbreitung/Marktchancen (M) & \textbf{1    } &  1/7 & \multicolumn{1}{r|}{1    } & 3 &  1/9 & 0.055 & 0.013 & 0.061 & 0.158 & 0.077 & 0.363 & 0.073 \\ \hline
Weiterentwicklung (W) & \multicolumn{1}{r|}{7    } & \textbf{1} & 5     & 4 &  1/9 & 0.382 & 0.094 & 0.303 & 0.211 & 0.077 & 1.067 & 0.213 \\ \hline
Verfügbarkeit von Experten (E) & \multicolumn{1}{r|}{1    } &  1/5 & \textbf{1} & 2 &  1/9 & 0.055 & 0.019 & 0.061 & 0.105 & 0.077 & 0.316 & 0.063 \\ \hline
Verwaltungskomfort (V) &  1/3 &  1/4 &  1/2 & \textbf{1} &  1/9 & 0.018 & 0.024 & 0.030 & 0.053 & 0.077 & 0.202 & 0.040 \\ \hline
Kinderschuhe Ausgereift (K) & \multicolumn{1}{r|}{9    } & \multicolumn{1}{r|}{9    } & \multicolumn{1}{r|}{9    } & \multicolumn{1}{r|}{9    } & \textbf{1} & 0.491 & 0.850 & 0.545 & 0.474 & 0.692 & 3.052 & 0.610 \\ \hline
Ci & 18.333 & 10.593 & 16.500 & 19.000 & 1.444 & \multicolumn{1}{r|}{1} & \multicolumn{1}{r|}{1} & \multicolumn{1}{r|}{1} & \multicolumn{1}{r|}{1} & \multicolumn{1}{r|}{1} & 5 & 1 \\ \hline
\end{longtable}
\label{}
\end{table}



\begin{center}
  \begin{tabular}{ | l | l | l | l | l | l | l | }
   \hline
    \cellcolor{gray}& \multicolumn{2}{|c|}{\cellcolor{gray}\color{white}Evaluationsmatrix} & \multicolumn{2}{|c|}{\cellcolor{gray}\color{white}Normalisierung} & \cellcolor{gray}& \cellcolor{gray}Gewicht\\  \cline{2-7}
   
    \multirow{-2}{*}{\cellcolor{gray}\color{white}Kosten}& \cellcolor{light-gray}A & \cellcolor{light-gray}L & \cellcolor{light-gray}A & \cellcolor{light-gray}L & \cellcolor{light-gray}& \cellcolor{light-gray}\\
    \hline
    \cellcolor{light-gray}Anschafftung & 2 & 3 &  &  &  & \\ \hline
    \cellcolor{light-gray}laufende Kosten & 5 & 6 &  & &  &\\ \hline  \hline
   \cellcolor{light-gray} ci & 8 & 9 & & & &\\
    \hline
    
  \end{tabular}
\end{center}

\section{Szenario 2 - Starkes Wachstum Daten / schwaches Wachstum der Abfragen}
Dieses Szenario beschreib den Soll-Zustand, wenn sich das Datenwachstum im vergleich zum Ist-Zustand stark steigert, aber der Datenzugriff auf gleichen Niveau hält wie in der Ist-Analyse hält. Dieses Szenario würde eintreffen, wenn sich die Kunden Anzahl oder die Anzahl Bilddaten pro Kund stark steigert, jedoch die Aufbereitung der Daten für den Druck gleich bleiben würde.

Datenwachstum in Prozent pro Monat: 

Datenwachstum in Tebibyte pro Monat: 6 Tebibyte

Speichervolumen in 36 Monaten: 218,5 Tebibyte

Bilder mit einer Speichervolumen von 1 Gigibyte: 221'184

\subsection{Datenzugriff}
Durch den Geringen Zugriff auf die Daten, muss auf die Daten nicht parallel von vielen Systemen zugegriffen werden können.

\subsection{Redundanz}
Die Daten sollen in doppelte oder dreifacher Redundanz gespeichert werden.

\subsection{Speicherkapazität}
Es soll 300 Tebibyte an Speicherkapazität zur Verfügung stehen.

\subsection{Verfügbarkeit}
Die Verfügbarkeit soll dem AEC-2 Standard von Harvard Research entsprechen.

\subsection{Daten Integrität}
Dem Kunden wird einen Qualität Standard gewährleistet, die Original gespeicherten Daten, sollen vor Veränderung geschützt werden, aus diesen Grund soll die Daten Integriät beim Zugriff auf die Original Daten sichergestellt sein.

\subsection{Lokalität}
Die Daten sollen in einer minimal Version an einen weiteren Standort als Backup gehalten werden.

\section{Szenario 3 - Starkes Wachstum Daten / starkes Wachstum der Abfragen}
Dieses Szenario beschreib den Soll-Zustand, wenn sich das Datenwachstum und den Datenzugriff im vergleich zum Ist-Zustand stark steigert. Dieses Szenario würde eintreffen, wenn sich die Kunden Anzahl oder die Anzahl Bilddaten pro Kund stark steigert und somit auch die Anfragen zur Aufbereitung der Daten für den Druck steigt.

Datenwachstum in Prozent pro Monat: 

Datenwachstum in Tebibyte pro Monat: 6 Tebibyte

Speichervolumen in 36 Monaten: 218,5 Tebibyte

Bilder mit einer Speichervolumen von 1 Gigibyte: 221'184

\subsection{Datenzugriff}
Auf die Daten müssen von verschiedenen Systemen gleichzeitig zugegriffen werden können.

\subsection{Redundanz}
Dem Kunden wird die Datensicherheit gewährleistet, aus diesen Grund sollen die Daten mindestens in doppelter oder dreifacher echter Redundanz gehalten werden.  Eine Redundanz durch Berechnung der Daten erfüllt diese Anforderung nicht sondern wird nur als Ergänzung für die Verfügbarkeit angesehen.

\subsection{Speicherkapazität}
Es soll 300 Tebibyte an Speicherkapazität zur Verfügung stehen.

\subsection{Verfügbarkeit}
Die Verfügbarkeit soll dem AEC-4 Standard von Harvard Research entsprechen. Bei hoher Kundenzahl und Speichervolumen, würde einen Unterbruch bei der Verfügbarkeit des Dienstes zu einen Repräsentation Schaden verursachen und Unsicherheit der Zuverlässigkeit bei den bestehenden Kunden in Bezug Ihrer Daten Verusachen.

\subsection{Daten Integrität}
Dem Kunden wird einen Qualität Standard gewährleistet, die Original gespeicherten Daten, sollen vor Veränderung geschützt werden, aus diesen Grund soll die Daten Integrität beim Zugriff auf die Original Daten sichergestellt sein.

\subsection{Lokalität}
Um eine Verfügbarkeit gemäss AEC-4 zu gewährleisten sollen die Daten Online über zwei Standorte zugreifbar sein. Eine Backup der Daten könnte an einen weiteren dritten Standort erfolgen.