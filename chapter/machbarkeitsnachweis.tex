\cleardoublepage
\chapter{Machbarkeitsnachweis}

Der selbe Hosting Provider des bestehenden Servers bietet eine Produkt an welches mit 15 SATA Festplatten mit einer Speicherkapazität von je 3 Terabyte an. Die Maximale Raid-5 Speicherkapazität \ref{eqn:maxRaid5-15disk} ist aufgrund der Anzahl Disk und der Speicherkapazität nicht die Ideale Konfiguration, da die Gefahr eines Doppelten Disk Ausfall durch die geringere MTTF \ref{eqn:MTBF15Disk}, welche durch die Zunahme der Anzahl Festplatten sinkt \ref{eqn:MTBF1Disk} und der höheren Rebuild Zeit steigt. Speichersystem Hersteller wie NetApp setze bei 
dieser Konfiguration auf Raid-6 welche doppelte Parität bietet und somit zwei Festplattenausfälle kompensieren können.

Festplattenkapazität in Tebibyte:
\begin{equation}
3   \, \mathrm{TB} =  3 * \frac{1000^4}{1024^4} = 2.7285  \, \mathrm{TiB}
\label{eqn:3TerrabyteTebibyte}
\end{equation}

Maximale Raid-5 Speicherkapazität:
\begin{equation}
(15 -1) * 2.7285  \, \mathrm{TiB} =  38.199 \, \mathrm{TiB}
\label{eqn:maxRaid5-15disk}
\end{equation}

MTBF 1 Festplatten (ST33000650SS): 
\begin{equation}1'200'000  \, \mathrm{h}
\label{eqn:MTBF1Disk}
\end{equation}

MTBF 15 Festplatten (Raid-5) (ST33000650SS): 
\begin{equation}
\frac{1'200'000  \, \mathrm{h}}{15}= 8'000  \, \mathrm{h}
\label{eqn:MTBF15Disk}
\end{equation}