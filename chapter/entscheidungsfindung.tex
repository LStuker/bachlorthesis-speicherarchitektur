%!TEX root=../documentation-bachlorthesis-speicherarchitektur-lstucker.tex

\cleardoublepage
\chapter{Entscheidungsfindung bei Evaluationen}\label{kab:Entscheidungsfindung}
\section{Grundlagen}
Evaluation ist der systematische Prozess der Datensammlung und Analyse, um eine Entscheidung treffen zu können.
\section{AHP}
Für das Evaluations- bzw. Bewertungsverfahren wurde das Analytische Hierachie-Prozess-Verfahren (AHP) angewendet. AHP wurde in den siebziger Jahren von Thomas L. Saaty zur Lösung mehrkriterieller Entscheidungsprobleme entwickelt und basiert unter anderem auf einem mathematischen Modell.
\cite{Reichardt2003}

Der Entscheidungsprozesse ist beim AHP wie der Name ausdrückt eben analytisch und hierarchisch. Die Analyse beruht auf mathematischen und logischen Entschlüssen. Auf das genaue mathematische Verfahren wird in dieser Arbeit nicht eingegangen. Sie kann in diverser Literatur nachgelesen werden. \cite{Reichardt2003}

Im AHP Verfahren werden alle Kriterien derselben Ebene in Paarvergleiche bewertet und anhand der 9-Punkte-Bewertungsskale aus der \reftab{tab:9PBewertungsskala} bzw. der umgekehrten Relation aus der \reftab{tab:UmgekehrteBewertungsskala} gewichtet.

Nach der Berechnung der Kriterien-Prioritätenbestimmung, werden die Alternativen in Paarvergleiche zu jedem Kriterium anhand derselben 9-Punkte-Bewertungsskale verglichen und bewertet. Anschliessend wird wiederum durch Berechnung der Gewinner ermittelt.

\begin{table}[htbp]
\caption{9-Punkte-Bewertungsskala \cite{Reichardt2003}}
\begin{tabular}{|c|L{3.5cm}|L{8.5cm}|}
\hline
\multicolumn{1}{|l|}{} & Definition & Interpretation \\ \hline
 1 &  gleiche Bedeutung &  Beide verglichenen Elemente haben die gleiche ""Bedeutung für das nächsthöhere Element. \\ \hline
3 &  etwas grössere "" ""Bedeutung & 
Erfahrung und Einschätzung sprechen für eine ""
etwas größere Bedeutung eines Elements im 
Vergleich zu einem anderen \\ \hline
5 &  erheblich grössere "" Bedeutung & 
Erfahrung und Einschätzung sprechen für eine "" 
erheblich größere Bedeutung eines Elements im ""
Vergleich zu einem anderen \\ \hline
7 &  sehr viel grössere ""Bedeutung & 
Die sehr viel größere Bedeutung eines Elements 
hat sich in der Vergangenheit klar gezeigt. \\ \hline
9 &  absolut dominierend &  Es handelt sich um den größtmöglichen ""
Bedeutungsunterschied zwischen zwei 
Elementen \\ \hline
\multicolumn{1}{|l|}{2,4,6,8} & Zwischenwerte &  \\ \hline
\end{tabular}
\label{tab:9PBewertungsskala}
\end{table}

\begin{table}[htbp]
\caption{Umgekehrte Relationen der Bewertungsskala \cite{Reichardt2003}}
\begin{tabular}{|c|L{3.5cm}|L{7.3cm}|}
\hline
\multicolumn{1}{|l|}{} & Definition & Interpretation \\ \hline
1 & gleiche Bedeutung & Beide verglichenen Elemente haben die gleiche 
Bedeutung für das nächsthöhere Element. \\ \hline
 1/3 & etwas geringere Bedeutung & Erfahrung und Einschätzung sprechen für eine 
etwas geringere Bedeutung eines Elements im 
Vergleich zu einem anderen.  \\ \hline
 1/5 & erheblich geringere Bedeutung & 
Erfahrung und Einschätzung sprechen für eine 
erheblich geringere Bedeutung eines Elements im 
Vergleich zu einem anderen \\ \hline
 1/7 & sehr viel geringere Bedeutung & 
Die sehr viel geringere Bedeutung eines Elements 
hat sich in der Vergangenheit klar gezeigt \\ \hline
 1/9 & absolut unterlegen & Es handelt sich um den größtmöglichen 
Bedeutungsunterschied zwischen zwei 
Elementen \\ \hline
\multicolumn{1}{|l|}{1/2, 1/4, 1/6, 1/8} & Zwischenwerte &  \\ \hline
\end{tabular}
\label{tab:UmgekehrteBewertungsskala}
\end{table}

\subsection{Software Unterstützung}
Der Berechnungsaufwand nimmt mit zunehmender Anzahl Alternativen und Kriterien zu. Aus diesem Grund ist es empfehlenswert, die Evaluation Software-unterstützt durchzuführen.

Alle Berechnungen in dieser Arbeit wurden mit der Software "'AHP Decision"' von "'True North Software"' durchgeführt.
