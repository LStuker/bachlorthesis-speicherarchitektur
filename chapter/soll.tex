%!TEX root=../documentation-bachlorthesis-speicherarchitektur-lstucker.tex
\cleardoublepage
\chapter{Soll-Analyse}
Bei der Soll-Analyse sollen die erarbeiteten Szenarien aus dem Szenarienbeschrieb berücksichtigt werden. 

\section{Szenario-1}\label{Soll-1}

\subsection{Verfügbarkeit}
Die Verfügbarkeit soll durchgehend von Speichersystem bis zum Web-Service mindestens dem AEC-2 Standard von Harvard Research entsprechen. Dies bedeutet, dass die Verfügbarkeit der Applikation, beziehungsweise der Daten, dass der Betrieb nur innerhalb festgelegter Zeitblöcke, beziehungsweise nur während der Hauptbetriebszeit minimal unterbrochen werden darf. Die bestehende Infrastruktur besteht nur aus einem einzelnen Web-Server. Dieser enthält gleichzeitig auch das Speichersystem mit seinen internen Festplatten, welche mit einem RAID-5 zusammengefasst sind. Eine defekte Komponente des Web-Servers würde das gesamte System zum Erliegen bringen. Wie in der Ist-Analyse beschrieben, erfüllt das Speichersystem für sich alleine zwar eine genügende Verfügbarkeit gemäss Definition von Harvard Research von AEC-2. Die restlichen Komponenten bzw. Softwarekomponenten des Web-Services, erfüllen diesen Anspruch allerdings nicht. Die bestehende Speicherarchitekur kann den Speicher nicht ohne Anpassung der Architektur, an weitere Web-Server zur Verfügung stellen.

Die Daten müssen mindesten in einfacher Redundanz vorhanden sein. 

Um den AEC-2 Standard zu erfüllen, ist es nicht notwendig, dass die \gls{Primearen-Daten} standortübergreifend verfügbar sind.

\subsection{Datenzugriff}
Die bestehende Speicherarchitektur konnte die bisherigen Datenzugriffe mit einem Web-Server zufriedenstellend erfüllen. Es wird davon ausgegangen, dass sich die Anzahl Datenzugriffe für die Bildaufbereitung und Speicherung von neuen Bilddaten nicht weiter steigert. Mit der erhöhten Anforderung bezüglich der Verfügbarkeit ist es notwendig, dass der Datenzugriff auf das Speichersystem von mindestens einem weiteren Web-Server erfolgen kann. 

Der \gls{POSIXIO} Zugriff soll nach Möglichkeit für die einfache Integration in die Web-Applikation unterstützt werden.

Der simultane Lesezugriff auf das gleiche Objekt muss unterstützt werden.

Der simultane Schreibvorgang auf ein Objekt muss nicht unterstützt werden.

\subsection{Speicherkapazität}
Für die Erfüllung der Speicheranforderungen des Szenario-1 muss das Speichersystem den Speicherausbau auf mindestens 16.1 Tebibyte unterstützen. In den 16.1 Tebibyte ist eine Reserve von 40\% für ein mögliches künftiges Wachstum oder als Migrationsreserve einkalkuliert. Nicht berücksichtigt ist eine gewollte Redundanz der Speicherkapazität.

Das Speichersystem soll ca. 400'000 Objekte verwalten können.

Das Speichersystem soll zudem die Speicherung von Objekten von bis zu 2 Gibibyte Grösse unterstützen.

\subsection{Datenqualität}

Die Selbstheilung von Objekten ist nicht gefordert kann aber als Option berücksichtigt werden.

Die standortübergreifende Datensicherung soll möglich sein.
Die konsistente Wiederherstellung der Daten aus einer Sicherung muss möglich sein. 

\subsection{Vergleich mit Ist-Zustand}
Im Vergleich der neuen Anforderungen mit dem Ist-Zustand ergeben sich folgende Vorteile:

\begin{itemize}
\item Durchgehende Verfügbarkeit nach AEC-2 Standard
\item Redundante Web-Server
\item Kein Single Point of Failure
\item Erfüllt die Anforderungen an die Speicherkapazität für die nächsten 3 Jahre
\item Ermöglicht mehrere gleichzeitige Lesezugriffe auf das gleiche Objekt
\end{itemize}

\section{Szenario-2}

\subsection{Verfügbarkeit}
Die Verfügbarkeit soll dem AEC-4 Standard von Harvard Research entsprechen. Das bedeutet, dass die Applikationen und Daten unterbrechungsfrei verfügbar sein müssen. ƒDer 24*7 Betrieb (24 Stunden, 7 Tage die Woche) muss gewährleistet sein. Bei hoher Kundenzahl und Speichervolumen würde ein Betriebsunterbruch der Web-Services grossen Imageschaden verursachen und die Kunden bezüglich der Zuverlässigkeit der benötigten Services verunsichern.

Die Daten müssen mindesten in einfacher, idealerweise in doppelter Redundanz vorhanden sein. 

Die \gls{Primearen-Daten} müssen mindestens an zwei Standorten gehalten werden.

\subsection{Datenzugriff}
Durch die gleichzeitig vielen Datenzugriffe muss es möglich sein, die Daten durch mehrere Web-Server zur Verfügung zu stellen.

Der \gls{POSIXIO} Zugriff soll nach Möglichkeit für die einfache Integration in die Web-Applikation unterstützt werden.

Der simultane Lesezugriff auf das gleiche Objekt muss möglich sein.

Der simultane Schreibvorgang auf ein Objekt muss nicht unterstützt werden.

\subsection{Speicherkapazität}
Für die Erfüllung der Speicheranforderungen des Szenario-2, muss das Speichersystem 306 Tebibyte unterstützen. In den 306 Tebibyte ist eine Reserve von 40\% für ein mögliches künftiges Wachstum oder als Migrationsreserve einkalkuliert. Nicht berücksichtigt ist eine gewollte Redundanz der Speicherkapazität.

Geht man von einem Datenzuwachs von 6 Tebibyte pro Monat aus, wird das Speichervolumen für alle Bilddaten nach 36 Monaten bereits 218,5 Tebibyte betragen. Berücksichtigt man zusätzliche Reserven von 40\%, so muss das Speichersystem mindestens 306 Tebibyte zur Verfügung stellen. Darin ist die Datenredundanz nicht eingerechnet.

Das Speichersystem soll 9'500'000 Objekte verwalten können.

Das Speichersystem soll zudem die Speicherung von Objekten von bis zu 2 Gibibyte Grösse unterstützen.

\subsection{Datenqualität}
Wegen der grossen Datenmenge soll die Selbstheilung von Objekten unterstützt werden. Damit verringert sich der Bedarf für die manuelle Wiederherstellung eines Objektes.

Die standortübergreifende Datensicherung soll möglich sein.
Die konsistente Wiederherstellung der Daten aus einer Sicherung muss möglich sein.

\subsection{Vergleich mit Ist-Zustand}
Im Vergleich der neuen Anforderungen mit dem Ist-Zustand ergeben sich folgende Vorteile:

\begin{itemize}
\item Durchgehende Verfügbarkeit nach AEC-7 Standard
\item Redundante Web-Server
\item Kein Single Point of Failure
\item Erfüllt die Anforderungen an die Speicherkapazität für die nächsten 3 Jahre
\item Unterstützung von Selbstheilung von korrupten Objekten
\item Ermöglicht mehrere gleichzeitige Lesezugriffe auf das gleiche Objekt
\end{itemize}