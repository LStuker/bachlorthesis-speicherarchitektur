\cleardoublepage
\chapter{Vorwort}
Die Hochschule für Technik sieht für das Fachstudium eine Semesterarbeit im zeitlichen Rahmen von 120 Stunden pro Student vor.
Aus persönlichem Interesse und weil es ein so wichtiger Bestandteil des Internets ausmacht, habe ich mich entschieden, eine webbasierte Suchapplikation mit Apache Solr\footnote{\url{http://lucene.apache.org/solr/}} und Nutch\footnote{\url{http://nutch.apache.org/}} zu erstellen.

\section{Die Bedeutung der Suche}
Die Möglichkeit im \ac{WWW} nach Informationen zu suchen ist heute essentiell und nicht mehr wegzudenken. Die Wichtigkeit der Suche bestätigt den Erfolg von Suchmaschinen wie Google, Yahoo und Bing und unsere Abhängigkeit von ihnen. Ohne eine Suchmöglichkeit wären die Unmengen an Informationen im Internet nur schwer zugänglich. Diese Problematik stellte Vannevar Bush in seinem viel zitierten Paper “As We May Think” bereits im Juli 1945 fest. Er beschreibt die Problematik beim Sammeln und Wiederfinden von Informationen in der Forschung: Die Mengen an Daten wachsen kontinuierlich, können aber nicht schnell genug verbreitet oder effizient wiedergefunden werden. Es wurden so viele Informationen durch Forschungen und Entdeckungen gesammelt, dass diese nicht schnell genug eingeordnet, verbreitet und bekanntgemacht werden konnten.

Für diese Informationsüberflutung (engl. Information Overload) suchte Vannevar Bush nach Lösungen. Die Suche, wie wir sie heute kennen, hat Bush nicht erläutert, aber er hat den Grundstein dafür gelegt.\cite{VannevarBush19450701}

Dem Thema des Wiederfindens von Informationen widmet sich das Fachgebiet \textit{Information Retrieval}.
Heute wissen wir, ohne eine Suchfunktion können wir uns im WWW nicht zurechtfinden. So mancher potentieller Kunde wäre in einem Onlineshop wie beispielsweise Amazon oder einem Auktionshaus wie Ricardo oder eBay ohne eine Suchfunktion verloren.

\begin{quotation}
\em Search is an integral part of peoples' online lives; people turn to search engines for help with a wide range of needs and desires, from satisfying idle curiousity to finding life-saving health remedies, from learning about medieval art history to finding video game solutions and pop music lyrics. \textbf{Web search engines are now the second most frequently used online computer application}, after email. Not long ago, most software applications did not contain a search module. Today, search is fully integrated into operating systems and is viewed as an essential part of most information systems.\end{quotation}\cite{SUI2009}

\section{Die Suche als Navigation}
Laut Miller's Gesetz \index{Miller\'s Law}\footnote{\url{http://psychclassics.yorku.ca/Miller/}} kann der Mensch sich etwa sieben Begriffe merken. Ein Navigation sollte daher sieben und höchstens neun Links beinhalten, die sich der Besucher merken kann. Neun Links sind sehr wenig für informationsreiche Webseiten. Es ist unmöglich jeden Bericht einer Nachrichten-Website wie beispielsweise \url{www.nzz.ch} und \url{www.tagesanzeiger.ch} direkt zu verlinken. Innerhalb von wenigen Tagen hätte die Navigation so viele Links, wie die Bibel Wörter und wäre nicht verwendbar. Eine Suchfunktion ist daher unumgänglich, um auf grossen Mengen an Informationen zugreiffen zu können.

\section{Fazit}
Um die Daten auf der Festplatte zu löschen, werden nur die Pointer entfernt. Obwohl die eigentlichen Daten noch physisch vorhanden sind, gelten sie als gelöscht. Eine Informationsreiche Website ohne Suchfunktion ist genauso "leer" wie eine Festplatte ohne Pointer.

Die Suche im Web ist essentiel und bei der stetig wachsenden Menge an Information, absolut notwendig. Aus diesem Grund lohnt sich die Investition in die Verbesserung der Suchfunktionalität im Web und vorallem auf der eigenen Website.

