%!TEX root=../documentation-bachlorthesis-speicherarchitektur-lstucker.tex

\cleardoublepage
\chapter{Vorwort}
"The Expanding Digital Universe" so bezeichnet das internationale US-amerikanische Marktforschung und Beratungsunternehmen 'International Data Corporation' (IDC) das starke Wachstum an generierten und gespeicherten Daten in einem Jahr. IDS hat seit Beginn ihrer Studie in 2007 festgestellt, das sich der Speicherbedarf alle zwei Jahre verdoppelt hat. Bleibt die Zunahme der Daten während den nächsten 10 Jahre konstant, haben wir so viele digitale Bits erstellt wie es Sterne im Universum gibt. Der grössere Teil der Daten werden unstrukturierte Daten sein (im Gegensatz zu verknüpften Daten eines Datenbanksystems), was das ganze so komplex erscheinen lässt wie das Universum. \cite{Gantz2011}

\section{Was bedeutet diese starke Wachstum für uns?}
Neben der zunehmenden Digitalisierung, hat sich auch das Volumen des Speicherplatzes der eingesetzten Speichermedien weiter entwickelt. Auf einem gegebenen Speichermedium können pro cm2 immer mehr Daten gespeichert werden. Die Bedürfnisse der Anwender konnten bisher nicht zufriedenstellend erfüllt werden. Die Entwicklungsabteilungen der Speichermedium Hersteller hinken deshalb den Marktanforderungen seit jeher hinterher. Heerscharen von Ingenieuren beschäftigen sich seit Beginn des digitalen Zeitalters deshalb mit der Entwicklung immer effizienteren Technologien, um die Flut der Daten speichern zu können und für viele Benutzer abrufbar und bearbeitbar zu halten. Neben hardwaretechnischen Fortschritten des Speichermediums, um die physischen Speichergrenzen mit höherer Datendichte zu überwinden, entwickelt die Industrie zum Beispiel immer effizientere Algorithmen, um mit Logik und Software die Daten ohne Verlust der wesentlichen Informationen höher komprimieren zu können, d.h. das zu speichernde Datenvolumen substanziell zu verkleinern. Die bereits realisierten Lösungen zeigen, dass die Grenzen noch nicht erreicht sind.

\section{Konsequenzen für den Service Provider}
Das Zürcher Startup Unternehmen "Reference Image AG" betreibt und entwickelt eine Webapplikation zur Speicherung, Archivierung, Verwaltung und Bearbeitung von qualitativen hochauflösenden digitalen Bildern. Zu Ihrem Kundensegment gehören Galerien, Museen, Künstler und Fotographen die sehr hohe Ansprüche an die Qualität Ihrer Bilder haben. Die Reference Image AG erlaubt die Speicherung von Bildgrössen bis zu 2Gbyte. Das heutige Speichersystem ist bereits zu über 50\% ausgelastet, so dass sich ein Überdenken der zukünftigen Speicherarchitektur aufdrängt. Tests basierend auf der NFS Speicherarchitektur haben ungenügende Performance ergeben. Das neue Speichersystem soll die Anforderungen an die folgenden Leistungskriterien erfüllen, welche im Soll-Konzept ausführlicher beschrieben sind: * Skalierbarkeit der Architektur (Anzahl Datenabfragen und Erweiterung der Speicherkapazität, bzw. des Datenvolumens) * Datendurchsatz / Performance * Datenverfügbarkeit (365 x 24) * Datenintegrität (Gewährleistung der Fälschungssicherheit der Daten, Bilder) * Datenqualität (die Originalauflösung der Bilddaten wird beibehalten, bzw. nicht komprimiert) * Wirtschaftlichkeit (attraktives Preis-/Leistungsverhältnis). Der Auftraggeber hat bezüglich der Skalierbarkeit festgelegt, dass die Anwendung verteilt über mehrere Standorte/Systeme betrieben werden soll. Da die Nutzung des Portals mit laufend wachsenden Besucherzahlen kontinuierlich steigt, wird die hohe Verfügbar der Daten immer wichtiger und entscheidend für den Erfolg der Unternehmung. Die Software und Hardware Architektur soll diesem Umstand Rechnung tragen. Die heutigen Infrastruktur-Betriebskosten sind tief. Die Investitionen in das neue System sollen aus den erwarteten Gebühreneinnahmen der nächsten 5 Jahren getragen werden können. Eine rasche Amortisation der Investitionen gibt dem Unternehmen den nötigen Freiraum, den künftigen Bedürfnissen der Kunden zeitnah und angemessen zu entsprechen. 

\section{Fazit}
Um unsere Zukünftigen unvorstellbaren bedarf an Speicherplatz decken zu können werden wir auch zukünftig, neue Lösungen für die Speicherung unserer Daten finden müssen.

