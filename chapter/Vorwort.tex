%!TEX root=../documentation-bachlorthesis-speicherarchitektur-lstucker.tex

\cleardoublepage
\chapter{Vorwort}
"The Expanding Digital Universe" so bezeichnet das Internationale US-amerikanscihe Marktforschung und Beratungsunternehmen International Data Corporation (\ac{IDC}) das starke Wachstum an generierten und gespeicherten Daten in einen Jahr. IDS hat seit beginn 2007 Ihrer Studie festgestellt, das sich der Speicherbedarf alle zwei Jahre verdoppelt hat. Bleibt die Zunahme der Daten die nächsten 10 Jahre konstant, haben wir so viele digitale Bits erstellt wie es Sterne im Universum gibt. Die grösste Teil an erstellten Daten sind unstrukturierte Daten, was das ganze gleich komplex macht wie das Universum.\cite{IDC}

\section{Was bedeutet diese starkte Wachstum für uns?}
Neben der zunehmenden Digitalisierung, hat sich auch das Volumen des Speicherplatzes der eingesetzten Speichermedien weiter entwickelt, die Bedürfnisse konnte aber bis anhin nie gedeckt werden. Weshalb wir uns Seit des digitalen Zeitalters damit beschäftigen wie wir unsere generierte digitalen Daten am besten Speichern. Um die Daten effizient Speichern und austauschen zu können haben, wir zum Beispiel Algorithmen entwickelt welche unsere Daten ohne Verlust der wesentlichen Informationen komprimieren, oder wir haben Verfahren entwickelt um die Speichergrenze eines einzelnen Medium zu überwinden.

\section{Konequenzen für Service Provider}
Das Zürcher Startup Unternehmen "Reference Image AG" betreibt und Entwickelt eine Webapplikation zur Speicherung, Archivierung, Verwalten und Wiederverwenden von qualitativen hochauflösenden digitalen Bildern. Zu Ihrem Kundensegment gehören Galerien, Museen, Künstler und Fotographen die sehr hohe Ansprüche an die Qualität Ihrer Bilder haben. Die Reference Image AG erlaubt die Speicherung von Bildgrössen bis zu 2Gbyte. Das heutige Speichersystem ist bereits zu über 50% ausgelastet, so dass sich ein Überdenken der zukünftigen Speicherarchitektur aufdrängt. Tests basierend auf NFS haben ungenügende Performance ergeben. Das neue Speichersystem soll die Anforderungen an die folgenden Leistungskriterien erfüllen, welche im Soll-Konzept ausführlicher beschrieben sind: * Skalierbarkeit der Architektur (Anzahl Datenabfragen und Erweiterung der Speicherkapazität, bzw. Datenvolumen) * Datendurchsatz / Performance * Datenverfügbarkeit (365 x 24) * Datenintegrität (Gewährleistung der Fälschungssicherheit der Daten, Bilder) * Datenqualität (die Originalauflösung der Bilddaten wird beibehalten, bzw. nicht komprimiert) * Wirtschaftlichkeit (attraktives Preis-/Leistungsverhältnis) Der Auftraggeber hat bezüglich der Skalierbarkeit festgelegt, dass die Anwendung verteilt betrieben werden soll. Da die Nutzung des Portals mit ständig wachsenden Besucherzahlen kontinuierlich steigt, wird die hohe Verfügbar der Daten immer wichtiger und entscheidend für den Erfolg der Unternehmung. Die Software und Hardware Architektur soll diesem Umstand Rechnung tragen. Die heutigen Infrastruktur-Betriebskosten sind tief. Die Investitionen in das neue System sollen aus den erwarteten Gebühreneinnahmen der nächsten 5 Jahren getragen werden können. Eine rasche Amortisation der Investitionen gibt dem Unternehmen den nötigen Freiraum, den Bedürfnissen der Kunden zeitnah und angemessen zu entsprechen. 

\section{Fazit}
Um unsere Zukünftigen unvorstellbaren bedarf an Speicherplatz decken zu können werden wir auch zukünftig ....

