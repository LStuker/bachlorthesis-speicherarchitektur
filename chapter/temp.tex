
\section{DAS}
\subsubsection{Datenverfügbarkeit / Redundanz}
Die Daten Redundanz kann mit RAID oder durch den Einsatz eines Logical Volume Manager gewährleistet werden. Die Daten können mit einen DAS System jedoch nur an einen Standort verfügbar gemacht werden.

\subsubsection{Skalierbarkeit Datenvolumen}
Einen Server können mehre Logical Unit zugeteilt werden. Durch den Einsatz eines Volume Manager können mehre Logical Unit zu einer grossen Logischen Volume zusammengefasst werden. 
Reicht die Kapazität eines DAS Speichersystems nicht aus können weiter DAS Systeme angeschlossen werden sofern der Server genügend freie Schnittstellen hat.

\subsubsection{Skalierbarkeit der Datenzugriffe}
Die Datenzugriffe durch mehre Server wird beim DAS durch die Anzahl verfügbaren Schnittstellen begrenzt.
\subsubsection{Integität}
\subsubsection{Durchsatz I/O}

\subsubsection{Lokalität}
Ein DAS kann seinen Speicher nur an Server, welche am selben Standort wie das DAS System befinden zur Verfügung stellen. In der Regel befinden sich die DAS Systeme in unmittelbare nähe zum Server.

\subsubsection{Backup}
Die Daten können mit Herkömmlichen Backup Verfahren gesichert werden.

\section{ISCSI}
\paragraph*{Standortübergreifend}
iSCSI-SANs können Standort übergreifend implementiert werden.
