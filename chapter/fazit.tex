%!TEX root=../documentation-bachlorthesis-speicherarchitektur-lstucker.tex
\cleardoublepage
\chapter{Fazit}

Für den Auftraggeber stehen unterschiedliche Speichertechnologien zur Verfügung. Doch welche der Speichertechnologien ist nun die Richtige Lösung für den Auftraggeber?

Die Evaluation hat gezeigt, dass der Online Speicher Amazon S3 von allen Gewählten Alternativen die beste Lösung ist. Für Amazon S3 sprach vor allen, die niedrigsten Gesamt kosten. Die Tiefe Kosten von Amazon S3 sind vor allem durch die fehlende Lohnkosten auf Seiten Arbeitgeber für den Betrieb der Speicherarchitektur zurück zu führen. Für Amazon S3 spricht aber auch, dass der Speicher nach Anmeldung sofort zur Verfügung steht und beliebig in der Speicherkapazität skaliert.

Ein Nachteil beim Einsatz von Amazon S3 könnte für den Auftraggeber die Bildaufbereitung sein. Die Daten müssten dazu über die Internetverbindung der Applikations-Server vom Speicher heruntergeladen werden um diese zu verarbeiten. Dadurch wird die Internet Verbindung der Applikations-Server zusätzlich belastet, die Transfer Geschwindigkeit währe tiefer als bei einer Lösung welche den Speicher über eine Lokales Netzwerk zur Verfügung stellt und es entstehen zusätzliche Kosten für den Datentransfer. Dieser Nachteil könnte aber allenfalls durch Amazon EC2 etwas Entschärft werden. Amazon EC2 bietet Online Rechenkapazität an für den Betrieb von Virtuelle Server. Der Transfer von Amazon S3 zu Amazon EC2 verursacht keine Kosten.

Neben Amazon S3 gibt es noch weitere Online Speicheranbieter, welche in dieser Evaluation nicht berücksichtigt wurden, aber für eine Definitive Entscheidung für eine Speicherlösungen noch in die Entscheidungsfindung mit einbezogen werden sollen.



\section{Erkenntnisse aus der Arbeit}

\subsection{Speichertechnologie}
Die Arbeit hat es mir ermöglicht für eine Reales Existierendes Projekt die Verschiedenen Speichertechnologien zu Vergleichen und mit ausgewählten Vertreter jeder Technologie eine Evaluation durchzuführen. Diese ermöglicht es mir Speichertechnologien kennen zu lernen mit welchen ich in der Vergangenheit keine Berührungspunkte hatte und mir somit zu diesen und den bereits bekannten Technologien neues Wissen anzueignen.

Bis anhin wurden der Speicherbedarf meist mit Blockbasierende oder Dateibasierende Speicherlösungen gedeckt. Diese Speicherlösungen stellen den Speicher meisten von einen Zentralen System zur Verfügung und adressieren die Daten als Block oder als Datei.
Die zunehmende Vernetzung, der Steigende Speicherbedarf und der bedarf aus den Gespeicherten Informationen mehr wertvolle Informationen zu gewinnen, stellen neue  Anforderungen an Speichersysteme, hinsichtlich Skalierung in der Speicherkapazität, der Performance, der Verwaltung und der Verteilung der Daten, welche mit den bisherigen Lösungen nicht immer befriedigend gelöst werden können. Wie die neue Anforderungen gelöst werden kann, haben Google und Amazon uns vorgemacht. Google ermöglichte es mit Ihrem System aus einer Informationsfluss ein brauchbares Suchergebnis zu liefern. Amazon ermöglicht es mit seinem Speichersystem eine enorme Speicherkapazität Ihren Kunden über das Internet zur Verfügung zu stellen. 

Von den Lösungen dieser Firmen inspiriert sind in den Vergangen Jahren Quelloffene Lösungen und dazu gehörige Community entstanden, wie Hadoop, Gluster, OpenStack um die bekanntesten davon zu nennen, die auch anderen Organisationen ermöglichen auf solche Techniken zuzugreifen um Ihre Speicheranforderungen zu decken. Das solche Lösungen Entwicklungspotential hat, deutet meiner Meinung nach, dass zunehmende Mediale Interesse an diesen Softwareprojekte und die Teilnahme der grössen Speicherhersteller an diesen, an. 

\subsection*{Evaluation}
Bevor die eigentliche Arbeit bereits begann, hatte ich insgeheim bereits einen persönlichen Favoriten bei den Speichertechnologie. Bei einer Evaluation ist es aber wichtig, dass die automatische Vorverurteilung die bei uns Menschen wahrscheinlich unvermeidbar ist, keinen Einfluss auf die Evaluation nimmt. In Nachhinein bin ich überzeugt, dass es aber dafür geeignete Verfahren notwendig sind, die diese so gut wie möglichst verhindert. Wenn ich die Evaluation mit einfachen auflisten der Vor und Nachteile jeder Technologie gemacht hätte, wäre eine Verhinderung schwer gewesen. Das AHP Verfahren mit seinen Hierarchischen Prozesse, zwang mich die Evaluationen in kleinen Schritten zu untersuchen, los gelöst von eigentlichen Endergebnis. Diese Hilft einem eine Neutralere Position einzunehmen und die Subjektivität zu vermeiden, da man nur einen spezifischen Vergleich zwischen Zwei Alternativen oder Kriterien durchführt. Am Ende ist der Subjektive Einfluss viel gerieren.

Das Verfahren ist meiner Einschätzung nach aber auch relative Aufwendig, besonderste mit zunehmenden Anzahl an Alternativen und Kriterien. Ich hatte den Aufwand, welche die beiden Szenerien und die Anzahl Alternativen bzw. Kriterien verursachen, stark unterschätzt. Die Lösung wahre einerseits gewesen Kriterien und Alternativen weg zu lassen oder mehr Zeit Aufwenden. Die aussage Kräftigkeit oder Genauigkeit der Evaluation ist aber auch schlussendlich von den gewählten Kriterien abhängig, weshalb ich froh bin dass ich die notwendige zusätzliche Zeit von Seiten der Schule bewilligt wurde.

Mit der Erfahrung aus der Arbeit, würde ich Empfehlen eine Evaluation nicht in Einzelarbeit durchzuführen sondern zu zweit oder allenfalls auch in einen kleinen Team. Grund dafür sehe ich, dass bei der Beurteilung der Diskussionspartner und dessen Beurteilung fehlt.

\section{Offene Fragen}

Fragen, die in dieser Arbeit nicht geklärt werden konnten.

\textbf{Wie hoch der Effektive Arbeitsaufwand ist um eine eigne OpenStack Object Storage Installation zu Betreiben?}

\textbf{Gibt es bereits oder wird es zukünftig eine echte Schweizer alternative zu Amazon S3 geben?}

\textbf{Werden die grossen Speicherhersteller, eigene Produkte für Verteilte Speicher anbieten?}

\textbf{Wird sich ein Echter Standard für den Zugriff auf Verteiler Speicher entwickeln?}

\textbf{Wie hoch wären die Kosten für die Bearbeitung der Bilder auf Amazon EC2?}


\section{Schlusswort}
Ich möchte mich hiermit an alle Beteiligen Personen ganz Herzlich bedanke, die mit Ihrer Engagement und Unterstützung mir es ermöglicht Haben diese Arbeit und meine Studium durch zufügen. 

Die Arbeit hat in mir, die Neugier in Verteile Systeme und Speicherlösungen geweckt. Ich finde es persönlich höchst Spannend, und würde mich gerne auch zukünftig weiter mit solchen Systemen beschäftigen. Ich gehe davon aus, dass der Bedarf an Verteile Speicherlösungen in Unternehmen zukünftig steigen wird. Meinen Eindruck ist aber, dass in der Schweiz der Bedarf noch nicht allzu gross ist, obwohl auch hier sicher grosses Potential vorhanden wäre. IT-Firmen und IT-Experten welche sich jetzt schon wissen aneignen. Haben wahrscheinliche mit den Wissensvorsprung die Chance sich in einen Wachsenden Markt Positionierungen zu können. 

Unsere Zukunft bleibt jeden falls höchstspannend. Wir werden in in dieser sicher noch viele weitere Bit an Informationen generieren um neben den Sternen in unser digitales Universum auch mit Planeten zu füllen. Es werden sich dann sicher häufen neue Spannende Fragen stellen, die es zu Beantworten gilt.