%!TEX root=../documentation-bachlorthesis-speicherarchitektur-lstucker.tex
\cleardoublepage
\chapter{Fazit}
\section{Erkenntnisse}

\subsection{Speichertechnologie}
Die Arbeit hat es mir ermöglicht für eine Reales Existierendes Projekt die Verschiedenen Speichertechnologien zu Vergleichen und mit ausgewählten Vertreter jeder Technologie eine Evaluation durchzuführen. Diese ermöglicht es mir Speichertechnologien kennen zu lernen mit welchen ich in der Vergangenheit keine Berührungspunkte hatte und mir somit zu diesen und den bereits bekannten Technologien neues Wissen anzueignen.

Bis anhin wurden der Speicherbedarf meist mit Blockbasierende oder Dateibasierende Speicherlösungen gedeckt. Diese Speicherlösungen stellen den Speicher meisten von einen Zentralen System zur Verfügung und adressieren die Daten als Block oder als Datei.
Die zunehmende Vernetzung, der Steigende Speicherbedarf und der bedarf aus den Gespeicherten Informationen mehr wertvolle Informationen zu gewinnen, stellen neue  Anforderungen an Speichersysteme, hinsichtlich Skalierung in der Speicherkapazität, der Performance, der Verwaltung und der Verteilung der Daten, welche mit den bisherigen Lösungen nicht immer befriedigend gelöst werden können. Wie die neue Anforderungen gelöst werden kann, haben Google und Amazon uns vorgemacht. Google ermöglichte es mit Ihrem System aus einer Informationsfluss ein brauchbares Suchergebnis zu liefern. Amazon ermöglicht es mit seinem Speichersystem eine enorme Speicherkapazität Ihren Kunden über das Internet zur Verfügung zu stellen. 

Von den Lösungen dieser Firmen inspiriert sind in den Vergangen Jahren Quelloffene Lösungen und dazu gehörige Community entstanden, wie Hadoop, Gluster, OpenStack um die bekanntesten davon zu nennen, die auch anderen Organisationen ermöglichen auf solche Techniken zuzugreifen um Ihre Speicheranforderungen zu decken. Das solche Lösungen Entwicklungspotential hat, deutet meiner Meinung nach, dass zunehmende Mediale Interesse an diesen Softwareprojekte und die Teilnahme der grössen Speicherhersteller an diesen, an. 

\subsection*{Evaluation}
Bevor die eigentliche Arbeit bereits begann, hatte ich insgeheim bereits einen persönlichen Favoriten bei den Speichertechnologie. Bei einer Evaluation ist es aber wichtig, dass die automatische Vorverurteilung die bei uns Menschen wahrscheinlich unvermeidbar ist, keinen Einfluss auf die Evaluation nimmt. In Nachhinein bin ich überzeugt, dass es aber dafür geeignete Verfahren notwendig sind, die diese so gut wie möglichst verhindert. Wenn ich die Evaluation mit einfachen auflisten der Vor und Nachteile jeder Technologie gemacht hätte, wäre eine Verhinderung schwer gewesen. Das AHP Verfahren mit seinen Hierarchischen Prozesse, zwang mich die Evaluationen in kleinen Schritten zu untersuchen, los gelöst von eigentlichen Endergebnis. Diese Hilft einem eine Neutralere Position einzunehmen und die Subjektivität zu vermeiden, da man nur einen spezifischen Vergleich zwischen Zwei Alternativen oder Kriterien durchführt. Am Ende ist der Subjektive Einfluss viel gerieren.

Das Verfahren ist meiner Einschätzung nach aber auch relative Aufwendig, besonderste mit zunehmenden Anzahl an Alternativen und Kriterien. Ich hatte den Aufwand, welche die beiden Szenerien und die Anzahl Alternativen bzw. Kriterien verursachen, stark unterschätzt. Die Lösung wahre einerseits gewesen Kriterien und Alternativen weg zu lassen oder mehr Zeit Aufwenden. Die aussage Kräftigkeit oder Genauigkeit der Evaluation ist aber auch schlussendlich von den gewählten Kriterien abhängig, weshalb ich froh bin dass ich die notwendige zusätzliche Zeit von Seiten der Schule bewilligt wurde.

Mit der Erfahrung aus der Arbeit, würde ich Empfehlen eine Evaluation nicht in Einzelarbeit durchzuführen sondern zu zweit oder allenfalls auch in einen kleinen Team. Grund dafür sehe ich, dass bei der Beurteilung der Diskussionspartner und dessen Beurteilung fehlt.

\section{Schlusswort}
Ich möchte mich hiermit an alle Beteiligen Personen ganz Herzlich bedanke, die mit Ihrer Engament und Unterstützung mir es ermöglicht Haben diese Arbeit und meine Studium durch zufügen. 

Die Arbeit hat in mir, die Neugier in Verteile Systeme und Speicherlösungen geweckt. Ich finde es persönlich höchst Spannend, und würde mich gerne auch zukünftig weiter mit solchen Systemen beschäftigen. Ich gehe davon aus, dass der Bedarf an Verteile Speicherlösungen in Unternehmen zukünftig steigen wird. Meinen Eindruck ist aber, dass in der Schweiz der Bedarf noch nicht allzu gross ist, obwohl auch hier sicher grosses Potential vorhanden wäre. IT-Firmen und IT-Experten welche sich jetzt schon wissen aneignen. Haben wahrscheinliche mit den Wissensvorsprung die Chance sich in einen Wachsenden Markt Positionierungen zu können. 

Unsere Zukunft bleibt jeden falls höchstspannend. Wir werden in in dieser sicher noch viele weitere Bit an Informationen generieren um neben den Sternen in unser digitales Universum auch mit Planeten zu füllen. Es werden sich dann sicher häufen neue Spannende Fragen stellen, die es zu Beantworten gilt.