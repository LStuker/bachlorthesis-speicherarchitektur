%!TEX root=../documentation-bachlorthesis-speicherarchitektur-lstucker.tex
\cleardoublepage
\chapter{Fazit}

Für den Auftraggeber stehen unterschiedliche Speichertechnologien zur Auswahl. Doch welche der Speichertechnologien ist nun die richtige Lösung für die definierten Anforderungen, welche dem Auftraggeber eine erfolgreiche Geschäftstätigkeit und dem Kunden eine optimale Plattform für deren Bilddaten für die nächsten 5 Jahre sichert und Investition und laufende Kosten rechtfertigt bzw. eine marktgerechte Leistung bietet?

Die Evaluation hat gezeigt, dass der Online Speicher Amazon S3 von den gewählten Optionen die beste Lösung ist. Für Amazon S3 sprach vor allem die niedrigen Gesamtkosten (TCO). Dieser Kostenlevel wird vor allem durch die eingesparten Lohnkosten für den Betrieb der Speicherarchitektur auf Seiten des Auftraggebers zurückzuführen. Für Amazon S3 spricht ferner die sofortige Verfügbarkeit der Speicherkapazität nach Anmeldung und dass diese beliebig nach Kundenbedürfnis skaliert, ohne teure Upgrades und Datenmigrationen in neue Systeme.

Ein Nachteil in den Funktionen von Amazon S3 könnte für den Auftraggeber die Bildaufbereitung sein. Die Daten müssten dazu über die Internetverbindung zum Applikations-Server heruntergeladen werden, und nach Bearbeitung wieder zum Speichersystem zurücktransferiert werden. Dadurch würde die Internetverbindung zum Applikations-Server zusätzlich belastet und Wartezeiten bei grossen Bilddateien könnten für den Kunden unangenehm werden. Idealerweise sollte der Applikations-Server über das lokale Netzwerk mit dem Speichersystem verbunden sein. Dieser Nachteil könnte allenfalls durch Amazon EC2 entschärft werden. Amazon EC2 bietet neu auch Online-Rechenkapazität für den Betrieb von virtuellen Servern an. Der Transfer von Amazon S3 zu Amazon EC2 verursacht keine Kosten und könnte die benötigte Performance für die Bildbearbeitung bieten.

Neben Amazon S3 bieten weitere Online Speicheranbieter ihre Dienste in der Cloud an, welche in dieser Evaluation nicht berücksichtigt wurden. Für eine definitive Entscheidung sollten solche Speicherlösungen in die Entscheidungsfindung einbezogen werden.



\section{Erkenntnisse aus der Arbeit}

\subsection{Speichertechnologie}
Die Arbeit hat es mir ermöglicht, für ein reales Projekt die verschiedenen Speichertechnologien in der Tiefe zu vergleichen und ausgewählte Vertreter jeder Technologie in die Evaluation einzubeziehen und mir neues, interessantes Wissen anzueignen. Die Einarbeitung in die neuen Speichertechnology-Themen war spannend und lehrreich und hat meinen beruflichen Horizont mit Sicherheit erweitert.

Bis anhin wurde der Speicherbedarf meistens mit block- oder dateibasierenden Speicherlösungen gedeckt. Diese Speicherlösungen stellen den Speicher meisten aus einem zentralen System zur Verfügung und adressieren die Daten als Block oder als Datei.
Die zunehmende Vernetzung, der enorm steigende Speicherbedarf sowie die Bedürfnisse aus den gespeicherten Daten mehr wertvolle Informationen zu gewinnen, stellen neue Herausforderungen an die Speichersysteme hinsichtlich Skalierung der Speicherkapazität, der Performance, der Verwaltung und der Verteilung der Daten. Die bisherigen konventionellen Lösungen stossen hier immer mehr an ihre Leistungsgrenzen. Wie man den neuen Anforderungen gerecht werden kann, haben Google und Amazon uns vorgemacht. Google ermöglichte es mit ihrem System aus einer täglich steigenden Flut an Daten in Rekordzeit ein brauchbares Suchergebnis zu liefern. Amazon ermöglicht es mit seinem Speichersystem eine enorme Speicherkapazität ihren Kunden bedarfsgerecht über das Internet zur Verfügung zu stellen. 

Von diesen Lösungen inspiriert, sind in den vergangenen Jahren neue quelloffene Lösungen und dazugehörende Communities entstanden, wie Hadoop, Gluster, OpenStack, um die bekanntesten davon zu nennen. Diese Entwicklung erlaubt nun auch anderen Organisationen diese neuen Technologien zu nutzen und in ihre eigenen Lösungen zu integrieren. Dass dieses Entwicklungspotential zu neuen Lösungen und Servicen führen wird, ist offensichtlich. Das zunehmende mediale Interesse an diesen Projekten und die Teilnahme der grossen Speicherhersteller wird meiner Meinung nach die Entwicklung weiter fördern, die schon heute in manchen Diskussion in den Chefetagen zu berücksichtigen wären. 

\subsection*{Evaluation}
Bevor die eigentliche Arbeit begann, hatte ich basierend auf meinem Wissen insgeheim bereits einen persönlichen Favoriten bei den Speichertechnologie ins Auge gefasst - das klassische Vorurteil auf einer schmalen Informationsbasis. Die Evaluationsarbeit hat dann aber gezeigt, dass die methodische, sachliche Wissensaufbereitung und -Bewertung zu einem ganz anderen Resultat führen kann und Vorurteile abbaut. Im Nachhinein bin ich überzeugt, dass es für die objektive, vorurteilsfreie Entscheidungsfindung notwendig ist, geeignete Verfahren einzusetzen. Wenn ich die Evaluation mit der einfachen Auflistung von den Vor- und Nachteilen jeder Technologie gemacht hätte, wäre dies wahrscheinlich nicht gelungen. Das AHP Verfahren mit seinen hierarchischen Gliederung und Prozessen zwang mich, die Evaluationen in kleinen Schritten zu untersuchen, losgelöst vom eigentlich anvisierten Ergebnis. Das tatsächliche Ergebnis ist nun zu meiner Freude objektiv und frei von subjektiv gefärbten Bewertungen entstanden, hinter das ich mich jederzeit stellen kann, weil ich auf allfällige Fragen auf den gut dokumentierten Prozess und die Teilergebnisse jederzeit zurückgreifen kann. 

Meiner Einschätzung nach ist das gewählte AHP-Verfahren mit zunehmender Anzahl von Alternativen und Kriterien relativ aufwendig. Ich hatte den Aufwand für meine Arbeit für die beiden gewählten Szenarien klar unterschätzt. Es wäre angebracht gewesen, für die Evaluation zwar einen möglichst vollständigen Katalog von Alternativen und Kriterien zu definieren, dann aber unter Berücksichtigung der vorgegeben Zeit, nur die klar relevanten Kriterien für den Vergleich zu bestimmen und auszuwerten. Diese Wahl beeinflusst natürlich bereits das Endergebnis und so bin ich doch froh mit den gewählten Kriterien ein vielleicht genaueres, umfassendes Ergebnis erarbeitet zu haben. Die dafür notwendige zusätzliche Zeit wurde zum Glück von der Schule bewilligt. Dies wird im Berufsleben nicht immer so sein.

Aus der Erfahrung dieser Arbeit würde ich empfehlen, eine solch wichtige Evaluation nicht im Alleingang durchzuführen, sondern zu zweit oder im kleinen Team die Arbeiten aufzuteilen und sich auszutauschen. Grund hierfür sehe ich im Vorteil, dass in den Evalutionsschritten und bei der Beurteilung ein Diskussionspartner eine echte Hilfe für die Qualität der Arbeit sein kann.

\section{Offene Fragen}

Fragen, die in dieser Arbeit nicht geklärt werden konnten.

\textbf{Wie hoch ist der effektive Arbeitsaufwand, um eine eigene OpenStack Object Storage Umgebung aufzubauen und zu betreiben?}

\textbf{Gibt es bereits oder wird es künftig eine echte schweizerische Alternative zu Amazon S3 geben?}

\textbf{Werden die grossen Speicherhersteller eigene Produkte für verteilte Speichersysteme anbieten?}

\textbf{Wird sich ein breit abgestützter und verlässlicher Standard für den Zugriff auf verteilte Speicher entwickeln?}

\textbf{Wie hoch wären die Kosten für die Bearbeitung der Bilder auf Amazon EC2?}


\section{Schlusswort}
Ich möchte mich hiermit bei allen beteiligten Personen ganz herzlich bedanken, die mit ihrem Engagement und Unterstützung es mir ermöglichten, diese Arbeit und meine Studium durchzuführen und zum Abschluss zu bringen. 

Die Arbeit hat in mir die Neugier geweckt, mich vermehrt in die Welt der verteilten Speichersysteme und Speicherlösungen zu vertiefen. Ich finde es persönlich höchst spannend und würde mich gerne auch zukünftig mit solchen Systemen beschäftigen. Ich bin überzeugt, dass der Bedarf an verteilten Speicherlösungen in Unternehmen steigen wird. Mein Eindruck ist aber, dass in der Schweiz der Bedarf für solche Lösungen noch in den Kinderschuhen ist, obwohl auch hier sicher ein grosses Potential vorhanden wäre. IT-Firmen und IT-Experten, welche sich jetzt schon das fachliche Wissen aneignen, haben wahrscheinliche mit dem Wissensvorsprung gute Chancen, sich in einem wachsenden Markt positionieren zu können. 

Das Thema ist noch für manchen neu, für mich bleibt es in jedem Falle höchst spannend die Entwicklung weiter zu verfolgen. 