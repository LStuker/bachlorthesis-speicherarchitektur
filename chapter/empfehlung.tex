%!TEX root=../documentation-bachlorthesis-speicherarchitektur-lstucker.tex
\cleardoublepage
\chapter{Empfehlung für den Auftraggeber}

Der Evaluations-Sieger ist bei beiden gesetzten Szenerien Amazon S3. Weshalb Amazon S3 in beiden Evaluationen die meisten Punkte erhielt, liegt unter anderem daran, dass der Speicher verbrauchsabhängig abgerechnet wird. Damit entfallen die Kosten für den Reservespeicher für die zukünftige Speicherbenutzung. Dies senkt die Kosten substantiell und führt dazu, dass sich diese Alternative für beide Szenarien gegenüber den Konkurrenten durchsetzen konnte, da sie mit dem Leistungskonzept auch alle anderen Anforderungen zuverlässig erfüllen konnte.

Es mag nicht überraschen, dass auch andere Anbieter und Web-2.0 Startup Unternehmen wie Dropbox, SmugMug, swisstopo, litmus, Mendeley, ElephantDrive, fotopedia und viele mehr sich für die Speicherlösung von Amazon S3 entschieden haben.

Die Verfügbarkeit der Daten stellt Amazon S3 durch eine Architektur sicher, welche alle System-Komponenten hoch-redundant eingebunden hat. So sind Daten im Amazon S3 Speicher mindestens dreifach-redundant gespeichert und über mehrere Datencenters verteilt gespeichert.

Ein Nachteil bei Amazon S3 sind die Kosten für die ausgehende Bandbreite. Wird eine Bild für den Druck aufbereitet, muss das Bild zuerst auf den Applikations-Server übertragen werden. Werden viele Bilder umgerechnet, können erhebliche Kosten die Folge sein. Ferner benötigt die Übertragung von grossen Dateien über das Internet im Vergleich zum internen Netzwerk mehr Benutzer-Wartezeit, was die Bearbeitungszeit pro Bild belastet. Amazon bietet neu die Möglichkeit, Rechenleistung in ihrem Netz zu mieten. Dies könnte eine solide Antwort für den beschriebenen Nachteil sein und ist zu prüfen.

Das Resultat dieser Arbeit hat Amazon S3 als einzigen Online Speicher berücksichtigt. Dies bedeutet nicht, dass Amazon S3 alle Bedürfnisse des Auftraggeber am besten erfüllt. Neben Amazon S3 gibt es weitere Anbieter wie Rackspace, welche ebenfalls Online-Speicher anbieten. Bei der Auswahl des Dienstleisters ist neben den Kosten auf die Dienstgütevereinbarung (DGV), oder auch Service-Level-Agreement (SLA) genannt, zu achten. Eine genauere Untersuchung der SLAs ist in diesem Zusammenhang wesentlich für die Entscheidungsfindung, speziell hinsichtlich der gesetzlichen Anforderungen im nationalen und internationalen Umfeld. Ferner ist darauf zu achten, dass der Dienstleister selbst wirtschaftlich gesund und stabil für den zukünftigen Markt fit aufgestellt ist. 

Bei der Auswahl des Online Speicher Anbieters sollte darauf geachtet werden, dass diese die Amazon S3 API oder Swift API (OpenStack Object Storage) unterstützen, da diese sich quasi zum inoffiziellen Standard entwickelt haben, welche vermehrt auch von anderen Speicheranbietern unterstützt werden.
 
Einen Wechsel von der aktuellen Speicherlösung ist nicht empfehlenswert, wenn die Annahmen bezüglich den geforderten Werte für die Speicherkapazität und Bildverarbeitungsanfragen aus dem Szenario-1 nicht zutreffen bzw. nicht übersteigen. Die Gesamtinvestitionen und laufenden Kosten für die neue Lösung wären nicht gerechtfertigt.

Stellt der Auftraggeber fest, dass zwar die angegebenen Werte für das Szenario-1 übertroffen werden, jedoch bleibt unklar in welchem Ausmass dies zu erwarten ist, dann ist zu empfehlen, den Evaluationsgewinner als neue Speicherlösung einzusetzen, da dieser die Speicherkapazität bedarfsgerecht und äusserst flexibel zur Verfügung stellen kann und keine unnötigen Ausgaben auslöst.

Sollten die zu erwarteten Mengen bezüglich Speicherkapazität und Bildbearbeitungsanfragen die angenommenen Werte von Szenario-2 übertreffen, dann wäre ein Wechsel auf OpenStack Object Storage eine zu prüfende Alternative, zumal die API von OpenStack Object Storage zum Amazon S3 kompatibel ist. Zum Beispiel könnte mit Amazon S3 gestartet und später bei Bedarf auf eine eigene OpenStack Objekt Storage Infrastruktur gewechselt werden. Damit stehen dem Auftraggeber alle Optionen zur Verfügung, um den Kunden bedarfsgerecht die beste Speicherlösung für die Bildanwendung anbieten zu können.

