%!TEX root=../documentation-bachlorthesis-speicherarchitektur-lstucker.tex

\cleardoublepage
\chapter{Szenarien Beschreibung}
Der Auftraggeber hat gewünscht, bei der Evaluation der Speicherlösungen mehrere Szenarien in der Volumenentwicklung über die Zeit zu berücksichtigen. 

Die beschriebenen Szenarien sind mit den Auftraggeber besprochen worden. Die folgenden Annahmen wurden nicht aufgrund eines vorhandenen Geschäftsplan aufgebaut und können von den definierten Geschäftszielen des Auftraggebers abweichen. 


\section{Szenario-1 Schwaches Datenvolumen und geringes Wachstum im Datenzugriff}\label{Szenario1}
Bei Szenario-1 wird angenommen, dass sich der Kundenzuwachs, welche die Dienstleistung zur Speicherung und Aufbereitung Ihrer Bilddaten verwenden, marginal steigert. In diesen Fall beträgt das durchschnittliche Datenwachstum 0.25 Tebibyte pro Monat und bewegt sich in einem vergleichbaren Umfang wie im bisherigen Umfang, siehe Beschrieb des Ist-Zustands.


Datenwachstum in Tebibyte pro Monat: 0.25 Tebibyte

Bilder\footnote{Durchschnittsgrösse von 100 Mebibyte pro Bild} Zuwachs pro Monat: 4'690 Bilder

Speichervolumen in 36 Monaten: 11.5 Tebibyte

Anzahl Bilder\footnotemark[\value{footnote}] in 36 Monaten: 168'821 Bilder

\section{Szenario-2 Starkes Wachstum der Daten / starkes Wachstum der Abfragen}
Im Szenario-2 wird davon ausgegangen, dass die Kundenbasis im Vergleich zum Ist-Zustand stark anwächst. Durch die substantielle Zunahme an Neukunden beträgt der durchschnittliche Datenzuwachs 6 Tebibyte pro Monat.


Datenwachstum in Tebibyte pro Monat: 6 Tebibyte

Bilder\footnotemark[\value{footnote}] Zuwachs pro Monat: 12'583 Bilder

Speichervolumen in 36 Monaten: 218,5 Tebibyte

Anzahl Bilder\footnotemark[\value{footnote}] in 36 Monaten: 458'228 Bilder
 
