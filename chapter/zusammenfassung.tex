%!TEX root=../documentation-bachlorthesis-speicherarchitektur-lstucker.tex

\cleardoublepage
\chapter{Zusammenfassung}

Der Auftraggeber eine Zürcher Startup Unternehmen betreibt und Entwickelt eine Webapplikation zur Speicherung, Archivierung, Verwalten und Wiederverwenden von qualitativen hochauflösenden digitalen Bildern. Zu Ihrem Kundensegment gehören Galerien, Museen, Künstler und Fotographen die sehr hohe Ansprüche an die Qualität Ihrer Bilder haben. 
Die Reference Image AG erlaubt ihren Kunden die Speicherung von Bilder mit einer Speicherkapazität von bis zu 2 Gigabyte.

Bis anhin würden die Daten auf einen einzelnen Server gespeichert, auf welchen auch die Webapplikation betrieben wird. Die Daten sind auf dem Server mittels "Redundant Array of independent Disks" RAID-5 System vor dem Ausfall einer Festplatte geschützt. Die Ist-Analyse hat gezeigt, dass der Speicher bereits zu über 50\% ausgelastet ist und bei gleichbleibenden Datenwachstum innert sieben Monate ausgeschöpft ist. Weshalb sich ein Überdenken der Zukünftigen Speicherarchitektur aufdrängt.

Es wird daher eine Speichersystem zu finden welche die Anforderungen bezüglich Skalierbarkeit in der Speicherkapazität und Anzahl Datenabfragen, im Datendurchsatzt, in der Datenverfügbarkeit, in der Datenintegrität und in der Wirtschaftlichkeit erfüllt. Bei der Suche nach einen neuen Speichersystem 

Die heutigen am Markt erhältlichen Speicherarchitekturen lassen sich in der obersten Kategorie in Block- (Block-based), Datei- (File-based) und Objekt- (Object-based) basierte adressierende Systeme unterteilen. Die Kategorien zur Einteilung der verschiedenen Lösungen lässt sich nicht exakt zuordnen, da einige Speicherlösungen aus einem Mix aus mehreren Kategorien bestehenden können. Die Block-basierende Speicherarchitektur ist wohl die traditionellste von allen und ist die am weitverbreiteste Form. Sie kommt in Desktop-Computern, Server-Systeme, Mobile-Telefone und Weiter Systeme die Datenspeichern vor. Block-basierende Speicher können über FibreChannel oder iSCSI über eine Speichernetzwerk (SAN) an mehreren Server-Systeme zur Verfügung gestellt werden. Mit dem Aufkommen von Desktop-Computern, wuchs der Bedarf Dateien an einen Zentralen Ort ablegen zu können und auf diesen von allen Desktop-Computer zuzugreifen. Aus diesen Grund wurden Datei-basierende Speicherarchitekturen entwickelt die Ihren Speicher über das Netzwerk teilen können, dazu gehören Network File System (NFS) und Common Internet File System (CIFS). Objekt-basierende Speicher werden zunehmend bei Speicherlösungen eingesetzt,  wo der bedarf an Speicherkapazität sehr hoch ist, wo bedarf an Datenanalyse besteht und wo der Speicher überall verfügbar sein muss.

Der Auftraggeber hat gewünscht bei der Evaluation mehre Szenarien zu berücksichtigen, die sich aufgrund der Speicherkapazität unterscheidet. Aus diesen Grund wurde ein Szenarion mit der Maximal Speicherkapazität von ... festgelegt und eines mit der Maximal Speicherkapazität von ....

Für die Evaluation wurde das Analytic Hierarchy Process (AHP) Verfahren eingesetzt. Bei AHP wird durch den hierachischen Analyseprozess die Evaluation strukturiert und . Beim AHP werden alle Kriterien der selben Hierarchie mit gleichen Oberziel einzeln jeweils in Paarvergleiche mit einer vorgegebene Wertungsskale zu einander gewichtet. Durch Mathematische Berechnung ergibt sich aus allen Gewichtungen aus den Paarvergleiche eine genauere Gesamt Gewichtung jedes einzelnen Kriteriums im Verhältnis zu den anderen Kriteren. In einen weiteren Schritt werden die definierten Alternativen ebenfalls in Paarvergleiche zu den einzelnen Kriterium verglichen und mit der selben Wertungsskale bewertet. Durch die Mathematische Berechnung der Paarvergleiche und die Gewichtung der Kriterien ergibt sich am Ende eine Gesamt Bewertung aller Alternativen zu einander. Der Rechenaufwand nimmt mit zunehmender Anzahl an Alternativen und Kriterien zu, weshalb die Evaluation mit Software Unterstützung durchgeführt wurde.

Für die Evaluation wurde jeweils einen Vertreter der Speicherarchitektur Kategorieren Block-basierend, Datei-basierenden und Objekt-basierend, einen Vertreter von Online Speicher und einen Vertreter der bisherigen Speicherarchitektur gewählt. 

Als Vertreter für Block-basiernde Speicher wurde eine Speichersystem von Hersteller NetApp gewählt welches seinen Speicher über iSCSI den Applikations-Server zur Verfügung stellt. Als Vertretet für Datei-basierende Speicher wurde das selbe Speichersystem von NetApp gewählt, welches aber seinen Speicher über NFS den Applikations-Server zur Verfügung stellt. Als Vertreter von Objekt-basiernde Speicher wurde die ebenfalls für Online Speicher eingesetzte OpenStack Object Storage gewählt. OpenStack Object Storage verwendet für die Speicherung gewöhnliche Computer-Systeme und erreicht durch redundante Verteilung der Daten eine hohe Verfügbarkeit. Dabei sind alle Daten auf mindestens drei Computer-Systeme gespeichert. Als Vertreter von Online Speicher wurde Amazon S3 gewählt.


Für das erste Szenarion wurden wegen den hohen Kosten der anderen Alternativen, nur der dedizierte Server von Hetzer und Amazon zugelassen. Bei zweiten Szenario wurden alle Alternativen berücksichtigte, mit Ausnahme von Hetzner welche kein Angebot hatte welche die benötigte Speicherkapazität hat. 

Der Evaluation Gewinner in beider Szenerien ist Amazon S3. Amazon S3 konnte, wegen den niedrigen Gesamtkosten, der beliebig Skalierung der Speicherkapazität, der hohen Redundanz und die Sicherstellung der Integrität der Daten Punkten. Als zweiter bei Szenario zwei wurde NetApp iSCSI dicht gefolgt von OpenStack Object Storage, welcher aufgrund der hohen Personal Kosten keinen besseren Platz erzielte.

Der Evaluation Gewinner Amazon S3 stellt für den Auftraggeber eine gut Wahl dar. Es ermöglicht Ihm zu jeden Zeitpunkt beliebig zu wachsen und er muss keine hohe Investitons Kosten in den Speichertätigen. Einen Nachteil für den Auftraggeber von Amazon S3, könnte die notwendige Übertragung der Bilddaten über das Internet für die Bearbeitung der Bilder für den Druck sein. Weshalb für Ihn trotzt den höheren Kosten, aber durch die schnellere Übertragung der Bilddaten zum Applikations-Server, die Lösung OpenStack Object Storage eine Alternative sein kann, wenn er durch Automation die Personal Kosten reduzieren kann.