\cleardoublepage
\chapter{Installation OpenStack Object Storage}
\section{Generelle Konfiguration}

\begin{lstlisting}[label=paketegenerell, language=Bash, caption=Installation generelle Host Pakete ]
sudo apt-get install swift openssh-server  rsync memcached python-netifaces python-xattr python-memcache 
\end{lstlisting}

\begin{lstlisting}[label=swiftconfdir, language=Bash, caption=Erstell /etc/swift Ordner und setzt Berechtigung ]
sudo mkdir -p /etc/swift
sudo chown -R swift:swift /etc/swift/
\end{lstlisting}


\begin{lstlisting}[label=swift.conf language=Bash, caption=Swift in /etc/swift/swift.conf konfigurieren]
[swift-hash]
# random unique string that can never change (DO NOT LOSE)
swift_hash_path_suffix = KmwBretYgombitrL
\end{lstlisting}

\section{Storage Nodes Konfiguration}

\begin{lstlisting}[label=paketenode, language=Bash, caption=Installation Data-Node Pakete ]
sudo apt-get install swift-account swift-container swift-object xfsprogs
\end{lstlisting}

begin{lstlisting}[label=partition, language=Bash, caption=Partition auf zweiter Festplatte erstellen ]
sudo fdisk /dev/sdb 
Device contains neither a valid DOS partition table, nor Sun, SGI or OSF disklabel
Building a new DOS disklabel with disk identifier 0xd5f71779.
Changes will remain in memory only, until you decide to write them.
After that, of course, the previous content won't be recoverable.

Warning: invalid flag 0x0000 of partition table 4 will be corrected by w(rite)

Command (m for help): m
Command action
   a   toggle a bootable flag
   b   edit bsd disklabel
   c   toggle the dos compatibility flag
   d   delete a partition
   l   list known partition types
   m   print this menu
   n   add a new partition
   o   create a new empty DOS partition table
   p   print the partition table
   q   quit without saving changes
   s   create a new empty Sun disklabel
   t   change a partition's system id
   u   change display/entry units
   v   verify the partition table
   w   write table to disk and exit
   x   extra functionality (experts only)

Command (m for help): n
Partition type:
   p   primary (0 primary, 0 extended, 4 free)
   e   extended
Select (default p): p
Partition number (1-4, default 1): 1
First sector (2048-41943039, default 2048): 
Using default value 2048
Last sector, +sectors or +size{K,M,G} (2048-41943039, default 41943039): 
Using default value 41943039

Command (m for help): w
The partition table has been altered!

Calling ioctl() to re-read partition table.
Syncing disks.
\end{lstlisting}

\begin{lstlisting}[label=xfs, language=Bash, caption=XFS Dateisystem in der Partition anlegen ]
sudo mkfs.xfs -i size=1024 /dev/sdb1
meta-data=/dev/sdb1              isize=1024   agcount=4, agsize=1310656 blks
         =                       sectsz=512   attr=2, projid32bit=0
data     =                       bsize=4096   blocks=5242624, imaxpct=25
         =                       sunit=0      swidth=0 blks
naming   =version 2              bsize=4096   ascii-ci=0
log      =internal log           bsize=4096   blocks=2560, version=2
         =                       sectsz=512   sunit=0 blks, lazy-count=1
realtime =none                   extsz=4096   blocks=0, rtextents=0
\end{lstlisting}

\begin{lstlisting}[label=mountconfig, language=Bash, caption=Mount Konfiguration anlegen ]
sudo echo "/dev/sdb1 /srv/node/sdb1 xfs noatime,nodiratime,nobarrier,logbufs=8 0 0" >> /etc/fstab
\end{lstlisting}

\begin{lstlisting}[label=MountPointDisk, language=Bash, caption=Mount Anlegen, Disk Mounten und Berechtigung setzen]
sudo mkdir -p /srv/node/sdb1
sudo mount /srv/node/sdb1
sudo chown -R swift:swift /srv/node
\end{lstlisting}

\begin{lstlisting}[label=rsyncConf, language=Bash, caption=Rsyncd in /etc/rsyncd.conf konfigurieren]
uid = swift
gid = swift
log file = /var/log/rsyncd.log
pid file = /var/run/rsyncd.pid
address = 172.16.251.90 

[account]
max connections = 2
path = /srv/node/
read only = false
lock file = /var/lock/account.lock

[container]
max connections = 2
path = /srv/node/
read only = false
lock file = /var/lock/container.lock

[object]
max connections = 2
path = /srv/node/
read only = false
lock file = /var/lock/object.lock
\end{lstlisting}

\begin{lstlisting}[label=configRsyncDefault, language=Bash, caption=Rsync Aktivieren in /etc/default/rsync]
RSYNC_ENABLE = true
\end{lstlisting}

\begin{lstlisting}[label=startRsynx, language=Bash, caption=Rsync Starten]
sudo service rsync start
\end{lstlisting}

\begin{lstlisting}[label=account-server.conf, language=Bash, caption=Account Server in /etc/swift/account-server.conf konfigurieren]
[DEFAULT]
bind_ip = 172.16.251.90  # IP Addresse des Node Servers
workers = 2

[pipeline:main]
pipeline = account-server

[app:account-server]
use = egg:swift#account

[account-replicator]

[account-auditor]

[account-reaper]
\end{lstlisting}


\begin{lstlisting}[label=container-server.conf, language=Bash, caption=Container Server in /etc/swift/container-server.conf konfigurieren]
[DEFAULT]
bind_ip = 172.16.251.90 # IP Addresse des Node Servers
workers = 2

[pipeline:main]
pipeline = container-server

[app:container-server]
use = egg:swift#container

[container-replicator]

[container-updater]

[container-auditor]
\end{lstlisting}


\begin{lstlisting}[label=object-server.conf, language=Bash, caption=Object Server in /etc/swift/object-server.conf konfigurieren]
[DEFAULT]
bind_ip = 172.16.251.90 # IP Addresse des Node Servers
workers = 2

[pipeline:main]
pipeline = object-server

[app:object-server]
use = egg:swift#object

[object-replicator]

[object-updater]

[object-auditor]
\end{lstlisting}

\section{Proxy Node Konfiguration}
\begin{lstlisting}[label=paketeProxy, language=Bash, caption=Installation Proxy-Node Pakete ]
sudo apt-get install swift-proxy memcached
\end{lstlisting}

\begin{lstlisting}[label=df, language=Bash, caption=X.509 Zertifikats in /etc/swift erstellen ]
sudo openssl req -new -x509 -nodes -out cert.crt -keyout cert.key
Generating a 1024 bit RSA private key
......++++++
.......................................++++++
writing new private key to 'cert.key'
-----
You are about to be asked to enter information that will be incorporated
into your certificate request.
What you are about to enter is what is called a Distinguished Name or a DN.
There are quite a few fields but you can leave some blank
For some fields there will be a default value,
If you enter '.', the field will be left blank.
-----
Country Name (2 letter code) [AU]:CH
State or Province Name (full name) [Some-State]:Zuerich
Locality Name (eg, city) []:Wallisellen
Organization Name (eg, company) [Internet Widgits Pty Ltd]:Stuker.biz
Organizational Unit Name (eg, section) []:Cloud Storage
Common Name (e.g. server FQDN or YOUR name) []:swift-c1.swift.stuker.biz
Email Address []:it@stuker.biz
\end{lstlisting}

\begin{lstlisting}[label=memcachedconf, language=Bash, caption=MemCached in /etc/memcached.conf konfigurieren ]
-l 127.0.0.1
in ändern
-l 172.16.251.80
\end{lstlisting}

\begin{lstlisting}[label=startMemcached, language=Bash, caption=Memcached starten]
sudo service memcached restart
\end{lstlisting}

\begin{lstlisting}[label=ring, language=Bash, caption=Account Container und Object Ring erstellen]
sudo swift-ring-builder account.builder create 18 3 1
sudo swift-ring-builder container.builder create 18 3 1
sudo swift-ring-builder object.builder create 18 3 1
\end{lstlisting}

\begin{lstlisting}[label=ring, language=Bash, caption=Server bzw. Speicher den Ringen hinzufügen]
sudo swift-ring-builder account.builder add z1-172.16.251.90:6002/sdb1 100
 sudo swift-ring-builder account.builder add z2-172.16.251.91:6002/sdb1 100
sudo swift-ring-builder account.builder add z3-172.16.251.92:6002/sdb1 100
sudo swift-ring-builder account.builder add z4-172.16.251.93:6002/sdb1 100
sudo swift-ring-builder account.builder add z5-172.16.251.94:6002/sdb1 100
sudo swift-ring-builder container.builder add z1-172.16.251.90:6001/sdb1 100
sudo swift-ring-builder container.builder add z2-172.16.251.91:6001/sdb1 100
sudo swift-ring-builder container.builder add z3-172.16.251.92:6001/sdb1 100
sudo swift-ring-builder container.builder add z4-172.16.251.93:6001/sdb1 100
sudo swift-ring-builder container.builder add z5-172.16.251.94:6001/sdb1 100
sudo swift-ring-builder object.builder add z1-172.16.251.90:6000/sdb1 100
sudo swift-ring-builder object.builder add z2-172.16.251.91:6000/sdb1 100
sudo swift-ring-builder object.builder add z3-172.16.251.92:6000/sdb1 100
sudo swift-ring-builder object.builder add z4-172.16.251.93:6000/sdb1 100
sudo swift-ring-builder object.builder add z5-172.16.251.94:6000/sdb1 100
sudo swift-ring-builder account.builder
sudo swift-ring-builder container.builder
sudo swift-ring-builder object.builder
\end{lstlisting}

\begin{lstlisting}[label=ring, language=Bash, caption=Ring rebalance]
sudo swift-ring-builder account.builder rebalance
sudo swift-ring-builder container.builder rebalance
sudo swift-ring-builder object.builder rebalance
\end{lstlisting}

\begin{lstlisting}[label=ring, language=Bash, caption=Dateien account.ring.gz, container.ring.gz, und object.ring.gz an alle Nodes verteilen und Berechtigung setzen]
  scp *.gz data01-c1-d1:/etc/swift
  scp *.gz data02-c1-d1:/etc/swift  
  scp *.gz data03-c1-d1:/etc/swift
  scp *.gz data04-c1-d2:/etc/swift
  scp *.gz data05-c1-d2:/etc/swift
  sudo chown -R swift:swift /etc/swift
\end{lstlisting}

\begin{lstlisting}[label=ring, language=Bash, caption=Proxy Dienst starten]
swift-init proxy start
\end{lstlisting}

\begin{lstlisting}[label=ring, language=Bash, caption=Auf allen Nodes die Dienste starten]
swift-init all start
\end{lstlisting}