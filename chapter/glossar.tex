%!TEX root=../documentation-bachlorthesis-speicherarchitektur-lstucker.tex

\newglossaryentry{RFC}{ name={RFC}, description={Request for Comments (RFC) sind Dokumente, über Internet, inklusive der technischen Spezifikation und Richtlinien, welche von der Organisation Internet Engineering Task Force entwickelt wurde. "'Das RFC wird erst nach erfolgter Diskussion unter der Aussicht des Internet Architecture Board (IAB) herausgegeben und fungiert als Quasistandard. Jedes RFC enthält eine eindeutige, vorlaufende Nummer, die kein zweites Mal zu gewiesen wird."' \cite{MicrosoftComputerLex}  \url{http://www.rfc-editor.org/}}}

\newglossaryentry{UDP}{ name={UDP}, description={  adfajsdfjadslkfjaödjfölaksdjfajsklfj }}

\newglossaryentry{IBM}{ name={IBM}, description={International Business Machines Corporation (kurz IBM) ist ein führendes unternehmen in Software, Hardware und IT-Dienstleistung Bereich }}

\newglossaryentry{HP}{ name={HP}, description={Hewlett-Packard Company (kurz HP), ist das umsatzstärkste IT-Unternehmen der Welt }}

\newglossaryentry{EMC}{ name={EMC}, description={EMC Corporation (kurz EMC), ist einer der Führenden Disk Array Speicher Hersteller }}

\newglossaryentry{Dell}{ name={Dell}, description={Dell Computer Corporation (kurz Dell), ist ein der grössten IT-Unternehmen der Welt }}

\newglossaryentry{CIFS}{ name={CIFS}, description={Common Internet File System (kurz CIFS) wurde 1996 von Microsoft eingeführt und beschreibt eine erweiterte Version von SMB. CIFS und SMB sind eine  Netzwerkdateisystem vergleichbar mit NFS und wird vorwiegend im MS Windows Bereich eingesetzt}}

\newglossaryentry{SSH}{ name={SSH}, description={Secure Shell (kurz. SSH) ist ein Programm bzw. Protokoll, welche es ermöglicht über eine Verschlüsselte Verbindung in einen entfernten Rechner über das Netzwerk bzw. Internet sich anzumelden und dort auf dem Rechner Kommandos auszuführen}}

\newglossaryentry{Primearen-Daten}{ name={Primären-Daten}, description={Die Primären-Daten sind die Orginal-Daten, auf welches das Rechensystem Zugriff hat, um die Daten auszulesen oder zu manipulieren}}

\newglossaryentry{IO}{ name={I/O}, description={IO ist die Englische Abkürzung für Input/Output was für Eingabe/Ausgabe steht. Unter Eingabe/Ausgabe versteht man die Kommunikation eines Information System. Zum Beispiel wird die Kommunikation von einer Festplatten mit dem Kontroller als Eingabe/Ausgabe bezeichnet}}

\newglossaryentry{XDR}{ name={XDR}, description={Die eXternal Data Representation (kurz XDR) Spezifikation stellt ein Standardisierte Verfahren zur Präsentation von gebräuchlichsten Daten Typen über das Netzwerk zur Verfügung. Dies löst das Problem der verschiedenen Byte-Reihenfolge (Big Endian), Speicherausrichtung auf unterschiedlichen Kommunikations Partner}}

\newglossaryentry{Hosting}{ name={Hosting}, description={Hosting versteht man die Unterbringung von Internetprojekten, die sich in der Regel auch öffentlich durch das Internet abrufen lassen. Diese Aufgabe übernehmen Internet-Dienstleistungsanbieter (Provider) die Web-Speicher, Datenbanken, E-Mail-Adressen und weitere Produkte anbieten und zum Austausch von Daten durch das Internet dienen.\url{https://de.wikipedia.org/wiki/Hosting} }}

\newglossaryentry{Provider}{ name={Hosting}, description={Provider zu deutsch auch Internetdienstanbieter oder Internetdienstleister  sind Anbieter von Diensten, Inhalten oder technischen Leistungen, die für die Nutzung oder den Betrieb von Inhalten und Diensten im Internet erforderlich sind.\url{https://de.wikipedia.org/wiki/Internetdienstanbieter} }}

\newglossaryentry{POSIX}{ name={POSIX}, description={Portable Operating System Interface (kurz POSIX) ist eine von IEEE entwickelter Standard, welche die Schnittstelle zwischen Applikation und Betriebsystem darstellt. Die aktuelle Version des Standards ist IEEE Std 1003.1-2008 \url{http://www.opengroup.org/austin/papers/posix_faq.html} }}

\newglossaryentry{POSIXIO}{ name={POSIX-IO}, description={POSIX IO (kein Offizellername) ist der Teil des POSIX Standard welche die IO Schnittstelle definiert}}

\newglossaryentry{FUSE}{ name={FUSE}, description={Filesystem in Userspace (kurz FUSE), ermöglicht die Implementierung eines voll Funktionsfähigen Dateisystem in Userspace. Normaler weise laufen 
FUSE wurde urspünglich Entwickelt um AVFS zu unterstützen, ist jedoch heute ein seperates Projekt. \url{http://fuse.sourceforge.net/}}} 

\newglossaryentry{RPC}{ name={RPC}, description={File locking erlaubt es einen Prozess den exklusiven Zugriff auf eine Datei oder teile einer Datei und zwingt ander Prozesse die auf die selbe Ressource zugreifen wollen zu warten bis das Locking aufgehoben wurde.} 

\newglossaryentry{FileLocking}{ name={FileLocking}, description={File locking erlaubt es einen Prozess den exklusiven Zugriff auf eine Datei oder teile einer Datei und zwingt ander Prozesse die auf die selbe Ressource zugreifen wollen zu warten bis das Locking aufgehoben wurde.} 

\newglossaryentry{API}{ name={API}, description={Application Programming Interface (kurz API) auch Anwendungsprogrammierschnittstelle genannt. "'Ein Satz an Routinen, die vom Betriebsystem des Computers für die Verwendung aus Anwendungsprogrammen heraus angeboten werden und diverse Dienste zur Verfügung stellen."' \cite{MicrosoftComputerLex} }}

\newglossaryentry{MIT{ name={MIT}, description={Die MIT Lizenz stammt von Massachusetts Institute of Technology und erlaub die die Verwendung von Software welche Quelloffen ist als auch software welche nicht Quell geschlossene ist. Die genauen Lizenz Bestimmungen sind unter folgenden URL zu finden \url{http://www.opensource.org/licenses/mit-license.php}}}

\newglossaryentry{GNU GPL{ name={GNU GPL}, description={Die GNU General Public License Lizenz auch GPL genannt stammt von der Free Software Foundation und regelt die Lizenzierung von Freie Software. Es gibt drei Versionen der GPL welche unter folgenden URL beschrieben sind \url{http://www.gnu.org/licenses/}}}

\newglossaryentry{Ruby{ name={Ruby}, description={Ruby ist eine interpretierte und objektorientierte Programmiersprache und beinhaltet einige bewährte Prinzipien wie z.B. "'DuckTyping"' und "'Principle of Least Suprice"'. Die Entwickler von Ruby stellen sich selber den Anspruch eine Programmiersprache zu schaffen, die durch Ihre Natürlichkeit einfach erlernbar ist und es den Programmierern ermöglicht, einfachen und übersichtlichen Code zu schreiben, welcher aber nicht seine Mächtigkeit und innere Komplexität verliert.
Ruby hat sich in den letzten Jahren von einer kaum beachteten Programmiersprache zu einem Publikums-Magneten entwickelt. Es gibt eine stetig wachsende offene Community "'Gemeinschaft"', welche sich und die Sprache durch Austausch von Erfahrungen und Ideen weiterbringen möchte.
Ein Grund für die hohe Bereitschaft der Community die Sprache Ruby weiter zu bringen ist der Umstand, dass die Programmiersprache vollständig OpenSource ist und unter der Lizenz der Ruby-License und GPL steht. Zudem ist die Sprache fast beliebig erweiterbar und bestehende Funktionen können einfach durch eigene Funktionen ausgetauscht werden}}

